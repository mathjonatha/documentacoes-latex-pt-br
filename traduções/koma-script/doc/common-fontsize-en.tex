% ======================================================================
% common-fontsize-en.tex
% Copyright (c) Markus Kohm, 2001-2022
%
% This file is part of the LaTeX2e KOMA-Script bundle.
%
% This work may be distributed and/or modified under the conditions of
% the LaTeX Project Public License, version 1.3c of the license.
% The latest version of this license is in
%   http://www.latex-project.org/lppl.txt
% and version 1.3c or later is part of all distributions of LaTeX
% version 2005/12/01 or later and of this work.
%
% This work has the LPPL maintenance status "author-maintained".
%
% The Current Maintainer and author of this work is Markus Kohm.
%
% This work consists of all files listed in MANIFEST.md.
% ======================================================================
%
% Paragraphs that are common for several chapters of the KOMA-Script guide
% Maintained by Markus Kohm
%
% ======================================================================

\KOMAProvidesFile{common-fontsize-en.tex}
                 [$Date: 2022-10-06 14:30:20 +0200 (Do, 06. Okt 2022) $
                  KOMA-Script guide (common paragraphs: fontsize)]
\translator{Markus Kohm\and Krickette Murabayashi\and Karl Hagen}

\section{Escolhendo o Tamanho da Fonte do Documento}
\seclabel{fontOptions}%
\BeginIndexGroup
\BeginIndex{}{font>size}%

\IfThisCommonFirstRun{%
  A fonte principal e seu tamanho são elementos centrais no design de um
  documento. Como afirmado no \autoref{cha:typearea}, a divisão da página
  entre a área de texto e as margens depende fundamentalmente deles. A fonte
  principal é aquela usada para a maior parte do texto em um documento. Todas
  as variações, seja em forma, espessura, inclinação ou tamanho, são
  relacionadas à fonte principal.%
}{%
  A informação na \autoref{sec:\ThisCommonFirstLabelBase.fontOptions}
  aplica-se igualmente a
  \IfThisCommonLabelBase{scrlttr2}{\Class{scrlttr2}\OnlyAt{scrlttr2}}%
  {este capítulo}. \IfThisCommonLabelBase{scrlttr2}{Por outro lado, o
    pacote \Package{scrletter} por si só não oferece seleção de tamanho de
    fonte, mas depende completamente da classe que você usa.}{} Portanto, se
  você já leu e entendeu a \autoref{sec:\ThisCommonFirstLabelBase.fontOptions},
  você pode \IfThisCommonLabelBase{scrlttr2}{continuar na
    \autopageref{sec:\ThisCommonLabelBase.fontOptions.end} no final desta
    seção. Se você usa \Package{scrletter}, você pode }{}%
  pular diretamente para a \autoref{sec:\ThisCommonLabelBase.fontOptions.next},
  \autopageref{sec:\ThisCommonLabelBase.fontOptions.next}.%
}

\begin{Declaration}
  \OptionVName{fontsize}{tamanho}
\end{Declaration}
Enquanto\IfThisCommonLabelBase{scrlttr2}{\OnlyAt{\Class{scrlttr2}}}{%
  \textnote{\KOMAScript{} vs. classes padrão}} as classes padrão suportam
apenas um número muito limitado de tamanhos de fonte,
\IfThisCommonLabelBase{scrlttr2}{\Class{scrlttr2}}{\KOMAScript} fornece a
habilidade de especificar qualquer \PName{tamanho} para a fonte principal.
Você também pode usar qualquer unidade \TeX{} conhecida como unidade para o
\PName{tamanho}. Se o \PName{tamanho} for especificado sem uma unidade,
assume-se que seja \PValue{pt}.\iffree{}{ O procedimento exato para
  configurar o tamanho da fonte está documentado para especialistas e usuários
  interessados na \autoref{sec:maincls-experts.addInfos},
  \DescPageRef{maincls-experts.option.fontsize}.}

Se você definir a opção dentro do documento, o tamanho da fonte principal e os
tamanhos de fonte dependentes dos comandos \Macro{tiny}, \Macro{scriptsize},
\Macro{footnotesize}, \Macro{small}, \Macro{normalsize}, \Macro{large},
\Macro{Large}, \Macro{LARGE}, \Macro{huge} e \Macro{Huge} são alterados. Isso
pode ser útil, por exemplo, se você quiser %
\IfThisCommonLabelBase{scrlttr2}{outra carta }{o apêndice }%
configurada em um tamanho de fonte menor.

Observe\textnote{Atenção!} que usar esta opção após
\IfThisCommonLabelBase{scrextend}{potencialmente carregar
  \hyperref[cha:typearea]{\Package{typearea}}\IndexPackage{typearea}%
  \important{\hyperref[cha:typearea]{\Package{typearea}}}}{carregar a classe}
não recalcula automaticamente a área de texto e as margens (veja
\DescRef{typearea.cmd.recalctypearea}\IndexCmd{recalctypearea},
\autoref{sec:typearea.typearea},
\DescPageRef{typearea.cmd.recalctypearea}). No entanto, se este recálculo for
executado, ele será baseado no tamanho atual da fonte principal. Os efeitos da
mudança do tamanho da fonte principal sobre outros pacotes carregados ou a
classe usada dependem desses pacotes e da classe.
\IfThisCommonLabelBase{maincls}{%
  Isso significa que você pode encontrar erros que não são culpa do
  \KOMAScript, e mesmo as próprias classes \KOMAScript{} não recalculam todos
  os comprimentos se o tamanho da fonte principal mudar após o carregamento da
  classe.%
}{%
  \IfThisCommonLabelBase{scrlttr2}{%
    Você pode encontrar erros que não são culpa do \KOMAScript{}, e além
    disso, a própria classe \Class{scrlttr2} não recalcula todos os
    comprimentos se o tamanho da fonte principal mudar após o carregamento da
    classe.%
  }{%
    Isso significa que você pode encontrar erros que não são culpa do
    \KOMAScript{}.%
  }%
}%

Esta\textnote{Atenção!} opção de modo algum deve ser interpretada como um
substituto para \Macro{fontsize} (veja \cite{latex:fntguide}). Além disso,
você não deve usá-la no lugar de um dos comandos de tamanho de fonte que são
relativos à fonte principal, de \Macro{tiny} a \Macro{Huge}. O uso dentro de
um parágrafo é, portanto, explicitamente proibido.
\IfThisCommonLabelBase{scrlttr2}{%
  Para \Class{scrlttr2} o padrão é \OptionValue{fontsize}{12pt}.

  \begin{Example}
    \phantomsection\label{sec:\ThisCommonLabelBase.fontOptions.end}%
    Suponha que a organização na carta de exemplo seja os \emph{``Amigos de
      Tamanhos de Fonte Nocivos''}, razão pela qual ela deveria ser
    configurada em 14\Unit{pt} em vez de 12\Unit{pt}. Você pode conseguir
    isso fazendo uma pequena alteração na primeira linha:%
    \lstinputcode[xleftmargin=1em]{letter-example-06-en.tex}%
    Alternativamente, a opção poderia ser definida como um argumento opcional
    para \DescRef{\LabelBase.env.letter}:
    \lstinputcode[xleftmargin=1em]{letter-example-05-en.tex}%
    Como a área de texto não é recalculada nesta mudança tardia do tamanho da
    fonte, os dois resultados diferem na \autoref{fig:scrlttr2.letter-05-06}.
    \begin{figure}
      \centering
      \frame{\includegraphics[width=.4\textwidth]{letter-example-05-en}}\quad
      \frame{\includegraphics[width=.4\textwidth]{letter-example-06-en}}
      \caption[{Exemplo: carta com endereço, saudação, texto, frase de
        encerramento, pós-escrito, anexos, lista de distribuição e tamanho de
        fonte nocivamente grande}]
      {resultado de uma carta curta com destinatário, abertura, texto,
      encerramento, pós-escrito, anexos, lista de distribuição e uma fonte
      nocivamente grande (a data é definida por padrão): na versão à
      esquerda, o tamanho da fonte foi definido pelo argumento opcional de
      \DescRef{\LabelBase.env.letter}; na da direita, foi usado o argumento
      opcional de \DescRef{\LabelBase.cmd.documentclass}}
      \label{fig:scrlttr2.letter-05-06}
    \end{figure}
  \end{Example}
  \ExampleEndFix
}{%
  \IfThisCommonLabelBase{maincls}{%
    \par
    \phantomsection\label{sec:\ThisCommonLabelBase.fontOptions.end}%
    O padrão para \Class{scrbook}, \Class{scrreprt} e \Class{scrartcl} é
    \OptionValue{fontsize}{11pt}.\textnote{\KOMAScript{} vs. classes padrão}
    Em contraste, o tamanho padrão nas classes padrão é \Option{10pt}.
    Você pode precisar considerar essa diferença se mudar de uma classe
    padrão para uma classe \KOMAScript{}\iffree{}{ ou ao usar a opção
      \DescRef{maincls-experts.option.emulatestandardclasses}%
      \IndexOption{emulatestandardclasses}}.%
  }{}%
}%
%
\EndIndexGroup
%
\EndIndexGroup


%%% Local Variables:
%%% mode: latex
%%% TeX-master: "scrguide-en.tex"
%%% coding: utf-8
%%% ispell-local-dictionary: "en_GB"
%%% eval: (flyspell-mode 1)
%%% End:
