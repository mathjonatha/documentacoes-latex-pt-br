% ======================================================================
% common-draftmode-en.tex
% Copyright (c) Markus Kohm, 2001-2022
%
% This file is part of the LaTeX2e KOMA-Script bundle.
%
% This work may be distributed and/or modified under the conditions of
% the LaTeX Project Public License, version 1.3c of the license.
% The latest version of this license is in
%   http://www.latex-project.org/lppl.txt
% and version 1.3c or later is part of all distributions of LaTeX
% version 2005/12/01 or later and of this work.
%
% This work has the LPPL maintenance status "author-maintained".
%
% The Current Maintainer and author of this work is Markus Kohm.
%
% This work consists of all files listed in MANIFEST.md.
% ======================================================================
%
% Paragraphs that are common for several chapters of the KOMA-Script guide
% Maintained by Markus Kohm
%
% ======================================================================

\KOMAProvidesFile{common-draftmode-en.tex}
                 [$Date: 2022-06-05 12:40:11 +0200 (So, 05. Jun 2022) $
                  KOMA-Script guide (common paragraphs)]
\translator{Markus Kohm\and Gernot Hassenpflug\and Krickette Murabayashi\and
  Karl Hagen}

\section{Modo de Rascunho}
\seclabel{draft}%
\BeginIndexGroup
\BeginIndex{}{modo de rascunho}%

\IfThisCommonFirstRun{}{%
  As informações em \autoref{sec:\ThisCommonFirstLabelBase.draft} aplicam-se
  igualmente a \IfThisCommonLabelBase{scrlttr2}{\Class{scrlttr2}%
	\OnlyAt{\Class{scrlttr2}}}{este capítulo}. Portanto, se você já leu
  e compreendeu \autoref{sec:\ThisCommonFirstLabelBase.draft}, pode pular
  diretamente para \autoref{sec:\ThisCommonLabelBase.draft.next} na
  \autopageref{sec:\ThisCommonLabelBase.draft.next}.%
  \IfThisCommonLabelBase{scrlttr2}{ O pacote \Package{scrletter} não
  	fornece um modo de rascunho próprio, mas depende da classe que você usa.}{}%
}

\IfThisCommonLabelBase{scrextend}{}{%
  Muitas classes e pacotes fornecem um modo de rascunho além do modo
  de composição normal. As diferenças entre esses dois são tão diversas quanto as
  classes e pacotes que oferecem essa distinção.%
  \IfThisCommonLabelBase{scrextend}{% Umbruchkorrekturtext
    \ O modo de rascunho de alguns pacotes também resulta em mudanças na saída
    que afetam o layout do documento. Esse não é o caso com
    \Package{scrextend}.%
  }{}%
}

\begin{Declaration}
  \OptionVName{draft}{chave simples}
  \OptionVName{overfullrule}{chave simples}
\end{Declaration}%
A opção \Option{draft}\IfThisCommonLabelBase{maincls}{%
  \ChangedAt{v3.00}{\Class{scrbook}\and \Class{scrartcl}\and
    \Class{scrreprt}}%
}{%
  \IfThisCommonLabelBase{scrlttr2}{%
    \ChangedAt{v3.00}{\Class{scrlttr2}}\OnlyAt{\Class{scrlttr2}}%
  }{}%
} distingue entre documentos sendo rascunhados e documentos
finalizados\Index{versão final}. A \PName{chave simples} pode ser um dos
valores padrão para chaves simples de \autoref{tab:truefalseswitch},
\autopageref{tab:truefalseswitch}. Se você ativar esta
opção\important{\OptionValue{draft}{true}}, pequenas caixas pretas serão exibidas
no final de linhas excessivamente longas. Essas caixas facilitam para o olho
destreinado localizar os parágrafos que requerem pós-processamento manual. Por outro lado,
o padrão, \OptionValue{draft}{false}, não mostra tais caixas. Aliás,
essas linhas frequentemente desaparecem quando você usa o
pacote \Package{microtype}\IndexPackage{microtype}\important{\Package{microtype}}
\cite{package:microtype}.

Já que\IfThisCommonLabelBase{maincls}{%
  \ChangedAt{v3.25}{\Class{scrbook}\and \Class{scrartcl}\and
    \Class{scrreprt}}%
}{%
  \IfThisCommonLabelBase{scrlttr2}{%
    \ChangedAt{v3.25}{\Class{scrlttr2}}%
  }{%
    \IfThisCommonLabelBase{scrextend}{%
      \ChangedAt{v3.25}{\Package{scrextend}}%
    }{}%
  }%
} a opção \Option{draft} pode levar a todo tipo de efeitos indesejados com
vários pacotes, o \KOMAScript{} permite que você controle essa marcação de linhas
excessivamente longas separadamente com a
opção \Option{overfullrule}\important{\OptionValue{overfullrule}{true}}. Se
esta opção estiver habilitada, o marcador é exibido novamente.%
%
\EndIndexGroup
%
\EndIndexGroup

%%% Local Variables:
%%% mode: latex
%%% TeX-master: "scrguide-en.tex"
%%% coding: utf-8
%%% ispell-local-dictionary: "en_GB"
%%% eval: (flyspell-mode 1)
%%% End:
