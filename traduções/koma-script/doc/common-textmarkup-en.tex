% ======================================================================
% common-textmarkup-en.tex
% Copyright (c) Markus Kohm, 2001-2025
%
% This file is part of the LaTeX2e KOMA-Script bundle.
%
% This work may be distributed and/or modified under the conditions of
% the LaTeX Project Public License, version 1.3c of the license.
% The latest version of this license is in
%   http://www.latex-project.org/lppl.txt
% and version 1.3c or later is part of all distributions of LaTeX
% version 2005/12/01 or later and of this work.
%
% This work has the LPPL maintenance status "author-maintained".
%
% The Current Maintainer and author of this work is Markus Kohm.
%
% This work consists of all files listed in MANIFEST.md.
% ======================================================================
%
% Paragraphs that are common for several chapters of the KOMA-Script guide
% Maintained by Markus Kohm
%
% ======================================================================

\KOMAProvidesFile{common-textmarkup-en.tex}
                 [$Date: 2025-02-12 11:47:33 +0100 (Mi, 12. Feb 2025) $
                  KOMA-Script guide (common paragraphs)]
\translator{Gernot Hassenpflug\and Markus Kohm\and Krickette Murabayashi\and
	Karl Hagen}

\section{Marcação de Texto}
\seclabel{textmarkup}%
\BeginIndexGroup%
\BeginIndex{}{text>markup}%
\BeginIndex{}{font}%

\IfThisCommonFirstRun{}{%
  As informações em \autoref{sec:\ThisCommonFirstLabelBase.textmarkup}
  aplicam-se amplamente a este capítulo. Portanto, se você já leu e compreendeu
  \autoref{sec:\ThisCommonFirstLabelBase.textmarkup}, você pode
  \IfThisCommonLabelBaseOneOf{scrextend,scrjura,scrlayer-notecolumn}{}{%
    limitar-se a examinar
    \autoref{tab:\ThisCommonLabelBase.fontelements},
    \autopageref{tab:\ThisCommonLabelBase.fontelements} e então }%
  pular para \autoref{sec:\ThisCommonLabelBase.textmarkup.next},
  \autopageref{sec:\ThisCommonLabelBase.textmarkup.next}.%
  \IfThisCommonLabelBase{scrextend}{\ Neste caso, porém,
    note\textnote{limitação} que \Package{scrextend} suporta apenas os
    elementos para o título do documento, o dictum, as notas de rodapé, e o
    ambiente \DescRef{maincls.env.labeling} de
    \autoref{tab:maincls.fontelements},
    \autopageref{tab:maincls.fontelements}. Embora o
    elemento \DescRef{maincls.fontelement.disposition} exista,
    \Package{scrextend} o utiliza apenas para o título do documento.%
  }{}%
}

% Umbruchkorrektur
\IfThisCommonLabelBaseOneOf{scrlayer-scrpage,scrextend}{}{%
  {\LaTeX} oferece diferentes possibilidades para marcação lógica e direta
  de texto\Index{logical markup}\Index{markup}. %
  \IfThisCommonLabelBaseOneOf{scrlttr2}{}{%
    Além da escolha da fonte, isso inclui comandos para escolher
    o tamanho e orientação da fonte. %
  } Para mais informações sobre os recursos de fonte padrão, consulte
  \cite{lshort}, \cite{latex:usrguide}, e \cite{latex:fntguide}.%
}

% Both commands are in LaTeX for a long time, so I do not describe them any
% longer despite \textsubscript is still defined by KOMA-Script if needed.
%\IfThisCommonLabelBaseOneOf{scrlayer-scrpage,scrjura,scrlayer-notecolumn}{%
%  \csname iftrue\endcsname}%
  \begin{Declaration}
    \Macro{textsuperscript}\Parameter{text}%
    \Macro{textsubscript}\Parameter{text}
  \end{Declaration}
  O núcleo do \LaTeX{} define o comando
  \Macro{textsuperscript}\IndexCmd{textsuperscript} para colocar texto em
  sobrescrito\Index{text>superscript}\Index{superscript}. Infelizmente,
  o próprio \LaTeX{}\textnote{\Latex~2015/01/01} não oferecia um comando para
  produzir texto em subscrito\Index{text>subscript}\Index{subscript} até
  a versão 2015/01/01. \KOMAScript{} define \Macro{textsubscript} para este
  propósito. %
  \ifthiscommonfirst
    \begin{Example}
      \phantomsection
      \xmpllabel{cmd.textsubscript}%
      Você está escrevendo um texto sobre metabolismo humano. De tempos em tempos,
      você precisa fornecer algumas fórmulas químicas simples nas quais os números
      estão em subscrito. Para permitir marcação lógica, você primeiro define no
      preâmbulo do documento ou em um pacote separado:
\begin{lstcode}
  \newcommand*{\molec}[2]{#1\textsubscript{#2}}
\end{lstcode}
      \newcommand*{\molec}[2]{#1\textsubscript{#2}}
      Usando isso você então escreve:
\begin{lstcode}
  A célula produz sua energia parcialmente da reação de \molec C6\molec
  H{12}\molec O6 e \molec O2 para produzir \molec H2\Molec O{} e
  \molec C{}\molec O2.  No entanto, arsênio (\molec{As}{}) tem um
  efeito bastante prejudicial no metabolismo.
\end{lstcode}
      A saída fica assim:
      \begin{ShowOutput}
        A célula produz sua energia parcialmente da reação de \molec C6\molec
        H{12}\molec O6 e \molec O2 para produzir \molec H2\molec O{} e
        \molec C{}\molec O2.  No entanto, arsênio (\molec{As}{}) tem um
        efeito bastante prejudicial no metabolismo.
      \end{ShowOutput}

      Algum tempo depois você decide que as fórmulas químicas devem ser
      compostas em fonte sem serifa. Agora você pode ver as vantagens de usar
      marcação lógica. Você só precisa redefinir o comando \Macro{molec}:
\begin{lstcode}
  \newcommand*{\molec}[2]{\textsf{#1\textsubscript{#2}}}
\end{lstcode}
      \renewcommand*{\molec}[2]{\textsf{#1\textsubscript{#2}}}
      Agora a saída em todo o documento muda para:
      \begin{ShowOutput}
        A célula produz sua energia parcialmente da reação de \molec
        C6\molec H{12}\molec O6 e \molec O2 para produzir \molec H2\molec
        O{} e \molec C{}\molec O2.  No entanto, arsênio (\molec{As}{}) tem
        um efeito bastante prejudicial no metabolismo.
      \end{ShowOutput}
    \end{Example}
    \iftrue
      \begin{Explain}
        O exemplo acima usa a notação ``\verb|\molec C6|''.
        Isso faz uso do fato de que argumentos consistindo de apenas um
        caractere não precisam ser colocados entre parênteses. É por isso que
        ``\verb|\molec C6|'' é semelhante a ``\verb|\molec{C}{6}|''. Você
        pode já estar familiarizado com essa notação de índices ou potências em
        ambientes matemáticos, como ``\verb|$x^2$|'' em vez de
        ``\verb|$x^{2}$|''
        para ``$x^2$''.
      \end{Explain}
    \else % maybe some time I've made an English book
      Usuários avançados podem encontrar informações sobre a razão pela qual o exemplo acima
      funciona a menos que você coloque todos os argumentos de \Macro{molec} entre chaves em
      \autoref{sec:experts.knowhow},
      \DescPageRef{experts.macroargs}.%
    \fi%
  \else%
    Você pode encontrar um exemplo que o utiliza em
    \autoref{sec:\ThisCommonFirstLabelBase.textmarkup},
    \PageRefxmpl{\ThisCommonFirstLabelBase.cmd.textsubscript}.
  \fi%
  \EndIndexGroup%
\fi


\IfThisCommonLabelBaseOneOf{scrlayer-scrpage,scrjura,scrlayer-notecolumn}{%
  \iffalse%
}{%
  \csname iftrue\endcsname%
}%
  \begin{Declaration}
    \OptionVName{sfdefaults}{simple switch}%
    \Macro{maybesffamily}%
    \Macro{textmaybesf}\Parameter{text}%
  \end{Declaration}
  Os%
  \IfThisCommonLabelBase{maincls}{%
    \ChangedAt{v3.39}{\Class{scrbook}\and \Class{scrreprt}\and
      \Class{scrartcl}}%
  }{%
    \IfThisCommonLabelBase{scrextend}{%
      \ChangedAt{v3.39}{\Package{screxend}}%
    }{%
      \IfThisCommonLabelBase{scrlttr2}{%
        \ChangedAt{v3.39}{\Class{scrlttr2}}%
      }{}%
    }%
  } comandos \Macro{maybesffamily} e \Macro{textmaybesf} comportam-se de forma diferente
  dependendo da configuração da opção \Option{sfdefaults}. Um dos valores
  padrão para chaves simples de \autoref{tab:truefalseswitch} pode ser usado
  aqui. Somente se esta opção estiver habilitada \Macro{maybesffamily} resultará em
  \Macro{sffamily} e \Macro{textmaybesf} usará \Macro{textsf}. Esta é também
  a configuração padrão. O próprio KOMA-Script usa \Macro{maybesffamily} nas
  configurações padrão dos elementos \IfThisCommonLabelBase{scrextend}{}{%
    \DescRef{\LabelBase.fontelement.descriptionlabel}, %
  }%
  \IfThisCommonLabelBase{scrlttr2}{%
    \DescRef{\LabelBase.fontelement.backaddress}, %
    \DescRef{\LabelBase.fontelement.refname}, %
    e \DescRef{\LabelBase.fontelement.lettertitle}%
  }{%
    \IfThisCommonLabelBase{scrextend}{%
      \DescRef{maincls.fontelement.disposition} %
    }{%
      \DescRef{\LabelBase.fontelement.disposition}, %
    }%
    e \DescRef{\LabelBase.fontelement.dictum}%
  }. \Macro{maybesffamily} pode assim também ser usado como parte dos
  \PName{commands} das instruções \DescRef{\LabelBase.cmd.setkomafont} e
  \DescRef{\LabelBase.cmd.addtokomafont} explicadas abaixo.%
  \EndIndexGroup%
  \fi


\begin{Declaration}
  \Macro{setkomafont}\Parameter{element}\Parameter{commands}%
  \Macro{addtokomafont}\Parameter{element}\Parameter{commands}%
  \Macro{usekomafont}\Parameter{element}
\end{Declaration}%
Com%
\IfThisCommonLabelBase{maincls}{%
  \ChangedAt{v2.8p}{\Class{scrbook}\and \Class{scrreprt}\and
    \Class{scrartcl}}%
}{} a ajuda dos comandos \Macro{setkomafont} e \Macro{addtokomafont},
você pode anexar \PName{commands} de estilo de fonte particulares que alteram
a aparência de um dado \PName{element}. Teoricamente, todas as instruções,
incluindo texto literal, podem ser usadas como \PName{commands}. Você
deve\textnote{Atenção!}, no entanto, limitar-se àquelas instruções que
realmente mudam apenas atributos de fonte. Estes são geralmente comandos como
\Macro{rmfamily}, \Macro{sffamily}, \Macro{ttfamily}, \Macro{upshape},
\Macro{itshape}, \Macro{slshape}, \Macro{scshape}, \Macro{mdseries},
\Macro{bfseries}, \Macro{normalfont}, bem como os comandos de tamanho de fonte
\Macro{Huge}, \Macro{huge}, \Macro{LARGE}, \Macro{Large}, \Macro{large},
\Macro{normalsize}, \Macro{small}, \Macro{footnotesize}, \Macro{scriptsize},
e \Macro{tiny}. Você pode encontrar estes comandos explicados em \cite{lshort},
\cite{latex:usrguide}, ou \cite{latex:fntguide}. Comandos de mudança de cor
como \Macro{normalcolor} (veja \cite{package:graphics} e
\cite{package:xcolor}) também são aceitáveis.%
\iffalse % Umbruchkorrekturtext
  \ O comportamento ao usar outros comandos, especialmente aqueles que levam a
  redefinições ou geram saída, é indefinido. Comportamento estranho é possível
  e não representa um bug.
\else
  \ O uso de outros comandos, em particular aqueles que redefinem coisas ou
  levam a saída, não é suportado. Comportamento estranho é possível nestes
  casos e não representa um bug.
\fi

O comando \Macro{setkomafont} fornece a um elemento uma definição completamente nova
de seu estilo de fonte. Em contraste, o comando \Macro{addtokomafont}
meramente estende uma definição existente. Você não deve usar nenhum comando
dentro do corpo do documento mas apenas no preâmbulo. Para exemplos de seu uso,
consulte as seções para o respectivo elemento.%
\IfThisCommonLabelBase{scrlayer-notecolumn}{}{%
  \ O nome e significado de cada elemento
  \IfThisCommonLabelBaseOneOf{scrlayer-scrpage,scrjura}{, bem como seus
    padrões,}{} estão listados em \IfThisCommonLabelBase{scrextend}{%
    \autoref{tab:maincls.fontelements}, \autopageref{tab:maincls.fontelements}
  }{%
    \autoref{tab:\ThisCommonLabelBase.fontelements} %
  }.%
  \IfThisCommonLabelBase{scrextend}{ %
    No entanto, em \Package{scrextend} apenas\textnote{limitação} os elementos listados
    para o título do documento, dictum, notas de rodapé, e o
    ambiente \DescRef{maincls.env.labeling} são suportados. Embora o
    elemento \DescRef{maincls.fontelement.disposition} exista,
    \Package{scrextend} o usa apenas para o título do documento.%
  }{%
    \IfThisCommonLabelBase{scrlayer-scrpage}{ %
      Os padrões especificados aplicam-se apenas se o elemento correspondente não
      já foi definido antes de carregar \Package{scrlayer-scrpage}. Por
      exemplo, as classes \KOMAScript{} definem
      \DescRef{maincls.fontelement.pageheadfoot}, e então
      \Package{scrlayer-scrpage} usa a configuração que encontra.%
    }{%
      \IfThisCommonLabelBase{scrjura}{}{ %
        Os valores padrão podem ser encontrados nas seções correspondentes.%
      }%
    }%
  }%
}%

\IfThisCommonLabelBaseOneOf{scrlttr2,scrextend}{% Umbruchvarianten
  O comando \Macro{usekomafont} pode ser usado para mudar o estilo de fonte atual
  para o \PName{Element} especificado.%
}{%
  Com o comando \Macro{usekomafont}, o estilo de fonte atual pode ser alterado
  para aquele definido para o \PName{element} especificado.%
}

\IfThisCommonLabelBase{maincls}{\iftrue}{\csname iffalse\endcsname}
  \begin{Example}
    \phantomsection\xmpllabel{cmd.setkomafont}%
    Suponha que você queira usar a mesma especificação de fonte para o elemento
    \DescRef{\ThisCommonLabelBase.fontelement.captionlabel}
    que é usada com
    \DescRef{\ThisCommonLabelBase.fontelement.descriptionlabel}. Isso pode ser
    facilmente feito com:
\begin{lstcode}
  \setkomafont{captionlabel}{%
    \usekomafont{descriptionlabel}%
  }
\end{lstcode}
    Você pode encontrar outros exemplos na explicação de cada elemento.
  \end{Example}

  \begin{desclist}
    \desccaption{%
      Elementos cujo estilo de fonte pode ser alterado em \Class{scrbook},
      \Class{scrreprt} ou \Class{scrartcl} com \Macro{setkomafont} e
      \Macro{addtokomafont}%
      \label{tab:maincls.fontelements}%
      \label{tab:scrextend.fontelements}%
    }{%
      Elementos cujo estilo de fonte pode ser alterado (\emph{continuação})%
    }%
    \feentry{author}{%
      \ChangedAt{v3.12}{\Class{scrbook}\and \Class{scrreprt}\and
        \Class{scrartcl}\and \Package{scrextend}}%
      autor do documento no título, i.\,e., o argumento de
      \DescRef{\ThisCommonLabelBase.cmd.author} quando
      \DescRef{\ThisCommonLabelBase.cmd.maketitle} é usado (veja
      \autoref{sec:maincls.titlepage}, \DescPageRef{\ThisCommonLabelBase.cmd.author})}%
    \feentry{caption}{texto de uma legenda de figura ou tabela (veja
      \autoref{sec:maincls.floats}, \DescPageRef{\ThisCommonLabelBase.cmd.caption})}%
    \feentry{captionlabel}{rótulo de uma legenda de figura ou tabela; aplicado em
      adição ao elemento \DescRef{\ThisCommonLabelBase.fontelement.caption}
      (veja \autoref{sec:maincls.floats},
      \DescPageRef{\ThisCommonLabelBase.cmd.caption})}%
    \feentry{chapter}{título do comando de seccionamento
      \DescRef{\ThisCommonLabelBase.cmd.chapter} (veja
      \autoref{sec:maincls.structure}, \DescPageRef{\ThisCommonLabelBase.cmd.chapter})}%
    \feentry{chapterentry}{%
      entrada no sumário para o comando de seccionamento
      \DescRef{\ThisCommonLabelBase.cmd.chapter} (veja
      \autoref{sec:maincls.toc}, \DescPageRef{\ThisCommonLabelBase.cmd.tableofcontents})}%
    \feentry{chapterentrydots}{%
      \ChangedAt{v3.15}{\Class{scrbook}\and \Class{scrreprt}}%
      pontos opcionais conectando entradas do sumário para o
      nível \DescRef{\ThisCommonLabelBase.cmd.chapter}, diferindo do
      elemento \DescRef{\ThisCommonLabelBase.fontelement.chapterentry},
      \Macro{normalfont} e \Macro{normalsize} (veja
      \autoref{sec:maincls.toc}, \DescPageRef{\ThisCommonLabelBase.cmd.tableofcontents})}%
    \feentry{chapterentrypagenumber}{%
      número de página da entrada do sumário para o comando de seccionamento
      \DescRef{\ThisCommonLabelBase.cmd.chapter}, diferindo do elemento
      \DescRef{\ThisCommonLabelBase.fontelement.chapterentry} (veja
      \autoref{sec:maincls.toc}, \DescPageRef{\ThisCommonLabelBase.cmd.tableofcontents})}%
    \feentry{chapterprefix}{%
      rótulo, e.\,g., ``Capítulo'', aparecendo antes do número do capítulo em ambos
      \OptionValueRef{maincls}{chapterprefix}{true} e
      \OptionValueRef{maincls}{appendixprefix}{true} (veja
      \autoref{sec:maincls.structure},
      \DescPageRef{\ThisCommonLabelBase.option.chapterprefix})}%
    \feentry{date}{%
      \ChangedAt{v3.12}{\Class{scrbook}\and \Class{scrreprt}\and
        \Class{scrartcl}\and \Package{scrextend}}%
      data do documento no título principal, i.\,e., o argumento de
      \DescRef{\ThisCommonLabelBase.cmd.date} quando
      \DescRef{\ThisCommonLabelBase.cmd.maketitle} é usado (veja
      \autoref{sec:maincls.titlepage}, \DescPageRef{\ThisCommonLabelBase.cmd.date})}%
    \feentry{dedication}{%
      \ChangedAt{v3.12}{\Class{scrbook}\and \Class{scrreprt}\and
        \Class{scrartcl}\and \Package{scrextend}}%
      página de dedicatória após o título principal, i.\,e., o argumento de
      \DescRef{\ThisCommonLabelBase.cmd.dedication} quando
      \DescRef{\ThisCommonLabelBase.cmd.maketitle} é usado (veja
      \autoref{sec:maincls.titlepage}, \DescPageRef{\ThisCommonLabelBase.cmd.dedication})}%
    \feentry{descriptionlabel}{rótulos, i.\,e., o argumento opcional de
      \DescRef{\ThisCommonLabelBase.cmd.item.description} no
      ambiente \DescRef{\ThisCommonLabelBase.env.description} (veja
      \autoref{sec:maincls.lists}, \DescPageRef{\ThisCommonLabelBase.env.description})}%
    \feentry{dictum}{dictum ou epígrafe (veja \autoref{sec:maincls.dictum},
      \DescPageRef{\ThisCommonLabelBase.cmd.dictum})}%
    \feentry{dictumauthor}{autor de um dictum ou epígrafe; aplicado em adição
      ao elemento \DescRef{\ThisCommonLabelBase.fontelement.dictum} (veja
      \autoref{sec:maincls.dictum}, \DescPageRef{\ThisCommonLabelBase.cmd.dictum})}%
    \feentry{dictumtext}{nome alternativo para
      \DescRef{\ThisCommonLabelBase.fontelement.dictum}}%
    \feentry{disposition}{todos os títulos de comando de seccionamento, i.\,e., os argumentos
      de \DescRef{\ThisCommonLabelBase.cmd.part} até
      \DescRef{\ThisCommonLabelBase.cmd.subparagraph} e
      \DescRef{\ThisCommonLabelBase.cmd.minisec}, incluindo o título do
      resumo; aplicado antes do elemento da respectiva unidade (veja
      \autoref{sec:maincls.structure}, \autopageref{sec:maincls.structure})}%
    \feentry{enumeratelabel}{%
      \ChangedAt{v3.44}{\Class{scrbook}\and \Class{scrreprt}\and
        \Class{scrartcl}}%
      Padrão para os números predefinidos do ambiente
      \DescRef{\ThisCommonLabelBase.env.enumerate} (veja
      \autoref{sec:maincls.lists}, \DescPageRef{\ThisCommonLabelBase.env.enumerate})}%
    \feentry{footnote}{texto e marcador de nota de rodapé (veja
      \autoref{sec:maincls.footnotes}, \DescPageRef{\ThisCommonLabelBase.cmd.footnote})}%
    \feentry{footnotelabel}{marcador para uma nota de rodapé; aplicado em adição ao
      elemento \DescRef{\ThisCommonLabelBase.fontelement.footnote} (veja
      \autoref{sec:maincls.footnotes}, \DescPageRef{\ThisCommonLabelBase.cmd.footnote})}%
    \feentry{footnotereference}{referência de nota de rodapé no texto (veja
      \autoref{sec:maincls.footnotes}, \DescPageRef{\ThisCommonLabelBase.cmd.footnote})}%
    \feentry{footnoterule}{%
      régua horizontal\ChangedAt{v3.07}{\Class{scrbook}\and
        \Class{scrreprt}\and \Class{scrartcl}} acima das notas de rodapé no final
      da área de texto (veja \autoref{sec:maincls.footnotes},
      \DescPageRef{\ThisCommonLabelBase.cmd.setfootnoterule})}%
    \feentry{itemizelabel}{%
      \ChangedAt{v3.33}{\Class{scrbook}\and \Class{scrreprt}\and
        \Class{scrartcl}}%
      Padrão para os símbolos predefinidos do ambiente
      \DescRef{\ThisCommonLabelBase.env.itemize} (veja
      \autoref{sec:maincls.lists}, \DescPageRef{\ThisCommonLabelBase.env.itemize})}%
    \feentry{labelenumi}{%
      \ChangedAt{v3.44}{\Class{scrbook}\and \Class{scrreprt}\and
        \Class{scrartcl}}%
      Fonte a ser usada na definição do número do item
      \DescRef{\ThisCommonLabelBase.cmd.labelenumi} (veja
      \autoref{sec:maincls.lists}, \DescPageRef{\ThisCommonLabelBase.env.enumerate})}%
    \feentry{labelenumii}{%
      \ChangedAt{v3.44}{\Class{scrbook}\and \Class{scrreprt}\and
        \Class{scrartcl}}%
      Fonte a ser usada na definição do número do item
      \DescRef{\ThisCommonLabelBase.cmd.labelenumii} (veja
      \autoref{sec:maincls.lists}, \DescPageRef{\ThisCommonLabelBase.env.enumerate})}%
    \feentry{labelenumiii}{%
      \ChangedAt{v3.44}{\Class{scrbook}\and \Class{scrreprt}\and
        \Class{scrartcl}}%
      Fonte a ser usada na definição do número do item
      \DescRef{\ThisCommonLabelBase.cmd.labelenumiii} (veja
      \autoref{sec:maincls.lists}, \DescPageRef{\ThisCommonLabelBase.env.enumerate})}%
    \feentry{labelenumiv}{%
      \ChangedAt{v3.44}{\Class{scrbook}\and \Class{scrreprt}\and
        \Class{scrartcl}}%
      Fonte a ser usada na definição do número do item
      \DescRef{\ThisCommonLabelBase.cmd.labelenumiv} (veja
      \autoref{sec:maincls.lists}, \DescPageRef{\ThisCommonLabelBase.env.enumerate})}%
    \feentry{labelinglabel}{rótulos, i.\,e., o argumento opcional de
      \DescRef{\ThisCommonLabelBase.cmd.item.labeling} no
      ambiente \DescRef{\ThisCommonLabelBase.env.labeling} (veja
      \autoref{sec:maincls.lists}, \DescPageRef{\ThisCommonLabelBase.env.labeling})}%
    \feentry{labelingseparator}{separador, i.\,e., o argumento opcional do
      ambiente \DescRef{\ThisCommonLabelBase.env.labeling}; aplicado em
      adição ao elemento
      \DescRef{\ThisCommonLabelBase.fontelement.labelinglabel} (veja
      \autoref{sec:maincls.lists}, \DescPageRef{\ThisCommonLabelBase.env.labeling})}%
    \feentry{labelitemi}{%
      \ChangedAt{v3.33}{\Class{scrbook}\and \Class{scrreprt}\and
        \Class{scrartcl}}%
      Fonte a ser usada na definição do símbolo do item
      \DescRef{\ThisCommonLabelBase.cmd.labelitemi} (veja
      \autoref{sec:maincls.lists}, \DescPageRef{\ThisCommonLabelBase.env.itemize})}%
    \feentry{labelitemii}{%
      \ChangedAt{v3.33}{\Class{scrbook}\and \Class{scrreprt}\and
        \Class{scrartcl}}%
      Fonte a ser usada na definição do símbolo do item
      \DescRef{\ThisCommonLabelBase.cmd.labelitemii} (veja
      \autoref{sec:maincls.lists}, \DescPageRef{\ThisCommonLabelBase.env.itemize})}%
    \feentry{labelitemiii}{%
      \ChangedAt{v3.33}{\Class{scrbook}\and \Class{scrreprt}\and
        \Class{scrartcl}}%
      Fonte a ser usada na definição do símbolo do item
      \DescRef{\ThisCommonLabelBase.cmd.labelitemiii} (veja
      \autoref{sec:maincls.lists}, \DescPageRef{\ThisCommonLabelBase.env.itemize})}%
    \feentry{labelitemiv}{%
      \ChangedAt{v3.33}{\Class{scrbook}\and \Class{scrreprt}\and
        \Class{scrartcl}}%
      Fonte a ser usada na definição do símbolo do item
      \DescRef{\ThisCommonLabelBase.cmd.labelitemiv} (veja
      \autoref{sec:maincls.lists}, \DescPageRef{\ThisCommonLabelBase.env.itemize})}%
    \feentry{minisec}{título de \DescRef{\ThisCommonLabelBase.cmd.minisec} (veja
      \autoref{sec:maincls.structure} ab \DescPageRef{\ThisCommonLabelBase.cmd.minisec})}%
    \feentry{pagefoot}{usado apenas se o pacote \Package{scrlayer-scrpage} foi
      carregado (veja \autoref{cha:scrlayer-scrpage},
      \DescPageRef{scrlayer-scrpage.fontelement.pagefoot})}%
    \feentry{pagehead}{nome alternativo para
      \DescRef{\ThisCommonLabelBase.fontelement.pageheadfoot}}%
    \feentry{pageheadfoot}{o cabeçalho e rodapé de uma página (veja
      \autoref{sec:maincls.pagestyle} de
      \autopageref{sec:maincls.pagestyle})}%
    \feentry{pagenumber}{número de página no cabeçalho ou rodapé (veja
      \autoref{sec:maincls.pagestyle})}%
    \feentry{pagination}{nome alternativo para
      \DescRef{\ThisCommonLabelBase.fontelement.pagenumber}}%
    \feentry{paragraph}{título do comando de seccionamento
      \DescRef{\ThisCommonLabelBase.cmd.paragraph} (veja
      \autoref{sec:maincls.structure}, \DescPageRef{\ThisCommonLabelBase.cmd.paragraph})}%
    \feentry{part}{título do comando de seccionamento \DescRef{\ThisCommonLabelBase.cmd.part},
      sem a linha contendo o número da parte (veja
      \autoref{sec:maincls.structure}, \DescPageRef{\ThisCommonLabelBase.cmd.part})}%
    \feentry{partentry}{%
      entrada do sumário para o comando de seccionamento
      \DescRef{\ThisCommonLabelBase.cmd.part} (veja \autoref{sec:maincls.toc},
      \DescPageRef{\ThisCommonLabelBase.cmd.tableofcontents})}%
    \feentry{partentrypagenumber}{%
      número de página da entrada do sumário para o comando de seccionamento
      \DescRef{\ThisCommonLabelBase.cmd.part}; aplicado em adição ao
      elemento \DescRef{\ThisCommonLabelBase.fontelement.partentry} (veja
      \autoref{sec:maincls.toc}, \DescPageRef{\ThisCommonLabelBase.cmd.tableofcontents})}%
    \feentry{partnumber}{linha contendo o número da parte em um título do
      comando de seccionamento \DescRef{\ThisCommonLabelBase.cmd.part} (veja
      \autoref{sec:maincls.structure}, \DescPageRef{\ThisCommonLabelBase.cmd.part})}%
    \feentry{publishers}{%
      \ChangedAt{v3.12}{\Class{scrbook}\and \Class{scrreprt}\and
        \Class{scrartcl}\and \Package{scrextend}}%
      editores do documento no título principal, i.\,e., o argumento de
      \DescRef{\ThisCommonLabelBase.cmd.publishers} quando
      \DescRef{\ThisCommonLabelBase.cmd.maketitle} é usado (veja
      \autoref{sec:maincls.titlepage}, \DescPageRef{\ThisCommonLabelBase.cmd.publishers})}%
    \feentry{section}{título do comando de seccionamento
      \DescRef{\ThisCommonLabelBase.cmd.section} (veja
      \autoref{sec:maincls.structure}, \DescPageRef{\ThisCommonLabelBase.cmd.section})}%
    \feentry{sectionentry}{%
      entrada do sumário para o comando de seccionamento
      \DescRef{\ThisCommonLabelBase.cmd.section} (disponível apenas em
      \Class{scrartcl}, veja \autoref{sec:maincls.toc},
      \DescPageRef{\ThisCommonLabelBase.cmd.tableofcontents})}%
    \feentry{sectionentrydots}{%
      \ChangedAt{v3.15}{\Class{scrartcl}}%
      pontos opcionais conectando entradas do sumário para o
      nível \DescRef{\ThisCommonLabelBase.cmd.section}, diferindo do
      elemento \DescRef{\ThisCommonLabelBase.fontelement.sectionentry},
      \Macro{normalfont} e \Macro{normalsize} (disponível apenas em
      \Class{scrartcl}, veja \autoref{sec:maincls.toc},
      \DescPageRef{\ThisCommonLabelBase.cmd.tableofcontents})}%
    \feentry{sectionentrypagenumber}{%
      número de página da entrada do sumário para o comando de seccionamento
      \DescRef{\ThisCommonLabelBase.cmd.section}; aplicado em adição ao
      elemento \DescRef{\ThisCommonLabelBase.fontelement.sectionentry} (disponível
      apenas em \Class{scrartcl}, veja \autoref{sec:maincls.toc},
      \DescPageRef{\ThisCommonLabelBase.cmd.tableofcontents})}%
    \feentry{sectioning}{nome alternativo para
      \DescRef{\ThisCommonLabelBase.fontelement.disposition}}%
    \feentry{subject}{%
      tópico do documento, i.\,e., o argumento de
      \DescRef{\ThisCommonLabelBase.cmd.subject} na página de título principal (veja
      \autoref{sec:maincls.titlepage}, \DescPageRef{\ThisCommonLabelBase.cmd.subject})}%
    \feentry{subparagraph}{título do comando de seccionamento
      \DescRef{\ThisCommonLabelBase.cmd.subparagraph} (veja
      \autoref{sec:maincls.structure},
      \DescPageRef{\ThisCommonLabelBase.cmd.subparagraph})}%
    \feentry{subsection}{título do comando de seccionamento
      \DescRef{\ThisCommonLabelBase.cmd.subsection} (veja
      \autoref{sec:maincls.structure}, \DescPageRef{\ThisCommonLabelBase.cmd.subsection})}%
    \feentry{subsubsection}{título do comando de seccionamento
      \DescRef{\ThisCommonLabelBase.cmd.subsubsection} (veja
      \autoref{sec:maincls.structure},
      \DescPageRef{\ThisCommonLabelBase.cmd.subsubsection})}%
    \feentry{subtitle}{%
      subtítulo do documento, i.\,e., o argumento de
      \DescRef{\ThisCommonLabelBase.cmd.subtitle} na página de título principal (veja
      \autoref{sec:maincls.titlepage}, \DescPageRef{\ThisCommonLabelBase.cmd.title})}%
    \feentry{title}{título principal do documento, i.\,e., o argumento de
      \DescRef{\ThisCommonLabelBase.cmd.title} (para detalhes sobre o tamanho do título
      veja a nota adicional no texto de
      \autoref{sec:maincls.titlepage} de \DescPageRef{\ThisCommonLabelBase.cmd.title})}%
    \feentry{titlehead}{%
      \ChangedAt{v3.12}{\Class{scrbook}\and \Class{scrreprt}\and
        \Class{scrartcl}\and \Package{scrextend}}%
      cabeçalho acima do título principal do documento, i.\,e., o argumento de
      \DescRef{\ThisCommonLabelBase.cmd.titlehead} quando
      \DescRef{\ThisCommonLabelBase.cmd.maketitle} é usado (veja
      \autoref{sec:maincls.titlepage}, \DescPageRef{\ThisCommonLabelBase.cmd.titlehead})}%
  \end{desclist}
\else
  \IfThisCommonLabelBase{scrextend}{\iftrue}{\csname iffalse\endcsname}
    \begin{Example}
      Suponha que você queira imprimir o título do documento em uma fonte serifada vermelha.
      Você pode fazer isso usando:
\begin{lstcode}
  \setkomafont{title}{\color{red}}
\end{lstcode}
      Você precisará do pacote \Package{color} ou do pacote \Package{xcolor} para
      o comando \Macro{color}\PParameter{red}. Usar \Macro{normalfont} é
      desnecessário neste caso porque já faz parte da definição do
      próprio título. Este\textnote{Atenção!} exemplo também precisa da
      opção \OptionValueRef{scrextend}{extendedfeature}{title} (veja
      \autoref{sec:scrextend.optionalFeatures},
      \DescPageRef{scrextend.option.extendedfeature}).
    \end{Example}
  \else
    \IfThisCommonLabelBase{scrlttr2}{%
      Você pode encontrar um exemplo geral que usa tanto \Macro{setkomafont} quanto
      \Macro{usekomafont} em \autoref{sec:maincls.textmarkup} em
      \PageRefxmpl{maincls.cmd.setkomafont}.

      \begin{desclist}
        \desccaption{%
          Elementos cujo estilo de fonte pode ser alterado na classe \Class{scrlttr2}
          ou no pacote \Package{scrletter} com os comandos
          \Macro{setkomafont} e \Macro{addtokomafont}%
          \label{tab:scrlttr2.fontelements}%
        }{%
          Elementos cujo estilo de fonte pode ser alterado (\emph{continuação})%
        }%
        \feentry{addressee}{nome e endereço do destinatário no campo de endereço
          (\autoref{sec:scrlttr2.firstpage},
          \DescPageRef{\ThisCommonLabelBase.option.addrfield})}%
        \feentry{backaddress}{%
          endereço de retorno para envelope com janela
          (\autoref{sec:scrlttr2.firstpage},
          \DescPageRef{\ThisCommonLabelBase.option.backaddress})}%
        \feentry{descriptionlabel}{%
          rótulo, i.\,e. o argumento opcional de
          \DescRef{\ThisCommonLabelBase.cmd.item.description}, em um
          ambiente \DescRef{\ThisCommonLabelBase.env.description}
          (\autoref{sec:scrlttr2.lists},
          \DescPageRef{\ThisCommonLabelBase.env.description})}%
        \feentry{enumeratelabel}{%
          \ChangedAt{v3.44}{\Class{scrlttr2}}%
          Padrão para os números predefinidos do ambiente
          \DescRef{\ThisCommonLabelBase.env.enumerate} (veja
          \autoref{sec:scrlttr2.lists}, \DescPageRef{\ThisCommonLabelBase.env.enumerate})}%
        \feentry{foldmark}{%
          marca de dobra na página do cabeçalho da carta; permite mudança de cor da linha
          (\autoref{sec:scrlttr2.firstpage},
          \DescPageRef{\ThisCommonLabelBase.option.foldmarks})}%
        \feentry{footnote}{%
          texto e marcador de nota de rodapé (\autoref{sec:scrlttr2.footnotes},
          \DescPageRef{\ThisCommonLabelBase.cmd.footnote})}%
        \feentry{footnotelabel}{%
          marcador de nota de rodapé; aplicado em adição ao
          elemento \DescRef{\ThisCommonLabelBase.fontelement.footnote}
          (\autoref{sec:scrlttr2.footnotes},
          \DescPageRef{\ThisCommonLabelBase.cmd.footnote})}%
        \feentry{footnotereference}{%
          referência de nota de rodapé no texto (\autoref{sec:scrlttr2.footnotes},
          \DescPageRef{\ThisCommonLabelBase.cmd.footnote})}%
        \feentry{footnoterule}{%
          régua horizontal\ChangedAt{v3.07}{\Class{scrlttr2}} acima das
          notas de rodapé no final da área de texto
          (\autoref{sec:maincls.footnotes},
          \DescPageRef{\ThisCommonLabelBase.cmd.setfootnoterule})}%
        \feentry{fromaddress}{%
          endereço do remetente no cabeçalho da carta
          (\autoref{sec:scrlttr2.firstpage},
          \DescPageRef{\ThisCommonLabelBase.variable.fromaddress})}%
        \feentry{fromname}{%
          nome do remetente no cabeçalho da carta, não incluindo \PValue{fromaddress}
          (\autoref{sec:scrlttr2.firstpage},
          \DescPageRef{\ThisCommonLabelBase.variable.fromname})}%
        \feentry{fromrule}{%
          régua horizontal no cabeçalho da carta; destinada para mudanças de cor
          (\autoref{sec:scrlttr2.firstpage},
          \DescPageRef{\ThisCommonLabelBase.option.fromrule})}%
        \feentry{itemizelabel}{%
          \ChangedAt{v3.33}{\Class{scrlttr2}}%
          Padrão para os símbolos predefinidos do ambiente
          \DescRef{\ThisCommonLabelBase.env.itemize} (veja
          \autoref{sec:scrlttr2.lists}, \DescPageRef{\ThisCommonLabelBase.env.itemize})}%
        \feentry{labelenumi}{%
          \ChangedAt{v3.44}{\Class{scrlttr2}}%
          Fonte a ser usada na definição do número do item
          \DescRef{\ThisCommonLabelBase.cmd.labelenumi} (veja
          \autoref{sec:scrlttr2.lists}, \DescPageRef{\ThisCommonLabelBase.env.enumerate})}%
        \feentry{labelenumii}{%
          \ChangedAt{v3.44}{\Class{scrlttr2}}%
          Fonte a ser usada na definição do número do item
          \DescRef{\ThisCommonLabelBase.cmd.labelenumii} (veja
          \autoref{sec:scrlttr2.lists}, \DescPageRef{\ThisCommonLabelBase.env.enumerate})}%
        \feentry{labelenumiii}{%
          \ChangedAt{v3.44}{\Class{scrlttr2}}%
          Fonte a ser usada na definição do número do item
          \DescRef{\ThisCommonLabelBase.cmd.labelenumiii} (veja
          \autoref{sec:scrlttr2.lists}, \DescPageRef{\ThisCommonLabelBase.env.enumerate})}%
        \feentry{labelenumiv}{%
          \ChangedAt{v3.44}{\Class{scrlttr2}}%
          Fonte a ser usada na definição do número do item
          \DescRef{\ThisCommonLabelBase.cmd.labelenumiv} (veja
          \autoref{sec:scrlttr2.lists}, \DescPageRef{\ThisCommonLabelBase.env.enumerate})}%
        \feentry{labelinglabel}{%
          rótulos, i.\,e. o argumento opcional de
          \DescRef{\ThisCommonLabelBase.cmd.item.labeling} no
          ambiente \DescRef{\ThisCommonLabelBase.env.labeling} %
          (veja \autoref{sec:scrlttr2.lists},
          \DescPageRef{\ThisCommonLabelBase.env.labeling})}%
        \feentry{labelingseparator}{%
          separador, i.\,e. o argumento opcional do
          ambiente \DescRef{\ThisCommonLabelBase.env.labeling}; aplicado em
          adição ao
          elemento \DescRef{\ThisCommonLabelBase.fontelement.labelinglabel}
          (veja \autoref{sec:scrlttr2.lists},
          \DescPageRef{\ThisCommonLabelBase.env.labeling})}%
        \feentry{labelitemi}{%
          \ChangedAt{v3.33}{\Class{scrlttr2}}%
          Fonte a ser usada na definição do símbolo do item
          \DescRef{\ThisCommonLabelBase.cmd.labelitemi} (veja
          \autoref{sec:scrlttr2.lists}, \DescPageRef{\ThisCommonLabelBase.env.itemize})}%
        \feentry{labelitemii}{%
          \ChangedAt{v3.33}{\Class{scrlttr2}}%
          Fonte a ser usada na definição do símbolo do item
          \DescRef{\ThisCommonLabelBase.cmd.labelitemii} (veja
          \autoref{sec:scrlttr2.lists}, \DescPageRef{\ThisCommonLabelBase.env.itemize})}%
        \feentry{labelitemiii}{%
          \ChangedAt{v3.33}{\Class{scrlttr2}}%
          Fonte a ser usada na definição do símbolo do item
          \DescRef{\ThisCommonLabelBase.cmd.labelitemiii} (veja
          \autoref{sec:scrlttr2.lists}, \DescPageRef{\ThisCommonLabelBase.env.itemize})}%
        \feentry{labelitemiv}{%
          \ChangedAt{v3.33}{\Class{scrlttr2}}%
          Fonte a ser usada na definição do símbolo do item
          \DescRef{\ThisCommonLabelBase.cmd.labelitemiv} (veja
          \autoref{sec:scrlttr2.lists}, \DescPageRef{\ThisCommonLabelBase.env.itemize})}%
        \feentry{pagefoot}{%
          dependendo do estilo de página usado após o
          elemento \DescRef{\ThisCommonLabelBase.fontelement.pageheadfoot} para
          o rodapé (\autoref{sec:\LabelBase.pagestyle},
          \DescPageRef{\LabelBase.fontelement.pagefoot})}%
        \feentry{pagehead}{%
          dependendo do estilo de página usado após o
          elemento \DescRef{\ThisCommonLabelBase.fontelement.pageheadfoot} para
          o cabeçalho (\autoref{sec:\LabelBase.pagestyle},
          \DescPageRef{\LabelBase.fontelement.pagefoot})}%
        \feentry{pageheadfoot}{%
          o cabeçalho e rodapé de uma página para todos os estilos de página que foram
          definidos usando \KOMAScript{} (\autoref{sec:\LabelBase.pagestyle},
          \DescPageRef{\ThisCommonLabelBase.fontelement.pageheadfoot})}%
        \feentry{pagenumber}{%
          número de página no cabeçalho ou rodapé %
          (\autoref{sec:\LabelBase.pagestyle},
          \DescPageRef{\ThisCommonLabelBase.fontelement.pagenumber})}%
        \feentry{pagination}{%
          nome alternativo para
          \DescRef{\ThisCommonLabelBase.fontelement.pagenumber}}%
        \feentry{placeanddate}{%
          \ChangedAt{v3.12}{\Class{scrlttr2}}%
          local e data, se uma linha de data for usada em vez de uma
          linha de referência normal (\autoref{sec:scrlttr2.firstpage},
          \DescPageRef{\ThisCommonLabelBase.variable.placeseparator})}%
        \feentry{refname}{%
          descrição ou título dos campos na linha de referência %
          (\autoref{sec:scrlttr2.firstpage},
          \DescPageRef{\ThisCommonLabelBase.variable.yourref})}%
        \feentry{refvalue}{%
          conteúdo dos campos na linha de referência %
          (\autoref{sec:scrlttr2.firstpage},
          \DescPageRef{\ThisCommonLabelBase.variable.yourref})}%
        \feentry{specialmail}{%
          tipo de entrega no campo de endereço %
          (\autoref{sec:scrlttr2.firstpage},
          \DescPageRef{\ThisCommonLabelBase.variable.specialmail})}%
        \feentry{lettersubject}{%
          \ChangedAt{v3.17}{\Class{scrlttr2}\and \Package{scrletter}}%
          assunto na abertura da carta %
          (\autoref{sec:scrlttr2.firstpage},
          \DescPageRef{\ThisCommonLabelBase.variable.subject})}%
        \feentry{lettertitle}{%
          \ChangedAt{v3.17}{\Class{scrlttr2}\and \Package{scrletter}}%
          título na abertura da carta %
          (\autoref{sec:scrlttr2.firstpage},
          \DescPageRef{\ThisCommonLabelBase.variable.title})}%
        \feentry{toaddress}{%
          variação do
          elemento \DescRef{\ThisCommonLabelBase.fontelement.addressee} para
          formatar o endereço do destinatário, não incluindo o nome, no
          campo de endereço (\autoref{sec:scrlttr2.firstpage},
          \DescPageRef{\ThisCommonLabelBase.variable.toaddress})}%
        \feentry{toname}{%
          variação do
          elemento \DescRef{\ThisCommonLabelBase.fontelement.addressee} para
          formatar o nome do destinatário no campo de endereço
          (\autoref{sec:scrlttr2.firstpage},
          \DescPageRef{\ThisCommonLabelBase.variable.toname})}%
      \end{desclist}
    }{%
      \IfThisCommonLabelBase{scrlayer-scrpage}{%
        \begin{desclist}
          \desccaption[{Elementos do \Package{scrlayer-scrpage} cujos estilos
            de fonte podem ser alterados com os comandos \Macro{setkomafont} e
            \Macro{addtokomafont}}]%
          {Elementos do \Package{scrlayer-scrpage} cujos estilos de fonte podem ser
            alterados com os comandos \Macro{setkomafont} e \Macro{addtokomafont},
            e seus padrões, se não foram definidos
            antes de carregar \Package{scrlayer-scrpage}%
            \label{tab:scrlayer-scrpage.fontelements}%
          }%
          {Elementos cujo estilo de fonte pode ser alterado (\emph{continuação})}%
          \feentry{footbotline}{%
            linha horizontal abaixo do rodapé de um estilo de página definido usando
            \Package{scrlayer-scrpage}. A fonte será aplicada após
            \Macro{normalfont}\IndexCmd{normalfont} e as fontes dos elementos
            \DescRef{\ThisCommonLabelBase.fontelement.pageheadfoot}%
            \IndexFontElement{pageheadfoot} e
            \DescRef{\ThisCommonLabelBase.fontelement.pagefoot}%
            \IndexFontElement{pagefoot}. Recomenda-se usar este elemento
            apenas para mudanças de cor.\par
            \mbox{Padrão: \emph{vazio}}%
          }%
          \feentry{footsepline}{%
            linha horizontal acima do rodapé de um estilo de página definido usando
            \Package{scrlayer-scrpage}. A fonte será aplicada após
            \Macro{normalfont}\IndexCmd{normalfont} e as fontes dos elementos
            \DescRef{\ThisCommonLabelBase.fontelement.pageheadfoot}%
            \IndexFontElement{pageheadfoot} e
            \DescRef{\ThisCommonLabelBase.fontelement.pagefoot}%
            \IndexFontElement{pagefoot}. Recomenda-se usar este elemento
            apenas para mudanças de cor.\par
            \mbox{Padrão: \emph{vazio}}%
          }%
          \feentry{headsepline}{%
            linha horizontal abaixo do cabeçalho de um estilo de página definido usando
            \Package{scrlayer-scrpage}. A fonte será aplicada após
            \Macro{normalfont}\IndexCmd{normalfont} e as fontes dos elementos
            \DescRef{\ThisCommonLabelBase.fontelement.pageheadfoot}%
            \IndexFontElement{pageheadfoot} e
            \DescRef{scrlayer-scrpage.fontelement.pagehead}%
            \IndexFontElement{pagehead}. Recomenda-se usar este elemento
            apenas para mudanças de cor.\par
            Padrão: \emph{vazio}%
          }%
          \feentry{headtopline}{%
            linha horizontal acima do cabeçalho de um estilo de página definido usando
            \Package{scrlayer-scrpage}. A fonte será aplicada após
            \Macro{normalfont}\IndexCmd{normalfont} e as fontes dos elementos
            \DescRef{\ThisCommonLabelBase.fontelement.pageheadfoot}%
            \IndexFontElement{pageheadfoot} e
            \DescRef{scrlayer-scrpage.fontelement.pagehead}%
            \IndexFontElement{pagehead}. Recomenda-se usar este elemento
            apenas para mudanças de cor.\par
            \mbox{Padrão: \emph{vazio}}%
          }%
          \feentry{pagefoot}{%
            conteúdo do rodapé da página de um estilo de página definido usando
            \Package{scrlayer-scrpage}. A fonte será aplicada após
            \Macro{normalfont}\IndexCmd{normalfont} e a fonte do elemento
            \DescRef{\ThisCommonLabelBase.fontelement.pageheadfoot}%
            \IndexFontElement{pageheadfoot}.\par
            \mbox{Padrão: \emph{vazio}}%
          }%
          \feentry{pagehead}{%
            conteúdo do cabeçalho da página de um estilo de página definido usando
            \Package{scrlayer-scrpage}. A fonte será aplicada após
            \Macro{normalfont}\IndexCmd{normalfont} e a fonte do elemento
            \DescRef{\ThisCommonLabelBase.fontelement.pageheadfoot}%
            \IndexFontElement{pageheadfoot}.\par
            \mbox{Padrão: \emph{vazio}}%
          }%
          \feentry{pageheadfoot}{%
            conteúdo do cabeçalho ou rodapé da página de um estilo de página definido
            usando \Package{scrlayer-scrpage}. A fonte será aplicada após
            \Macro{normalfont}\IndexCmd{normalfont}.\par
            \mbox{Padrão: \Macro{normalcolor}\Macro{slshape}}%
          }%
          \feentry{pagenumber}{%
            paginação definida com
            \DescRef{\ThisCommonLabelBase.cmd.pagemark}. Se você redefinir
            \DescRef{\ThisCommonLabelBase.cmd.pagemark}, você deve garantir
            que sua redefinição também use
            \Macro{usekomafont}\PParameter{pagenumber}!\par
            \mbox{Padrão: \Macro{normalfont}}%
          }%
        \end{desclist}
      }{%
        \IfThisCommonLabelBase{scrjura}{%
          \begin{table}
            \caption{Elementos cujos estilos de fonte do \Package{scrjura} podem ser
              alterados com \Macro{setkomafont} e \Macro{addtokomafont},
              incluindo suas configurações padrão}%
            \label{tab:scrjura.fontelements}%
            \begin{desctabular}
              \feentry{Clause}{%
                alias para \FontElement{\PName{environment name}.Clause}
                dentro de um ambiente de contrato, por exemplo
                \FontElement{contract.Clause} dentro de
                \DescRef{\ThisCommonLabelBase.env.contract}; se nenhum
                elemento correspondente está definido,
                \FontElement{contract.Clause} é usado%
              }%
              \feentry{contract.Clause}{%
                cabeçalho de um parágrafo dentro de
                \DescRef{\ThisCommonLabelBase.env.contract} (veja
                \autoref{sec:\ThisCommonLabelBase.contract},
                \DescPageRef{\ThisCommonLabelBase.fontelement.contract.Clause});
                \par
                \mbox{Padrão: \Macro{sffamily}\Macro{bfseries}\Macro{large}}%
              }%
              \entry{\DescRef{\ThisCommonLabelBase.fontelement./Name/.Clause}}{%
                \IndexFontElement[indexmain]{\PName{name}.Clause}%
                cabeçalho de um parágrafo dentro de um ambiente \PName{name}
                definido com
                \DescRef{\ThisCommonLabelBase.cmd.DeclareNewJuraEnvironment}
                desde que a configuração tenha sido feita com \Option{ClauseFont} ou
                o item tenha sido subsequentemente definido (veja
                \autoref{sec:\ThisCommonLabelBase.newenv},
                \DescPageRef{\ThisCommonLabelBase.fontelement./Name/.Clause});
                \par
                \mbox{Padrão: \emph{nenhum}}%
              }%
              \feentry{parnumber}{%
                números de parágrafo dentro de um ambiente de contrato (veja
                \autoref{sec:\ThisCommonLabelBase.par},
                \DescPageRef{\ThisCommonLabelBase.fontelement.parnumber});\par
                \mbox{Padrão: \emph{vazio}}%
              }%
              \feentry{sentencenumber}{%
                \ChangedAt{v3.26}{\Package{scrjura}}%
                número da sentença de \DescRef{\ThisCommonLabelBase.cmd.Sentence}
                (veja \autoref{sec:\ThisCommonLabelBase.sentence},
                \DescPageRef{%
                  \ThisCommonLabelBase.fontelement.sentencenumber});\par
                \mbox{Padrão: \emph{vazio}}%
              }%
            \end{desctabular}
          \end{table}
        }{%
          \IfThisCommonLabelBase{scrlayer-notecolumn}{}{%
            \InternalCommonFileUsageError%
          }%
        }%
      }%
    }%
  \fi%
\fi
\EndIndexGroup


\begin{Declaration}
  \Macro{usefontofkomafont}\Parameter{element}%
  \Macro{useencodingofkomafont}\Parameter{element}%
  \Macro{usesizeofkomafont}\Parameter{element}%
  \Macro{usefamilyofkomafont}\Parameter{element}%
  \Macro{useseriesofkomafont}\Parameter{element}%
  \Macro{useshapeofkomafont}\Parameter{element}
\end{Declaration}
Às vezes\ChangedAt{v3.12}{\Class{scrbook}\and \Class{scrreprt}\and
  \Class{scrartcl}\and \Package{scrextend}}, embora isso não seja recomendado,
a configuração de fonte de um elemento é usada para configurações que não estão realmente
relacionadas à fonte. Se você quiser aplicar apenas a configuração de fonte de um elemento
mas não essas outras configurações, você pode usar \Macro{usefontofkomafont} em vez de
\DescRef{\ThisCommonLabelBase.cmd.usekomafont}. Isso ativará o tamanho da fonte
e espaço entre linhas, a codificação da fonte, a família da fonte, a série da fonte,
e a forma da fonte de um elemento, mas nenhuma configuração adicional desde que essas
configurações adicionais sejam locais.

Você também pode mudar para um único desses atributos usando um dos outros
comandos. Note que \Macro{usesizeofkomafont} usa tanto o tamanho da fonte
quanto o espaço entre linhas.%
%
\IfThisCommonLabelBase{scrextend}{% Umbruchvariante!
}{%
  \IfThisCommonLabelBase{scrjura}{%
    \par%
    No entanto, o uso indevido das configurações de fonte é fortemente desencorajado (veja
    \autoref{sec:maincls-experts.fonts},
    \DescPageRef{maincls-experts.cmd.addtokomafontrelaxlist})!%
  }{%
    \par%
    No entanto, você não deve tomar esses comandos como legitimando a inserção
    de comandos arbitrários na configuração de fonte de um elemento. Fazê-lo pode levar
    rapidamente a erros (veja \autoref{sec:maincls-experts.fonts},
    \DescPageRef{maincls-experts.cmd.addtokomafontrelaxlist}).%
  }%
}%
\EndIndexGroup
%
\EndIndexGroup

%%% Local Variables:
%%% mode: latex
%%% TeX-master: "scrguide-en.tex"
%%% coding: utf-8
%%% ispell-local-dictionary: "en_GB"
%%% eval: (flyspell-mode 1)
%%% End:
