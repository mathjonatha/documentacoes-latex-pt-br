% ======================================================================
% common-typearea-en.tex
% Copyright (c) Markus Kohm, 2001-2022
%
% This file is part of the LaTeX2e KOMA-Script bundle.
%
% This work may be distributed and/or modified under the conditions of
% the LaTeX Project Public License, version 1.3c of the license.
% The latest version of this license is in
%   http://www.latex-project.org/lppl.txt
% and version 1.3c or later is part of all distributions of LaTeX
% version 2005/12/01 or later and of this work.
%
% This work has the LPPL maintenance status "author-maintained".
%
% The Current Maintainer and author of this work is Markus Kohm.
%
% This work consists of all files listed in MANIFEST.md.
% ======================================================================
%
% Paragraphs that are common for several chapters of the KOMA-Script guide
% Maintained by Markus Kohm
%
% ======================================================================

\KOMAProvidesFile{common-typearea-en.tex}
                 [$Date: 2022-06-05 12:40:11 +0200 (So, 05. Jun 2022) $
                  KOMA-Script guide (common paragraphs: typearea)]
\translator{Markus Kohm\and Krickette Murabayashi\and Karl Hagen}

\section{Layout da Página}
\seclabel{typearea}
\BeginIndexGroup
\BeginIndex{}{page>layout}

Cada página de um documento consiste em diferentes elementos de layout, tais como as
margens, o cabeçalho, o rodapé, a área de texto, a coluna de notas marginais e as
distâncias entre esses elementos. O \KOMAScript{} distingue adicionalmente a página
inteira, também conhecida como papel, e a página visível. Sem dúvida, a separação da
página nesses diferentes componentes é uma das características básicas de uma
classe\IfThisCommonLabelBase{scrlttr2}{\OnlyAt{scrlttr2}}{}. O \KOMAScript{} delega
este trabalho ao pacote
\hyperref[cha:typearea]{\Package{typearea}}\IndexPackage{typearea}. Este pacote
também pode ser usado com outras classes. As classes \KOMAScript{}, entretanto,
carregam \Package{typearea} por conta própria. Portanto, não é necessário nem
sensato carregar o pacote explicitamente com \Macro{usepackage} ao usar uma classe
\KOMAScript{}. Veja também \autoref{sec:\ThisCommonLabelBase.options},
\autopageref{sec:\ThisCommonLabelBase.options}.

Algumas configurações das classes \KOMAScript{} afetam o layout da página e vice-versa.
Esses efeitos são documentados nas configurações correspondentes.

Para mais informações sobre a escolha do formato do papel, a divisão da página em
margens e área de texto, e a escolha entre composição em uma ou duas colunas, consulte
a documentação do pacote
\hyperref[cha:typearea]{\Package{typearea}}\IndexPackage{typearea}. Você pode
encontrá-la no \autoref{cha:typearea}, começando na
\autopageref{cha:typearea}.

%%% Local Variables:
%%% mode: latex
%%% TeX-master: "scrguide-en.tex"
%%% coding: utf-8
%%% ispell-local-dictionary: "en_GB"
%%% eval: (flyspell-mode 1)
%%% End:
