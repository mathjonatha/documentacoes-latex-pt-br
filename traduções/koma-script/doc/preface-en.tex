% ======================================================================
% preface-en.tex
% Copyright (c) Markus Kohm, 2008-2022
%
% This file is part of the LaTeX2e KOMA-Script bundle.
%
% This work may be distributed and/or modified under the conditions of
% the LaTeX Project Public License, version 1.3c of the license.
% The latest version of this license is in
%   http://www.latex-project.org/lppl.txt
% and version 1.3c or later is part of all distributions of LaTeX
% version 2005/12/01 or later and of this work.
%
% This work has the LPPL maintenance status "author-maintained".
%
% The Current Maintainer and author of this work is Markus Kohm.
%
% This work consists of all files listed in MANIFEST.md.
% ======================================================================

\KOMAProvidesFile{preface-en.tex}
                 [$Date: 2025-08-19 15:21:00 +0200 (Di, 19. Aug 2025) $
                  preface to dedicated version]
\translator{Markus Kohm\and DeepL}

\addchap{Prefácio ao \KOMAScript~3.46}

Enquanto trabalhava no \KOMAScript{} 3.46, recebi a má notícia de que Axel
Sommerfeldt pretende descontinuar a manutenção do pacote \Package{caption}
e, portanto, não mais aceitará sugestões de mudanças e melhorias. Isto é
relevante para o \KOMAScript{} na medida em que, após algumas melhorias no
processamento de \Macro{caption}, tornou-se aparente que o \Package{caption}
também precisa de algumas pequenas alterações para manter e melhorar a
compatibilidade com o \KOMAScript{}. De forma alguma as melhorias no
processamento de \Macro{caption} dentro do \KOMAScript{} deveriam ter um
impacto negativo na compatibilidade do pacote \Package{caption}. A
consequência é que o próprio pacote \Package{caption} não mais garante a
compatibilidade, mas sim o \KOMAScript{} reage ao carregamento do
\Package{caption} para então estabelecer a compatibilidade por si mesmo.

Como em muitos outros casos, isto naturalmente significa mais trabalho para
mim em termos tanto de desenvolvimento quanto de testes. Juntamente com as
incontáveis mudanças que recentemente tornaram-se necessárias devido ao rápido
desenvolvimento do próprio \LaTeX{}, isto infelizmente significa que alguns
trabalhos urgentes no \KOMAScript{} estão sendo constantemente adiados. Vários
outros projetos, que estão por fim se tornando cada vez mais importantes para
a continuidade do \KOMAScript{}, não puderam ser trabalhados novamente.
Infelizmente, isto também inclui o suporte a marcação.

Apesar disso, esta versão do \KOMAScript{} conseguiu avançar ainda mais na
revisão urgentemente necessária da distribuição de tarefas entre classes e
pacotes. Alguns outros problemas também foram abordados, mas estes podem levar
a quebras de página alteradas em documentos de duas colunas com títulos de
capítulos. Veja o site »\href{https://sourceforge.net/p/koma-script/wiki-en/Releases/}
{Important notes and changes}« e, em particular,
»\href{https://sourceforge.net/p/koma-script/wiki-en/Release%203.46/}
{Known issues and important changes in \KOMAScript{} 3.46}«.

Meus agradecimentos vão principalmente para minha esposa. Ela amortece todas
as minhas experiências desagradáveis na Internet. Ela também me aturou por mais
de 30~anos quando estou mais uma vez sem responder porque estou completamente
imerso no \KOMAScript{} ou em alguns problemas do \LaTeX{}. O fato de que posso
me dar ao luxo de investir uma quantidade tão insana de tempo em um projeto
como este também é graças à minha esposa.

\bigskip\noindent
Markus Kohm, Neckarhausen em agosto de 2025.

\endinput

%%% Local Variables: 
%%% mode: latex
%%% TeX-master: "scrguide-en.tex"
%%% coding: utf-8
%%% ispell-local-dictionary: "en_GB"
%%% eval: (flyspell-mode 1)
%%% End: 
