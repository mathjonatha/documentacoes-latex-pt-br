% ======================================================================
% common-pagestylemanipulation-en.tex
% Copyright (c) Markus Kohm, 2013-2022
%
% This file is part of the LaTeX2e KOMA-Script bundle.
%
% This work may be distributed and/or modified under the conditions of
% the LaTeX Project Public License, version 1.3c of the license.
% The latest version of this license is in
%   http://www.latex-project.org/lppl.txt
% and version 1.3c or later is part of all distributions of LaTeX
% version 2005/12/01 or later and of this work.
%
% This work has the LPPL maintenance status "author-maintained".
%
% The Current Maintainer and author of this work is Markus Kohm.
%
% This work consists of all files listed in MANIFEST.md.
% ======================================================================
%
% Text that is common for several chapters of the KOMA-Script guide
% Maintained by Markus Kohm
%
% ============================================================================

\KOMAProvidesFile{common-pagestylemanipulation-en.tex}
                 [$Date: 2023-09-18 09:02:03 +0200 (Mo, 18. Sep 2023) $
                  KOMA-Script guide (common paragraph:
                                     Setting up defined page styles)]
\translator{Markus Kohm\and Jana Schubert\and Karl Hagen}

\section{Manipulando Estilos de Página}
\seclabel{pagestyle.content}
\BeginIndexGroup
\BeginIndex{}{page>style}

\IfThisCommonLabelBase{scrlayer}{%
  Embora o \Package{scrlayer} em si não defina estilos de página concretos com
  conteúdo\,---\,os estilos de página mencionados anteriormente
  \DescRef{\LabelBase.pagestyle.@everystyle@} e \PageStyle{empty} são
  inicialmente definidos sem quaisquer camadas, isto é, sem conteúdo\,---\,, o
  pacote fornece algumas opções e comandos para manipular seus conteúdos.%
}{%
  \IfThisCommonLabelBase{scrlayer-scrpage}{%
    \autoref{sec:scrlayer-scrpage.predefined.pagestyles} explica como os estilos
    de página \DescRef{\LabelBase.pagestyle.scrheadings} e
    \DescRef{\LabelBase.pagestyle.plain.scrheadings} são definidos e como esses
    padrões podem ser alterados. Mas tópicos como criar títulos correntes,
    alterar as larguras do cabeçalho e rodapé, e colocar linhas horizontais
    acima ou abaixo do cabeçalho ou rodapé ainda precisam ser descritos. Embora
    essas capacidades sejam na verdade parte do pacote
    \hyperref[cha:scrlayer]{\Package{scrlayer}}%
    \important{\hyperref[cha:scrlayer]{\Package{scrlayer}}}, elas serão
    explicadas abaixo porque esses recursos básicos do
    \hyperref[cha:scrlayer]{\Package{scrlayer}} constituem uma parte importante
    do \IfThisCommonLabelBase{scrlayer-scrpage}{\Package{scrlayer-scrpage}%
    }{%
      \hyperref[cha:scrlayer-scrpage]{\Package{scrlayer-scrpage}}%
    }.%
  }{%
    \IfThisCommonLabelBase{scrlayer-scrpage-experts}{%
      Esta seção é um complemento à
      \autoref{sec:scrlayer.pagestyle.content}. Ela descreve recursos que podem
      ser muito complicados para iniciantes.%
    }{\InternalCommonFileUsageError}%
  }%
}

\IfThisCommonLabelBase{scrlayer-scrpage-experts}{\iffalse}{%
  \csname iftrue\endcsname}%
  \begin{Declaration}
    \IfThisCommonLabelBase{scrlayer}{%
      \Option{automark}
      \OptionVName{autooneside}{switch simples}
      \Option{manualmark}
    }{}%
    \Macro{automark}\OParameter{nível de seção da marca direita}
                    \Parameter{nível de seção da marca esquerda}
    \Macro{automark*}\OParameter{nível de seção da marca direita}
                    \Parameter{nível de seção da marca esquerda}
    \Macro{manualmark}
  \end{Declaration}
  \IfThisCommonLabelBase{scrlayer-scrpage}{%
    \begin{Explain}
      Tanto nas classes padrão do \LaTeX{} quanto nas classes \KOMAScript{}, a
      decisão de usar títulos correntes automáticos ou estáticos\Index{títulos
      correntes>automáticos}\Index{títulos correntes>estáticos} é feita usando
      o estilo de página apropriado. Os títulos correntes repetem algum texto
      descritivo, como um título, que seja apropriado à página ou coluna,
      geralmente no cabeçalho, mais raramente no rodapé. Como já explicado em
      \autoref{sec:maincls.pagestyle}, você obtém títulos correntes automáticos
      com \DescRef{maincls.pagestyle.headings}\IndexPagestyle{headings}%
      \important{\DescRef{maincls.pagestyle.headings}}

      Nas classes de artigo\OnlyAt{\Class{article}\and \Class{scrartcl}}
      \Class{article} ou \hyperref[cha:maincls]{\Class{scrartcl}}, o estilo de
      página \PageStyle{headings}\IndexPagestyle{headings} usa o título da
      seção, que é o argumento obrigatório ou opcional de
      \DescRef{maincls.cmd.section}, para o título corrente%
      \textnote{título corrente automático} de documentos de um lado. Documentos
      de dois lados usam esse título de seção como \emph{marca esquerda} e o
      título da subseção como \emph{marca direita}. A marca esquerda é impressa,
      como o nome indica, em páginas à esquerda (verso). A marca direita é
      impressa em páginas à direita (recto)\,---\,na impressão de um lado isto
      significa em todas\,---\,as páginas. As classes por padrão também excluem
      a marca direita sempre que colocam o título da seção na marca esquerda.

      As classes report e book \OnlyAt{\Class{report}\and
      	\Class{scrreprt}\and \Class{book}\and \Class{scrbook}} começam um nível
      acima. Assim, elas usam o título do capítulo como marca direita na
      impressão de um lado. Na impressão de dois lados, o título do capítulo é a
      marca esquerda e o título da seção é a marca direita.

      Se você usar \DescRef{maincls.pagestyle.myheadings}%
      \IndexPagestyle{myheadings}%
      \important{\DescRef{maincls.pagestyle.myheadings}}\textnote{título
      	corrente manual}, as marcas no cabeçalho da página ainda existem, e os
      números de página são colocados da mesma maneira, mas os comandos de seção
      não mais definem as marcas automaticamente. Você pode defini-las
      manualmente\important{%
      	\DescRef{\ThisCommonLabelBase.cmd.markright}\\
      	\DescRef{\ThisCommonLabelBase.cmd.markboth}} usando os comandos
      \DescRef{\ThisCommonLabelBase.cmd.markright} e
      \DescRef{\ThisCommonLabelBase.cmd.markboth}, que são descritos mais adiante
      nesta seção.
    \end{Explain}\par%
    Esta distinção foi eliminada pelo %
    \iffalse \Package{scrpage2} e agora também pelo \fi%
    \hyperref[cha:scrlayer]{\Package{scrlayer}}\IndexPackage{scrlayer}. Em vez
    de distinguir entre títulos correntes\textnote{títulos correntes automáticos
      e manuais} automáticos e manuais pelo estilo de página selecionado,
    existem dois novos comandos: \Macro{automark} e \Macro{manualmark}.%
  }{%
    \IfThisCommonLabelBase{scrlayer}{%
      Para a maioria das classes, a escolha de um estilo de página\,---\,geralmente
      \PageStyle{headings} ou \PageStyle{myheading}\,---\,determina se os títulos
      correntes são criados automaticamente ou manualmente.
% TODO: Fix new translation
      Com o \Package{scrlayer} a distinção é feita com os dois comandos
      \Macro{automark} e \Macro{manualmark}.%
% :ODOT
    }{\InternalCommonFileUsageError}%
  }%

  O comando \Macro{manualmark}\important{\Macro{manualmark}} muda para marcas
  manuais e desativa o preenchimento automático das marcas. Em contraste,
  \Macro{automark}\important{\Macro{automark}} e \Macro{automark*} definem quais
  níveis de seção devem ser usados para definir a marca automaticamente. O
  argumento opcional é o \PName{nível de seção da marca direita}, o argumento
  obrigatório é o \PName{nível de seção da marca esquerda}. Os argumentos devem
  sempre ser o nome de um nível de seção como \PValue{part}, \PValue{chapter},
  \PValue{section}, \PValue{subsection}, \PValue{subsubsection},
  \PValue{paragraph}, ou \PValue{subparagraph}.

  Normalmente, o nível superior deve ser usado para a marca esquerda e o nível
  inferior para a marca direita. Isto é apenas uma convenção e não um requisito,
  mas faz sentido.

  Por favor, note\textnote{Atenção!} que nem toda classe fornece títulos
  correntes para cada nível de seção. Por exemplo, as classes
  padrão\textnote{\KOMAScript{} vs. classes padrão} nunca usam
  \DescRef{maincls.cmd.part} no cabeçalho. As classes \KOMAScript{}, por outro
  lado, suportam todos os níveis.

  A diferença entre \Macro{automark} e \Macro{automark*}%
  \important{\Macro{automark*}} é que \Macro{automark} sobrescreve todos os
  comandos anteriores para definir a marca automaticamente, enquanto
  \Macro{automark*} muda apenas o comportamento dos níveis de seção especificados
  em seus argumentos.%
  \IfThisCommonLabelBase{scrlayer-scrpage}{}{ %
    Com este recurso você pode lidar com casos relativamente complexos.%
  }% Umbruchoptimierung
  \IfThisCommonLabelBase{scrlayer-scrpage}{\iftrue}{\csname
    iffalse\endcsname}%
    \iffalse% Umbruchkorrekturtext
      \iffree{}{\par
        O pacote obsoleto
        \Package{scrpage2}\IndexPackage{scrpage2}\important{\Package{scrpage2}}
        compreende tanto \Macro{manualmark} quanto \Macro{automark}, mas não
        \Macro{automark*}. Portanto, os exemplos a seguir não são completamente
        transferíveis ao usar \Package{scrpage2}.%
      }%
    \fi
    %
    \begin{Example}
      \phantomsection\xmpllabel{mark}%
      Suponha que você queira que os títulos de capítulo sejam usados como
      título corrente de páginas pares e o título da seção seja o título
      corrente de páginas ímpares, como de costume. Mas em páginas ímpares você
      também quer que os títulos de capítulo sejam usados como título corrente
      até que a primeira seção apareça. Para fazer isso, você primeiro precisa
      carregar \IfThisCommonLabelBase{scrlayer-scrpage}{%
      	\Package{scrlayer-scrpage}%
      }{%
      	\hyperref[cha:scrlayer-scrpage]{\Package{scrlayer-scrpage}}%
      }
      e selecionar o estilo de página \DescRef{\LabelBase.pagestyle.scrheadings},
      então o documento começa com:
\begin{lstcode}
  \documentclass{scrbook}
  \usepackage{scrlayer-scrpage}
  \pagestyle{scrheadings}
\end{lstcode}
      Em seguida, certifique-se de que os títulos de capítulo definem tanto a
      marca esquerda quanto a direita:
\begin{lstcode}
  \automark[chapter]{chapter}
\end{lstcode}
      Então o título da seção também deve definir marcas direitas:
\begin{lstcode}
   \automark*[section]{}
\end{lstcode}
      Aqui a versão com asterisco é usada, já que o comando \Macro{automark}
      anterior deve permanecer em efeito. Adicionalmente, o argumento obrigatório
      para o \PName{nível de seção da marca esquerda} está vazio porque esta
      marca deve permanecer inalterada.

      Tudo que falta agora é um pouco de conteúdo do documento para mostrar o
      resultado:
\begin{lstcode}
  \usepackage{lipsum}
  \begin{document}
  \chapter{Título do Capítulo}
  \lipsum[1-20]
  \section{Título da Seção}
  \lipsum[21-40]
  \end{document}
\end{lstcode}
      Usamos o pacote extremamente útil \Package{lipsum}\IndexPackage{lipsum}
      para gerar algum texto fictício com o comando
      \Macro{lipsum}\IndexCmd{lipsum}.

      Se você testar o exemplo, verá que a primeira página do capítulo aparece,
      como de costume, sem um título corrente, já que essa página usa
      automaticamente o estilo de página \PageStyle{plain}
      \DescRef{\LabelBase.pagestyle.plain.scrheadings} (veja o
      \DescRef{maincls.cmd.chapterpagestyle} em
      \DescPageRef{maincls.cmd.chapterpagestyle}). As páginas~2--4 têm os
      títulos de capítulo no título corrente. Após o título da seção na
      página~4, o título corrente da página~5 muda para esse título de seção. A
      partir desta página até o fim, o título corrente alterna de página a
      página entre os títulos de capítulo e seção.%
    \end{Example}
  \fi

  \IfThisCommonLabelBase{scrlayer}{}{%
  \begin{Declaration}
    \Option{automark}
    \OptionVName{autooneside}{switch simples}
    \Option{manualmark}
  \end{Declaration}}
  Em vez dos comandos descritos anteriormente, você também pode usar as opções
  \Option{manualmark}\important{\Option{manualmark}\\\Option{automark}} e
  \Option{automark} para alternar entre títulos correntes automáticos e manuais.
  \Option{automark} sempre usa o padrão\textnote{padrão}
  \IfThisCommonLabelBase{scrlayer-scrpage}{\iftrue}{\csname
    iffalse\endcsname}%
    \lstinline|\automark[section]{chapter}| para classes com
    \DescRef{maincls.cmd.chapter} e
    \lstinline|\automark[subsection]{section}| para outras classes.
  \else
\begin{lstcode}
  \automark[section]{chapter}
\end{lstcode}
    para classes com \DescRef{maincls.cmd.chapter} e
\begin{lstcode}
  \automark[subsection]{section}
\end{lstcode}
    para outras classes.
  \fi

  \IfThisCommonLabelBaseOneOf{scrlayer,scrlayer-scrpage}{%
    Na impressão de um lado, você normalmente quer apenas os níveis de seção
    superiores para fornecer o título corrente.%
  }{%
    Na impressão de um lado, você normalmente não quer que o nível de seção
    inferior influencie a marca direita. Em vez disso, você quer que o nível de
    seção superior, que apareceria apenas na marca esquerda na impressão de dois
    lados, seja o título corrente de todas as páginas.%
  } A opção padrão \Option{autooneside}\important{\Option{autooneside}}
  corresponde a este comportamento. A opção aceita os valores para
  switches simples listados em \autoref{tab:truefalseswitch},
  \autopageref{tab:truefalseswitch}. Se você desativar esta opção, os
  argumentos opcional e obrigatório de \Macro{automark} e \Macro{automark*}
  novamente controlarão o título corrente na impressão de um lado.%
  \IfThisCommonLabelBase{scrlayer-scrpage}{\iftrue}{\csname
    iffalse\endcsname}%
    \begin{Example}
      \phantomsection\xmpllabel{mark.oneside}%
      Suponha que você tem um relatório de um lado mas quer títulos correntes
      similares àqueles no exemplo anterior do livro. Especificamente, os
      títulos de capítulo devem ser usados como título corrente até que a
      primeira seção apareça. A partir daí, o título da seção deve ser usado.
      Então modificamos o exemplo anterior um pouco:
\begin{lstcode}
  \documentclass{scrreprt}
  \usepackage[autooneside=false]{scrlayer-scrpage}
  \pagestyle{scrheadings}
  \automark[section]{chapter}
  \usepackage{lipsum}
  \begin{document}
  \chapter{Título do Capítulo}
  \lipsum[1-20]
  \section{Título da Seção}
  \lipsum[21-40]
  \end{document}
\end{lstcode}
      Como você pode ver, um comando \DescRef{\LabelBase.cmd.automark*} não é
      necessário neste caso. Você deve tentar o exemplo com \Option{autooneside}
      definido como \PValue{true}, ou remover a opção, para comparação. Você
      notará uma diferença no título corrente a partir da página~4.
    \end{Example}
  \fi

  Note\textnote{Atenção!} que meramente carregar o pacote não tem qualquer
  efeito sobre se títulos correntes automáticos ou manuais são usados, ou que
  tipo de títulos de seção preenchem as marcas. Apenas usando explicitamente a
  opção \Option{automark} ou \Option{manualmark}, ou o comando
  \DescRef{\LabelBase.cmd.automark} ou \DescRef{\LabelBase.cmd.manualmark}, as
  condições aqui serão inicializadas.%
  \IfThisCommonLabelBase{scrlayer}{\par%
    Você pode encontrar informações adicionais de segundo plano e exemplos de
    como usar esses comandos na documentação do pacote \Package{scrlayer}
    \hyperref[cha:scrlayer-scrpage]{\Package{scrlayer-scrpage}}%
    em \autoref{sec:scrlayer-scrpage.pagestyle.content}, começando em
    \DescPageRef{scrlayer-scrpage.cmd.manualmark}.%
  }{%
    \EndIndexGroup%
  }%
  \EndIndexGroup


  \IfThisCommonLabelBase{scrlayer}{% already described at \layercontentsmeasure
  }{%
    \begin{Declaration}
      \OptionVName{draft}{switch simples}
    \end{Declaration}
    Esta opção do \KOMAScript{} aceita os valores para switches simples listados
    em \autoref{tab:truefalseswitch}, \autopageref{tab:truefalseswitch}. Se esta
    opção estiver ativa, todos os elementos dos estilos de página também
    mostrarão réguas\index{régua}%
    \IfThisCommonLabelBase{scrlayer}{ para}{%
          . Isto pode às vezes ser útil durante}
     o processo de rascunho.%
    \IfThisCommonLabelBase{scrlayer-scrpage}{ %
      Se esta opção foi definida globalmente (veja o argumento opcional de
      \DescRef{\LabelBase.cmd.documentclass}) mas você não quer as réguas, você
      pode desativá-las para este pacote sozinho usando
      \OptionValue{draft}{false} como argumento opcional de
      \DescRef{\LabelBase.cmd.usepackage} ao carregar o pacote.%
    }{}%
    \EndIndexGroup%
  }%

  \begin{Declaration}
    \Macro{MakeMarkcase}\Parameter{texto}
    \OptionVName{markcase}{valor}
  \end{Declaration}
  Títulos correntes automáticos, mas não os manuais, usam \Macro{MakeMarkcase}
  para sua saída. Se o comando não foi definido, por exemplo, pela classe ao
  carregar \IfThisCommonLabelBase{scrlayer}{%
    \Package{scrlayer}%
  }{%
    \hyperref[cha:scrlayer]{\Package{scrlayer}}%
  }, ele é definido por padrão para produzir o argumento \PName{texto} sem
  alterações. Mas o padrão pode ser alterado redefinindo \Macro{MakeMarkcase}.
  Usar a opção \DescRef{\LabelBase.option.markcase}%
  \IndexOption{markcase}\important{\DescRef{\LabelBase.option.markcase}} com um
  dos valores da \autoref{tab:scrlayer-scrpage.markcase} também redefine
  \Macro{MakeMarkcase}.

  \IfThisCommonLabelBase{scrlayer}{%
    Devido à qualidade tipográfica pobre da rotina primitiva de
    capitalização (veja a explicação de
    \DescRef{scrlayer-scrpage.option.markcase} em
    \autoref{sec:scrlayer-scrpage.pagestyle.content},
    \autopageref{expl:scrlayer-scrpage.MakeUppercase}) o autor do
    \KOMAScript{} recomenda que você evite composição em maiúsculas para títulos
    correntes.%
  }{%
    Infelizmente,\phantomsection\label{expl:\ThisCommonLabelBase.MakeUppercase}
    o comando \LaTeX{} para converter texto em maiúsculas\Index{maiúsculas},
    \Macro{MakeUppercase}\IndexCmd{MakeUppercase}, não produz bons resultados
    porque ele não espaça os caracteres nem equilibra as linhas adequadamente.
    Isto certamente se deve em parte ao fato de que uma conversão
    tipograficamente correta para maiúsculas requer analisar os glifos para
    contabilizar as diferentes formas de letra \iffree{e suas combinações}{}
    enquanto equilibra o bloco. Portanto, recomendo que você evite composição em
    maiúsculas para títulos correntes.%
  } Isto geralmente é possível com \OptionValue{markcase}{used}\important{%
    \OptionValue{markcase}{used}}\IndexOption[indexmain]{markcase~=used}.
  Contudo, algumas classes inserem \Macro{MarkUppercase}, ou até mesmo o comando
  \TeX{} \Macro{uppercase}, nos títulos correntes. Para esses casos, você pode
  usar a opção \OptionValue{markcase}{noupper}%
  \important{\OptionValue{markcase}{noupper}}%
  \IndexOption[indexmain]{markcase~=noupper}. Isto também desativará
  \Macro{MakeUppercase} e \Macro{uppercase} dentro dos títulos correntes.

  Você pode encontrar todos os valores válidos para \Option{markcase} em
  \autoref{tab:scrlayer-scrpage.markcase}%
  \IfThisCommonLabelBase{scrlayer-scrpage}{}{,
    \autopageref{tab:scrlayer-scrpage.markcase}}.%
  \IfThisCommonLabelBase{scrlayer-scrpage}{%
    \begin{table}
      \centering
      \caption[Valores disponíveis para a opção \Option{markcase}]{Valores
        disponíveis para a opção \Option{markcase} para selecionar composição em
        maiúsculas/minúsculas em títulos correntes automáticos}%
      \label{tab:\ThisCommonLabelBase.markcase}%
      \begin{desctabular}
        \pventry{lower}{\IndexOption[indexmain]{markcase~=lower}%
          redefine \DescRef{\LabelBase.cmd.MakeMarkcase} para converter os
          títulos correntes automáticos em letras minúsculas usando
          \Macro{MakeLowercase}.%
        }\\[-1.7ex]
        \pventry{upper}{\IndexOption[indexmain]{markcase~=upper}%
          redefine \DescRef{\LabelBase.cmd.MakeMarkcase} para converter os
          títulos correntes automáticos em letras maiúsculas usando
          \Macro{MakeUppercase}.%
        }\\[-1.7ex]
        \pventry{title}{\IndexOption[indexmain]{markcase~=title}%
          \IfThisCommonLabelBase{scrlayer}{%
            \ChangedAt{v3.41}{\Package{scrlayer}}}{%
            \IfThisCommonLabelBase{scrlayer-scrpage}{%
              \ChangedAt{v3.41}{\Package{scrlayer-scrpage}}}}{}%
          redefine \DescRef{\LabelBase.cmd.MakeMarkcase} para converter os
          títulos correntes automáticos em maiúsculas de título usando
          \Macro{MakeTitlecase}. Isto é feito separadamente para o número e o
          texto. Se \Macro{MakeTitlecase} não estiver definido, isto é, porque
          você está usando um kernel \LaTeX{} antigo, você receberá uma mensagem
          de aviso e a opção será ignorada.%
        }\\[-1.7ex]
        \pventry{used}{\IndexOption[indexmain]{markcase~=used}%
          redefine \DescRef{\LabelBase.cmd.MakeMarkcase} para usar títulos
          correntes automáticos sem quaisquer mudanças de caixa.%
        }\\[-1.7ex]
        \entry{\PValue{ignoreuppercase}, \PValue{nouppercase},
          \PValue{ignoreupper},
          \PValue{noupper}}{\IndexOption[indexmain]{markcase~=noupper}%
          redefine não apenas \DescRef{\LabelBase.cmd.MakeMarkcase} mas também
          \Macro{MakeUppercase} e \Macro{uppercase} localmente aos títulos
          correntes para deixar os títulos correntes automáticos inalterados.%
        }%
      \end{desctabular}
    \end{table}
  }{}%
  \EndIndexGroup
\fi


\IfThisCommonLabelBase{scrlayer-scrpage}{\iffalse}{\csname iftrue\endcsname}
  \begin{Declaration}
    \Macro{righttopmark}
    \Macro{rightbotmark}
    \Macro{rightfirstmark}
    \Macro{rightmark}
    \Macro{lefttopmark}
    \Macro{leftbotmark}
    \Macro{leftfirstmark}
    \Macro{leftmark}
  \end{Declaration}
  O \LaTeX\ChangedAt{v3.16}{\Package{scrlayer}} tipicamente usa uma marca
  \TeX{} de duas partes para estilos de página. Títulos correntes podem acessar
  a parte esquerda dessa marca com \DescRef{scrlayer.cmd.leftmark}%
  \important{\DescRef{scrlayer.cmd.leftmark}}\IndexCmd{leftmark} e a parte
  direita com \DescRef{scrlayer.cmd.rightmark}%
  \important{\DescRef{scrlayer.cmd.rightmark}}\IndexCmd{rightmark}. De fato,
  provavelmente foi pretendido usar \DescRef{scrlayer.cmd.leftmark} para o
  título corrente de páginas à esquerda (pares) e
  \DescRef{scrlayer.cmd.rightmark} para o título corrente de páginas à direita
  (ímpares) de documentos de dois lados. Na impressão de um lado, contudo, as
  classes padrão nem mesmo definem a parte esquerda da marca.

  O \TeX{} em si conhece três maneiras de acessar uma marca. O \Macro{botmark}%
  \IndexCmd{botmark}\important{\Macro{botmark}} é a última marca válida da página
  mais recente que foi construída. Se nenhuma marca foi definida na página, ela
  corresponde à última marca definida nas páginas que já foram enviadas. O
  comando \LaTeX{} \DescRef{scrlayer.cmd.leftmark} usa precisamente esta marca,
  então ele retorna a parte esquerda da última marca da página. Isto corresponde
  exatamente a \Macro{leftbotmark}\important{\Macro{leftbotmark}}. Por
  comparação, \Macro{rightbotmark}\important{\Macro{rightbotmark}} imprime a
  parte direita desta marca.

  \Macro{firstmark}\IndexCmd{firstmark}\important{\Macro{firstmark}} é a primeira
  marca da última página que foi construída. Esta é a primeira marca que foi
  definida na página. Se nenhuma marca foi definida na página, ela corresponde à
  última marca das páginas que já foram enviadas. O comando \LaTeX{}
  \DescRef{scrlayer.cmd.rightmark} usa precisamente esta marca, então ele retorna
  a parte direita da primeira marca da página. Isto corresponde exatamente a
  \Macro{rightfirstmark}\important{\Macro{rightfirstmark}}. Por comparação,
  \Macro{leftfirstmark}\important{\Macro{leftfirstmark}} imprime a parte esquerda
  desta marca.

  \Macro{topmark}\IndexCmd{topmark}\important{\Macro{topmark}} é o conteúdo que
  \Macro{botmark} tinha antes de construir a página atual. O \LaTeX{} em si não o
  usa. No entanto, \IfThisCommonLabelBase{scrlayer}{%
    \Package{scrlayer}%
  }{%
    \hyperref[cha:scrlayer]{\Package{scrlayer}}%
  } fornece \Macro{lefttopmark}\important{\Macro{lefttopmark}} para acessar a
  parte esquerda desta marca e \Macro{righttopmark}%
  \important{\Macro{righttopmark}} para acessar a parte direita.

  Note\textnote{Atenção!} que as partes esquerda e direita da marca só podem ser
  definidas juntas. Mesmo se você usar
  \DescRef{scrlayer.cmd.markright}\IndexCmd{markright} para mudar apenas a parte
  direita, a parte esquerda será definida novamente (inalterada). Consequentemente,
  na impressão de dois lados, usando o estilo de página \PageStyle{headings}%
  \important{\PageStyle{headings}}\IndexPagestyle{headings}, os níveis de seção
  superiores sempre fazem ambas as partes. Por exemplo,
  \DescRef{scrlayer.cmd.chaptermark} usa \DescRef{scrlayer.cmd.markboth} com um
  argumento direito vazio neste caso. Esta é a razão pela qual
  \DescRef{scrlayer.cmd.rightmark} ou \Macro{rightfirstmark} sempre mostra um
  valor vazio em páginas que iniciam um capítulo, mesmo se houve um
  \DescRef{scrlayer.cmd.sectionmark} ou \DescRef{maincls.cmd.section} na mesma
  página para fazer a parte direita da marca.
  Por favor, note\textnote{Atenção!} que usar qualquer um desses comandos para
  mostrar a parte esquerda ou direita da marca como parte do conteúdo da página
  pode levar a resultados inesperados. Eles são realmente destinados para uso no
  cabeçalho ou rodapé de um estilo de página apenas. Portanto, eles devem sempre
  ser parte do conteúdo de uma camada ao usar \IfThisCommonLabelBase{scrlayer}{%
    \Package{scrlayer}%
  }{%
    \hyperref[cha:scrlayer]{\Package{scrlayer}}%
  }. Mas não importa se a camada está restrita ao fundo ou ao primeiro plano, já
  que todas as camadas são enviadas após construir a página atual.

  Se você precisar de mais informações sobre o mecanismo de marca
  \iffree{do \TeX{}}{\unskip}, por favor, dê uma olhada em
  \cite[capítulo~23]{knuth:texbook}. O tópico é marcado lá como um assunto para
  verdadeiros especialistas. \IfThisCommonLabelBase{scrlayer}{% Umbruchkorrektur
  Então, se a explicação acima o confundiu, por favor, não se preocupe com isso.}{}%
  \EndIndexGroup
\fi


\IfThisCommonLabelBase{scrlayer-scrpage-experts}{\iffalse}{%
  \csname iftrue\endcsname}%
  \begin{Declaration}
    \IfThisCommonLabelBase{scrlayer-scrpage}{%
      \Macro{leftmark}
      \Macro{rightmark}
    }{}%
    \Macro{headmark}
    \Macro{pagemark}
  \end{Declaration}
  \IfThisCommonLabelBase{scrlayer-scrpage}{%
    Se você quer se afastar dos estilos de página predefinidos, você tipicamente
    precisa decidir onde colocar o conteúdo das marcas. Com
    \Macro{leftmark}\important{\Macro{leftmark}} você pode definir o que
    aparecerá na marca esquerda quando a página for enviada.

    Similarmente, você pode usar \Macro{rightmark}\important{\Macro{rightmark}}
    para definir o conteúdo da marca direita.\iffree{}{ Para informações sobre
      algumas sutilezas ao usar esses comandos, veja mais adiante
      \DescRef{maincls-experts.cmd.rightmark} em
      \autoref{sec:maincls-experts.addInfos},
      \DescPageRef{maincls-experts.cmd.rightmark}.}

  }{}%

  Você pode tornar a vida mais fácil com
  \Macro{headmark}\important{\Macro{headmark}}. Esta extensão do
  \IfThisCommonLabelBase{scrlayer}{%
    \Package{scrlayer}%
  }{%
    \hyperref[cha:scrlayer]{\Package{scrlayer}}%
  } é uma forma abreviada que resolve para \Macro{leftmark} ou \Macro{rightmark}
  dependendo se a página atual é par ou ímpar.

  O comando \Macro{pagemark}\important{\Macro{pagemark}} não tem nada a ver com o
  mecanismo de marca do \TeX. Ele é usado para produzir um número de página
  formatado. \BeginIndex{FontElement}{pagenumber}\LabelFontElement{pagenumber}%
  A fonte do elemento \FontElement{pagenumber}\important{\FontElement{pagenumber}}
  será usada para a saída. Isto pode ser alterado usando os comandos
  \Macro{setkomafont} ou \DescRef{maincls.cmd.addtokomafont} (veja também
  \autoref{sec:maincls.textmarkup},
  \DescPageRef{maincls.cmd.setkomafont}).%
  \EndIndex{FontElement}{pagenumber}%
  \IfThisCommonLabelBase{scrlayer-scrpage}{\iftrue}{%
    \par %
    Se você estiver interessado em um exemplo mostrando o uso dos comandos
    \Macro{headmark} e \Macro{pagemark}, veja
    \autoref{sec:scrlayer-scrpage.pagestyle.content},
    \PageRefxmpl{scrlayer-scrpage.cmd.headmark}. Internamente, o pacote
    \IfThisCommonLabelBase{scrlayer-scrpage}{%
    	\Package{scrlayer-scrpage}%
    }{%
    	\hyperref[cha:scrlayer-scrpage]{\Package{scrlayer-scrpage}}%
    } usa muitos desses recursos do
    \IfThisCommonLabelBase{scrlayer-scrpage}{%
    	\hyperref[cha:scrlayer]{\Package{scrlayer}}%
    }{%
    	\Package{scrlayer}%
    }.%
    \csname iffalse\endcsname}%
    \begin{Example}
      \phantomsection\xmpllabel{cmd.headmark}%
      Suponha que você quer que o título corrente seja alinhado à margem esquerda
      e o número da página à margem direita na impressão de um lado. O seguinte
      exemplo mínimo funcional faz exatamente isso:
\begin{lstcode}
  \documentclass{scrreprt}
  \usepackage{blindtext}
  \usepackage[automark]{scrlayer-scrpage}
  \pagestyle{scrheadings}
  \ihead{\headmark}
  \ohead*{\pagemark}
  \chead{}
  \cfoot[]{}
  \begin{document}
  \blinddocument
  \end{document}
\end{lstcode}
      O pacote \Package{blindtext}\IndexPackage{blindtext} e seu comando
      \Macro{blinddocument}\IndexCmd{blinddocument} foram usados aqui para gerar
      rapidamente conteúdo de documento de amostra para o exemplo.

      Os comandos \DescRef{scrlayer-scrpage.cmd.ihead}\IndexCmd{ihead} e
      \DescRef{scrlayer-scrpage.cmd.ohead*}\IndexCmd{ohead*} configuram as marcas
      desejadas. A variante com asterisco \DescRef{scrlayer-scrpage.cmd.ohead*} %
      \iffalse% Umbruchvarianten
        configura a marca de número de página não apenas nas páginas definidas
        com o estilo de página
        \DescRef{\LabelBase.pagestyle.scrheadings}%
        \IndexPagestyle{scrheadings} mas também aquelas definidas com o estilo
        \PageStyle{plain} %
      \else%
        também configura o número de página com o %
      \fi%
      \DescRef{\LabelBase.pagestyle.plain.scrheadings}%
      \IndexPagestyle{plain.scrheadings} estilo de página usado na primeira
      página de um capítulo.%

      Porque esses estilos de página têm marcas predefinidas no centro do
      cabeçalho e rodapé, esses elementos são limpos usando
      \DescRef{scrlayer-scrpage.cmd.chead} e \DescRef{scrlayer-scrpage.cmd.cfoot}
      com argumentos vazios. Alternativamente, você poderia usar
      \DescRef{scrlayer-scrpage-experts.cmd.clearpairofpagestyles}
      \IndexCmd{clearpairofpagestyles} \emph{antes} de
      \DescRef{scrlayer-scrpage.cmd.ihead}. Você encontrará este comando descrito
      em \autoref{sec:scrlayer-scrpage-experts.pagestyle.pairs}.
    \end{Example}

    Por favor, note\textnote{Atenção!} que o argumento opcional vazio de
    \DescRef{scrlayer-scrpage.cmd.cfoot} no exemplo acima não é o mesmo que omitir
    o argumento opcional. Você deve tentar você mesmo e dar uma olhada na
    diferença no rodapé da primeira página.%
  \fi

  \IfThisCommonLabelBase{scrlayer-scrpage}{% Umbruchvarianten
    Usuários avançados podem encontrar mais comandos de definição de marca%
  }{%
    Se as opções para as marcas descritas acima não são suficientes, comandos
    adicionais para usuários avançados são documentados%
  } %
  começando em \IfThisCommonLabelBase{scrlayer-scrpage}{%
    \DescPageRef{scrlayer-scrpage-experts.cmd.righttopmark}}{%
    \DescPageRef{\ThisCommonLabelBase.cmd.righttopmark}}.%
  \iffalse% Umbruchkorrektur
    \ Por exemplo, lá você pode encontrar
    \DescRef{scrlayer-scrpage-experts.cmd.leftfirstmark} e
    \DescRef{scrlayer-scrpage-experts.cmd.rightbotmark}, que são bastante úteis
    para documentos como léxicos.%
  \fi%
  \EndIndexGroup


  \begin{Declaration}
    \Macro{partmarkformat}%
    \Macro{chaptermarkformat}%
    \Macro{sectionmarkformat}%
    \Macro{subsectionmarkformat}%
    \Macro{subsubsectionmarkformat}%
    \Macro{paragraphmarkformat}%
    \Macro{subparagraphmarkformat}
  \end{Declaration}
  As classes \KOMAScript{} e o pacote \IfThisCommonLabelBase{scrlayer}{%
    \Package{scrlayer}%
  }{%
    \hyperref[cha:scrlayer]{\Package{scrlayer}}%
  } tipicamente usam esses comandos internamente para formatar os números de
  seção. Eles também suportam o mecanismo \DescRef{maincls.cmd.autodot} das
  classes \KOMAScript{}. Se desejado, esses comandos podem ser redefinidos para
  alcançar uma formatação diferente dos números de seção.%
  \IfThisCommonLabelBase{scrlayer-scrpage}{\iftrue}{%
    \ Veja o exemplo em \autoref{sec:scrlayer-scrpage.pagestyle.content},
    \PageRefxmpl{scrlayer-scrpage.cmd.sectionmarkformat} para mais informações.%
    \csname iffalse\endcsname%
  }%
    \begin{Example}
      \phantomsection\xmpllabel{cmd.sectionmarkformat}%
      \iftrue
        Por exemplo, se você quer ter títulos correntes sem um número de seção,
        é assim que você faz:
      \else
        Suponha que você quer que títulos de seção sem o número da seção apareçam
        no título corrente. Isto pode ser realizado facilmente com o seguinte:
      \fi
\begin{lstcode}
  \renewcommand*{\sectionmarkformat}{}
\end{lstcode}
    \end{Example}
    \ExampleEndFix
  \fi%
  \EndIndexGroup


  \begin{Declaration}
    \Macro{partmark}\Parameter{Texto}%
    \Macro{chaptermark}\Parameter{Texto}%
    \Macro{sectionmark}\Parameter{Texto}%
    \Macro{subsectionmark}\Parameter{Texto}%
    \Macro{subsubsectionmark}\Parameter{Texto}%
    \Macro{paragraphmark}\Parameter{Texto}%
    \Macro{subparagraphmark}\Parameter{Texto}
  \end{Declaration}
  A maioria das classes usa esses comandos internamente para definir as marcas de
  acordo com os comandos de seção. O argumento deve conter o texto sem o número da
  unidade de seção. O número é automaticamente determinado usando o nível de seção
  atual se você usar títulos numerados.

  Contudo\textnote{Atenção!}, nem todas as classes usam tal comando para cada
  nível de seção. As classes padrão\textnote{\KOMAScript{} vs. classes padrão},
  por exemplo, não chamam \Macro{partmark}
  \IfThisCommonLabelBase{scrlayer-scrpage}{%
    ao usar um comando \Macro{part}}{%
    , enquanto as classes \KOMAScript{} naturalmente suportam \Macro{partmark}
    também}.

  Se você redefinir esses comandos, certifique-se\textnote{Atenção!} de verificar
  se os números serão produzidos via \DescRef{maincls.counter.secnumdepth} antes
  de definir o número mesmo se você não mudar o contador
  \DescRef{maincls.counter.secnumdepth} você mesmo, porque pacotes e classes
  podem fazê-lo localmente e confiar no manuseio correto de
  \DescRef{maincls.counter.secnumdepth}.

  O pacote \IfThisCommonLabelBase{scrlayer}{%
    \Package{scrlayer}%
  }{%
    \hyperref[cha:scrlayer]{\Package{scrlayer}}%
  } também redefine esses comandos sempre que você usar
  \DescRef{scrlayer.cmd.automark} ou \DescRef{scrlayer.cmd.manualmark} ou as
  opções correspondentes, para ativar ou desativar os títulos correntes desejados.%
  \EndIndexGroup


  \begin{Declaration}
    \Macro{markleft}\Parameter{marca esquerda}
    \Macro{markright}\Parameter{marca direita}
    \Macro{markboth}\Parameter{marca esquerda}\Parameter{marca direita}
    \Macro{markdouble}\Parameter{marca}
  \end{Declaration}
  Independentemente de você estar trabalhando com títulos correntes manuais ou
  automáticos, você sempre pode mudar o conteúdo da \PName{marca esquerda} ou da
  \PName{marca direita} usando esses comandos. Note que a marca à esquerda
  resultante de \Macro{leftmark}\IndexCmd{leftmark}%
  \important{\Macro{leftmark}} será a última marca colocada na página
  correspondente, enquanto a marca à direita resultante de
  \Macro{rightmark}\IndexCmd{rightmark}\important{\Macro{rightmark}} é a primeira
  marca colocada na página correspondente. Para mais detalhes, veja
  \iffree{}{a explicação adicional de \DescRef{maincls-experts.cmd.rightmark} em
    \autoref{sec:maincls-experts.addInfos},
    \DescPageRef{maincls-experts.cmd.rightmark} ou} para
  \DescRef{scrlayer.cmd.rightfirstmark}\IfThisCommonLabelBase{scrlayer}{}{ em
  	\autoref{sec:scrlayer.pagestyle.content}},
  \DescPageRef{scrlayer.cmd.rightfirstmark}.

  Se você estiver usando títulos correntes manuais\Index{título corrente>manual},
  as marcas permanecem válidas até que sejam explicitamente substituídas ao
  reutilizar os comandos correspondentes. Contudo, se você estiver usando títulos
  correntes automáticos, as marcas podem se tornar inválidas com o próximo título
  de seção, dependendo da configuração automática.

  Você também pode usar esses comandos em conjunto com as versões com asterisco
  dos comandos de seção.%
  \IfThisCommonLabelBase{scrlayer-scrpage}{\iftrue}{%
    \ Você pode encontrar exemplos detalhados ilustrando o uso de \Macro{markboth}
    com o pacote derivado do \IfThisCommonLabelBase{scrlayer-scrpage}{%
    	\hyperref[cha:scrlayer]{\Package{scrlayer}}%
    }{%
    	\Package{scrlayer}%
    } \IfThisCommonLabelBase{scrlayer-scrpage}{%
    	\Package{scrlayer-scrpage}%
    }{%
    	\hyperref[cha:scrlayer-scrpage]{\Package{scrlayer-scrpage}}%
    } em
    \autoref{sec:scrlayer-scrpage.pagestyle.content},
    \PageRefxmpl{scrlayer-scrpage.cmd.markboth}.%
    \csname iffalse\endcsname%
  }%
    \begin{Example}
      \phantomsection\xmpllabel{cmd.markboth}%
      Suponha que você escreva um prefácio de várias páginas colocado logo antes
      do sumário mas não aparecendo nele. Contudo, já que você usa linhas
      divisórias em seu cabeçalho, você quer um título corrente para este
      prefácio:
\begin{lstcode}
  \documentclass[headsepline]{book}
  \usepackage{scrlayer-scrpage}
  \pagestyle{scrheadings}
  \usepackage{blindtext}
  \begin{document}
  \chapter*{Prefácio}
  \markboth{Prefácio}{Prefácio}
  \blindtext[20]
  \tableofcontents
  \blinddocument
  \end{document}
\end{lstcode}
      À primeira vista, isto parece produzir o resultado desejado. Dando uma
      segunda olhada, contudo, você pode ver que o título corrente
      ``\texttt{Prefácio}'' não aparece em letras maiúsculas, ao contrário dos
      outros títulos correntes. Mas isso é fácil de mudar:
\begin{lstcode}
  \documentclass[headsepline]{book}
  \usepackage{scrlayer-scrpage}
  \pagestyle{scrheadings}
  \usepackage{blindtext}
  \begin{document}
  \chapter*{Prefácio}
  \markboth{\MakeMarkcase{Prefácio}}{\MakeMarkcase{Prefácio}}
  \blindtext[20]
  \tableofcontents
  \blinddocument
  \end{document}
\end{lstcode}
      Usar o comando \DescRef{\LabelBase.cmd.MakeMarkcase} resulta em obter a
      mesma caixa de letras que para títulos correntes automáticos.

      Agora, vamos mover o \DescRef{maincls.cmd.tableofcontents} na frente do
      prefácio e remover o comando \Macro{markboth}. Você descobrirá que o
      prefácio agora tem o título corrente ``\texttt{SUMÁRIO}''. Isto se deve a
      uma peculiaridade de \DescRef{maincls.cmd.chapter*} (veja também
      \autoref{sec:maincls.structure} em
      \DescPageRef{maincls.cmd.chapter*}). Se você não quer um título corrente
      aqui, você pode facilmente realizar isto passando dois argumentos vazios
      para \Macro{markboth}:
\begin{lstcode}
  \documentclass[headsepline]{book}
  \usepackage{scrlayer-scrpage}
  \pagestyle{scrheadings}
  \usepackage{blindtext}
  \begin{document}
  \tableofcontents
  \chapter*{Prefácio}
  \markboth{}{}
  \blindtext[20]
  \blinddocument
  \end{document}
\end{lstcode}
    \end{Example}
  \fi%
% TODO: Fix new translation
  O comando\ChangedAt{v3.28}{\Package{scrlayer}}
  \Macro{markdouble}\important{\Macro{markdouble}} muda a marca esquerda e a
  marca direita para o mesmo conteúdo. Então
  \Macro{markdouble}\Parameter{marca} é uma forma mais curta de
  \Macro{markboth}\Parameter{marca}\Parameter{marca} com dois argumentos
  idênticos.%
% :ODOT
  \EndIndexGroup
\fi


\IfThisCommonLabelBase{scrlayer-scrpage}{\iffalse}{\csname iftrue\endcsname}
  \begin{Declaration}
    \Macro{GenericMarkFormat}\Parameter{nome do nível de seção}
  \end{Declaration}
  Por padrão, este comando é usado para formatar todos os números de seção em
  títulos correntes abaixo do nível de subseção, e para classes sem
  \DescRef{maincls.cmd.chapter}, também para os níveis de seção e subseção, a
  menos que os respectivos comandos de marca para esses níveis sejam definidos
  antes de carregar \Package{scrlayer}. O comando faz com que o pacote use
  \Macro{@seccntmarkformat}\IndexCmd{@seccntmarkformat}%
  \important{\Macro{@seccntmarkformat}} se este comando interno estiver definido,
  como está nas classes \KOMAScript{}. Caso contrário, \Macro{@seccntformat}%
  \IndexCmd{@seccntformat}\important{\Macro{@seccntformat}} será usado, que é
  fornecido pelo kernel \LaTeX{}. O argumento obrigatório do comando contém o
  nome de um comando de seção, como \PValue{chapter} ou \PValue{section}
  \emph{sem} o prefixo de barra invertida.

  Ao redefinir este comando, você pode mudar o formato de número padrão para todos
  os comandos de seção que o usam. Classes também podem mudar a formatação padrão
  definindo este comando.%
  \IfThisCommonLabelBase{scrlayer-scrpage-experts}{\iftrue}{%
    \par %
    Um exemplo detalhado ilustrando a interação do comando
    \Macro{GenericMarkFormat} com o comando
    \DescPageRef{\ThisCommonLabelBase.cmd.chaptermark} e
    \DescRef{\ThisCommonLabelBase.cmd.sectionmarkformat} ou
    \DescRef{\ThisCommonLabelBase.cmd.subsectionmarkformat} ao usar o pacote
    derivado do \IfThisCommonLabelBase{scrlayer}{%
    	\Package{scrlayer}%
    }{%
    	\hyperref[cha:scrlayer]{\Package{scrlayer}}%
    } \IfThisCommonLabelBase{scrlayer-scrpage}{%
    	\Package{scrlayer-scrpage}%
    }{%
    	\hyperref[cha:scrlayer-scrpage]{\Package{scrlayer-scrpage}}%
    } é mostrado em \autoref{sec:scrlayer-scrpage-experts.pagestyle.content},
    \PageRefxmpl{scrlayer-scrpage-experts.cmd.GenericMarkFormat}.%
    \csname iffalse\endcsname}%
    \begin{Example}
      \phantomsection
      \xmpllabel{cmd.GenericMarkFormat}%
      Suponha que você quer que os números de seção de todos os níveis no título
      corrente de um artigo sejam impressos em branco dentro de uma caixa preta.
      Como os comandos \Macro{sectionmarkformat} e \Macro{subsectionmarkformat} do
      pacote \Package{scrlayer} são definidos com \Macro{GenericMarkFormat} para
      artigos usando a classe padrão \LaTeX{} \Class{article}, você precisa
      redefinir apenas este único comando:
\begin{lstcode}
  \documentclass{article}
  \usepackage{blindtext}
  \usepackage[automark]{scrlayer-scrpage}
  \pagestyle{scrheadings}
  \usepackage{xcolor}
  \newcommand*{\numberbox}[1]{%
    \colorbox{black}{\strut~\textcolor{white}{#1}~}}
  \renewcommand*{\GenericMarkFormat}[1]{%
    \protect\numberbox{\csname the#1\endcsname}\enskip}
  \begin{document}
  \blinddocument
  \end{document}
\end{lstcode}
      A cor foi alterada usando o pacote \Package{xcolor}\IndexPackage{xcolor}.
      Mais detalhes podem ser encontrados no manual desse pacote (veja
      \cite{package:xcolor}).

      Este exemplo também usa um strut invisível. Qualquer introdução detalhada ao
      \LaTeX{} deve explicar o comando relacionado \Macro{strut}.

      Uma macro auxiliar, \Macro{numberbox}, foi definida para formatar o número
      dentro de uma caixa. Este comando é prefixado na redefinição de
      \Macro{GenericMarkFormat} por \Macro{protect} a fim de protegê-lo da
      expansão. Isto é necessário porque caso contrário a conversão para letra
      maiúscula de \Macro{MakeUppercase} para o título corrente resultaria em
      usar as cores ``\texttt{BLACK}'' e ``\texttt{WHITE}'' em vez de
      ``\texttt{black}'' e ``\texttt{white}'', e essas cores estão indefinidas.
      Alternativamente, você poderia definir \Macro{numberbox} usando
      \Macro{DeclareRobustCommand*} em vez de \Macro{newcommand*} e omitir
      \Macro{protect} (veja \cite{latex:clsguide}).

      Se você quisesse alcançar o mesmo efeito com uma classe \KOMAScript{} ou com
      as classes padrão \LaTeX{} \Class{book} ou \Class{report}, você também teria
      que redefinir, respectivamente,
      \DescRef{scrlayer.cmd.sectionmarkformat}\IndexCmd{sectionmarkformat}%
      \important{\DescRef{scrlayer.cmd.sectionmarkformat}} e
      \DescRef{scrlayer.cmd.subsectionmarkformat}%
      \IndexCmd{subsectionmarkformat}%
      \important{\DescRef{scrlayer.cmd.subsectionmarkformat}}, ou
      \DescRef{scrlayer.cmd.chaptermarkformat}%
      \IndexCmd{chaptermarkformat}%
      \important{\DescRef{scrlayer.cmd.chaptermarkformat}} e
      \DescRef{scrlayer.cmd.sectionmarkformat}%
      \IndexCmd{sectionmarkformat}%
      \important{\DescRef{scrlayer.cmd.sectionmarkformat}}, porque estes não são
      por padrão definidos usando \Macro{GenericMarkFormat}:
\begin{lstcode}
  \documentclass[headheight=19.6pt]{scrbook}
  \usepackage{blindtext}
  \usepackage[automark]{scrlayer-scrpage}
  \pagestyle{scrheadings}
  \usepackage{xcolor}
  \newcommand*{\numberbox}[1]{%
    \colorbox{black}{\strut~\textcolor{white}{#1}~}}
  \renewcommand*{\GenericMarkFormat}[1]{%
    \protect\numberbox{\csname the#1\endcsname}\enskip}
  \renewcommand*{\chaptermarkformat}{\GenericMarkFormat{chapter}}
  \renewcommand*{\sectionmarkformat}{\GenericMarkFormat{section}}
  \begin{document}
  \blinddocument
  \end{document}
\end{lstcode}
      % TODO: New translation
      Aqui, a opção \DescRef{typearea.option.headheight} é usada para eliminar o
      aviso que também foi relatado no exemplo anterior.%
      % :ODOT
    \end{Example}
  \fi%
  \EndIndexGroup
\fi


\IfThisCommonLabelBase{scrlayer-scrpage}{\iffalse}{\csname iftrue\endcsname}
  \begin{Declaration}
    \Macro{@mkleft}\Parameter{marca esquerda}%
    \Macro{@mkright}\Parameter{marca direita}%
    \Macro{@mkdouble}\Parameter{marca}%
    \Macro{@mkboth}\Parameter{marca esquerda}\Parameter{marca direita}
  \end{Declaration}
  Dentro de classes e pacotes, você pode querer usar apenas títulos correntes se
  títulos correntes automáticos estiverem ativos (veja a opção
  \DescRef{scrlayer.option.automark} e o comando
  \DescRef{scrlayer.cmd.automark} em \DescPageRef{scrlayer-scrpage.cmd.automark}).
  Nas classes padrão \LaTeX{}, isto só funciona com \Macro{@mkboth}. Este comando
  corresponde a \Macro{@gobbletwo}, que simplesmente consome ambos os argumentos
  obrigatórios, ou \DescRef{scrlayer.cmd.markboth}, que define tanto a
  \PValue{marca esquerda} quanto a \PValue{marca direita}. Pacotes como
  \Package{babel} também mudam \Macro{\@mkboth}, por exemplo, para habilitar a
  troca de idioma no título corrente.

  Contudo, se você quer mudar apenas a \PName{marca esquerda} ou a \PName{marca
  direita} sem mudar a outra, não há comando correspondente. O pacote
  \IfThisCommonLabelBase{scrlayer}{%
    \Package{scrlayer}%
  }{%
    \hyperref[cha:scrlayer]{\Package{scrlayer}}%
  } em si precisa de tais comandos para implementar títulos correntes automáticos.
  Então, se \Macro{@mkleft}, para definir apenas a marca esquerda, ou
  \Macro{@mkright}, para definir apenas a marca direita, ou \Macro{@mkdouble},
  para definir ambas as marcas com o mesmo conteúdo, estiver indefinido ao carregar
  \IfThisCommonLabelBase{scrlayer}{%
    \Package{scrlayer}%
  }{%
    \hyperref[cha:scrlayer]{\Package{scrlayer}}%
  }, este pacote os definirá. Esta definição usa o estado de \Macro{@mkboth} como
  uma indicação de se usar títulos correntes automáticos. Os comandos definirão as
  marcas apenas no caso de títulos correntes automáticos.

  Autores de classes e pacotes também podem recorrer a esses comandos conforme
  apropriado se eles quiserem definir a marca esquerda ou direita apenas se
  títulos correntes automáticos estiverem ativados.%
  \EndIndexGroup%
\fi%

\IfThisCommonLabelBase{scrlayer}{%
  \par
  Para mais informações sobre manipular o conteúdo dos estilos de página, veja
  também \autoref{sec:scrlayer-scrpage.pagestyle.content} começando em
  \autopageref{sec:scrlayer-scrpage.pagestyle.content}.%
}{}%
\EndIndexGroup

%%% Local Variables:
%%% mode: latex
%%% TeX-master: "scrguide-en.tex"
%%% coding: utf-8
%%% ispell-local-dictionary: "en_GB"
%%% eval: (flyspell-mode 1)
%%% End:

% LocalWords:  scrlayer
