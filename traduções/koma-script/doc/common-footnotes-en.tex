% ======================================================================
% common-footnotes-en.tex
% Copyright (c) Markus Kohm, 2001-2022
%
% This file is part of the LaTeX2e KOMA-Script bundle.
%
% This work may be distributed and/or modified under the conditions of
% the LaTeX Project Public License, version 1.3c of the license.
% The latest version of this license is in
%   http://www.latex-project.org/lppl.txt
% and version 1.3c or later is part of all distributions of LaTeX
% version 2005/12/01 or later and of this work.
%
% This work has the LPPL maintenance status "author-maintained".
%
% The Current Maintainer and author of this work is Markus Kohm.
%
% This work consists of all files listed in MANIFEST.md.
% ======================================================================
%
% Paragraphs that are common for several chapters of the KOMA-Script guide
% Maintained by Markus Kohm
%
% ======================================================================

\KOMAProvidesFile{common-footnotes-en.tex}
                 [$Date: 2022-06-05 12:40:11 +0200 (So, 05. Jun 2022) $
                  KOMA-Script guide (common paragraphs: Footnotes)]
\translator{Markus Kohm\and Krickette Murabayashi\and Karl Hagen}

\section{Notas de Rodapé}
\seclabel{footnotes}%
\BeginIndexGroup
\BeginIndex{}{footnotes}%

\IfThisCommonFirstRun{}{%
  As informações em \autoref{sec:\ThisCommonFirstLabelBase.footnotes} se aplicam
  igualmente a este capítulo. Portanto, se você já leu e entendeu
  \autoref{sec:\ThisCommonFirstLabelBase.footnotes}, você pode pular para
  \autopageref{sec:\ThisCommonLabelBase.footnotes.next},
  \autopageref{sec:\ThisCommonLabelBase.footnotes.next}.%
  \IfThisCommonLabelBase{scrlttr2}{ %
    Se você não usa uma classe \KOMAScript{}, \Package{scrletter}%
    \OnlyAt{\Package{scrletter}} depende do
    pacote \hyperref[cha:scrextend]{\Package{scrextend}}\IndexPackage{scrextend}%
    \important{\hyperref[cha:scrextend]{\Package{scrextend}}}.
    Portanto, veja também \autoref{sec:scrextend.footnotes},
    \autopageref{sec:scrextend.footnotes} ao usar \Package{scrletter}.%
    \iffalse% Umbruchkorrekturtext
    \ Note particularmente que neste caso algumas extensões típicas do \KOMAScript{}
    não estão ativas por padrão\textnote{padrão}. Em vez disso, as
    notas de rodapé fazem uso da classe usada, ou do núcleo do \LaTeX{}.%
    \fi%
  }{}%
}

\IfThisCommonLabelBase{maincls}{%
  Ao contrário\textnote{\KOMAScript{} vs. classes padrão} das classes padrão,
  o \KOMAScript{} oferece a capacidade de configurar o formato do bloco de
  notas de rodapé.%
}{%
  \IfThisCommonLabelBase{scrlttr2}{%
    Você pode encontrar os comandos básicos para definir notas de rodapé em
    qualquer introdução ao \LaTeX, por exemplo \cite{lshort}. \KOMAScript{}%
    \textnote{\KOMAScript{} vs. classes padrão} fornece recursos adicionais
    para alterar o formato do bloco de notas de rodapé. %
    \iffalse % Umbruchoptimierung

      Se notas de rodapé devem ser permitidas em cartas depende muito do
      tipo de carta e seu layout. Por exemplo, você não deve permitir que
      notas de rodapé se sobreponham visualmente ao rodapé do papel timbrado ou
      sejam confundidas com a lista de cópias de cortesia. Fazer isso é
      responsabilidade do usuário.%

      Como as notas de rodapé são raramente usadas em cartas, exemplos nesta
      seção foram omitidos. Se você precisa de exemplos, você pode encontrá-los em
      \autoref{sec:\ThisCommonFirstLabelBase.footnotes},
      \autopageref{sec:\ThisCommonFirstLabelBase.footnotes}.%
    \fi%
  }{%
    \IfThisCommonLabelBase{scrextend}{%
      As capacidades de notas de rodapé das classes \KOMAScript{} também são
      fornecidas pelo \Package{scrextend}. Por padrão, a formatação das
      notas de rodapé é deixada para a classe usada. Isso muda assim que você
      emite o comando \DescRef{\ThisCommonLabelBase.cmd.deffootnote},
      que é explicado em detalhes em
      \DescPageRef{\ThisCommonLabelBase.cmd.deffootnote}.

      As opções para ajustar a linha divisória acima das notas de rodapé,
      no entanto, não são fornecidas pelo \Package{scrextend}.%
    }{\InternalCommonFileUsageError}%
  }%
}%

\begin{Declaration}
  \OptionVName{footnotes}{configuração}
  \Macro{multfootsep}
\end{Declaration}
\IfThisCommonLabelBase{scrextend}{Muitas classes marcam notas de rodapé }{%
  Notas de rodapé %
  \IfThisCommonLabelBase{maincls}{%
    \ChangedAt{v3.00}{\Class{scrbook}\and \Class{scrreprt}\and
      \Class{scrartcl}}%
  }{%
    \IfThisCommonLabelBase{scrlttr2}{%
      \ChangedAt{v3.00}{\Class{scrlttr2}}%
    }{}%
  }%
  são marcadas %
}%
por padrão no texto com um pequeno número sobrescrito. Se várias notas de
rodapé aparecem em sucessão no mesmo ponto, isso dá a impressão de que há
uma nota de rodapé com um número grande em vez de múltiplas notas de rodapé
(por exemplo, nota de rodapé 12 em vez de notas de rodapé 1 e 2).
Com\important{\OptionValue{footnotes}{multiple}}
\OptionValue{footnotes}{multiple}\IndexOption{footnotes=~multiple}, notas de
rodapé que seguem umas às outras diretamente são separadas com um delimitador.
O delimitador padrão em \Macro{multfootsep}\important{\Macro{multfootsep}} é
definido como uma vírgula sem espaço:
\begin{lstcode}
  \newcommand*{\multfootsep}{,}
\end{lstcode}
Isso pode ser redefinido.

Todo o mecanismo é compatível com o
pacote \Package{footmisc}\IndexPackage{footmisc}\important{\Package{footmisc}},
versões~5.3d a 5.5b (veja \cite{package:footmisc}). Ele afeta
marcadores de notas de rodapé colocados usando
\DescRef{\ThisCommonLabelBase.cmd.footnote}\IndexCmd{footnote}, bem como
aqueles colocados diretamente com
\DescRef{\ThisCommonLabelBase.cmd.footnotemark}\IndexCmd{footnotemark}.

Você pode voltar ao padrão
\OptionValue{footnotes}{nomultiple} a qualquer momento usando o
comando \DescRef{\ThisCommonLabelBase.cmd.KOMAoptions} ou
\DescRef{\ThisCommonLabelBase.cmd.KOMAoption}. No entanto, se você
encontrar algum problema ao usar outro pacote que altera as notas de rodapé,
você não deve usar esta opção, nem deve alterar a \PName{configuração} em
qualquer lugar dentro do documento.

Um resumo dos valores de \PName{configuração} disponíveis para \Option{footnotes}
pode ser encontrado em \autoref{tab:\ThisCommonFirstLabelBase.footnotes}%
\IfThisCommonFirstRun{%
  .
  \begin{table}
    \caption[{Valores disponíveis para a opção \Option{footnotes}}]
    {Valores disponíveis para a opção \Option{footnotes} para configurar notas de rodapé}
    \label{tab:\ThisCommonLabelBase.footnotes}
    \begin{desctabular}
      \pventry{multiple}{%
        Marcadores de notas de rodapé consecutivos serão separados por
        \DescRef{\ThisCommonLabelBase.cmd.multfootsep}\IndexCmd{multfootsep}.%
        \IndexOption{footnotes~=\textKValue{multiple}}}%
      \pventry{nomultiple}{%
        Marcadores de notas de rodapé consecutivos serão tratados como
        notas de rodapé únicas e não separados uns dos outros.%
        \IndexOption{footnotes~=\textKValue{nomultiple}}}%
    \end{desctabular}
  \end{table}%
}{,
  \autopageref{tab:\ThisCommonFirstLabelBase.footnotes}.%
}%
%
\EndIndexGroup


\begin{Declaration}
  \Macro{footnote}\OParameter{número}\Parameter{texto}%
  \Macro{footnotemark}\OParameter{número}%
  \Macro{footnotetext}\OParameter{número}\Parameter{texto}%
  \Macro{multiplefootnoteseparator}%
\end{Declaration}%
Notas de rodapé no \KOMAScript{} são produzidas, como são nas classes padrão,
com o comando \Macro{footnote}, ou alternativamente o par de comandos
\Macro{footnotemark} e \Macro{footnotetext}. Como nas classes padrão,
é possível que ocorra uma quebra de página dentro de uma nota de rodapé.
Normalmente isso acontece se o marcador de nota de rodapé é colocado tão perto
da parte inferior de uma página que não deixa ao \LaTeX{} nenhuma escolha senão
mover a nota de rodapé para a próxima página.
Ao contrário\textnote{\KOMAScript{} vs. classes padrão}
\IfThisCommonLabelBase{maincls}{%
  \ChangedAt{v3.00}{\Class{scrbook}\and \Class{scrreprt}\and
    \Class{scrartcl}}%
}{%
  \IfThisCommonLabelBase{scrlttr2}{%
    \ChangedAt{v3.00}{\Class{scrlttr2}}%
  }{}%
} %
das classes padrão, o \KOMAScript{} pode reconhecer e separar notas de rodapé
consecutivas automaticamente.
Veja\important{\DescRef{\ThisCommonLabelBase.option.footnotes}} a opção
documentada anteriormente \DescRef{\ThisCommonLabelBase.option.footnotes}.

Se em vez disso você quiser colocar este delimitador manualmente, você pode
fazê-lo chamando \Macro{multiplefootnoteseparator}. No entanto, os usuários não
devem redefinir este comando, pois ele contém não apenas o delimitador, mas
também a formatação do delimitador, por exemplo a seleção de tamanho de fonte e
o sobrescrito. O delimitador em si é armazenado no comando
\DescRef{\ThisCommonLabelBase.cmd.multfootsep}%
\important{\DescRef{\ThisCommonLabelBase.cmd.multfootsep}}%
\IndexCmd{multfootsep} descrito anteriormente.

\IfThisCommonFirstRun{\iftrue}{%
  Você pode encontrar exemplos e dicas adicionais em
  \autoref{sec:\ThisCommonFirstLabelBase.footnotes} de
  \PageRefxmpl{\ThisCommonFirstLabelBase.cmd.footnote}.%
  \csname iffalse\endcsname }%
  \begin{Example}
    \phantomsection\xmpllabel{cmd.footnote}%
    Suponha que você queira colocar duas notas de rodapé após uma única palavra.
    Primeiro você tenta
\begin{lstcode}
  Palavra\footnote{1ª nota de rodapé}\footnote{2ª nota de rodapé}
\end{lstcode}
    Vamos supor que as notas de rodapé estejam numeradas 1 e 2. Como os dois
    números de notas de rodapé seguem um ao outro diretamente, isso cria a
    impressão de que a palavra tem apenas uma nota de rodapé numerada 12. Você
    pode alterar este comportamento usando
\begin{lstcode}
  \KOMAoptions{footnotes=multiple}
\end{lstcode}
    para habilitar o reconhecimento automático de sequências de notas de rodapé.
    Alternativamente, você pode usar
\begin{lstcode}
  palavra\footnote{Nota de rodapé 1}%
  \multiplefootnoteseparator
  \footnote{Nota de rodapé 2}
\end{lstcode}
    Isso deve dar a você o resultado desejado mesmo se a detecção automática
    falhar ou não puder ser usada por alguma razão.

    Agora suponha que você também queira que os números de notas de rodapé sejam
    separados não apenas por uma vírgula, mas por uma vírgula e um espaço. Neste
    caso, escreva
\begin{lstcode}
  \renewcommand*{\multfootsep}{,\nobreakspace}
\end{lstcode}
    no preâmbulo do seu documento.
    \Macro{nobreakspace}\IndexCmd{nobreakspace} foi usado aqui em vez de um
    espaço normal para evitar quebras de parágrafo ou página dentro da sequência
    de notas de rodapé.
  \end{Example}%
\fi%
%
\EndIndexGroup


\begin{Declaration}
  \Macro{footref}\Parameter{referência}
\end{Declaration}
Às vezes\IfThisCommonLabelBase{maincls}{%
  \ChangedAt{v3.00}{\Class{scrbook}\and \Class{scrreprt}\and
    \Class{scrartcl}}%
}{%
  \IfThisCommonLabelBase{scrlttr2}{%
    \ChangedAt{v3.00}{\Class{scrlttr2}}%
  }{}} você tem uma nota de rodapé em um documento à qual há várias referências
no texto. Uma maneira inconveniente de compor isso seria usar
\DescRef{\ThisCommonLabelBase.cmd.footnotemark} para definir o número diretamente.
A desvantagem deste método é que você precisa saber o número e
manualmente definir cada comando \DescRef{\ThisCommonLabelBase.cmd.footnotemark}.
E se o número mudar porque você adiciona ou remove uma nota de rodapé anterior,
você terá que alterar cada \DescRef{\ThisCommonLabelBase.cmd.footnotemark}.
O \KOMAScript{} portanto oferece o mecanismo \Macro{label}\IndexCmd{label}%
\important{\Macro{label}} para lidar com tais casos. Depois de colocar um
\Macro{label} dentro da nota de rodapé, você pode usar \Macro{footref} para
definir todos os outros marcadores para esta nota de rodapé no texto.
\IfThisCommonFirstRun{\iftrue}{\csname iffalse\endcsname}%
  \begin{Example}
    \phantomsection\xmpllabel{cmd.footref}%
    Você está escrevendo um texto no qual você deve criar uma nota de rodapé
    cada vez que um nome de marca ocorre, indicando que é uma marca registrada.
    Você pode escrever, por exemplo,
\begin{lstcode}
  Empresa SplishSplash\footnote{Este é um nome comercial registrado.
    Todos os direitos são reservados.\label{refnote}}
  produz não apenas SplishPlump\footref{refnote}
  mas também SplishPlash\footref{refnote}.
\end{lstcode}
    Isso produzirá o mesmo marcador de nota de rodapé três vezes, mas apenas um
    texto de nota de rodapé. O primeiro marcador de nota de rodapé é produzido
    pelo próprio \DescRef{\ThisCommonLabelBase.cmd.footnote}, e os dois
    marcadores de notas de rodapé seguintes são produzidos pelos comandos
    \Macro{footref} adicionais. O texto da nota de rodapé será produzido por
    \DescRef{\ThisCommonLabelBase.cmd.footnote}.
  \end{Example}
\fi%
Ao definir marcadores de notas de rodapé com o mecanismo \Macro{label}, quaisquer
alterações nos números de notas de rodapé exigirão pelo menos duas execuções do
\LaTeX{} para garantir números corretos para todos os marcadores \Macro{footref}.%
\IfThisCommonLabelBaseOneOf{scrlttr2,scrextend}{\par%
  Você pode encontrar um exemplo de como usar \Macro{footref} em
  \autoref{sec:\ThisCommonFirstLabelBase.footnotes} em
  \PageRefxmpl{\ThisCommonFirstLabelBase.cmd.footref}. %
}{}%
\IfThisCommonLabelBase{scrlttr2}{}{%
  \par
  Note\textnote{Atenção!} que declarações como \Macro{ref}\IndexCmd{ref}
  ou \Macro{pageref}\IndexCmd{pageref} são frágeis e portanto você deve
  colocar \Macro{protect}\IndexCmd{protect} na frente delas se elas aparecerem em
  argumentos móveis como títulos. %
}%
A propósito, a partir\IfThisCommonLabelBase{maincls}{%
  \ChangedAt{v3.33}{\Class{scrbook}\and \Class{scrreprt}\and
    \Class{scrartcl}\and \Package{scrextend}}%
}{%
  \IfThisCommonLabelBase{scrlttr2}{%
    \ChangedAt{v3.33}{\Class{scrlttr2}}%
  }{}%
} %
de \LaTeX{} 2021-05-01, o comando é fornecido pelo próprio \LaTeX{}.%
\EndIndexGroup


\begin{Declaration}
  \Macro{deffootnote}\OParameter{largura da marca}\Parameter{recuo}%
                     \Parameter{recuo do parágrafo}\Parameter{definição}%
  \Macro{deffootnotemark}\Parameter{definição}%
  \Macro{thefootnotemark}
\end{Declaration}%
\IfThisCommonLabelBase{maincls}{As classes \KOMAScript{} definem}{\KOMAScript{}
  define}\textnote{\KOMAScript{} vs. classes padrão} notas de rodapé ligeiramente
diferentes do que as classes padrão fazem. Como nas classes padrão, a
marca de nota de rodapé no texto é renderizada com números pequenos e
sobrescritos. A mesma formatação é usada na própria nota de rodapé. A marca na
nota de rodapé é composta alinhada à direita em uma caixa com uma largura de
\PName{largura da marca}. A primeira linha da nota de rodapé segue diretamente.

Todas as linhas subsequentes serão recuadas pelo comprimento de \PName{recuo}.
Se o parâmetro opcional \PName{largura da marca} não for especificado, ele
assume o padrão de \PName{recuo}. Se a nota de rodapé consiste em mais de um
parágrafo, a primeira linha de cada parágrafo é recuada pelo valor de
\PName{recuo do parágrafo}.

\autoref{fig:\ThisCommonFirstLabelBase.deffootnote} %
\IfThisCommonFirstRun{}{em
  \autopageref{fig:\ThisCommonFirstLabelBase.deffootnote} }{}%
mostra os diferentes parâmetros%
\IfThisCommonLabelBase{maincls}{ novamente}{}%
. A configuração padrão das classes \KOMAScript{} é a seguinte:
\IfThisCommonLabelBase{scrextend}{\iftrue}{\csname iffalse\endcsname}%
\begin{lstcode}
  \deffootnote[1em]{1.5em}{1em}{%
    \textsuperscript{\thefootnotemark}}
\end{lstcode}
\else
\begin{lstcode}
  \deffootnote[1em]{1.5em}{1em}{%
    \textsuperscript{\thefootnotemark}%
  }
\end{lstcode}
\fi%
\Macro{textsuperscript} controla tanto o
sobrescrito quanto o tamanho de fonte menor. O comando \Macro{thefootnotemark}
contém a marca de nota de rodapé atual sem qualquer formatação.%
\IfThisCommonLabelBase{scrextend}{ %
  O pacote \Package{scrextend}, por outro lado, não altera as
  configurações padrão de notas de rodapé da classe que você está usando.
  Simplesmente carregar o pacote, portanto, não deve levar a quaisquer
  alterações na formatação de marcadores de notas de rodapé ou texto de notas
  de rodapé. Para usar as configurações padrão das classes \KOMAScript{}
  com \Package{scrextend}, você deve alterar as configurações acima
  você mesmo. Por exemplo, você pode inserir a linha de código acima imediatamente
  após carregar o pacote \Package{scrextend}.%
}{}%

\IfThisCommonLabelBase{maincls}{%
  \begin{figure}
%  \centering
    \KOMAoption{captions}{bottombeside}
    \setcapindent{0pt}%
    \begin{captionbeside}
      [{Parâmetros que controlam o layout de notas de rodapé}]%
      {\label{fig:\ThisCommonLabelBase.deffootnote}\hspace{0pt plus 1ex}%
        Parâmetros que controlam o layout de notas de rodapé}%
      [l]
      \setlength{\unitlength}{1mm}
      \begin{picture}(100,22)
        \thinlines
        % frame of following paragraph
        \put(5,0){\line(1,0){90}}
        \put(5,0){\line(0,1){5}}
        \put(10,5){\line(0,1){5}}\put(5,5){\line(1,0){5}}
        \put(95,0){\line(0,1){10}}
        \put(10,10){\line(1,0){85}}
        % frame of first paragraph
        \put(5,11){\line(1,0){90}}
        \put(5,11){\line(0,1){5}}
        \put(15,16){\line(0,1){5}}\put(5,16){\line(1,0){10}}
        \put(95,11){\line(0,1){10}}
        \put(15,21){\line(1,0){80}}
        % box of the footnote mark
        \put(0,16.5){\framebox(14.5,4.5){\mbox{}}}
        % description of paragraphs
        \put(45,16){\makebox(0,0)[l]{\textsf{primeiro parágrafo de uma nota de rodapé}}}
        \put(45,5){\makebox(0,0)[l]{\textsf{próximo parágrafo de uma nota de rodapé}}}
        % help lines
        \thicklines
        \multiput(0,0)(0,3){7}{\line(0,1){2}}
        \multiput(5,0)(0,3){3}{\line(0,1){2}}
        % parameters
        \put(2,7){\vector(1,0){3}}
        \put(5,7){\line(1,0){5}}
        \put(15,7){\vector(-1,0){5}}
        \put(15,7){\makebox(0,0)[l]{\small\PName{recuo do parágrafo}}}
        %
        \put(-3,13){\vector(1,0){3}}
        \put(0,13){\line(1,0){5}}
        \put(10,13){\vector(-1,0){5}}
        \put(10,13){\makebox(0,0)[l]{\small\PName{recuo}}}
        %
        \put(-3,19){\vector(1,0){3}}
        \put(0,19){\line(1,0){14.5}}
        \put(19.5,19){\vector(-1,0){5}}
        \put(19.5,19){\makebox(0,0)[l]{\small\PName{largura da marca}}}
      \end{picture}
    \end{captionbeside}
  \end{figure}}

\BeginIndexGroup
\BeginIndex{FontElement}{footnote}\LabelFontElement{footnote}%
\BeginIndex{FontElement}{footnotelabel}\LabelFontElement{footnotelabel}%
A nota de rodapé\IfThisCommonLabelBase{maincls}{%
  \ChangedAt{v2.8q}{\Class{scrbook}\and \Class{scrreprt}\and
    \Class{scrartcl}%
}}{}, incluindo a marca de nota de rodapé, usa a fonte especificada no
elemento \FontElement{footnote}\important{\FontElement{footnote}}. Você pode
alterar a fonte da marca de nota de rodapé separadamente usando os
comandos \DescRef{\ThisCommonLabelBase.cmd.setkomafont} e
\DescRef{\ThisCommonLabelBase.cmd.addtokomafont} (veja
\autoref{sec:\ThisCommonLabelBase.textmarkup},
\DescPageRef{\ThisCommonLabelBase.cmd.setkomafont})
para o elemento \FontElement{footnotelabel}\important{\FontElement{footnotelabel}}.
Veja também \autoref{tab:\ThisCommonLabelBase.fontelements},
\autopageref{tab:\ThisCommonLabelBase.fontelements}.
A configuração padrão é nenhuma alteração na fonte.%
\IfThisCommonLabelBase{scrextend}{ %
  No entanto, com \Package{scrextend} estes elementos só alterarão as fontes
  se as notas de rodapé forem tratadas pelo pacote, isto é, após usar
  \Macro{deffootnote}.%
}{} Por favor, não use indevidamente este elemento para outros propósitos, por
exemplo para definir as notas de rodapé alinhadas à direita (veja também
\DescRef{\LabelBase.cmd.raggedfootnote},
\DescPageRef{\LabelBase.cmd.raggedfootnote}).

\BeginIndex{FontElement}{footnotereference}%
\LabelFontElement{footnotereference}%
A marca de nota de rodapé no texto é definida separadamente da marca na
frente da nota de rodapé real. Isso é feito com
\Macro{deffootnotemark}. A configuração padrão é:
\begin{lstcode}
  \deffootnotemark{%
    \textsuperscript{\thefootnotemark}}
\end{lstcode}
Com\IfThisCommonLabelBase{maincls}{%
  \ChangedAt{v2.8q}{\Class{scrbook}\and \Class{scrreprt}\and
    \Class{scrartcl}}%
}{} este padrão, a fonte para o
elemento \FontElement{footnotereference}\important{\FontElement{footnotereference}}
é usada (veja \autoref{tab:\ThisCommonLabelBase.fontelements},
\autopageref{tab:\ThisCommonLabelBase.fontelements}). Assim, as marcas de notas de
rodapé no texto e na própria nota de rodapé são idênticas. Você pode alterar a
fonte com os comandos \DescRef{\ThisCommonLabelBase.cmd.setkomafont} e
\DescRef{\ThisCommonLabelBase.cmd.addtokomafont} (veja
\autoref{sec:\ThisCommonLabelBase.textmarkup},
\DescPageRef{\ThisCommonLabelBase.cmd.setkomafont}).

\IfThisCommonFirstRun{\iftrue}{\csname iffalse\endcsname}%
  \begin{Example}
    \phantomsection
    \xmpllabel{cmd.deffootnote}%
    Um\textnote{Dica!} recurso que é frequentemente solicitado são marcadores
    de notas de rodapé que não estão em sobrescrito nem em uma fonte menor.
    Eles não devem tocar o texto da nota de rodapé, mas ser separados por um
    pequeno espaço. Você pode conseguir isso da seguinte forma:
\begin{lstcode}
  \deffootnote{1em}{1em}{\thefootnotemark\ }
\end{lstcode}
    Isso definirá o marcador de nota de rodapé e espaço subsequente alinhados à
    direita em uma caixa de largura 1\Unit{em}. As linhas do texto de nota de
    rodapé que seguem também são recuadas 1\Unit{em} da margem esquerda.

    Outro\textnote{Dica!} layout que é frequentemente solicitado são marcadores
    de notas de rodapé alinhados à esquerda. Você pode obtê-los com a seguinte
    definição:
\begin{lstcode}
  \deffootnote{1.5em}{1em}{%
      \makebox[1.5em][l]{\thefootnotemark}}
\end{lstcode}

    Se, no entanto você quiser alterar a fonte para todas as notas de rodapé,
    por exemplo para sans serif, isso pode ser facilmente feito com os comandos
    \DescRef{\ThisCommonLabelBase.cmd.setkomafont} e
    \DescRef{\ThisCommonLabelBase.cmd.addtokomafont} (veja
    \autoref{sec:\ThisCommonLabelBase.textmarkup},
    \DescPageRef{\ThisCommonLabelBase.cmd.setkomafont}):
\begin{lstcode}
  \setkomafont{footnote}{\sffamily}
\end{lstcode}
  \end{Example}%
  \IfThisCommonLabelBase{scrextend}{}{%
    Como os exemplos mostram, {\KOMAScript} permite uma ampla variedade de
    formatos de notas de rodapé diferentes com esta interface de usuário simples.%
  }%
\fi%
\IfThisCommonFirstRun{}{%
  Para exemplos, veja \autoref{sec:\ThisCommonFirstLabelBase.footnotes},
  \PageRefxmpl{\ThisCommonFirstLabelBase.cmd.deffootnote}.%
}{}%
%
\EndIndexGroup
\EndIndexGroup

\IfThisCommonLabelBase{scrextend}{\iffalse}{\csname iftrue\endcsname}
\begin{Declaration}
  \Macro{setfootnoterule}\OParameter{espessura}\Parameter{comprimento}%
\end{Declaration}%
Geralmente,\IfThisCommonLabelBase{maincls}{%
  \ChangedAt{v3.06}{\Class{scrbook}\and \Class{scrreprt}\and
    \Class{scrartcl}}%
}{%
  \IfThisCommonLabelBase{scrlttr2}{%
    \ChangedAt{v3.06}{\Class{scrlttr2}}%
  }{%
    \IfThisCommonLabelBase{scrextend}{%
      \ChangedAt{v3.06}{\Package{scrextend}}%
    }{}%
  }%
} uma régua horizontal é definida entre a área de texto e a área de notas de
rodapé, mas normalmente esta régua não se estende por toda a largura da área
de composição. Com \Macro{setfootnoterule}, você pode definir a espessura e o
comprimento exatos da régua. Neste caso, os parâmetros \PName{espessura} e
\PName{comprimento} são avaliados apenas ao definir a própria régua. Se o
argumento opcional \PName{espessura} foi omitido, a espessura da régua não será
alterada. Argumentos vazios para \PName{espessura} ou \PName{comprimento}
também são permitidos e não alteram os parâmetros correspondentes. Usar valores
absurdos resultará em mensagens de aviso tanto ao definir quanto ao usar os
parâmetros.

\BeginIndexGroup
\BeginIndex{FontElement}{footnoterule}\LabelFontElement{footnoterule}%
Você pode%
\IfThisCommonLabelBase{maincls}{%
  \ChangedAt{v3.07}{\Class{scrbook}\and \Class{scrreprt}\and
    \Class{scrartcl}}%
}{%
  \IfThisCommonLabelBase{scrlttr2}{%
    \ChangedAt{v3.07}{\Class{scrlttr2}}%
  }{%
    \IfThisCommonLabelBase{scrextend}{%
      \ChangedAt{v3.07}{\Package{scrextend}}%
    }{}%
  }%
} %
alterar a cor da régua com o
elemento \FontElement{footnoterule}\important{\FontElement{footnoterule}} usando
os comandos \DescRef{\ThisCommonLabelBase.cmd.setkomafont} e
\DescRef{\ThisCommonLabelBase.cmd.addtokomafont} (veja
\autoref{sec:\ThisCommonLabelBase.textmarkup},
\DescPageRef{\ThisCommonLabelBase.cmd.setkomafont}). O padrão é nenhuma alteração
de fonte ou cor. Para alterar a cor, você também deve carregar um pacote de
cores como
\Package{xcolor}\IndexPackage{xcolor}\important{\Package{xcolor}}.%
\EndIndexGroup
\EndIndexGroup
\fi

\begin{Declaration}
  \Macro{raggedfootnote}
\end{Declaration}
Por padrão%
\IfThisCommonLabelBase{maincls}{%
  \ChangedAt{v3.23}{\Class{scrbook}\and \Class{scrreprt}\and
    \Class{scrartcl}}%
}{%
  \IfThisCommonLabelBase{scrlttr2}{%
    \ChangedAt{v3.23}{\Class{scrlttr2}}%
  }{%
    \IfThisCommonLabelBase{scrextend}{%
      \ChangedAt{v3.23}{\Package{scrextend}}%
    }{}%
  }%
} %
o \KOMAScript{} justifica notas de rodapé assim como nas classes padrão.
Mas\IfThisCommonLabelBase{scrextend}{%
  \ se você usar \DescRef{\LabelBase.cmd.deffootnote}%
  \important{\DescRef{\LabelBase.cmd.deffootnote}}%
  \IndexCmd{deffootnote}%
}{%
  \textnote{\KOMAScript{} vs. classes padrão}%
} você também pode alterar a justificação separadamente do resto do
documento redefinindo \Macro{raggedfootnote}. Definições válidas são
\Macro{raggedright}\IndexCmd{raggedright},
\Macro{raggedleft}\IndexCmd{raggedleft},
\Macro{centering}\IndexCmd{centering}, \Macro{relax}\IndexCmd{relax} ou uma
definição vazia, que é o padrão. Os comandos de alinhamento do
pacote \Package{ragged2e}\IndexPackage{ragged2e} também são válidos (veja
\cite{package:ragged2e}).  \IfThisCommonLabelBase{scrextend}{%
  Você pode encontrar um exemplo adequado em
  \autoref{sec:\ThisCommonFirstLabelBase.footnotes},
  \PageRefxmpl{\ThisCommonFirstLabelBase.cmd.raggedfootnote}.%
  \iffalse }{\csname iftrue\endcsname}%
  \begin{Example}
    \phantomsection\xmpllabel{cmd.raggedfootnote}%
    Suponha que você esteja usando notas de rodapé apenas para fornecer referências
    a links muito longos, onde quebras de linha produziriam resultados ruins se
    justificadas. Você pode usar
\begin{lstcode}
  \let\raggedfootnote\raggedright
\end{lstcode}
    no preâmbulo do seu documento para alternar para notas de rodapé alinhadas à
    direita.
  \end{Example}%
\fi
\EndIndexGroup

\begin{Declaration}
  \DoHook{footnote/text/begin}%
  \DoHook{footnote/text/end}%
\end{Declaration}
\BeginIndex{}{hook}%
Para\ChangedAt{v3.36}{\Class{scrbook}\and \Class{scrreprt}\and
  \Class{scrartcl}\and \Package{scrextend}} especialistas também há dois ganchos
do tipo \emph{do-hook} (veja \autoref{sec:scrbase.hooks} de
\autopageref{sec:scrbase.hooks}). O primeiro destes é usado no início
de \Macro{@makefntext} antes de
\DescRef{\LabelBase.cmd.raggedfootnote} ser executado. O segundo no final
antes do parágrafo ser finalizado. Atualmente nenhum gancho é usado pelo
próprio \KOMAScript{}.%
\EndIndexGroup
%
\EndIndexGroup


%%% Local Variables:
%%% mode: latex
%%% coding: utf-8
%%% ispell-local-dictionary: "en_GB"
%%% eval: (flyspell-mode 1)
%%% TeX-master: "../guide"
%%% End:
