% ======================================================================
% common-titles-en.tex
% Copyright (c) Markus Kohm, 2001-2022
%
% This file is part of the LaTeX2e KOMA-Script bundle.
%
% This work may be distributed and/or modified under the conditions of
% the LaTeX Project Public License, version 1.3c of the license.
% The latest version of this license is in
%   http://www.latex-project.org/lppl.txt
% and version 1.3c or later is part of all distributions of LaTeX
% version 2005/12/01 or later and of this work.
%
% This work has the LPPL maintenance status "author-maintained".
%
% The Current Maintainer and author of this work is Markus Kohm.
%
% This work consists of all files listed in MANIFEST.md.
% ======================================================================
%
% Paragraphs that are common for several chapters of the KOMA-Script guide
% Maintained by Markus Kohm
%
% ======================================================================

\KOMAProvidesFile{common-titles-en.tex}
                 [$Date: 2022-06-05 12:40:11 +0200 (So, 05. Jun 2022) $
                  KOMA-Script guide (common paragraphs)]

\translator{Gernot Hassenpflug\and Markus Kohm\and Krickette Murabayashi\and
	Karl Hagen}

\section{Títulos de Documentos}
\seclabel{titlepage}%
\BeginIndexGroup
\BeginIndex{}{title}%

\IfThisCommonFirstRun{}{%
  Esta informação na \autoref{sec:\ThisCommonFirstLabelBase.titlepage}
  aplica-se amplamente a este capítulo. Portanto, se você já leu e compreendeu
  a \autoref{sec:\ThisCommonFirstLabelBase.titlepage}, pode pular para a
  \autoref{sec:\ThisCommonLabelBase.titlepage.next},
  \autopageref{sec:\ThisCommonLabelBase.titlepage.next}.%
}%
\IfThisCommonLabelBase{scrextend}{\iftrue}{\csname iffalse\endcsname}%
  \ Contudo,\textnote{Atenção!} as capacidades do \Package{scrextend}
  para lidar com o título do documento fazem parte dos recursos avançados
  opcionais. Portanto, eles só estão disponíveis se
  \OptionValueRef{\ThisCommonLabelBase}{extendedfeature}{title} for
  selecionado ao carregar o pacote (veja
  \autoref{sec:\ThisCommonLabelBase.optionalFeatures},
  \DescRef{\ThisCommonLabelBase.option.extendedfeature}).

  Além disso, o \Package{scrextend} não pode ser usado com uma classe
  \KOMAScript{}. Por isso, você deve substituir
\begin{lstcode}
  \documentclass{scrbook}
\end{lstcode}
  por
\begin{lstcode}
  \documentclass{book}
  \usepackage[extendedfeature=title]{scrextend}
\end{lstcode}
  para todos os exemplos da \autoref{sec:maincls.titlepage}, se quiser
  experimentá-los com o \Package{scrextend}.
\fi

\IfThisCommonLabelBase{scrextend}{}{%
  Em geral, distinguimos dois tipos de títulos de documentos. Primeiro, há as
  páginas de título. Estas incluem o título do documento, juntamente com
  informações adicionais como o autor, em uma página separada. Além da página
  de título principal, pode haver várias outras páginas de título, como a
  antetítulo ou falso título, dados da editora, dedicatória e assim por diante.
  Segundo, há o título na página. Este tipo de título aparece no topo de uma
  nova página, geralmente a primeira, e é especialmente enfatizado. Ele também
  pode ser acompanhado por informações adicionais, mas será seguido por mais
  material na mesma página, por exemplo, por um resumo, o sumário ou mesmo
  uma seção.%
}%

\begin{Declaration}
  \OptionVName{titlepage}{simple switch}%
  \OptionValue{titlepage}{firstiscover}
  \Macro{coverpagetopmargin}
  \Macro{coverpageleftmargin}
  \Macro{coverpagerightmargin}
  \Macro{coverpagebottommargin}
\end{Declaration}%
Esta opção\IfThisCommonLabelBase{maincls}{%
	\ChangedAt{v3.00}{\Class{scrbook}\and \Class{scrreprt}\and
		\Class{scrartcl}}%
}{} determina se devem ser usadas páginas de título de
documento\Index{title>pages} ou títulos na página\Index{title>in-page}
ao usar \DescRef{\ThisCommonLabelBase.cmd.maketitle} (veja
\DescPageRef{\ThisCommonLabelBase.cmd.maketitle}). Qualquer valor de
\autoref{tab:truefalseswitch}, \autopageref{tab:truefalseswitch} pode ser
usado para \PName{simple switch}.

Com a opção \OptionValue{titlepage}{true}%
\important{\OptionValue{titlepage}{true}}
\IfThisCommonLabelBase{scrextend}{}{ou \Option{titlepage}},
invocar \DescRef{\ThisCommonLabelBase.cmd.maketitle} cria títulos em
páginas separadas. Estas páginas são compostas dentro de um ambiente
\DescRef{\ThisCommonLabelBase.env.titlepage}, e normalmente não têm
cabeçalho nem rodapé. Comparado ao {\LaTeX} padrão, o {\KOMAScript}
expande significativamente o tratamento dos títulos. Estes elementos
adicionais podem ser encontrados nas páginas seguintes.

Em contraste, com a opção
\OptionValue{titlepage}{false}\important{\OptionValue{titlepage}{false}},
invocar \DescRef{\ThisCommonLabelBase.cmd.maketitle} cria um título
\emph{na página}. Isto significa que o título é especialmente enfatizado,
mas pode ser seguido por mais material na mesma página, por exemplo, um
resumo ou uma seção.%

A terceira escolha,%
\IfThisCommonLabelBase{maincls}{%
  \ChangedAt{v3.12}{\Class{scrbook}\and \Class{scrreprt}\and
    \Class{scrartcl}}%
}{%
  \IfThisCommonLabelBase{scrextend}{%
    \ChangedAt{v3.12}{\Package{scrextend}}%
  }{\InternalCommonFileUseError}%
} \OptionValue{titlepage}{firstiscover}%
\important{\OptionValue{titlepage}{firstiscover}} não apenas ativa páginas
de título, mas também imprime a primeira página de título de
\DescRef{\ThisCommonLabelBase.cmd.maketitle}\IndexCmd{maketitle}, ou seja,
o antetítulo ou o título principal, como uma página de
capa\Index{cover}. Qualquer outra configuração da opção \Option{titlepage}
cancelará esta configuração. As margens\important{\Macro{coverpage\dots margin}}
da página de capa são dadas por \Macro{coverpagetopmargin},
\Macro{coverpageleftmargin}, \Macro{coverpagerightmargin} e
\Macro{coverpagebottommargin}. Os padrões destes dependem dos comprimentos
de \Length{topmargin}\IndexLength{topmargin} e
\Length{evensidemargin}\IndexLength{evensidemargin} e podem ser alterados
com \Macro{renewcommand}.

\IfThisCommonLabelBase{maincls}{%
  O padrão das classes \Class{scrbook} e \Class{scrreprt} é usar páginas
  de título. A classe \Class{scrartcl}, por outro lado, usa títulos na
  página por padrão.%
}{%
  \IfThisCommonLabelBase{scrextend}{%
    O padrão depende da classe usada e o \Package{scrextend} o reconhece
    de uma forma compatível com a classe padrão. Se uma classe não
    configurar um padrão comparável, será um título na página.%
  }{\InternalCommonFileUsageError}%
}%
%
\EndIndexGroup


\begin{Declaration}
  \begin{Environment}{titlepage}\end{Environment}%
\end{Declaration}%
As classes padrão e o {\KOMAScript} compõem todas as páginas de título em
um ambiente especial: o ambiente \Environment{titlepage}. Este ambiente
sempre inicia uma nova página\,---\,na impressão em dois lados, uma nova
página direita\,---\,e em modo de coluna única. Para esta página, o estilo
é alterado para \DescRef{maincls.cmd.thispagestyle}%
\PParameter{\DescRef{maincls.pagestyle.empty}}, de modo que nem o número
da página nem o cabeçalho corrente sejam impressos. No final do ambiente,
a página é automaticamente enviada para saída. Se você não puder usar o
layout automático das páginas de título fornecido por
\DescRef{\ThisCommonLabelBase.cmd.maketitle}, descrito a seguir, você deve
criar um novo com a ajuda deste ambiente.

\IfThisCommonFirstRun{\iftrue}{%
  Um exemplo simples de uma página de título com \Environment{titlepage} é
  mostrado na \autoref{sec:\ThisCommonFirstLabelBase.titlepage} na
  \PageRefxmpl{\ThisCommonFirstLabelBase.env.titlepage}%
  \csname iffalse\endcsname%
}%
  \begin{Example}
    \phantomsection\xmpllabel{env.titlepage}
    Suponha que você queira uma página de título na qual apenas a palavra
    ``Eu'' apareça no topo à esquerda, tão grande quanto possível e em
    negrito\,---\,sem autor, sem data, nada mais. O documento seguinte cria
    exatamente isso:
\begin{lstcode}
  \documentclass{scrbook}
  \begin{document}
  \begin{titlepage}
    \textbf{\Huge Eu}
  \end{titlepage}
  \end{document}
\end{lstcode}
    É simples, não é?
  \end{Example}
\fi%
\EndIndexGroup


\begin{Declaration}
  \Macro{maketitle}\OParameter{page number}
\end{Declaration}%
Enquanto\textnote{\KOMAScript{} vs. classes padrão} as classes padrão
produzem no máximo uma página de título que pode ter três itens (título,
autor e data), com o \KOMAScript{} o \Macro{maketitle} pode produzir até
seis páginas. Em contraste com as classes padrão, o \Macro{maketitle} no
{\KOMAScript} aceita um argumento numérico opcional. Se for usado, este
número é o número da página da primeira página de título. Este número de
página não é impresso, mas afeta a numeração subsequente. Você deve
definitivamente escolher um número ímpar, porque caso contrário a contagem
toda fica confusa. Na minha opinião, há apenas duas aplicações úteis para o
argumento opcional. Por um lado, você poderia dar o número de página lógico
-1 ao antetítulo\Index[indexmain]{half-title} para dar à página de título
completa o número 1. Por outro lado, você poderia usá-lo para começar em um
número de página mais alto, por exemplo, 3, 5 ou 7, para acomodar outras
páginas de título adicionadas pela editora. O argumento opcional é ignorado
para títulos \emph{na página}. Você pode alterar o estilo de página de tal
página de título redefinindo o macro
\DescRef{\ThisCommonLabelBase.cmd.titlepagestyle} (veja
\autoref{sec:maincls.pagestyle}, \DescPageRef{maincls.cmd.titlepagestyle}).

Os comandos seguintes não levam imediatamente ao envio dos títulos. A
composição e o envio das páginas de título são sempre feitos por
\Macro{maketitle}. Note também que \Macro{maketitle} não deve ser usado
dentro de um ambiente \DescRef{\ThisCommonLabelBase.env.titlepage}.
Como\textnote{Atenção!} mostrado nos exemplos, você deve usar
\Macro{maketitle} ou \DescRef{\ThisCommonLabelBase.env.titlepage}, mas não
ambos.

Os comandos seguintes apenas definem o conteúdo do título. Portanto, eles
devem ser usados antes de \Macro{maketitle}. Não é, contudo, necessário e,
ao usar o pacote \Package{babel}\IndexPackage{babel}, não recomendado,
incluir estes no preâmbulo antes de \Macro{begin}\PParameter{document}
(veja \cite{package:babel}). Você pode encontrar exemplos
\IfThisCommonFirstRun{nas descrições dos outros comandos nesta seção}{na
  \autoref{sec:\ThisCommonFirstLabelBase.titlepage}, começando na
  \PageRefxmpl{\ThisCommonFirstLabelBase.cmd.extratitle}}.


\begin{Declaration}
  \Macro{extratitle}\Parameter{half-title}
  \Macro{frontispiece}\Parameter{frontispiece}
\end{Declaration}%
\begin{Explain}%
  Em tempos anteriores, o livro interno muitas vezes não era protegido da
  sujeira por uma capa. Esta função era então assumida pela primeira página
  do livro, que geralmente tinha apenas um título curto, conhecido como o
  \emph{antetítulo}. Hoje em dia, a página extra muitas vezes aparece antes
  do título principal real e contém informações sobre a editora, número de
  série e informações similares.
\end{Explain}
Com o {\KOMAScript}, é possível incluir uma página antes da página de
título real. O \PName{half-title} pode ser um texto arbitrário\,---\,até
vários parágrafos. O conteúdo do \PName{half-title} é impresso pelo
{\KOMAScript} sem formatação adicional. Sua organização é completamente
deixada ao usuário. O verso do antetítulo\IfThisCommonLabelBase{maincls}{%
  \ChangedAt{v3.25}{\Class{scrbook}\and\Class{scrreprt}\and\Class{scrartcl}}%
}{%
  \IfThisCommonLabelBase{scrextend}{%
    \ChangedAt{v3.25}{\Package{scrextend}}%
  }{\ThisCommonLabelBaseFailure}%
} é o frontispício. O antetítulo é composto em sua própria página mesmo
quando títulos na página são usados. A saída do antetítulo definido com
\Macro{extratitle} ocorre como parte do título produzido por
\DescRef{\ThisCommonLabelBase.cmd.maketitle}.

\IfThisCommonFirstRun{\iftrue}{%
  Um exemplo de uma página de título simples com antetítulo e título
  principal é mostrado na \autoref{sec:\ThisCommonFirstLabelBase.titlepage}
  na \PageRefxmpl{\ThisCommonFirstLabelBase.cmd.extratitle}%
  \csname iffalse\endcsname%
}%
  \begin{Example}
    \phantomsection\xmpllabel{cmd.extratitle}
    Voltemos ao exemplo anterior e suponhamos que o espartano ``Eu'' seja o
    antetítulo. O título completo ainda deve seguir o antetítulo. Você pode
    proceder da seguinte forma:
\begin{lstcode}
  \documentclass{scrbook}
  \begin{document}
    \extratitle{\textbf{\Huge Eu}}
    \title{Sou eu}
    \maketitle
  \end{document}
\end{lstcode}
    Você pode centralizar o antetítulo horizontalmente e colocá-lo um pouco
    mais abaixo na página:
\begin{lstcode}
  \documentclass{scrbook}
  \begin{document}
    \extratitle{\vspace*{4\baselineskip}
      \begin{center}\textbf{\Huge Eu}\end{center}}
    \title{Sou eu}
    \maketitle
  \end{document}
\end{lstcode}
    O comando\textnote{Atenção!} \DescRef{\ThisCommonLabelBase.cmd.title}
    é necessário para fazer os exemplos acima funcionarem corretamente. Ele
    é explicado a seguir.
  \end{Example}
\fi%
\EndIndexGroup


\begin{Declaration}
  \Macro{titlehead}\Parameter{title head}%
  \Macro{subject}\Parameter{subject}%
  \Macro{title}\Parameter{title}%
  \Macro{subtitle}\Parameter{subtitle}%
  \Macro{author}\Parameter{author}%
  \Macro{date}\Parameter{date}%
  \Macro{publishers}\Parameter{publisher}%
  \Macro{and}%
  \Macro{thanks}\Parameter{footnote}
\end{Declaration}%
Há sete elementos disponíveis para o conteúdo da página de título
principal. A página de título principal é impressa como parte das páginas
de título criadas por \DescRef{\ThisCommonLabelBase.cmd.maketitle},
enquanto as definições dadas aqui se aplicam apenas aos respectivos
elementos.

\BeginIndexGroup\BeginIndex{FontElement}{titlehead}%
\LabelFontElement{titlehead}%
O\important{\Macro{titlehead}} \PName{title head}%
\Index[indexmain]{title>head} é definido com o comando \Macro{titlehead}.
Ele ocupa toda a largura do texto, no topo da página, em justificação
normal, e pode ser livremente projetado pelo usuário. Ele usa o elemento de
fonte\important{\FontElement{titlehead}} de mesmo nome (veja
\autoref{tab:\ThisCommonFirstLabelBase.mainTitle},
\autopageref{tab:\ThisCommonFirstLabelBase.mainTitle}).%
\EndIndexGroup

\BeginIndexGroup\BeginIndex{FontElement}{subject}\LabelFontElement{subject}%
O\important{\Macro{subject}} \PName{subject}\Index[indexmain]{subject} é
impresso com o elemento de fonte\important{\FontElement{subject}} de mesmo
nome imediatamente acima do \PName{title}.%
\EndIndexGroup

\BeginIndexGroup\BeginIndex{FontElement}{title}\LabelFontElement{title}%
O\important{\Macro{title}} \PName{title} é composto em um tamanho de fonte
muito grande. Junto\IfThisCommonLabelBase{maincls}{%
  \ChangedAt{v2.8p}{\Class{scrbook}\and \Class{scrreprt}\and
    \Class{scrartcl}}}{} com o tamanho da fonte, o elemento de fonte
\FontElement{title}\IndexFontElement[indexmain]{title}%
\important{\FontElement{title}} é aplicado (veja
\autoref{tab:\ThisCommonFirstLabelBase.mainTitle},
\autopageref{tab:\ThisCommonFirstLabelBase.mainTitle}).%
\EndIndexGroup

\BeginIndexGroup\BeginIndex{FontElement}{subtitle}\LabelFontElement{subtitle}%
O\important{\Macro{subtitle}}
\PName{subtitle}\IfThisCommonLabelBase{maincls}{%
  \ChangedAt{v2.97c}{\Class{scrbook}\and \Class{scrreprt}\and
    \Class{scrartcl}}}{} é composto logo abaixo do título usando o elemento
de fonte\important{\FontElement{subtitle}} de mesmo nome (veja
\autoref{tab:\ThisCommonFirstLabelBase.mainTitle},
\autopageref{tab:\ThisCommonFirstLabelBase.mainTitle}).%
\EndIndexGroup

\BeginIndexGroup\BeginIndex{FontElement}{author}\LabelFontElement{author}%
Abaixo\important{\Macro{author}} do \PName{subtitle} aparece o
\PName{author}\Index[indexmain]{author}. Vários autores podem ser
especificados no argumento de \Macro{author}. Eles devem ser separados por
\Macro{and}\important{\Macro{and}}. A saída usa o elemento de
fonte\important{\FontElement{author}} de mesmo nome. (veja
\autoref{tab:\ThisCommonFirstLabelBase.mainTitle},
\autopageref{tab:\ThisCommonFirstLabelBase.mainTitle}).%
\EndIndexGroup

\BeginIndexGroup\BeginIndex{FontElement}{date}\LabelFontElement{date}%
Abaixo\important{\Macro{date}} do autor ou autores aparece a
data\Index{date} na fonte do elemento de mesmo nome. O valor padrão é a
data atual, como produzida por \Macro{today}\IndexCmd{today}. O comando
\Macro{date} aceita informações arbitrárias\,---\,até um argumento vazio. A
saída usa o elemento de fonte\important{\FontElement{date}} de mesmo nome
(veja \autoref{tab:\ThisCommonFirstLabelBase.mainTitle},
\autopageref{tab:\ThisCommonFirstLabelBase.mainTitle}).%
\EndIndexGroup

\BeginIndexGroup\BeginIndex{FontElement}{publishers}%
\LabelFontElement{publishers}%
Finalmente\important{\Macro{publishers}} vem o
\PName{publisher}\Index[indexmain]{publisher}. Claro que este comando
também pode ser usado para qualquer outra informação de menor importância.
Se necessário, o comando \Macro{parbox} pode ser usado para compor esta
informação sobre toda a largura da página como um parágrafo regular em vez
de centralizá-la. Deve então ser considerado equivalente ao cabeçalho do
título. Note, contudo, que este campo é colocado acima de quaisquer notas
de rodapé existentes. A saída usa o elemento de
fonte\important{\FontElement{publishers}} de mesmo nome (veja
\autoref{tab:\ThisCommonFirstLabelBase.mainTitle},
\autopageref{tab:\ThisCommonFirstLabelBase.mainTitle}).%
\EndIndexGroup

Notas de rodapé\important{\Macro{thanks}}\Index{footnotes} na página de
título são produzidas não com \Macro{footnote}, mas com \Macro{thanks}.
Elas servem tipicamente para notas associadas aos autores. Símbolos são
usados como marcadores de notas de rodapé em vez de números.
Note\textnote{Atenção!} que \Macro{thanks} deve ser usado dentro do
argumento de outro comando, como no argumento \PName{author} do comando
\Macro{author}.
\IfThisCommonLabelBase{scrextend}{%
  Contudo, para que o elemento
  \DescRef{\ThisCommonLabelBase.fontelement.footnote} esteja ciente do
  pacote \Package{scrextend}, não apenas a extensão de título precisa ser
  ativada, ela também deve ser configurada para usar notas de rodapé com
  este pacote (veja a introdução à
  \autoref{sec:\ThisCommonLabelBase.footnotes},
  \autopageref{sec:\ThisCommonLabelBase.footnotes}). Caso contrário, a
  fonte especificada pela classe ou outros pacotes usados para as notas de
  rodapé será usada.%
}{}%

Para%
\IfThisCommonLabelBase{maincls}{%
  \ChangedAt{v3.12}{\Class{scrbook}\and \Class{scrreprt}\and
    \Class{scrartcl}}%
}{%
  \IfThisCommonLabelBase{scrextend}{%
    \ChangedAt{v3.12}{\Package{scrextend}}%
  }{\InternalCommonFileUsageError}%
} a saída dos elementos de título, a fonte\textnote{font} pode ser
configurada usando os comandos
\DescRef{\ThisCommonLabelBase.cmd.setkomafont} e
\DescRef{\ThisCommonLabelBase.cmd.addtokomafont} (veja
\autoref{sec:\ThisCommonLabelBase.textmarkup},
\DescPageRef{\ThisCommonLabelBase.cmd.setkomafont}). Os padrões estão
listados na \autoref{tab:\ThisCommonFirstLabelBase.titlefonts}%
\IfThisCommonFirstRun{}{%
  , \autopageref{tab:\ThisCommonFirstLabelBase.titlefonts}%
}.%
\IfThisCommonFirstRun{%
  \begin{table}
%  \centering
%  \caption
    \KOMAoptions{captions=topbeside}%
    \setcapindent{0pt}%
%  \addtokomafont{caption}{\raggedright}%
    \begin{captionbeside}
      [{Padrões de fonte para os elementos do título}]
      {\label{tab:\ThisCommonLabelBase.titlefonts}%
        \hspace{0pt plus 1ex}Padrões de fonte para os elementos do título}
      [l]
      \begin{tabular}[t]{ll}
        \toprule
        Nome do elemento & Padrão \\
        \midrule
        \FontElement{author} & \Macro{Large} \\
        \FontElement{date} & \Macro{Large} \\
        \FontElement{dedication} & \Macro{Large} \\
        \FontElement{publishers} & \Macro{Large} \\
        \FontElement{subject} &
                                \Macro{normalfont}\Macro{normalcolor}%
                                \Macro{bfseries}\Macro{Large} \\
        \FontElement{subtitle} &
                                 \DescRef{\ThisCommonLabelBase.cmd.usekomafont}%
                                 \PParameter{title}\Macro{large} \\
        \FontElement{title} &
                              \DescRef{\ThisCommonLabelBase.cmd.usekomafont}%
                              \PParameter{disposition} \\
        \FontElement{titlehead} & \\
        \bottomrule
      \end{tabular}
    \end{captionbeside}
  \end{table}%
}{}%

Com exceção de \PName{title head} e quaisquer notas de rodapé, toda a saída
é centralizada horizontalmente. %
\iffree{%
  \IfThisCommonLabelBase{scrextend}{A formatação de cada elemento é}{Estes
    detalhes são} resumida brevemente na
  \autoref{tab:\ThisCommonFirstLabelBase.mainTitle}\IfThisCommonFirstRun{}{%
    , \autopageref{tab:\ThisCommonFirstLabelBase.mainTitle}}.%
}{%
  \IfThisCommonLabelBase{scrextend}{%
    O alinhamento de elementos individuais também pode ser encontrado na
    \autoref{tab:\ThisCommonFirstLabelBase.mainTitle}\IfThisCommonFirstRun{}{%
      , \autopageref{tab:\ThisCommonFirstLabelBase.mainTitle}}.%
  }{%
    Para um resumo, veja \autoref{tab:\ThisCommonFirstLabelBase.mainTitle}.%
  }%
}%
\IfThisCommonFirstRun{%
  \begin{table}
    % \centering
    \KOMAoptions{captions=topbeside}%
    \setcapindent{0pt}%
    % \caption
    \begin{captionbeside}[Título principal]{%
        \hspace{0pt plus 1ex}%
        Fonte e posicionamento horizontal dos elementos na página de título
        principal na ordem de sua posição vertical de cima para baixo quando
        compostos com \DescRef{\ThisCommonLabelBase.cmd.maketitle}}
      [l]
      \setlength{\tabcolsep}{.85\tabcolsep}% Umbruchoptimierung
      \begin{tabular}[t]{llll}
        \toprule
        Elemento   & Comando            & Fonte              & Alinhamento   \\
        \midrule
        Cabeçalho do título & \Macro{titlehead}  & \DescRef{\ThisCommonLabelBase.cmd.usekomafont}\PParameter{titlehead} & justificado \\
        Assunto    & \Macro{subject}    & \DescRef{\ThisCommonLabelBase.cmd.usekomafont}\PParameter{subject} & centralizado \\
        Título     & \Macro{title}      & \DescRef{\ThisCommonLabelBase.cmd.usekomafont}\PParameter{title}\Macro{huge}  & centralizado  \\
        Subtítulo  & \Macro{subtitle}   & \DescRef{\ThisCommonLabelBase.cmd.usekomafont}\PParameter{subtitle}  & centralizado \\
        Autores    & \Macro{author}     & \DescRef{\ThisCommonLabelBase.cmd.usekomafont}\PParameter{author}  & centralizado \\
        Data       & \Macro{date}       & \DescRef{\ThisCommonLabelBase.cmd.usekomafont}\PParameter{date}  & centralizado  \\
        Editora    & \Macro{publishers} & \DescRef{\ThisCommonLabelBase.cmd.usekomafont}\PParameter{publishers} & centralizado \\
        \bottomrule
      \end{tabular}
    \end{captionbeside}
    \label{tab:maincls.mainTitle}
  \end{table}
}{}

Note\textnote{Atenção!} que para o título principal, \Macro{huge} será
usado após o elemento de mudança de fonte
\DescRef{\ThisCommonLabelBase.fontelement.title}\IndexFontElement{title}.
Portanto, você não pode alterar o tamanho do título principal usando
\DescRef{\ThisCommonLabelBase.cmd.setkomafont} ou
\DescRef{\ThisCommonLabelBase.cmd.addtokomafont}.%

\IfThisCommonFirstRun{\iftrue}{%
  Um exemplo de uma página de título usando todos os elementos oferecidos
  pelo \KOMAScript{} é mostrado na
  \autoref{sec:\ThisCommonFirstLabelBase.titlepage} na
  \PageRefxmpl{\ThisCommonFirstLabelBase.maintitle}.%
  \csname iffalse\endcsname%
}%
  \begin{Example}
    \phantomsection\xmpllabel{maintitle}%
    Suponha que você esteja escrevendo uma dissertação. A página de título
    deve ter o nome e endereço da universidade no topo, alinhado à
    esquerda, e o semestre, alinhado à direita. Como de costume, um título
    incluindo autor e data de submissão deve ser dado. O orientador também
    deve ser indicado, juntamente com o fato de que o documento é uma
    dissertação. Você pode fazer isso da seguinte forma:
\begin{lstcode}
  \documentclass{scrbook}
  \usepackage[english]{babel}
  \begin{document}
  \titlehead{{\Large Universidade Invisível
      \hfill 2º Semestre 2002\\}
    Instituto de Análise Superior\\
    Rua Mitológica\\
    34567 Mundo Etéreo}
  \subject{Dissertação}
  \title{Simulação de espaço digital com o DSP\,56004}
  \subtitle{Curto mas doce?}
  \author{Jorge Confuso}
  \date{30 de Fevereiro de 2002}
  \publishers{Orientador Prof. Dr. João Excêntrico}
  \maketitle
  \end{document}
\end{lstcode}
  \end{Example}%
\fi

\begin{Explain}
  Um equívoco comum diz respeito à função da página de título completa. Ela
  é frequentemente erroneamente assumida como sendo a capa\Index{cover} ou
  sobrecapa. Portanto, é frequentemente esperado que a página de título não
  siga o layout normal para composição em dois lados, mas tenha margens
  esquerda e direita igualmente grandes.

  Mas se você pegar um livro e abri-lo, rapidamente encontrará pelo menos
  uma página de título dentro da capa, dentro do chamado bloco do livro.
  Precisamente estas páginas de título são produzidas por
  \DescRef{\ThisCommonLabelBase.cmd.maketitle}.

  Como é o caso com o antetítulo, a página de título completa pertence ao
  bloco do livro e, portanto, deve ter o mesmo layout de página que o
  restante do documento. Uma capa é na verdade algo que você deve criar em
  um documento separado. Afinal, ela muitas vezes tem um formato muito
  distinto. Ela também pode ser projetada com a ajuda de um programa
  gráfico ou DTP. Um documento separado também deve ser usado porque a capa
  será impressa em um meio diferente, como papelão, e possivelmente com
  outra impressora.

  No entanto, desde o \KOMAScript~3.12, a primeira página de título emitida
  por \DescRef{\ThisCommonLabelBase.cmd.maketitle} pode ser formatada como
  uma página de capa com margens diferentes. Mudanças nas margens desta
  página não afetam as outras margens. Para mais informações sobre esta
  opção, veja
  \OptionValueRef{\ThisCommonLabelBase}{titlepage}{firstiscover}%
  \IndexOption{titlepage~=\textKValue{firstiscover}} na
  \DescPageRef{\ThisCommonLabelBase.option.titlepage}.
\end{Explain}
%
\EndIndexGroup


\begin{Declaration}
  \Macro{uppertitleback}\Parameter{titlebackhead}%
  \Macro{lowertitleback}\Parameter{titlebackfoot}
\end{Declaration}%
Na\textnote{\KOMAScript{} vs. classes padrão} impressão em dois lados, as
classes padrão deixam o verso da página de título vazio. Contudo, com o
{\KOMAScript} o verso da página de título completa pode ser usado para
outras informações. Há exatamente dois elementos que o usuário pode
formatar livremente: \PName{titlebackhead}\Index{title>back}%
\Index{title>verso} e \PName{titlebackfoot}. O cabeçalho pode se estender
ao rodapé e vice-versa. \iffree{Usando este guia como exemplo, o aviso
  legal foi composto com a ajuda do comando \Macro{uppertitleback}.}{As
  informações da editora deste livro, por exemplo, poderiam ter sido
  compostas com \Macro{uppertitleback} ou \Macro{lowertitleback}.}%
%
\EndIndexGroup


\begin{Declaration}
  \Macro{dedication}\Parameter{dedication}
\end{Declaration}%
O {\KOMAScript} oferece sua própria página de dedicatória. Esta
dedicatória\Index{dedication} é centralizada e composta por padrão com uma
fonte ligeiramente maior\textnote{font}.
\BeginIndexGroup\BeginIndex{FontElement}{dedication}%
\LabelFontElement{dedication}
A%
\IfThisCommonLabelBase{maincls}{%
  \ChangedAt{v3.12}{\Class{scrbook}\and \Class{scrreprt}\and
    \Class{scrartcl}}%
}{%
  \IfThisCommonLabelBase{scrextend}{%
    \ChangedAt{v3.12}{\Package{scrextend}}%
  }{\InternalCommonFileUseError}%
}\important{\FontElement{dedication}} configuração de fonte exata para o
elemento \FontElement{dedication}, que é tirada da
\autoref{tab:\ThisCommonFirstLabelBase.titlefonts},
\autopageref{tab:\ThisCommonFirstLabelBase.titlefonts}, pode ser alterada
com os comandos \DescRef{\ThisCommonLabelBase.cmd.setkomafont} e
\DescRef{\ThisCommonLabelBase.cmd.addtokomafont} (veja
\autoref{sec:\ThisCommonLabelBase.textmarkup},
\DescPageRef{\ThisCommonLabelBase.cmd.setkomafont}).%
\EndIndexGroup

\IfThisCommonFirstRun{\iftrue}{%
  Um exemplo com todas as páginas de título fornecidas pelo \KOMAScript{} é
  mostrado na \autoref{sec:\ThisCommonFirstLabelBase.titlepage} na
  \PageRefxmpl{\ThisCommonFirstLabelBase.fulltitle}.%
  \csname iffalse\endcsname%
}%
  \begin{Example}
    \phantomsection\xmpllabel{fulltitle}%
    Suponha que você tenha escrito um livro de poesia e queira dedicá-lo ao
    seu cônjuge. Uma solução seria assim:
\begin{lstcode}
  \documentclass{scrbook}
  \usepackage[english]{babel}
  \begin{document}
  \extratitle{\textbf{\Huge Apaixonado}}
  \title{Apaixonado}
  \author{Príncipe Coração de Ferro}
  \date{1412}
  \lowertitleback{Este livro de poemas foi composto com%
       a ajuda do {\KOMAScript} e {\LaTeX}}
  \uppertitleback{Editora Auto-zombaria}
  \dedication{À minha querida galinha-selvagem\\
    em amor eterno\\
    do seu ratinho-do-campo.}
  \maketitle
  \end{document}
\end{lstcode}
    Por favor, use seus próprios apelidos carinhosos favoritos para
    personalizá-lo.
  \end{Example}%
\fi%
\EndIndexGroup
%
\EndIndexGroup
%
\EndIndexGroup

%%% Local Variables:
%%% mode: latex
%%% TeX-master: "scrguide-en.tex"
%%% coding: utf-8
%%% ispell-local-dictionary: "en_GB"
%%% eval: (flyspell-mode 1)
%%% End:
