% ======================================================================
% scrlayer-scrpage-en.tex
% Copyright (c) Markus Kohm, 2013-2023
%
% This file is part of the LaTeX2e KOMA-Script bundle.
%
% This work may be distributed and/or modified under the conditions of
% the LaTeX Project Public License, version 1.3c of the license.
% The latest version of this license is in
%   http://www.latex-project.org/lppl.txt
% and version 1.3c or later is part of all distributions of LaTeX 
% version 2005/12/01 or later and of this work.
%
% This work has the LPPL maintenance status "author-maintained".
%
% The Current Maintainer and author of this work is Markus Kohm.
%
% This work consists of all files listed in MANIFEST.md.
% ======================================================================
%
% Chapter about scrlayer-scrpage of the KOMA-Script guide
%
% ============================================================================

\KOMAProvidesFile{scrlayer-scrpage-en.tex}%
                 [$Date: 2023-04-06 09:06:52 +0200 (Do, 06. Apr 2023) $
                  KOMA-Script guide (chapter: scrlayer-scrpage)]
\translator{Markus Kohm\and Jana Schubert\and Jens Hühne\and Karl Hagen}

\chapter[{Cabeçalhos e Rodapés com \Package{scrlayer-scrpage}}]
  {Cabeçalhos\ChangedAt{v3.12}{\Package{scrlayer-scrpage}} e
  Rodapés com \Package{scrlayer-scrpage}}
\labelbase{scrlayer-scrpage}
%
\BeginIndexGroup
\BeginIndex{Package}{scrlayer-scrpage}%
\begin{Explain}
  Até a versão 3.11b do \KOMAScript, o pacote \Package{scrpage2}%
  \IndexPackage[indexmain]{scrpage2}\important{\Package{scrpage2}} era
  a forma recomendada de personalizar cabeçalhos e rodapés além das opções
  fornecidas pelos estilos de página \PageStyle{headings}, \PageStyle{myheadings},
  \PageStyle{plain}, e \PageStyle{empty} das classes \KOMAScript{}.
  \iffalse%
  O pacote ainda mais antigo \Package{scrpage}\IndexPackage{scrpage} foi
  marcado como obsoleto em 2001 e removido da distribuição regular do \KOMAScript{}
  em 2013.\par
  \fi%
  Desde 2013, o pacote \hyperref[cha:scrlayer]{\Package{scrlayer}}%
  \important{\hyperref[cha:scrlayer]{\Package{scrlayer}}}%
  \IndexPackage{scrlayer} foi incluído como um módulo básico do
  \KOMAScript. Este pacote fornece um modelo de camadas e um novo modelo
  de estilo de página baseado nele. Porém, a interface do pacote é quase
  demasiado flexível e, consequentemente, não é fácil para o usuário médio
  compreendê-la. Para mais informações sobre esta interface, consulte
  \autoref{cha:scrlayer} em \autoref{part:forExperts}. Contudo, algumas das
  opções que são na verdade parte do \Package{scrlayer}, e que por isso são
  retomadas novamente naquele capítulo, também estão documentadas aqui porque
  são necessárias para usar \Package{scrlayer-scrpage}.

  Muitos usuários já estão familiarizados com os comandos do \Package{scrpage2}.
  Por esta razão, \Package{scrlayer-scrpage} fornece um método para
  manipular cabeçalhos e rodapés que é baseado em \Package{scrlayer}, é
  amplamente compatível com \Package{scrpage2}, e ao mesmo tempo expande
  significativamente a interface do usuário. Se você já está familiarizado com
  \Package{scrpage2} e se abstém de chamar diretamente seus comandos internos,
  você pode normalmente usar \Package{scrlayer-scrpage} como um substituto
  direto. Isto também se aplica à maioria dos exemplos que usam \Package{scrpage2}
  encontrados em livros \LaTeX{} ou na Internet.%
  \iffalse%
    \iffree{}{\par Com o lançamento deste livro, \Package{scrlayer-scrpage}
      para \KOMAScript{} é recomendado como a ferramenta de escolha quando se
      trata de alterar os estilos de página predefinidos. O uso do pacote
      obsoleto \Package{scrpage2}\IndexPackage[indexmain]{scrpage2}%
      \important{\Package{scrpage2}} agora é obsoleto. Portanto, os
      comandos para este pacote desatualizado não fazem mais parte deste livro.
      Se você encontrar documentos mais antigos que ainda usam \Package{scrpage2},
      considere mudar para \Package{scrlayer-scrpage}. Não obstante, este
      capítulo contém algumas dicas para usar \Package{scrpage2}.}%
  \fi
\end{Explain}

Além de \Package{scrlayer-scrpage}\iffree{ ou \Package{scrpage2}}{},
você também poderia usar \Package{fancyhdr}\IndexPackage{fancyhdr} (veja
\cite{package:fancyhdr}) para configurar os cabeçalhos e rodapés das páginas.
Porém, este pacote não tem suporte para várias características \KOMAScript{},
por exemplo o esquema de elementos (veja \DescRef{\LabelBase.cmd.setkomafont},
\DescRef{\LabelBase.cmd.addtokomafont}, e
\DescRef{\LabelBase.cmd.usekomafont} em \autoref{sec:maincls.textmarkup},
\DescPageRef{maincls.cmd.setkomafont}) ou o formato de numeração configurável
para cabeçalhos dinâmicos (veja a opção \DescRef{maincls.option.numbers} e,
por exemplo, \DescRef{\LabelBase.cmd.chaptermarkformat} em
\autoref{sec:maincls.structure}, \DescPageRef{maincls.option.numbers} e
\DescPageRef{maincls.cmd.chaptermarkformat}). Por esta razão, se você está
usando uma classe \KOMAScript{}, deve usar o novo pacote
\Package{scrlayer-scrpage}. \iffree{Se você tiver problemas, ainda pode usar
\Package{scrpage2}.}{\ignorespaces} Naturalmente, você também pode usar
\Package{scrlayer-scrpage} com outras classes, como as padrões do \LaTeX{}.

Além das características descritas neste capítulo, \Package{scrlayer-scrpage}
fornece várias funções adicionais que provavelmente só interessam a um número
muito pequeno de usuários e, portanto, estão descritas em
\autoref{cha:scrlayer-scrpage-experts} de \autoref{part:forExperts}, começando
em \autopageref{cha:scrlayer-scrpage-experts}. Não obstante, se as opções
descritas em \autoref{part:forAuthors} forem insuficientes para seus propósitos,
você deve examinar \autoref{cha:scrlayer-scrpage-experts}.

\LoadCommonFile{options} % \section{Early or late Selection of Options}

\LoadCommonFile{headfootheight} % \section{Header and Footer Height}

\LoadCommonFile{textmarkup} % \section{Text Markup}

\section{Usando Estilos de Página Predefinidos}
\seclabel{predefined.pagestyles}

A forma mais fácil de criar cabeçalhos e rodapés personalizados com
\Package{scrlayer-scrpage} é usar um dos estilos de página predefinidos.
%
\iffalse % Umbruchoptimierung
  Esta seção introduz os estilos de página definidos pelo
  pacote \Package{scrlayer-scrpage} quando ele é carregado. Também explica os
  comandos que você pode usar para criar configurações básicas para estes estilos
  de página. Você pode encontrar opções adicionais, comandos e informações
  de contexto nas seções seguintes e em
  \autoref{sec:scrlayer-scrpage-experts.pagestyle.pairs} em
  \autoref{part:forExperts}.%
\fi

\begin{Declaration}
  \PageStyle{scrheadings}%
  \PageStyle{plain.scrheadings}
\end{Declaration}
O pacote \Package{scrlayer-scrpage} fornece dois estilos de página que você pode
reconfigurar ao seu gosto. O primeiro estilo de página é
\PageStyle{scrheadings}\important{\PageStyle{scrheadings}}, que se destina
a ser um estilo de página com cabeçalhos correntes. Seus padrões são similares
ao estilo de página \PageStyle{headings}\IndexPagestyle{headings} das classes
padrão \LaTeX{} ou \KOMAScript{}. Você pode configurar exatamente o que aparece
no cabeçalho ou rodapé com os comandos e opções descritos abaixo.

O segundo estilo de página é \PageStyle{plain.scrheadings}%
\important{\PageStyle{plain.scrheadings}}, que se destina a ser um estilo
sem cabeçalho corrente. Seus padrões se assemelham aos do
estilo de página \PageStyle{plain}\IndexPagestyle{plain} das classes padrão ou
\KOMAScript{}. Você pode configurar exatamente o que aparece no cabeçalho ou
rodapé com os comandos e opções descritos abaixo.

Naturalmente, você poderia configurar \PageStyle{scrheadings} para ser um estilo
de página sem cabeçalho corrente e \PageStyle{plain.scrheadings} para ser um estilo
de página com cabeçalho corrente. É, contudo, aconselhável seguir as convenções
mencionadas acima. Os dois estilos de página se influenciam mutuamente. Uma vez que
você aplique um desses estilos de página, \PageStyle{scrheadings} ficará acessível
como \PageStyle{headings}\important{\PageStyle{headings}}%
\IndexPagestyle{headings} e o estilo de página \PageStyle{plain.scrheadings}
ficará acessível como \PageStyle{plain}\important{\PageStyle{plain}}%
\IndexPagestyle{plain}. Assim, se você usar uma classe ou pacote que mude
automaticamente entre \PageStyle{headings} e \PageStyle{plain}, você apenas
precisa selecionar \PageStyle{scrheadings} ou \PageStyle{plain.scrheadings} uma vez.
Patches diretos às classes ou pacotes correspondentes não são necessários. Este par
de estilos de página pode servir como uma substituição direta para
\PageStyle{headings} e \PageStyle{plain}. Se você precisar de mais esses pares,
consulte \autoref{sec:scrlayer-scrpage-experts.pagestyle.pairs} em
\autoref{part:forExperts}.%
\EndIndexGroup


\begin{Declaration}
  \Macro{lehead}\OParameter{plain.scrheadings content}%
                \Parameter{scrheadings content}%
  \Macro{cehead}\OParameter{plain.scrheadings content}%
                \Parameter{scrheadings content}%
  \Macro{rehead}\OParameter{plain.scrheadings content}%
                \Parameter{scrheadings content}%
  \Macro{lohead}\OParameter{plain.scrheadings content}%
                \Parameter{scrheadings content}%
  \Macro{cohead}\OParameter{plain.scrheadings content}%
                \Parameter{scrheadings content}%
  \Macro{rohead}\OParameter{plain.scrheadings content}%
                \Parameter{scrheadings content}
\end{Declaration}
Você pode definir o conteúdo do cabeçalho para os estilos de página
\DescRef{\LabelBase.pagestyle.plain.scrheadings} e
\DescRef{\LabelBase.pagestyle.scrheadings} com estes comandos.
O argumento opcional define o conteúdo de um elemento do
estilo de página \DescRef{\LabelBase.pagestyle.plain.scrheadings}, enquanto o
argumento obrigatório define o conteúdo do elemento correspondente do
estilo de página \DescRef{\LabelBase.pagestyle.scrheadings}.

O conteúdo de páginas pares\,---\,ou de mão\,---\,esquerda\textnote{páginas de mão esquerda}
pode ser definido com \Macro{lehead}, \Macro{cehead}, e \Macro{rehead}. O
``\texttt{e}'' que aparece como a segunda letra dos nomes dos comandos significa
``\emph{even}'' (par).

O conteúdo de páginas ímpares\,---\,ou de mão\,---\,direita\textnote{páginas de mão direita}
pode ser definido com \Macro{lohead}, \Macro{cohead}, e \Macro{rohead}. O
``\texttt{o}'' que aparece como a segunda letra dos nomes dos comandos significa
``\emph{odd}'' (ímpar).

Observe\textnote{Atenção!} que em impressão unilateral, apenas páginas de mão direita
existem, e \LaTeX{} as designa como páginas ímpares independentemente do número da página.

Cada cabeçalho consiste de um elemento alinhado à\textnote{alinhado à esquerda} esquerda que pode
ser definido com \Macro{lehead} ou \Macro{lohead}. O ``\texttt{l}'' que aparece como
a primeira letra dos nomes dos comandos significa ``\emph{left}'' (esquerda).

Similarmente, cada cabeçalho tem um elemento centralizado\textnote{centralizado} que pode ser definido
com \Macro{cehead} ou \Macro{cohead}. O ``\texttt{c}'' que aparece como a
primeira letra dos nomes dos comandos significa ``\emph{centred}'' (centralizado).

Similarmente, cada cabeçalho tem um elemento alinhado à\textnote{alinhado à direita} direita que
pode ser definido com \Macro{rehead} ou \Macro{rohead}. O ``\texttt{r}'' que aparece
como a primeira letra dos nomes dos comandos significa ``\emph{right
aligned}'' (alinhado à direita).

\BeginIndexGroup
\BeginIndex{FontElement}{pagehead}\LabelFontElement{pagehead}%
\BeginIndex{FontElement}{pageheadfoot}\LabelFontElement{pageheadfoot}%
Estes elementos não possuem atributos individuais de fonte que você possa
alterar usando os comandos \DescRef{\LabelBase.cmd.setkomafont} e
\DescRef{\LabelBase.cmd.addtokomafont} (veja \autoref{sec:maincls.textmarkup},
\DescPageRef{maincls.cmd.setkomafont}). Em vez disso, eles usam um elemento
nomeado \FontElement{pagehead}. Antes deste elemento ser aplicado, o
elemento \FontElement{pageheadfoot} também será aplicado. Veja
\autoref{tab:scrlayer-scrpage.fontelements} para os padrões destes
elementos.%
\EndIndexGroup

O significado de cada comando para cabeçalhos em impressão bilateral é ilustrado
em \autoref{fig:scrlayer-scrpage.head}.%
%
\begin{figure}[tp]
  \centering
  \begin{picture}(\textwidth,30mm)(0,-10mm)
    \thinlines
    \small\ttfamily
    % left/even page
    \put(0,20mm){\line(1,0){.49\textwidth}}%
    \put(0,0){\line(0,1){20mm}}%
    \multiput(0,0)(0,-1mm){10}{\line(0,-1){.5mm}}%
    \put(.49\textwidth,5mm){\line(0,1){15mm}}%
    \put(.05\textwidth,10mm){%
      \color{ImageRed}%
      \put(-.5em,0){\line(1,0){4em}}%
      \multiput(3.5em,0)(.25em,0){5}{\line(1,0){.125em}}%
      \put(-.5em,0){\line(0,1){\baselineskip}}%
      \put(-.5em,\baselineskip){\line(1,0){4em}}%
      \multiput(3.5em,\baselineskip)(.25em,0){5}{\line(1,0){.125em}}%
      \makebox(4em,5mm)[l]{\Macro{lehead}}%
    }%
    \put(.465\textwidth,10mm){%
      \color{ImageBlue}%
      \put(-4em,0){\line(1,0){4em}}%
      \multiput(-4em,0)(-.25em,0){5}{\line(1,0){.125em}}%
      \put(0,0){\line(0,1){\baselineskip}}%
      \put(-4em,\baselineskip){\line(1,0){4em}}%
      \multiput(-4em,\baselineskip)(-.25em,0){5}{\line(1,0){.125em}}%
      \put(-4.5em,0){\makebox(4em,5mm)[r]{\Macro{rehead}}}%
    }%
    \put(.2525\textwidth,10mm){%
      \color{ImageGreen}%
      \put(-2em,0){\line(1,0){4em}}%
      \multiput(2em,0)(.25em,0){5}{\line(1,0){.125em}}%
      \multiput(-2em,0)(-.25em,0){5}{\line(1,0){.125em}}%
      \put(-2em,\baselineskip){\line(1,0){4em}}%
      \multiput(2em,\baselineskip)(.25em,0){5}{\line(1,0){.125em}}%
      \multiput(-2em,\baselineskip)(-.25em,0){5}{\line(1,0){.125em}}%
      \put(-2em,0){\makebox(4em,5mm)[c]{\Macro{cehead}}}%
    }%
    % right/odd page
    \put(.51\textwidth,20mm){\line(1,0){.49\textwidth}}%
    \put(.51\textwidth,5mm){\line(0,1){15mm}}%
    \put(\textwidth,0){\line(0,1){20mm}}%
    \multiput(\textwidth,0)(0,-1mm){10}{\line(0,-1){.5mm}}%
    \put(.5325\textwidth,10mm){%
      \color{ImageBlue}%
      \put(0,0){\line(1,0){4em}}%
      \multiput(4em,0)(.25em,0){5}{\line(1,0){.125em}}%
      \put(0,0){\line(0,1){\baselineskip}}%
      \put(0em,\baselineskip){\line(1,0){4em}}%
      \multiput(4em,\baselineskip)(.25em,0){5}{\line(1,0){.125em}}%
      \put(.5em,0){\makebox(4em,5mm)[l]{\Macro{lohead}}}%
    }%
    \put(.965\textwidth,10mm){%
      \color{ImageRed}%
      \put(-4em,0){\line(1,0){4em}}%
      \multiput(-4em,0)(-.25em,0){5}{\line(1,0){.125em}}%
      \put(0,0){\line(0,1){\baselineskip}}%
      \put(-4em,\baselineskip){\line(1,0){4em}}%
      \multiput(-4em,\baselineskip)(-.25em,0){5}{\line(1,0){.125em}}%
      \put(-4.5em,0){\makebox(4em,5mm)[r]{\Macro{rohead}}}%
    }%
    \put(.75\textwidth,10mm){%
      \color{ImageGreen}%
      \put(-2em,0){\line(1,0){4em}}%
      \multiput(2em,0)(.25em,0){5}{\line(1,0){.125em}}%
      \multiput(-2em,0)(-.25em,0){5}{\line(1,0){.125em}}%
      \put(-2em,\baselineskip){\line(1,0){4em}}%
      \multiput(2em,\baselineskip)(.25em,0){5}{\line(1,0){.125em}}%
      \multiput(-2em,\baselineskip)(-.25em,0){5}{\line(1,0){.125em}}%
      \put(-2em,0){\makebox(4em,5mm)[c]{\Macro{cohead}}}%
    }%
    % commands for both pages
    \color{ImageBlue}%
    \put(.5\textwidth,0){\makebox(0,\baselineskip)[c]{\Macro{ihead}}}%
    \color{ImageGreen}%
    \put(.5\textwidth,-5mm){\makebox(0,\baselineskip)[c]{\Macro{chead}}}
    \color{ImageRed}%
    \put(.5\textwidth,-10mm){\makebox(0,\baselineskip)[c]{\Macro{ohead}}}
    \put(\dimexpr.5\textwidth-2em,.5\baselineskip){%
      \color{ImageBlue}%
      \put(0,0){\line(-1,0){1.5em}}%
      \put(-1.5em,0){\vector(0,1){5mm}}%
      \color{ImageGreen}%
      \put(0,-1.25\baselineskip){\line(-1,0){\dimexpr .25\textwidth-2em\relax}}%
      \put(-\dimexpr
      .25\textwidth-2em\relax,-1.25\baselineskip){\vector(0,1){\dimexpr
          5mm+1.25\baselineskip\relax}}
      \color{ImageRed}%
      \put(0,-2.5\baselineskip){\line(-1,0){\dimexpr .45\textwidth-4em\relax}}%
      \put(-\dimexpr
      .45\textwidth-4em\relax,-2.5\baselineskip){\vector(0,1){\dimexpr
          5mm+2.5\baselineskip\relax}}
    }%
    \put(\dimexpr.5\textwidth+2em,.5\baselineskip){%
      \color{ImageBlue}%
      \put(0,0){\line(1,0){1.5em}}%
      \put(1.5em,0){\vector(0,1){5mm}}%
      \color{ImageGreen}%
      \put(0,-1.25\baselineskip){\line(1,0){\dimexpr .25\textwidth-2em\relax}}
      \put(\dimexpr
      .25\textwidth-2em\relax,-1.25\baselineskip){\vector(0,1){\dimexpr
          5mm+1.25\baselineskip\relax}}
      \color{ImageRed}%
      \put(0,-2.5\baselineskip){\line(1,0){\dimexpr .45\textwidth-4em\relax}}
      \put(\dimexpr
      .45\textwidth-4em\relax,-2.5\baselineskip){\vector(0,1){\dimexpr
          5mm+2.5\baselineskip\relax}}
   }%
  \end{picture}
  \caption[Comandos para definir o cabeçalho da página]%
          {O significado dos comandos para definir o conteúdo dos cabeçalhos
          mostrado em um esquema de duas páginas}
  \label{fig:scrlayer-scrpage.head}
\end{figure}
%
\begin{Example}
  Suponha que você esteja escrevendo um artigo curto e queira que o nome do autor
  apareça no lado esquerdo da página e o título do artigo apareça
  à direita. Você pode escrever, por exemplo:
\begin{lstcode}
  \documentclass{scrartcl}
  \usepackage{scrlayer-scrpage}
  \lohead{John Doe}
  \rohead{Page style with \KOMAScript}
  \pagestyle{scrheadings}
  \begin{document}
  \title{Page styles with \KOMAScript}
  \author{John Doe}
  \maketitle
  \end{document}
\end{lstcode}
  Mas o que acontece? Na primeira página há apenas um número de página no
  rodapé, enquanto o cabeçalho permanece vazio!

  A explicação é simples: A classe \Class{scrartcl}, como a classe padrão
  \Class{article}, muda para o estilo de página \PageStyle{plain} para a
  página que contém o título. Após o comando
  \DescRef{maincls.cmd.pagestyle}\PParameter{scrheadings} no preâmbulo do
  nosso exemplo, isto na verdade se refere ao
  estilo de página \DescRef{\LabelBase.pagestyle.plain.scrheadings}. O padrão para
  este estilo de página quando usando uma classe \KOMAScript{} é um cabeçalho de página vazio e
  um número de página no rodapé. No exemplo, os argumentos opcionais de
  \Macro{lohead} e \Macro{rohead} são omitidos, portanto o
  estilo de página \DescRef{\LabelBase.pagestyle.plain.scrheadings} permanece
  inalterado e o resultado para a primeira página é na verdade correto.

  O uso explícito de \DescRef{maincls.cmd.pagestyle}\PParameter{scrheadings}
  nem sequer é necessário. O pacote já executa este comando por si mesmo quando
  carregado, então ele define automaticamente o estilo de página para
  \DescRef{\LabelBase.pagestyle.scrheadings}\IndexPagestyle{scrheadings}. Isto
  também muda não apenas o estilo de página
  \DescRef{maincls.pagestyle.headings}\IndexPagestyle{headings} automaticamente
  para \DescRef{\LabelBase.pagestyle.scrheadings}, mas também
  \DescRef{maincls.pagestyle.plain}\IndexPagestyle{plain} para
  \DescRef{\LabelBase.pagestyle.plain.scrheadings}%
  \IndexPagestyle{plain.scrheadings}.

  Agora adicione texto suficiente ao exemplo após \DescRef{maincls.cmd.maketitle}
  para que uma segunda página seja impressa. Você pode simplesmente adicionar
  \Macro{usepackage}\PParameter{lipsum}\IndexPackage{lipsum} ao preâmbulo do
  documento e \Macro{lipsum}\IndexCmd{lipsum} abaixo
  \DescRef{maincls.cmd.maketitle}. Você verá que o cabeçalho da segunda
  página agora contém o autor e o título do documento como desejávamos.

  Para comparação, você também deve adicionar o argumento opcional para
  \Macro{lohead} e \Macro{rohead}. Altere o exemplo como segue:
\begin{lstcode}
  \documentclass{scrartcl}
  \usepackage{scrlayer-scrpage}
  \lohead[John Doe]
         {John Doe}
  \rohead[Page style with \KOMAScript]
         {Page style with \KOMAScript}
  \begin{document}
  \title{Page styles with \KOMAScript}
  \author{John Doe}
  \maketitle
  \end{document}
\end{lstcode}
  Agora você tem um cabeçalho na primeira página logo acima do próprio título.
  Isto é porque você reconfigurou o estilo de página
  \DescRef{\LabelBase.pagestyle.plain.scrheadings} com os dois argumentos
  opcionais. Como você provavelmente aprecia, seria melhor deixar este estilo de página
  inalterado, já que um cabeçalho corrente acima do título do documento é
  bastante incômodo.
  
  A propósito, como uma alternativa para configurar
  \DescRef{\LabelBase.pagestyle.plain.scrheadings} você poderia, se estivesse
  usando uma classe \KOMAScript{}, ter alterado o estilo de página para páginas que
  contêm cabeçalhos de título. Veja \DescRef{maincls.cmd.titlepagestyle}%
  \important{\DescRef{maincls.cmd.titlepagestyle}}%
  \IndexCmd{titlepagestyle} em \autoref{sec:maincls.pagestyle},
  \DescPageRef{maincls.cmd.titlepagestyle}.
\end{Example}

Observe\textnote{Atenção!} que você nunca deve colocar um título de seção
ou número de seção diretamente no cabeçalho usando um desses
comandos. Devido à forma assincronizada como \TeX{} apresenta e produz
páginas, fazer assim pode facilmente resultar no número errado ou texto de título no
cabeçalho corrente. Em vez disso, você deve usar o mecanismo de marca, idealmente em
conjunto com os procedimentos explicados na próxima seção.%
\EndIndexGroup

\begin{Declaration}
  \Macro{lehead*}\OParameter{plain.scrheadings content}%
                \Parameter{scrheadings content}%
  \Macro{cehead*}\OParameter{plain.scrheadings content}%
                \Parameter{scrheadings content}%
  \Macro{rehead*}\OParameter{plain.scrheadings content}%
                \Parameter{scrheadings content}%
  \Macro{lohead*}\OParameter{plain.scrheadings content}%
                \Parameter{scrheadings content}%
  \Macro{cohead*}\OParameter{plain.scrheadings content}%
                \Parameter{scrheadings content}%
  \Macro{rohead*}\OParameter{plain.scrheadings content}%
                \Parameter{scrheadings content}
\end{Declaration}
As versões com asterisco\ChangedAt{v3.14}{\Package{scrlayer-scrpage}} dos
comandos previamente descritos diferem das versões ordinárias apenas se você
omite o argumento opcional \PName{plain.scrheadings content}. Neste caso,
a versão sem asterisco não altera o conteúdo de
\DescRef{\LabelBase.pagestyle.plain.scrheadings}. A versão com asterisco, por outro lado,
usa o argumento obrigatório \PName{scrheading content}
para \DescRef{\LabelBase.pagestyle.plain.scrheadings} também. Então se ambos
argumentos devem ser iguais, você pode simplesmente usar a versão com asterisco com apenas
o argumento obrigatório.%

\begin{Example}
  Você pode encurtar o exemplo anterior usando as versões com asterisco de
  \DescRef{\LabelBase.cmd.lohead} e \DescRef{\LabelBase.cmd.rohead}:
\begin{lstcode}
  \documentclass{scrartcl}
  \usepackage{scrlayer-scrpage}
  \lohead*{John Doe}
  \rohead*{Page style with \KOMAScript}
  \begin{document}
  \title{Page styles with \KOMAScript}
  \author{John Doe}
  \maketitle
  \end{document}
\end{lstcode}%
\end{Example}%
\EndIndexGroup
\ExampleEndFix


\begin{Declaration}
  \Macro{lefoot}\OParameter{plain.scrheadings content}%
                \Parameter{scrheadings content}%
  \Macro{cefoot}\OParameter{plain.scrheadings content}%
                \Parameter{scrheadings content}%
  \Macro{refoot}\OParameter{plain.scrheadings content}%
                \Parameter{scrheadings content}%
  \Macro{lofoot}\OParameter{plain.scrheadings content}%
                \Parameter{scrheadings content}%
  \Macro{cofoot}\OParameter{plain.scrheadings content}%
                \Parameter{scrheadings content}%
  \Macro{rofoot}\OParameter{plain.scrheadings content}%
                \Parameter{scrheadings content}
\end{Declaration}
Você pode definir o conteúdo do rodapé para os estilos de página
\DescRef{\LabelBase.pagestyle.scrheadings} e
\DescRef{\LabelBase.pagestyle.plain.scrheadings} com estes comandos. O
argumento opcional define o conteúdo de um elemento de
\DescRef{\LabelBase.pagestyle.plain.scrheadings}, enquanto o argumento obrigatório
define o conteúdo do elemento correspondente de
\DescRef{\LabelBase.pagestyle.scrheadings}.

O conteúdo de páginas pares\,---\,ou de mão\,---\,esquerda\textnote{página de mão esquerda}
é definido com \Macro{lefoot}, \Macro{cefoot}, e \Macro{refoot}. O
``\texttt{e}'' que aparece como a segunda letra dos nomes dos comandos significa
``\emph{even}'' (par).

O conteúdo de páginas ímpares\,---\,ou de mão\,---\,direita\textnote{página de mão direita}
é definido com \Macro{lofoot}, \Macro{cofoot}, e \Macro{rofoot}. O
``\texttt{o}'' que aparece como a segunda letra dos nomes dos comandos significa
``\emph{odd}'' (ímpar).

Observe\textnote{Atenção!} que em impressão unilateral, apenas páginas de mão direita
existem, e \LaTeX{} as designa como páginas ímpares independentemente do número da página.

Cada rodapé consiste de um elemento alinhado à\textnote{alinhado à esquerda} esquerda que pode
ser definido com \Macro{lefoot} ou \Macro{lofoot}. O ``\texttt{l}'' que aparece como
a primeira letra dos nomes dos comandos significa ``\emph{left}'' (esquerda).

Similarmente, cada rodapé tem um elemento centralizado\textnote{centralizado} que pode ser definido
com \Macro{cefoot} ou \Macro{cofoot}. O ``\texttt{c}'' na primeira letra
dos nomes dos comandos significa ``\emph{centred}'' (centralizado).

Similarmente, cada rodapé tem um elemento alinhado à\textnote{alinhado à direita} direita que
pode ser definido com \Macro{refoot} ou \Macro{rofoot}. O ``\texttt{r}'' na
primeira letra dos nomes dos comandos significa ``\emph{right aligned}'' (alinhado à direita).

\BeginIndexGroup
\BeginIndex{FontElement}{pagefoot}\LabelFontElement{pagefoot}%
\BeginIndex{FontElement}{pageheadfoot}\LabelFontElement[foot]{pageheadfoot}%
Entretanto, estes elementos não possuem atributos individuais de fonte que possam ser
alterados com os comandos \DescRef{\LabelBase.cmd.setkomafont} e
\DescRef{\LabelBase.cmd.addtokomafont} (veja
\autoref{sec:maincls.textmarkup}, \DescPageRef{maincls.cmd.setkomafont}).
Em vez disso, eles usam um elemento nomeado
\FontElement{pagefoot}\important{\FontElement{pagefoot}}. Antes deste elemento
ser aplicado, o elemento de fonte
\FontElement{pageheadfoot}\important{\FontElement{pageheadfoot}} também é
aplicado. Veja \autoref{tab:scrlayer-scrpage.fontelements} para os padrões das
fontes destes elementos.%
\EndIndexGroup

O significado de cada comando para rodapés em impressão bilateral é ilustrado
em \autoref{fig:scrlayer-scrpage.foot}.%
%
\begin{figure}[bp]
  \centering
  \begin{picture}(\textwidth,30mm)
    \thinlines
    \small\ttfamily
    % left page
    \put(0,0){\line(1,0){.49\textwidth}}%
    \put(0,0){\line(0,1){20mm}}%
    \multiput(0,20mm)(0,1mm){10}{\line(0,1){.5mm}}%
    \put(.49\textwidth,0){\line(0,1){15mm}}%
    \put(.05\textwidth,5mm){%
      \color{ImageRed}%
      \put(-.5em,0){\line(1,0){4em}}%
      \multiput(3.5em,0)(.25em,0){5}{\line(1,0){.125em}}%
      \put(-.5em,0){\line(0,1){\baselineskip}}%
      \put(-.5em,\baselineskip){\line(1,0){4em}}%
      \multiput(3.5em,\baselineskip)(.25em,0){5}{\line(1,0){.125em}}%
      \makebox(4em,5mm)[l]{\Macro{lefoot}}%
    }%
    \put(.465\textwidth,5mm){%
      \color{ImageBlue}%
      \put(-4em,0){\line(1,0){4em}}%
      \multiput(-4em,0)(-.25em,0){5}{\line(1,0){.125em}}%
      \put(0,0){\line(0,1){\baselineskip}}%
      \put(-4em,\baselineskip){\line(1,0){4em}}%
      \multiput(-4em,\baselineskip)(-.25em,0){5}{\line(1,0){.125em}}%
      \put(-4.5em,0){\makebox(4em,5mm)[r]{\Macro{refoot}}}%
    }%
    \put(.2525\textwidth,5mm){%
      \color{ImageGreen}%
      \put(-2em,0){\line(1,0){4em}}%
      \multiput(2em,0)(.25em,0){5}{\line(1,0){.125em}}%
      \multiput(-2em,0)(-.25em,0){5}{\line(1,0){.125em}}%
      \put(-2em,\baselineskip){\line(1,0){4em}}%
      \multiput(2em,\baselineskip)(.25em,0){5}{\line(1,0){.125em}}%
      \multiput(-2em,\baselineskip)(-.25em,0){5}{\line(1,0){.125em}}%
      \put(-2em,0){\makebox(4em,5mm)[c]{\Macro{cefoot}}}%
    }%
    % right page
    \put(.51\textwidth,0){\line(1,0){.49\textwidth}}%
    \put(.51\textwidth,0){\line(0,1){15mm}}%
    \put(\textwidth,0){\line(0,1){20mm}}%
    \multiput(\textwidth,20mm)(0,1mm){10}{\line(0,1){.5mm}}%
    \put(.5325\textwidth,5mm){%
      \color{ImageBlue}%
      \put(0,0){\line(1,0){4em}}%
      \multiput(4em,0)(.25em,0){5}{\line(1,0){.125em}}%
      \put(0,0){\line(0,1){\baselineskip}}%
      \put(0em,\baselineskip){\line(1,0){4em}}%
      \multiput(4em,\baselineskip)(.25em,0){5}{\line(1,0){.125em}}%
      \put(.5em,0){\makebox(4em,5mm)[l]{\Macro{lofoot}}}%
    }%
    \put(.965\textwidth,5mm){%
      \color{ImageRed}%
      \put(-4em,0){\line(1,0){4em}}%
      \multiput(-4em,0)(-.25em,0){5}{\line(1,0){.125em}}%
      \put(0,0){\line(0,1){\baselineskip}}%
      \put(-4em,\baselineskip){\line(1,0){4em}}%
      \multiput(-4em,\baselineskip)(-.25em,0){5}{\line(1,0){.125em}}%
      \put(-4.5em,0){\makebox(4em,5mm)[r]{\Macro{rofoot}}}%
    }%
    \put(.75\textwidth,5mm){%
      \color{ImageGreen}%
      \put(-2em,0){\line(1,0){4em}}%
      \multiput(2em,0)(.25em,0){5}{\line(1,0){.125em}}%
      \multiput(-2em,0)(-.25em,0){5}{\line(1,0){.125em}}%
      \put(-2em,\baselineskip){\line(1,0){4em}}%
      \multiput(2em,\baselineskip)(.25em,0){5}{\line(1,0){.125em}}%
      \multiput(-2em,\baselineskip)(-.25em,0){5}{\line(1,0){.125em}}%
      \put(-2em,0){\makebox(4em,5mm)[c]{\Macro{cofoot}}}%
    }%
    % both pages
    \color{ImageBlue}%
    \put(.5\textwidth,15mm){\makebox(0,\baselineskip)[c]{\Macro{ifoot}}}%
    \color{ImageGreen}%
    \put(.5\textwidth,20mm){\makebox(0,\baselineskip)[c]{\Macro{cfoot}}}
    \color{ImageRed}%
    \put(.5\textwidth,25mm){\makebox(0,\baselineskip)[c]{\Macro{ofoot}}}
    \put(\dimexpr.5\textwidth-2em,.5\baselineskip){%
      \color{ImageBlue}%
      \put(0,15mm){\line(-1,0){1.5em}}%
      \put(-1.5em,15mm){\vector(0,-1){5mm}}%
      \color{ImageGreen}%
      \put(0,20mm){\line(-1,0){\dimexpr .25\textwidth-2em\relax}}%
      \put(-\dimexpr .25\textwidth-2em\relax,20mm){\vector(0,-1){10mm}}%
      \color{ImageRed}%
      \put(0,25mm){\line(-1,0){\dimexpr .45\textwidth-4em\relax}}%
      \put(-\dimexpr .45\textwidth-4em\relax,25mm){\vector(0,-1){15mm}}%
    }%
    \put(\dimexpr.5\textwidth+2em,.5\baselineskip){%
      \color{ImageBlue}%
      \put(0,15mm){\line(1,0){1.5em}}%
      \put(1.5em,15mm){\vector(0,-1){5mm}}%
      \color{ImageGreen}%
      \put(0,20mm){\line(1,0){\dimexpr .25\textwidth-2em\relax}}%
      \put(\dimexpr .25\textwidth-2em\relax,20mm){\vector(0,-1){10mm}}%
      \color{ImageRed}%
      \put(0,25mm){\line(1,0){\dimexpr .45\textwidth-4em\relax}}%
      \put(\dimexpr .45\textwidth-4em\relax,25mm){\vector(0,-1){15mm}}%
    }%
  \end{picture}
  \caption[Comandos para definir o rodapé da página]%
          {O significado dos comandos para definir o conteúdo do
            rodapé mostrado em um esquema de duas páginas}%
  \label{fig:scrlayer-scrpage.foot}
\end{figure}
%
\begin{Example}
  Vamos retornar ao exemplo do artigo curto. Digamos que você queira
  especificar o editor no lado esquerdo do rodapé. Você alteraria o
  exemplo acima para:
\begin{lstcode}
  \documentclass{scrartcl}
  \usepackage{scrlayer-scrpage}
  \lohead{John Doe}
  \rohead{Page style with \KOMAScript}
  \lofoot{Smart Alec Publishing}
  \usepackage{lipsum}
  \begin{document}
  \title{Page styles with \KOMAScript}
  \author{John Doe}
  \maketitle
  \lipsum
  \end{document}
\end{lstcode}
  Uma vez mais, o editor não é impresso na primeira página com o título.
  A razão é a mesma do exemplo com
  \DescRef{\LabelBase.cmd.lohead} acima. E a solução para colocar o
  editor na primeira página é similar:
\begin{lstcode}
  \lofoot[Smart Alec Publishing]
         {Smart Alec Publishing}
\end{lstcode}
  Agora você decide\textnote{mudança de fonte}\important{\FontElement{pageheadfoot}}%
  \IndexFontElement{pageheadfoot} que o cabeçalho e rodapé devem usar uma
  fonte reta mas menor no lugar da fonte inclinada padrão:
\begin{lstcode}
  \setkomafont{pageheadfoot}{\small}
\end{lstcode}
  Além disso, o cabeçalho, mas não o rodapé, deve ser em negrito:
\begin{lstcode}
  \setkomafont{pagehead}{\bfseries}
\end{lstcode}
  É importante\textnote{Atenção!} que este comando não ocorra até
  depois que \Package{scrpage-scrlayer} tenha sido carregado porque a classe
  \KOMAScript{} define \DescRef{\LabelBase.fontelement.pagehead} como um apelido para
  \DescRef{\LabelBase.fontelement.pageheadfoot}. Apenas ao carregar
  \Package{scrpage-scrlayer}, \DescRef{\LabelBase.fontelement.pagehead}
  se tornará um elemento independente de
  \DescRef{\LabelBase.fontelement.pageheadfoot}.

  Agora adicione um \Macro{lipsum} a mais e a opção
  \Option{twoside}\IndexOption{twoside}\important{\Option{twoside}}
  ao carregar \Class{scrartcl}. Primeiro de tudo, você verá o número da página
  se mover do centro para a margem externa do rodapé da página, devido aos
  padrões alterados de \DescRef{\LabelBase.pagestyle.scrheadings} e
  \DescRef{\LabelBase.pagestyle.plain.scrheadings} para impressão bilateral com
  uma classe \KOMAScript{}.

  Simultaneamente, o autor, título do documento, e editor desaparecerão da
  página~2. Eles aparecem apenas na página~3. Isto é porque apenas usamos
  comandos para páginas ímpares. Você pode reconhecer isto pelo ``\texttt{o}'' na
  segunda posição dos nomes dos comandos.

  Agora, poderíamos simplesmente copiar aqueles comandos e trocar o ``\texttt{o}'' com
  um ``\texttt{e}'' para definir o conteúdo de páginas \emph{pares}. Mas com
  impressão bilateral, faz mais sentido usar elementos espelhados-invertidos,
  i.\,e. o elemento esquerdo de uma página par deve se tornar o elemento direito da
  página ímpar e vice-versa. Para alcançar isto, também trocamos a primeira
  letra ``\texttt{l}'' com ``\texttt{r}'':
\begin{lstcode}
  \documentclass[twoside]{scrartcl}
  \usepackage{scrlayer-scrpage}
  \lohead{John Doe}
  \rohead{Page style with \KOMAScript}
  \lofoot[Smart Alec Publishing]
         {Smart Alec Publishing}
  \rehead{John Doe}
  \lohead{Page style with \KOMAScript}
  \refoot[Smart Alec Publishing]
         {Smart Alec Publishing}
  \usepackage{lipsum}
  \begin{document}
  \title{Page styles with \KOMAScript}
  \author{John Doe}
  \maketitle
  \lipsum\lipsum
  \end{document}
\end{lstcode}
\end{Example}
%
Como é um pouco incômodo definir páginas esquerda e direita separadamente em
casos como no exemplo anterior, uma solução mais simples para este caso comum é
introduzida abaixo.

Permita-me uma vez mais uma nota importante:\textnote{Atenção!} você nunca deve
colocar um título de seção ou número de seção diretamente no rodapé usando
um desses comandos. Devido à forma assincronizada como \TeX{} apresenta e produz
páginas, fazer assim pode facilmente resultar no número errado ou texto de título
no cabeçalho corrente. Em vez disso, você deve usar o mecanismo de marca, idealmente em
conjunto com os procedimentos explicados na próxima seção.%
\EndIndexGroup


\begin{Declaration}
  \Macro{lefoot*}\OParameter{plain.scrheadings content}%
                \Parameter{scrheadings content}%
  \Macro{cefoot*}\OParameter{plain.scrheadings content}%
                \Parameter{scrheadings content}%
  \Macro{refoot*}\OParameter{plain.scrheadings content}%
                \Parameter{scrheadings content}%
  \Macro{lofoot*}\OParameter{plain.scrheadings content}%
                \Parameter{scrheadings content}%
  \Macro{cofoot*}\OParameter{plain.scrheadings content}%
                \Parameter{scrheadings content}%
  \Macro{rofoot*}\OParameter{plain.scrheadings content}%
                \Parameter{scrheadings content}
\end{Declaration}
As versões com asterisco\ChangedAt{v3.14}{\Package{scrlayer-scrpage}} dos
comandos previamente descritos diferem apenas se você omite o argumento opcional
\OParameter{plain.scrheadings content}. Neste caso, a versão sem asterisco não altera
o conteúdo de \DescRef{\LabelBase.pagestyle.plain.scrheadings}. A versão com asterisco, por outro lado,
usa o argumento obrigatório \PName{scrheading content} para
\DescRef{\LabelBase.pagestyle.plain.scrheadings} também. Então se ambos argumentos
devem ser iguais, você pode simplesmente usar a versão com asterisco com apenas
o argumento obrigatório.%

\begin{Example}
  Você pode encurtar o exemplo anterior usando as versões com asterisco de
  \DescRef{\LabelBase.cmd.lofoot} e \DescRef{\LabelBase.cmd.refoot}:
\begin{lstcode}
  \documentclass[twoside]{scrartcl}
  \usepackage{scrlayer-scrpage}
  \lohead{John Doe}
  \rohead{Page style with \KOMAScript}
  \lofoot*{Smart Alec Publishing}
  \rehead{John Doe}
  \lohead{Page style with \KOMAScript}
  \refoot*{Smart Alec Publishing}
  \usepackage{lipsum}
  \begin{document}
  \title{Page styles with \KOMAScript}
  \author{John Doe}
  \maketitle
  \lipsum\lipsum
  \end{document}
\end{lstcode}
\end{Example}
%
\EndIndexGroup
\ExampleEndFix


\begin{Declaration}
  \Macro{ohead}\OParameter{plain.scrheadings content}%
                \Parameter{scrheadings content}%
  \Macro{chead}\OParameter{plain.scrheadings content}%
                \Parameter{scrheadings content}%
  \Macro{ihead}\OParameter{plain.scrheadings content}%
                \Parameter{scrheadings content}%
  \Macro{ofoot}\OParameter{plain.scrheadings content}%
                \Parameter{scrheadings content}%
  \Macro{cfoot}\OParameter{plain.scrheadings content}%
                \Parameter{scrheadings content}%
  \Macro{ifoot}\OParameter{plain.scrheadings content}%
                \Parameter{scrheadings content}
\end{Declaration}
Para configurar os cabeçalhos e rodapés para impressão bilateral com os
comandos previamente descritos, você teria que configurar os lados esquerdo e direito
separadamente um do outro. Na maioria dos casos, porém, os lados esquerdo e direito
são mais ou menos simétricos. Um item que aparece no lado esquerdo de uma
página par deve aparecer no lado direito de uma página ímpar e vice-versa. Elementos
centralizados geralmente estão centralizados em ambos os lados.

Para simplificar a definição de tais estilos de página simétricos,
\Package{scrlayer-scrpage} tem atalhos. O comando \Macro{ohead}
corresponde a uma chamada tanto para \DescRef{\LabelBase.cmd.lehead} como para
\DescRef{\LabelBase.cmd.rohead}. O comando \Macro{chead} corresponde a uma
chamada tanto para \DescRef{\LabelBase.cmd.cehead} como para
\DescRef{\LabelBase.cmd.cohead}. E o comando \Macro{ihead} corresponde a
uma chamada tanto para \DescRef{\LabelBase.cmd.rehead} como para
\DescRef{\LabelBase.cmd.lohead}. O mesmo aplica-se aos comandos equivalentes
para o rodapé da página. Um esboço destas relações também pode ser encontrado em
\autoref{fig:scrlayer-scrpage.head} em \autopageref{fig:scrlayer-scrpage.head}
e \autoref{fig:scrlayer-scrpage.foot} em
\autopageref{fig:scrlayer-scrpage.foot}.
%
\begin{Example}
  Você pode simplificar o exemplo anterior usando os novos comandos:
\begin{lstcode}
  \documentclass[twoside]{scrartcl}
  \usepackage{scrlayer-scrpage}
  \ihead{John Doe}
  \ohead{Page style with \KOMAScript}
  \ifoot[Smart Alec Publishing]
        {Smart Alec Publishing}
  \usepackage{lipsum}
  \begin{document}
  \title{Page styles with \KOMAScript}
  \author{John Doe}
  \maketitle
  \lipsum\lipsum
  \end{document}
\end{lstcode}
\iffalse%
  Como você pode ver, você pode usar metade do número de comandos mas obter o mesmo
  resultado. %
\fi%
\end{Example}%
Porque a impressão unilateral trata todas as páginas como páginas ímpares, estes comandos são
sinônimos dos comandos correspondentes do lado direito quando em modo unilateral.
Portanto, na maioria dos casos, você precisará apenas destes seis comandos em vez dos
doze descritos antes.

Permita-me uma vez mais uma nota importante:\textnote{Atenção!} você nunca deve
colocar um título de seção ou número de seção diretamente no rodapé usando um desses
comandos. Devido à forma assincronizada como \TeX{} apresenta e produz
páginas, fazer assim pode facilmente resultar no número errado ou texto de título
no cabeçalho corrente. Em vez disso, você deve usar o mecanismo de marca, idealmente em
conjunto com os procedimentos explicados na próxima seção.%
\EndIndexGroup


\begin{Declaration}
  \Macro{ohead*}\OParameter{plain.scrheadings content}%
                \Parameter{scrheadings content}%
  \Macro{chead*}\OParameter{plain.scrheadings content}%
                \Parameter{scrheadings content}%
  \Macro{ihead*}\OParameter{plain.scrheadings content}%
                \Parameter{scrheadings content}%
  \Macro{ofoot*}\OParameter{plain.scrheadings content}%
                \Parameter{scrheadings content}%
  \Macro{cfoot*}\OParameter{plain.scrheadings content}%
                \Parameter{scrheadings content}%
  \Macro{ifoot*}\OParameter{plain.scrheadings content}%
                \Parameter{scrheadings content}
\end{Declaration}
Os comandos previamente descritos também têm versões com asterisco
\ChangedAt{v3.14}{\Package{scrlayer-scrpage}} que diferem apenas se você
omite o argumento opcional \OParameter{plain.scrheadings content}. Neste
caso, a versão sem asterisco não altera o conteúdo de
\DescRef{\LabelBase.pagestyle.plain.scrheadings}. A versão com asterisco,
por outro lado, também usa o argumento obrigatório \PName{scrheadings
content} para \DescRef{\LabelBase.pagestyle.plain.scrheadings}. Então se ambos
argumentos devem ser iguais, você pode simplesmente usar a versão com asterisco com apenas
o argumento obrigatório.%

\begin{Example}
  Você pode encurtar o exemplo anterior usando a versão com asterisco de
  \DescRef{\LabelBase.cmd.ifoot}:
\begin{lstcode}
  \documentclass[twoside]{scrartcl}
  \usepackage{scrlayer-scrpage}
  \ihead{John Doe}
  \ohead{Page style with \KOMAScript}
  \ifoot*{Smart Alec Publishing}
  \usepackage{lipsum}
  \begin{document}
  \title{Page styles with \KOMAScript}
  \author{John Doe}
  \maketitle
  \lipsum\lipsum
  \end{document}
\end{lstcode}%
\end{Example}%
\EndIndexGroup


\begin{Declaration}
  \OptionVName{pagestyleset}{setting}
\end{Declaration}
\BeginIndex{Option}{pagestyleset~=KOMA-Script}%
\BeginIndex{Option}{pagestyleset~=standard}%
Os exemplos acima referem-se várias vezes às configurações padrão dos estilos de página
\DescRef{\LabelBase.pagestyle.scrheadings}\IndexPagestyle{scrheadings}
e \DescRef{\LabelBase.pagestyle.plain.scrheadings}%
\IndexPagestyle{plain.scrheadings}. De fato, \Package{scrlayer-scrpage}
atualmente fornece dois padrões diferentes para estes estilos de página. Você pode
selecioná-los manualmente com a opção \Option{pagestyleset}.

A configuração \PValue{KOMA-Script}\important{\OptionValue{pagestyleset}{KOMA-Script}}
seleciona os padrões, que também são definidos automaticamente se a
opção não for especificada e uma classe \KOMAScript{} for detectada. Em impressão bilateral,
\DescRef{\LabelBase.pagestyle.scrheadings} usa cabeçalhos correntes alinhados
externamente no cabeçalho e números de página alinhados externamente no rodapé.
Em impressão unilateral, o cabeçalho corrente será impresso no
meio do cabeçalho e o número de página no meio do rodapé. Letras maiúsculas
e minúsculas são usadas nos cabeçalhos correntes automáticos como
aparecem nos títulos de seção. Isto corresponde à opção
\OptionValueRef{\LabelBase}{markcase}{used}\IndexOption{markcase~=used}%
\important{\OptionValueRef{\LabelBase}{markcase}{used}}. O
estilo de página \DescRef{\LabelBase.pagestyle.plain.scrheadings} não tem cabeçalhos
correntes, mas os números de página são impressos da mesma forma.

Entretanto, se a classe \hyperref[cha:scrlttr2]{\Class{scrlttr2}}%
\important{\hyperref[cha:scrlttr2]{\Class{scrlttr2}}}%
\IndexClass{scrlttr2} for detectada, as configurações padrão são baseadas nos
estilos de página daquela classe. Veja \autoref{sec:scrlttr2.pagestyle},
\autopageref{sec:scrlttr2.pagestyle}.

A configuração \PValue{standard}\important{\OptionValue{pagestyleset}{standard}}
seleciona padrões que correspondem aos estilos de página das classes
padrão. Isto também é ativado automaticamente se a opção não tiver sido
especificada e nenhuma classe \KOMAScript{} for detectada. Neste caso, para impressão bilateral,
\DescRef{\LabelBase.pagestyle.scrheadings} usa cabeçalhos correntes
alinhados internamente no cabeçalho, e os números de página serão impressos\,---\,também
no cabeçalho\,---\,alinhados externamente. A impressão unilateral usa as mesmas configurações,
mas como apenas páginas de mão direita existem neste modo, o cabeçalho corrente sempre
será alinhado à esquerda e o número de página alinhado à direita. Os cabeçalhos
correntes automáticos\,---\,apesar de objeções tipográficas consideráveis\,---\,são
convertidos em letras maiúsculas, como seria com a opção
\OptionValueRef{\LabelBase}{markcase}{upper}\IndexOption{markcase~=upper}%
\important{\OptionValueRef{\LabelBase}{markcase}{upper}}. Em impressão unilateral,
o estilo de página \DescRef{\LabelBase.pagestyle.plain.scrheadings}
difere consideravelmente de \DescRef{\LabelBase.pagestyle.scrheadings} porque
o número de página é impresso no meio do rodapé.
Diferentemente\textnote{\KOMAScript{} vs. classes padrão} do estilo de página \PageStyle{plain}
nas classes padrão,
\DescRef{\LabelBase.pagestyle.plain.scrheadings} omite o número de página em
impressão bilateral. As classes padrão imprimem o número de página no meio
do rodapé, o que não corresponde ao resto dos estilos de página em impressão
bilateral.
\iffalse % Umbruchkorrekturtext
  Se você quiser o número de página de volta
\begin{lstcode}
  \cfoot[\pagemark]{}
\end{lstcode}
  vai restaurá-lo. %
\fi%
O cabeçalho corrente é omitido em \DescRef{\LabelBase.pagestyle.plain.scrheadings}.

Observe\textnote{Atenção!} que usar esta opção ativa o
estilo de página \DescRef{\LabelBase.pagestyle.scrheadings}\IndexPagestyle{scrheadings}%
\important{\DescRef{\LabelBase.pagestyle.scrheadings}}.
\iffalse% Umbruchkorrektur
  Isto também se aplica se você usar a opção dentro do documento.%
\fi
%
\EndIndexGroup


\LoadCommonFile{pagestylemanipulation} % \section{Manipulating Defined Page Styles}

\begin{Declaration}
  \OptionVName{headwidth}{width\textup{:}offset\textup{:}offset}%
  \OptionVName{footwidth}{width\textup{:}offset\textup{:}offset}
\end{Declaration}
Por padrão, o cabeçalho\Index{cabeçalho>largura} e rodapé\Index{rodapé>largura} são
tão largos quanto a área de texto. Entretanto, você pode mudar isto usando estas
opções \KOMAScript{}. O valor \PName{width} é a largura desejada do
cabeçalho ou rodapé. O \PName{offset} define até onde o cabeçalho ou rodapé
deve ser movido em direção à margem externa\,---\,em impressão unilateral para a
direita\,---. Todos os três\ChangedAt{v3.14}{\Package{scrlayer-scrpage}}
valores são opcionais e podem ser omitidos. Se você omite um valor, você também pode omitir
o dois-pontos associado à sua esquerda. Se apenas um \PName{offset} for
especificado, é usado para páginas pares e ímpares. Caso contrário, o primeiro
\PName{offset} é usado para páginas ímpares e o segundo \PName{offset} para páginas pares em
modo bilateral. Se você usar apenas um valor sem dois-pontos, este será o
\PName{width}.

Para ambos \PName{width} e o \PName{offset} você pode usar qualquer valor de comprimento válido,
comprimento \LaTeX{}, dimensão \TeX{}, ou espaço \TeX{}. Além disso, você pode
usar uma expressão de dimensão \eTeX{} com as operações aritméticas básicas
\texttt{+}, \texttt{-}, \texttt{*}, \texttt{/}, e parênteses. Veja
\cite[section~3.5]{manual:eTeX} para mais informações sobre tais expressões. Veja
\autoref{sec:scrlayer-scrpage.options} para mais informações sobre usar um
comprimento \LaTeX{} como um valor de opção. O \PName{width} também pode ser um dos
valores simbólicos mostrados em \autoref{tab:scrlayer-scrpage.symbolic.values}.

Por padrão, o cabeçalho e o rodapé são a largura da área de texto. O padrão
\PName{offset} depende da \PName{width} selecionada. A impressão unilateral
típicamente usa metade da diferença entre \PName{width} e a
largura da área de texto. Isto centraliza o cabeçalho horizontalmente acima da área de
texto. A impressão bilateral, por outro lado, usa apenas um terço da
diferença entre \PName{width} e a largura da área de texto. Entretanto, se
\PName{width} for a largura de toda a área de texto e da coluna de notas
marginais, o padrão \PName{offset} será zero. Se isto é muito complicado
para você, deve simplesmente especificar o \PName{offset} desejado você mesmo.
%
\begin{table}
  \centering
  \caption[Valores simbólicos para as opções \Option{headwidth} e \Option{footwidth}
  ]{Valores simbólicos disponíveis para o valor \PName{width} das opções
    \Option{headwidth} e \Option{footwidth}}
  \label{tab:scrlayer-scrpage.symbolic.values}
  \begin{desctabular}
    \entry{\PValue{foot}}{%
      a largura atual do rodapé%
    }%
    \entry{\PValue{footbotline}}{%
      o comprimento atual da linha horizontal abaixo do rodapé%
    }%
    \entry{\PValue{footsepline}}{%
      o comprimento atual da linha horizontal acima do rodapé%
    } \entry{\PValue{head}}{%
      a largura atual do cabeçalho%
    }%
    \entry{\PValue{headsepline}}{%
      o comprimento atual da linha horizontal abaixo do cabeçalho%
    }%
    \entry{\PValue{headtopline}}{%
      o comprimento atual da linha horizontal acima do cabeçalho%
    }%
    \entry{\PValue{marginpar}}{%
      a largura da coluna de notas marginais incluindo a distância
      entre a área de texto e a coluna de notas marginais%
    }%
    \entry{\PValue{page}}{%
      a largura da página levando em conta qualquer correção de encadernação definida
      com a ajuda do pacote \Package{typearea} (veja a
      opção \DescRef{typearea.option.BCOR} em
      \autoref{sec:typearea.typearea}, \DescPageRef{typearea.option.BCOR})%
    }%
    \entry{\PValue{paper}}{%
      a largura do papel sem considerar nenhuma correção de encadernação%
    }%
    \entry{\PValue{text}}{%
      a largura da área de texto%
    }%
    \entry{\PValue{textwithmarginpar}}{%
      a largura da área de texto incluindo a coluna de notas marginais e
      a distância entre as duas (Nota: apenas neste caso o
      padrão para \PName{offset} é zero)%
    }%
  \end{desctabular}
\end{table}
%
\EndIndexGroup


\begin{Declaration}
  \OptionVName{headtopline}{thickness\textup{:}length}%
  \OptionVName{headsepline}{thickness\textup{:}length}%
  \OptionVName{footsepline}{thickness\textup{:}length}%
  \OptionVName{footbotline}{thickness\textup{:}length}
\end{Declaration}
\BeginIndex{Option}{headtopline~=\PName{thickness\textup{:}length}}%
\BeginIndex{Option}{headsepline~=\PName{thickness\textup{:}length}}%
\BeginIndex{Option}{footsepline~=\PName{thickness\textup{:}length}}%
\BeginIndex{Option}{footbotline~=\PName{thickness\textup{:}length}}%
The \KOMAScript{} classes provide only one separation line below the header
and another above the footer, and you can only switch these lines on or off.
But the \Package{scrlayer-scrpage} package also lets you place lines above the
header and below the footer. And for all four lines, you can not only switch
them on an off but also configure their \PName{length} and \PName{thickness}.

Both values are optional. If you omit the \PName{thickness}, a default value
of 0.4\Unit{pt} is used, producing a so-called \emph{hairline}. If you omit
the \PName{length}, the width of the header or footer will be used. If you
omit both, you can also omit the colon. If you use only one value without
colon, this is the \PName{thickness}.

Of course, the \PName{length} can be not just shorter than the current width
of the header or footer but also longer. See also the options 
\DescRef{\LabelBase.option.ilines}\IndexOption{ilines}%
\important{\DescRef{\LabelBase.option.ilines},
\DescRef{\LabelBase.option.clines}, \DescRef{\LabelBase.option.olines}},
\DescRef{\LabelBase.option.clines}\IndexOption{clines} and
\DescRef{\LabelBase.option.olines}\IndexOption{olines} later in this section.

\BeginIndexGroup
\BeginIndex{FontElement}{headtopline}\LabelFontElement{headtopline}%
\BeginIndex{FontElement}{headsepline}\LabelFontElement{headsepline}%
\BeginIndex{FontElement}{footsepline}\LabelFontElement{footsepline}%
\BeginIndex{FontElement}{footbotline}\LabelFontElement{footbotline}%
In addition to the length and thickness, you can also change the colour of the
lines. Initially the colour depends on the colour of the header or footer. In
addition to this, however, the settings of the corresponding elements
\important[i]{\FontElement{headtopline}\\
  \FontElement{headsepline}\\
  \FontElement{footsepline}\\
  \FontElement{footbotline}} \FontElement{headtopline},
\FontElement{headsepline}, \FontElement{footsepline} and
\FontElement{footbotline} are applied. You can
change these using the \DescRef{\LabelBase.cmd.setkomafont} or
\DescRef{\LabelBase.cmd.addtokomafont} commands (see
\autoref{sec:maincls.textmarkup}, \DescPageRef{maincls.cmd.setkomafont}).
By default these elements are empty, so they do not change the current font or
colour. Font changes at this point, unlike colour changes, make little sense
anyway and are therefore not recommended for these elements.%
\EndIndexGroup
%
\EndIndexGroup


\begin{Declaration}
  \OptionVName{plainheadtopline}{simple switch}%
  \OptionVName{plainheadsepline}{simple switch}%
  \OptionVName{plainfootsepline}{simple switch}%
  \OptionVName{plainfootbotline}{simple switch}
\end{Declaration}
You can use these options to apply the settings for the lines to the
\PageStyle{plain} page style. You can find the available values for
\PName{simple switch} in \autoref{tab:truefalseswitch} on
\autopageref{tab:truefalseswitch}. If one of these options is activated, the
\PageStyle{plain} page style will use the line settings given by the options
and commands described above. If the option is deactivated, the
\PageStyle{plain} will not show the corresponding line.%
\EndIndexGroup


\begin{Declaration}
  \Option{ilines}%
  \Option{clines}%
  \Option{olines}
\end{Declaration}
As previously explained, dividing lines for the header or footer can be longer
or shorter than the width of the header or footer respectively. But the
question remains how these lines are aligned. By default, all lines are
aligned to the left margin in one-sided printing and to the inner margin in
two-sided printing. This corresponds to using the \Option{ilines} option.
Alternatively, you can use the \Option{clines} option to centre the lines with
respect to the width of the header or footer, or the \Option{olines} option to
align them to the outer (or right) margin.%
\EndIndexGroup
%
\EndIndexGroup

%%% Local Variables: 
%%% mode: latex
%%% TeX-master: "scrguide-en.tex"
%%% coding: utf-8
%%% ispell-local-dictionary: "en_GB"
%%% eval: (flyspell-mode 1)
%%% End: 
