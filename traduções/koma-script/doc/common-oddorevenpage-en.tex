% ======================================================================
% common-oddorevenpage-en.tex
% Copyright (c) Markus Kohm, 2001-2022
%
% This file is part of the LaTeX2e KOMA-Script bundle.
%
% This work may be distributed and/or modified under the conditions of
% the LaTeX Project Public License, version 1.3c of the license.
% The latest version of this license is in
%   http://www.latex-project.org/lppl.txt
% and version 1.3c or later is part of all distributions of LaTeX
% version 2005/12/01 or later and of this work.
%
% This work has the LPPL maintenance status "author-maintained".
%
% The Current Maintainer and author of this work is Markus Kohm.
%
% This work consists of all files listed in MANIFEST.md.
% ======================================================================
%
% Paragraphs that are common for several chapters of the KOMA-Script guide
% Maintained by Markus Kohm
%
% ======================================================================

\KOMAProvidesFile{common-oddorevenpage-en.tex}
                 [$Date: 2022-06-05 12:40:11 +0200 (So, 05. Jun 2022) $
                  KOMA-Script guide (common paragraph: Detection of Odd and
                                     Even Pages)]
\translator{Markus Kohm\and Krickette Murabayashi\and Karl Hagen}

\section{Detectando Páginas Ímpares e Pares}
\seclabel{oddOrEven}%
\BeginIndexGroup%
\BeginIndex{}{page>odd}%
\BeginIndex{}{page>even}%

\IfThisCommonFirstRun{}{%
  As informações em \autoref{sec:maincls.oddOrEven} aplicam-se igualmente a este
  capítulo. Portanto, se você já leu e compreendeu
  \autoref{sec:\ThisCommonFirstLabelBase.oddOrEven}, pode pular para
  \autopageref{sec:\ThisCommonLabelBase.oddOrEven.next},
  \autopageref{sec:\ThisCommonLabelBase.oddOrEven.next}.%
}

Em documentos de dois lados, distinguimos páginas esquerdas e direitas. Páginas esquerdas sempre
têm um número de página par, e páginas direitas sempre têm um número de página ímpar. %
\IfThisCommonLabelBase{maincls}{%
  Identificar páginas direitas e esquerdas é equivalente a identificar páginas
  de numeração par ou ímpar, e por isso normalmente nos referimos a elas como páginas pares e ímpares
  neste \iffree{guia}{livro}.

  % Umbruchkorrekturtext
  \iftrue%
    Em documentos de um lado, a distinção entre páginas esquerdas e direitas não
    existe. No entanto, ainda existem páginas com números de página pares e ímpares.%
  \fi%
}{%
  \IfThisCommonLabelBase{scrlttr2}{%
    Como regra, cartas serão compostas em um lado. No entanto, se você precisar imprimir
    cartas usando ambos os lados do papel ou, em casos excepcionais, estiver
    gerando cartas realmente de dois lados, pode ser útil saber se você
    está atualmente em uma página par ou ímpar.%
  }{}%
}


\begin{Declaration}
  \Macro{Ifthispageodd}\Parameter{parte verdadeira}\Parameter{parte falsa}
\end{Declaration}%
Se\IfThisCommonLabelBase{maincls}{%
  \ChangedAt{v3.28}{\Class{scrbook}\and \Class{scrreprt}\and
    \Class{scrartcl}}%
}{%
  \IfThisCommonLabelBase{scrlttr2}{%
    \ChangedAt{v3.28}{\Class{scrlttr2}}%
  }{%
    \IfThisCommonLabelBase{scrextend}{%
      \ChangedAt{v3.28}{\Package{scrextend}}%
    }{}%
  }%
} %
você quiser determinar se o texto aparece em uma página par ou ímpar,
o \KOMAScript{} fornece o comando \Macro{Ifthispageodd}. O argumento \PName{parte
	verdadeira} é executado somente se você estiver atualmente em uma página ímpar.
Caso contrário, o argumento \PName{parte falsa} é executado.
%
\IfThisCommonLabelBase{scrextend}{\iffalse}{\csname iftrue\endcsname}%
\begin{Example}
  Suponha que você simplesmente queira mostrar se um texto será colocado em uma
  página par ou ímpar. Você pode conseguir isso
  usando{\phantomsection\xmpllabel{Ifthispageodd}}
\begin{lstcode}
  Esta página tem um número de página
  \Ifthispageodd{ímpar}{par}.
\end{lstcode}
  Isso resulta na saída
  \begin{quote}
    Esta página tem um número de página \Ifthispageodd{ímpar}{par}.
  \end{quote}
\end{Example}
\fi

Como o comando \Macro{Ifthispageodd} usa um mecanismo que é muito
similar a um rótulo e uma referência a ele, pelo menos duas execuções do {\LaTeX} são
necessárias após cada mudança no texto. Somente então a decisão estará
correta. Na primeira execução, uma heurística é usada para fazer a escolha inicial.

Em \autoref{sec:maincls-experts.addInfos},
\DescPageRef{maincls-experts.cmd.Ifthispageodd}, usuários avançados podem encontrar mais
informações sobre os problemas de detectar páginas esquerdas e direitas, ou números de
página pares e ímpares.%
\IfThisCommonLabelBase{scrextend}{%
  Um exemplo para \Macro{Ifthispageodd} é mostrado em
  \PageRefxmpl{maincls.Ifthispageodd} em \autoref{sec:maincls.oddOrEven}.%
}{}%
%
\EndIndexGroup
%
\EndIndexGroup

%%% Local Variables:
%%% mode: latex
%%% TeX-master: "scrguide-en.tex"
%%% coding: utf-8
%%% ispell-local-dictionary: "en_GB"
%%% eval: (flyspell-mode 1)
%%% End: 
