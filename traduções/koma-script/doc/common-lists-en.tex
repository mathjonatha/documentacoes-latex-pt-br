% ======================================================================
% common-lists-en.tex
% Copyright (c) Markus Kohm, 2001-2025
%
% This file is part of the LaTeX2e KOMA-Script bundle.
%
% This work may be distributed and/or modified under the conditions of
% the LaTeX Project Public License, version 1.3c of the license.
% The latest version of this license is in
%   http://www.latex-project.org/lppl.txt
% and version 1.3c or later is part of all distributions of LaTeX
% version 2005/12/01 or later and of this work.
%
% This work has the LPPL maintenance status "author-maintained".
%
% The Current Maintainer and author of this work is Markus Kohm.
%
% This work consists of all files listed in MANIFEST.md.
% ======================================================================
%
% Paragraphs that are common for several chapters of the KOMA-Script guide
% Maintained by Markus Kohm
%
% ======================================================================

\KOMAProvidesFile{common-lists-en.tex}
                 [$Date: 2025-06-04 18:44:23 +0200 (Mi, 04. Jun 2025) $
                  KOMA-Script guide (common paragraphs)]
\translator{Gernot Hassenpflug\and Markus Kohm\and Krickette Murabayashi\and
  Karl Hagen}

\section{Listas}
\seclabel{lists}%
\BeginIndexGroup
\BeginIndex{}{lists}%

\IfThisCommonFirstRun{}{%
  As informações em \autoref{sec:\ThisCommonFirstLabelBase.lists} aplicam-se
  igualmente a este capítulo. Portanto, se você já leu e compreendeu
  \autoref{sec:\ThisCommonFirstLabelBase.lists}, pode pular para
  \autoref{sec:\ThisCommonLabelBase.lists.next},
  \autopageref{sec:\ThisCommonLabelBase.lists.next}.%
  \IfThisCommonLabelBaseOneOf{scrextend,scrlttr2}{ %
    \IfThisCommonLabelBase{scrlttr2}{%
      O pacote \Package{scrletter}\OnlyAt{\Package{scrletter}} não
      define nenhum ambiente de lista por si mesmo, mas os deixa para a classe
      utilizada. Se esta não for uma classe \KOMAScript{}, ele carregará
      \hyperref[cha:scrextend]{\Package{scrextend}}\IndexPackage{scrextend}%
      \important{\hyperref[cha:scrextend]{\Package{scrextend}}}. %
    }{}%
    No entanto, o pacote \Package{scrextend} define apenas os
    ambientes \DescRef{\ThisCommonLabelBase.env.labeling},
    \DescRef{\ThisCommonLabelBase.env.addmargin} e
    \DescRef{\ThisCommonLabelBase.env.addmargin*}. Todos os outros ambientes
    de lista ficam sob responsabilidade da classe utilizada.%
  }{}%
}

\IfThisCommonLabelBase{scrextend}{}{%
  Tanto o {\LaTeX} quanto as classes padrão\textnote{\KOMAScript{} vs. classes
    padrão} oferecem diferentes ambientes para listas. Naturalmente, o {\KOMAScript}
  também oferece todos esses ambientes, embora ligeiramente modificados ou estendidos em
  alguns casos. Em geral, todas as listas\,---\,mesmo aquelas de diferentes
  tipos\,---\,podem ser aninhadas até quatro níveis de profundidade. Do ponto de vista
  tipográfico, qualquer coisa além disso não faria sentido, pois listas de mais de três níveis
  não podem ser facilmente compreendidas. Nesses casos, recomendo\textnote{Dica!} que você
  divida uma lista grande em várias listas menores.%
}

\IfThisCommonFirstRun{}{%
  Como listas são elementos padrão do \LaTeX{}, exemplos foram omitidos
  nesta seção. No entanto, você pode encontrar exemplos em
  \autoref{sec:\ThisCommonFirstLabelBase.lists},
  \autopageref{sec:\ThisCommonFirstLabelBase.lists} ou em qualquer tutorial de \LaTeX{}.%
}

\IfThisCommonLabelBase{scrextend}{\iffalse}{\csname iftrue\endcsname}
  \begin{Declaration}
    \begin{Environment}{itemize}
      \begin{Body}
        \Macro{item} \dots
        \BodyDots
      \end{Body}
    \end{Environment}
    \Macro{labelitemi}
    \Macro{labelitemii}
    \Macro{labelitemiii}
    \Macro{labelitemiv}
  \end{Declaration}%
  \IfThisCommonLabelBase{scrlttr2}{\OnlyAt{\Class{scrlttr2}}}{}%
  A forma mais simples de uma lista é a lista não ordenada\textnote{lista não ordenada},
  \Environment{itemize}. %
  \iffalse % Umbruckoptimierungstext
  Os usuários de um certo pacote de processamento de texto desprezado frequentemente se referem a
  esta forma de lista como \emph{marcadores}. Presumivelmente, eles não podem imaginar
  que, dependendo do nível, um símbolo diferente de um grande ponto pode ser usado
  para introduzir cada item. %
  \fi%
  Dependendo do nível, as classes \KOMAScript{} usam as seguintes marcas:
  ``{\labelitemi}'', ``{\labelitemii}'', ``{\labelitemiii}'' e
  ``{\labelitemiv}''. A definição desses símbolos é especificada nos
  comandos \Macro{labelitemi}, \Macro{labelitemii}, \Macro{labelitemiii} e
  \Macro{labelitemiv}, todos os quais você pode redefinir usando
  \Macro{renewcommand}.
  \BeginIndex{FontElement}{itemizelabel}\LabelFontElement{itemizelabel}%
  \BeginIndex{FontElement}{labelitemi}\LabelFontElement{labelitemi}%
  \BeginIndex{FontElement}{labelitemii}\LabelFontElement{labelitemii}%
  \BeginIndex{FontElement}{labelitemiii}\LabelFontElement{labelitemiii}%
  \BeginIndex{FontElement}{labelitemiv}\LabelFontElement{labelitemiv}%
  Com as classes \KOMAScript{}, as
  fontes\Index{fonte>estilo}\ChangedAt{v3.33}{\Class{scrbook}\and
    \Class{scrreprt}\and \Class{scrartcl}} usadas para formatar os símbolos dos
  diferentes níveis podem ser alteradas usando
  \DescRef{\ThisCommonLabelBase.cmd.setkomafont} e
  \DescRef{\ThisCommonLabelBase.cmd.addtokomafont} (veja
  \autoref{sec:\ThisCommonLabelBase.textmarkup},
  \DescPageRef{\ThisCommonLabelBase.cmd.setkomafont}) para os elementos
  \FontElement{labelitemi}\important{\FontElement{labelitemi}},
  \FontElement{labelitemii}\important{\FontElement{labelitemii}},
  \FontElement{labelitemiii}\important{\FontElement{labelitemiii}} e
  \FontElement{labelitemiv}\important{\FontElement{labelitemiv}}. Por padrão,
  todos usam a configuração de fonte para o elemento
  \FontElement{itemizelabel}\important{\FontElement{itemizelabel}}. Apenas o
  elemento \FontElement{labelitemii} usa adicionalmente \Macro{bfseries}. O
  padrão de \FontElement{itemizelabel} em si é \Macro{normalfont}.%
  \EndIndex{FontElement}{labelitemiv}%
  \EndIndex{FontElement}{labelitemiii}%
  \EndIndex{FontElement}{labelitemii}%
  \EndIndex{FontElement}{labelitemi}%
  \EndIndex{FontElement}{itemizelabel}
  Cada item é introduzido com \Macro{item}.%
  \IfThisCommonFirstRun{\iftrue}{\csname iffalse\endcsname}
    \begin{Example}
      \phantomsection\xmpllabel{env.itemize}%
      Você tem uma lista simples que está aninhada em vários níveis. Você escreve,
      por exemplo:
\begin{lstcode}
  \minisec{Veículos no jogo}
  \begin{itemize}
    \item aviões
    \begin{itemize}
      \item biplano
      \item aviões de transporte
      \begin{itemize}
        \item monomotor
        \begin{itemize}
          \item a jato
          \item a hélice
        \end{itemize}
        \item bimotor
        \begin{itemize}
	      \item a jato
		  \item a hélice
		\end{itemize}
      \end{itemize}
      \item helicópteros
    \end{itemize}
    \item motocicletas
    \item automóveis
    \begin{itemize}
      \item carros de corrida
      \item carros de passeio
      \item caminhões
    \end{itemize}
    \item bicicletas
  \end{itemize}
\end{lstcode}
      Como saída você obtém:
      \begin{ShowOutput}[\baselineskip]
        \minisec{Veículos no jogo}
        \begin{itemize}
        \item aviões
          \begin{itemize}
          \item biplanos
          \item aviões de transporte
            \begin{itemize}
            \item monomotor
              \begin{itemize}
              \item a jato
              \item a hélice
              \end{itemize}
            \item bimotor
              \begin{itemize}
              \item a jato
              \item a hélice
              \end{itemize}
            \end{itemize}
          \item helicópteros
          \end{itemize}
        \item motocicletas
          % \begin{itemize}
          % \item historically accurate
          % \item futuristic, not real
          % \end{itemize}
        \item automóveis
          \begin{itemize}
          \item carros de corrida
          \item carros de passeio
          \item caminhões
          \end{itemize}
        \item bicicletas
        \end{itemize}
      \end{ShowOutput}
    \end{Example}
  \fi
  %
  \EndIndexGroup


  \begin{Declaration}
    \begin{Environment}{enumerate}\labelsuffix[enumerate]
      \begin{Body}
        \Macro{item} \dots
        \BodyDots
      \end{Body}
    \end{Environment}
    \Macro{theenumi}%
    \Macro{theenumii}%
    \Macro{theenumiii}%
    \Macro{theenumiv}%
    \Macro{labelenumi}%
    \Macro{labelenumii}%
    \Macro{labelenumiii}%
    \Macro{labelenumiv}
  \end{Declaration}%
  \IfThisCommonLabelBase{scrlttr2}{\OnlyAt{\Class{scrlttr2}}}{}A lista
  numerada\textnote{lista numerada} também é muito comum e já é fornecida pelo
  núcleo do {\LaTeX}. A numeração\Index{numeração} difere de acordo com o
  nível, com números arábicos, letras minúsculas, algarismos romanos minúsculos e letras
  maiúsculas, respectivamente. O estilo de numeração é definido com os comandos
  \Macro{theenumi} até \Macro{theenumiv}. O formato de saída é determinado
  pelos comandos \Macro{labelenumi} até \Macro{labelenumiv}. Enquanto a letra
  minúscula do segundo nível é seguida por um parêntese direito, os valores de
  todos os outros níveis são seguidos por um ponto.
  \BeginIndex{FontElement}{enumeratelabel}\LabelFontElement{enumeratelabel}%
  \BeginIndex{FontElement}{labelenumi}\LabelFontElement{labelenumi}%
  \BeginIndex{FontElement}{labelenumii}\LabelFontElement{labelenumii}%
  \BeginIndex{FontElement}{labelenumiii}\LabelFontElement{labelenumiii}%
  \BeginIndex{FontElement}{labelenumiv}\LabelFontElement{labelenumiv}%
  Com as classes \KOMAScript{}, as
  fontes\Index{fonte>estilo}\IfThisCommonLabelBase{scrlttr2}{\ChangedAt{v3.44}{\Class{scrlttr2}\and
      \Package{scrletter}}}{\ChangedAt{v3.44}{\Class{scrbook}\and
      \Class{scrreprt}\and \Class{scrartcl}}} usadas para formatar os números
  respectivos rótulos para os diferentes níveis podem ser alteradas usando
  \DescRef{\ThisCommonLabelBase.cmd.setkomafont} e
  \DescRef{\ThisCommonLabelBase.cmd.addtokomafont} (veja
  \autoref{sec:\ThisCommonLabelBase.textmarkup},
  \DescPageRef{\ThisCommonLabelBase.cmd.setkomafont}) para os elementos
  \FontElement{labelenumi}\important{\FontElement{labelenumi}},
  \FontElement{labelenumii}\important{\FontElement{labelenumii}} e
  \FontElement{labelenumiii}\important{\FontElement{labelenumiii}} e
  \FontElement{labelenumiv}\important{\FontElement{labelenumiv}}. Por padrão,
  todos usam a configuração de fonte para o elemento
  \FontElement{enumeratelabel}\important{\FontElement{enumeratelabel}}. O
  padrão de \FontElement{enumeratelabel} em si é vazio, o que significa que não há
  mudança da fonte atual.%
  \EndIndex{FontElement}{labelenumiv}%
  \EndIndex{FontElement}{labelenumiii}%
  \EndIndex{FontElement}{labelenumii}%
  \EndIndex{FontElement}{labelenumi}%
  \EndIndex{FontElement}{enumeratelabel} Cada item é introduzido com
  \Macro{item}.%
  \IfThisCommonFirstRun{\iftrue}{\csname iffalse\endcsname}
    \begin{Example}
      \phantomsection\xmpllabel{env.enumerate}%
      Vamos encurtar o exemplo anterior e substituir o
      ambiente \DescRef{\ThisCommonLabelBase.env.itemize} por um
      ambiente \Environment{enumerate}:
      \begin{ShowOutput}[\baselineskip]
        \minisec{Veículos no jogo}
        \begin{enumerate}
        \item aviões
          \begin{enumerate}
          \item biplanos
          \item aviões de transporte
            \begin{enumerate}
            \item monomotor
              \begin{enumerate}
              \item a jato\label{xmp:maincls.jets}
              \item a hélice
              \end{enumerate}
            \item bimotor
            \end{enumerate}
          % \item helicópteros
          \end{enumerate}
          \item motocicletas
          \begin{enumerate}
             \item historicamente precisos
             \item futuristas, não reais
          \end{enumerate}
        %\item automóveis
        %  \begin{enumerate}
        %  \item carros de corrida
        %  \item carros particulares
        %  \item caminhões
        %  \end{enumerate}
        %\item bicicletas
        \end{enumerate}
      \end{ShowOutput}
      Dentro da lista, você pode definir rótulos da maneira normal com
      \Macro{label} e então referenciá-los com \Macro{ref}.
      No exemplo acima, um rótulo foi definido após os aviões de transporte
      monomotor a jato com
      ``\Macro{label}\PParameter{xmp:jets}''. O valor de \Macro{ref} é então
      ``\ref{xmp:maincls.jets}''.
    \end{Example}
  \fi%
  %
  \EndIndexGroup


  \begin{Declaration}
    \begin{Environment}{description}\labelsuffix[description]
      \begin{Body}
        \Macro{item}\OParameter{palavra-chave} \dots
        \BodyDots
      \end{Body}
    \end{Environment}
  \end{Declaration}%
  \IfThisCommonLabelBase{scrlttr2}{\OnlyAt{\Class{scrlttr2}}}{}Outra forma de
  lista é a lista de descrição\textnote{lista de descrição}. Ela serve principalmente para
  descrever itens ou palavras-chave individuais. O próprio item é especificado como um
  parâmetro opcional em \Macro{item}. %
  \BeginIndex{FontElement}{descriptionlabel}%
  \LabelFontElement{descriptionlabel}%
  A fonte\Index{fonte>estilo}\ChangedAt{v2.8p}{%
    \Class{scrbook}\and \Class{scrreprt}\and \Class{scrartcl}}%
  \ usada para formatar a palavra-chave pode ser alterada com os comandos
  \DescRef{\ThisCommonLabelBase.cmd.setkomafont} e
  \DescRef{\ThisCommonLabelBase.cmd.addtokomafont} (veja
  \autoref{sec:\ThisCommonLabelBase.textmarkup},
  \DescPageRef{\ThisCommonLabelBase.cmd.setkomafont}) para o
  elemento \FontElement{descriptionlabel}\important{\FontElement{descriptionlabel}}
  (veja \autoref{tab:\ThisCommonLabelBase.fontelements},
  \autopageref{tab:\ThisCommonLabelBase.fontelements}). O padrão é
  \IfThisCommonLabelBase{maincls}{%
    \ChangedAt{v3.39}{\Class{scrbook}\and \Class{scrreprt}\and
      \Class{scrartcl}}%
  }{%
    \IfThisCommonLabelBase{scrextend}{%
      \ChangedAt{v3.39}{\Package{screxend}}%
    }{%
      \IfThisCommonLabelBase{scrlttr2}{%
        \ChangedAt{v3.39}{\Class{scrlttr2}}%
      }{}%
    }%
  }%
  \DescRef{\LabelBase.cmd.maybesffamily}\IndexCmd{maybesffamily}\linebreak[1]%
  \Macro{bfseries}.%
  \IfThisCommonFirstRun{\iftrue}{\csname iffalse\endcsname}
    \begin{Example}
      \phantomsection\xmpllabel{env.description}%
      Você quer que as palavras-chave sejam impressas em negrito e na fonte normal em vez de
      negrito e sem serifa. Usando
\begin{lstcode}
  \setkomafont{descriptionlabel}{\normalfont\bfseries}
\end{lstcode}
      você redefine a fonte adequadamente.

      Um exemplo de lista de descrição é a saída dos estilos de página
      listados em \autoref{sec:maincls.pagestyle}. O código (abreviado) é:
\begin{lstcode}
  \begin{description}
    \item[empty] é o estilo de página sem cabeçalho ou rodapé.
    \item[plain] é o estilo de página sem títulos correntes.
    \item[headings] é o estilo de página com títulos correntes.
    \item[myheadings] é o estilo de página para títulos manuais.
  \end{description}
\end{lstcode}
      Isso produz:
      \begin{ShowOutput}
        \begin{description}
        \item[empty] é o estilo de página sem cabeçalho ou rodapé.
        \item[plain] é o estilo de página sem títulos correntes.
        \item[headings] é o estilo de página com títulos correntes.
        \item[myheadings] é o estilo de página para títulos manuais.
        \end{description}
      \end{ShowOutput}
    \end{Example}
  \fi%
  %
  \EndIndexGroup%
\fi

\begin{Declaration}
  \begin{Environment}{labeling}\OParameter{delimitador}
    \Parameter{padrão mais largo}
    \labelsuffix[labeling]
    \begin{Body}
      \Macro{item}\OParameter{palavra-chave}\dots
      \BodyDots
    \end{Body}
  \end{Environment}
\end{Declaration}%
Outra forma de lista de descrição\textnote{lista de descrição alternativa} está
disponível apenas nas classes {\KOMAScript}%
\IfThisCommonLabelBase{scrextend}{ e no \Package{scrextend} }{%
  \IfThisCommonLabelBase{scrlttr2}{ e no
    \hyperref[cha:scrextend]{\Package{scrextend}}}{}%
}%
: o ambiente \Environment{labeling}. Ao contrário do
\IfThisCommonLabelBase{scrextend}{%
  \DescRef{\ThisCommonFirstLabelBase.env.description}%
}{%
  \DescRef{\ThisCommonLabelBase.env.description} descrito acima%
}, você pode especificar um padrão para \Environment{labeling} cujo comprimento
determina a indentação de todos os itens. Além disso, você pode colocar um
\PName{delimitador} opcional entre o item e sua descrição. %
\BeginIndexGroup
\BeginIndex{FontElement}{labelinglabel}\LabelFontElement{labelinglabel}%
\BeginIndex{FontElement}{labelingseparator}%
\LabelFontElement{labelingseparator}%
A fonte\Index{fonte>estilo}%
\IfThisCommonLabelBase{maincls}{%
  \ChangedAt{v3.02}{\Class{scrbook}\and \Class{scrreprt}\and
    \Class{scrartcl}}%
}{%
  \IfThisCommonLabelBase{scrlttr2}{%
    \ChangedAt{v3.02}{\Class{scrlttr2}}%
  }{%
    \IfThisCommonLabelBase{scrextend}{%
      \ChangedAt{v3.02}{\Package{scrextend}}%
    }{\InternalCommonFileUsageError}%
  }%
} usada para formatar o item e o separador pode ser alterada com os comandos
\DescRef{\ThisCommonLabelBase.cmd.setkomafont} e
\DescRef{\ThisCommonLabelBase.cmd.addtokomafont} (veja
\autoref{sec:\ThisCommonLabelBase.textmarkup},
\DescPageRef{\ThisCommonLabelBase.cmd.setkomafont}) para os elementos
\FontElement{labelinglabel} e \FontElement{labelingseparator} (veja
\autoref{tab:\ThisCommonLabelBase.fontelements},
\autopageref{tab:\ThisCommonLabelBase.fontelements}).
\IfThisCommonFirstRun{\iftrue}{\par\csname iffalse\endcsname}
  \begin{Example}
    \phantomsection\xmpllabel{env.labeling}%
    \IfThisCommonLabelBase{scrextend}{%
      Um pequeno exemplo de uma lista como esta pode ser escrito da seguinte forma:%
    }{%
      Alterando ligeiramente o exemplo do
      ambiente \DescRef{\ThisCommonLabelBase.env.description}, poderíamos
      escrever o seguinte:%
    }%
\begin{lstcode}
  \setkomafont{labelinglabel}{\ttfamily}
  \setkomafont{labelingseparator}{\normalfont}
  \begin{labeling}[~--]{myheadings}
    \item[empty]
      Estilo de página sem cabeçalho ou rodapé
    \item[plain]
      Estilo de página para início de capítulos sem títulos
    \item[headings]
      Estilo de página para títulos correntes
    \item[myheadings]
      Estilo de página para títulos manuais
  \end{labeling}
\end{lstcode}
    O resultado é este:
    \begin{ShowOutput}
      \setkomafont{labelinglabel}{\ttfamily}
      \setkomafont{labelingseparator}{\normalfont}
      \begin{labeling}[~--]{myheadings}
      \item[empty]
        Estilo de página sem cabeçalho ou rodapé
      \item[plain]
        Estilo de página para início de capítulos sem títulos
      \item[headings]
        Estilo de página para títulos correntes
      \item[myheadings]
        Estilo de página para títulos manuais
      \end{labeling}
    \end{ShowOutput}
    Como este exemplo mostra, você pode definir um comando de mudança de fonte da maneira
    usual. Mas se você não quer que a fonte do separador seja alterada da
    mesma forma que a fonte do rótulo, você tem que definir a fonte do
    separador também.
  \end{Example}
\fi%
\EndIndexGroup
Originalmente, este ambiente foi implementado para coisas como ``Premissa,
Evidência, Prova'', ou ``Dado, Encontrar, Solução'' que são frequentemente usadas em
apostilas de aulas. Atualmente, no entanto, o ambiente tem aplicações muito diferentes.
Por exemplo, o ambiente para exemplos neste guia foi
definido com o ambiente \Environment{labeling}.%
%
\EndIndexGroup


\IfThisCommonLabelBase{scrextend}{\iffalse}{\csname iftrue\endcsname}
  \begin{Declaration}
    \begin{Environment}{verse}\end{Environment}
  \end{Declaration}%
  \IfThisCommonLabelBase{scrlttr2}{\OnlyAt{\Class{scrlttr2}}}{} O
  ambiente \Environment{verse}\textnote{verso} normalmente não é percebido
  como um ambiente de lista porque você não trabalha com comandos \Macro{item}.
  Em vez disso, quebras de linha fixas são usadas dentro do ambiente \Environment{flushleft}.
  Internamente, no entanto, tanto as classes padrão quanto o
  {\KOMAScript} implementam-no como um ambiente de lista.

  Em geral, o ambiente \Environment{verse} é usado para
  poesia\Index{poesia}. As linhas são indentadas tanto à esquerda quanto à direita. Linhas
  individuais de verso são terminadas por uma quebra de linha fixa: \verb|\\|. Os versos são compostos como
  parágrafos, separados por uma linha vazia. Frequentemente também
  \Macro{medskip}\IndexCmd{medskip} ou \Macro{bigskip}\IndexCmd{bigskip} é
  usado em vez disso. Para evitar uma quebra de página no final de uma linha de verso você pode,
  como de costume, inserir \verb|\\*| em vez de \verb|\\|.
  \IfThisCommonFirstRun{\iftrue}{\csname iffalse\endcsname}
    \begin{Example}
      \phantomsection\xmpllabel{env.verse}%
      \iffalse
        Como exemplo, as primeiras linhas de ``Chapeuzinho Vermelho e o
        Lobo'' de Roald Dahl:
\begin{lstcode}
  \begin{verse}
    As soon as Wolf began to feel\\*
    that he would like a decent meal,\\*
    He went and knocked on Grandma's door.\\*
    When Grandma opened it, she saw\\*
    The sharp white teeth, the horrid grin,\\*
    And Wolfie said, `May I come in?'
  \end{verse}
\end{lstcode}
        O resultado é o seguinte:
        \begin{ShowOutput}
          \begin{verse}
            As soon as Wolf began to feel\\*
            That he would like a decent meal,\\*
            He went and knocked on Grandma's door.\\*
            When Grandma opened it, she saw\\*
            The sharp white teeth, the horrid grin,\\*
            And Wolfie said, `May I come in?'
          \end{verse}
        \end{ShowOutput}
      \else
        Como exemplo, o soneto de Emma Lazarus do pedestal da Liberdade
        Iluminando o Mundo\footnote{As linhas do poema de Roald Dahl
          ``Chapeuzinho Vermelho e o Lobo'', que foram usadas em versões
          anteriores, foram substituídas, porque nestes tempos certos
          políticos ao redor do mundo realmente parecem precisar de ``O Novo
          Colosso'' como lembrete urgente.}:
\begin{lstcode}
  \begin{verse}
    Not like the brazen giant of Greek fame\\*
    With conquering limbs astride from land to land\\*
    Here at our sea-washed, sunset gates shall stand\\*
    A mighty woman with a torch, whose flame\\*
    Is the imprisoned lightning, and her name\\*
    Mother of Exiles. From her beacon-hand\\*
    Glows world-wide welcome; her mild eyes command\\*
    The air-bridged harbor that twin cities frame.\\*
    ``Keep, ancient lands, your storied pomp!'' cries she\\*
    With silent lips. ``Give me your tired, your poor,\\*
    Your huddled masses yearning to breathe free,\\*
    The wretched refuse of your teeming shore.\\*
    Send these, the homeless, tempest-tossed to me:\\*
    I lift my lamp beside the golden door.''
  \end{verse}
\end{lstcode}
        O resultado é o seguinte:
        \begin{ShowOutput}
          \begin{verse}
            Not like the brazen giant of Greek fame\\*
            With conquering limbs astride from land to land\\*
            Here at our sea-washed, sunset gates shall stand\\*
            A mighty woman with a torch, whose flame\\*
            Is the imprisoned lightning, and her name\\*
            Mother of Exiles. From her beacon-hand\\*
            Glows world-wide welcome; her mild eyes command\\*
            The air-bridged harbor that twin cities frame.\\*
            ``Keep, ancient lands, your storied pomp!'' cries she\\*
            With silent lips. ``Give me your tired, your poor,\\*
            Your huddled masses yearning to breathe free,\\*
            The wretched refuse of your teeming shore.\\*
            Send these, the homeless, tempest-tossed to me:\\*
            I lift my lamp beside the golden door.''
          \end{verse}
        \end{ShowOutput}
      \fi
      No entanto, se você tiver linhas de verso muito longas onde uma quebra de linha
      ocorre dentro de uma linha de verso:
\begin{lstcode}
  \begin{verse}
    Tanto o filósofo quanto o proprietário
    sempre têm algo para consertar.\\*
    \bigskip
    Não confie em um homem, meu filho, que lhe diz
    que nunca mentiu.
  \end{verse}
\end{lstcode}
      \begin{ShowOutput}
        \begin{verse}
    	  Tanto o filósofo quanto o proprietário sempre têm algo para
          consertar.\\
          \bigskip Não confie em um homem, meu filho, que lhe diz que nunca
          mentiu.
        \end{verse}
      \end{ShowOutput}
      neste caso \verb|\\*| não pode impedir que uma quebra de página ocorra dentro de um
      verso em tal quebra de linha. Para evitar tal quebra de página, uma mudança de
      \Macro{interlinepenalty}\IndexCmd{interlinepenalty} teria que ser
      inserida no início do ambiente:
\begin{lstcode}
  \begin{verse}\interlinepenalty 10000
    Tanto o filósofo quanto o proprietário
    sempre têm algo para consertar.\\
    \bigskip
    Não confie em um homem, meu filho, que lhe diz
    que nunca mentiu.
  \end{verse}
\end{lstcode}
      \iftrue% Umbruchkorrekturtext
        Aqui estão dois ditados que devem sempre ser considerados quando confrontados
        com perguntas aparentemente estranhas sobre {\LaTeX} ou suas explicações
        acompanhantes:
\begin{lstcode}
  \begin{verse}
    A little learning is a dangerous thing.\\*
    Drink deep, or taste not the Pierian Spring;\\
    \bigskip
    Our judgments, like our watches, none\\*
    go just alike, yet each believes his own.
  \end{verse}
\end{lstcode}
        \begin{ShowOutput}
          \iffree{}{\vskip-.8\baselineskip}% Umbruchkorrektur
          \begin{verse}
            A little learning is a dangerous thing.\\*
            Drink deep, or taste not the Pierian Spring;\\
            \bigskip
            Our judgments, like our watches, none\\*
            go just alike, yet each believes his own.
          \end{verse}
        \end{ShowOutput}
      \fi
      Aliás, \Macro{bigskip} foi usado nestes exemplos para separar dois
      ditados.
    \end{Example}
  \fi
  %
  \EndIndexGroup

  \iffalse% Umbruchkorrekturvarianten
    \begin{Declaration}
  	  \begin{Environment}{quote}\end{Environment}
  	\end{Declaration}%
  	\IfThisCommonLabelBase{scrlttr2}{\OnlyAt{\Class{scrlttr2}}}{}Este é
  	internamente também um ambiente de lista\textnote{citação em bloco com espaçamento
  	de parágrafo} e pode ser encontrado tanto nas classes padrão quanto no
  	{\KOMAScript}. O conteúdo do ambiente é composto totalmente justificado.
    O ambiente é frequentemente usado para formatar citações\Index{citações} mais longas.
    Parágrafos dentro do ambiente são distinguidos com espaço vertical.%
    \EndIndexGroup

    \begin{Declaration}
      \begin{Environment}{quotation}\end{Environment}
    \end{Declaration}%
    \IfThisCommonLabelBase{scrlttr2}{\OnlyAt{\Class{scrlttr2}}}{}Este
    ambiente\textnote{citação em bloco com indentação de parágrafo} é comparável ao
    \DescRef{\ThisCommonLabelBase.env.quote}. Enquanto
    os parágrafos do \DescRef{\ThisCommonLabelBase.env.quote} são indicados por
    espaçamento vertical, \Environment{quotation} indenta a primeira linha de cada
    parágrafo horizontalmente. Isso também se aplica ao primeiro parágrafo de um
    ambiente \Environment{quotation}. Se você quiser evitar a
    indentação lá, você deve precedê-lo com o
    comando \Macro{noindent}\IndexCmd{noindent}.%
  \else
    \begin{Declaration}
      \begin{Environment}{quote}\end{Environment}
      \begin{Environment}{quotation}\end{Environment}
    \end{Declaration}%
    \IfThisCommonLabelBase{scrlttr2}{\OnlyAt{\Class{scrlttr2}}}{} Estes dois
    ambientes\textnote{citações em bloco} também são definidos internamente como ambientes de
    lista e podem ser encontrados tanto nas classes padrão quanto nas classes {\KOMAScript}.
    Ambos os ambientes usam texto justificado que é indentado tanto à
    esquerda quanto à direita. Frequentemente eles são usados para separar
    citações\Index{citações} mais longas do texto principal. A diferença entre
    os dois está na maneira em que os parágrafos são compostos. Enquanto
    os parágrafos do \Environment{quote} são distinguidos por espaço vertical, nos
    parágrafos do \Environment{quotation}, a primeira linha é indentada. Isso também
    se aplica à primeira linha de um ambiente \Environment{quotation}% Umbruchkorrektur
    \IfThisCommonLabelBase{maincls}{%
      , a menos que seja precedido por \Macro{noindent}\IndexCmd{noindent}.%
    }{%
      \IfThisCommonLabelBase{scrlttr2}{%
        . Se você quiser evitar a indentação lá, você deve precedê-la
        com o comando \Macro{noindent}\IndexCmd{noindent}.%
      }{\InternalCommonFileUsageError}%
    }%
  \fi % Umbruchkorrekturvarianten
  \IfThisCommonFirstRun{\iftrue}{\csname iffalse\endcsname}
    \begin{Example}
      \phantomsection\xmpllabel{env.quote}%
      Você quer destacar uma pequena anedota. Você escreve o seguinte
      ambiente \Environment{quotation} para isso:%
\begin{lstcode}
  Um pequeno exemplo de uma curta anedota:
  \begin{quotation}
    O velho ano estava ficando marrom; o Vento Oeste estava
    chamando;

    Tom pegou a folha de faia caindo na floresta.
    ``Peguei o dia feliz que as brisas me sopraram!
    Por que esperar até o próximo ano? Vou pegá-lo quando me agradar.
    Assim vou consertar meu barco e viajar conforme o acaso
    oeste abaixo do riacho de salgueiros, seguindo minhas fantasias!''

    Passarinho sentou no galho. ``Opa, Tom! Eu te ouço.
    Tenho um palpite, tenho um palpite de onde suas fantasias te levam.
    Devo ir, devo ir, levar-lhe uma mensagem para te encontrar?''
  \end{quotation}
\end{lstcode}
      O resultado é:
      \begin{ShowOutput}
        Um pequeno exemplo de uma curta anedota:
        \begin{quotation}
          O velho ano estava ficando marrom; o Vento Oeste estava
          chamando;

          Tom pegou a folha de faia caindo na floresta.
          ``Peguei o dia feliz que as brisas me sopraram!
          Por que esperar até o próximo ano? Vou pegá-lo quando me agradar.
          Assim vou consertar meu barco e viajar conforme o acaso
          oeste abaixo do riacho de salgueiros, seguindo minhas fantasias!''

          Passarinho sentou no galho. ``Opa, Tom! Eu te ouço.
          Tenho um palpite, tenho um palpite de onde suas fantasias te levam.
          Devo ir, devo ir, levar-lhe uma mensagem para te encontrar?''
        \end{quotation}
      \end{ShowOutput}
      %
      Usando um ambiente \Environment{quote} em vez disso você obtém:
      %
      \begin{ShowOutput}
        Um pequeno exemplo de uma curta anedota:
        \begin{quote}\setlength{\parskip}{4pt plus 2pt minus 2pt}
          O velho ano estava ficando marrom; o Vento Oeste estava
          chamando;

          Tom pegou a folha de faia caindo na floresta.
          ``Peguei o dia feliz que as brisas me sopraram!
          Por que esperar até o próximo ano? Vou pegá-lo quando me agradar.
          Assim vou consertar meu barco e viajar conforme o acaso
          oeste abaixo do riacho de salgueiros, seguindo minhas fantasias!''

          Passarinho sentou no galho. ``Opa, Tom! Eu te ouço.
          Tenho um palpite, tenho um palpite de onde suas fantasias te levam.
          Devo ir, devo ir, levar-lhe uma mensagem para te encontrar?''
        \end{quote}
      \end{ShowOutput}
      %
    \end{Example}
  \fi
  %
  \EndIndexGroup
\fi

\begin{Declaration}
  \begin{Environment}{addmargin}
                     \OParameter{indentação esquerda}\Parameter{indentação}
  \end{Environment}
  \begin{Environment}{addmargin*}
                     \OParameter{indentação interna}\Parameter{indentação}
  \end{Environment}
\end{Declaration}
Como \IfThisCommonLabelBase{scrextend}{%
  \DescRef{\ThisCommonFirstLabelBase.env.quote} e
  \DescRef{\ThisCommonFirstLabelBase.env.quotation}, que estão disponíveis nas
  classes padrão e nas classes \KOMAScript{}}{%
  \DescRef{\ThisCommonLabelBase.env.quote} e
  \DescRef{\ThisCommonLabelBase.env.quotation}%
}, o ambiente \Environment{addmargin} altera a margem\Index{margem}.
No entanto, ao contrário dos dois primeiros ambientes, \Environment{addmargin} permite que o
usuário altere a largura da indentação. Além dessa mudança, este
ambiente não altera a indentação da primeira linha nem o
espaçamento vertical entre parágrafos.

Se apenas o argumento obrigatório \PName{indentação} for fornecido, tanto a margem
esquerda quanto a direita são expandidas por este valor. Se o argumento opcional
\PName{indentação esquerda} for fornecido também, então o valor \PName{indentação
  esquerda} é usado para a margem esquerda em vez de \PName{indentação}.

A variante com asterisco \Environment{addmargin*}%
\important{\Environment{addmargin*}} difere da versão normal apenas no
modo de duas faces. Além disso, a diferença só ocorre se o argumento
opcional \PName{indentação interna} for usado. Neste caso, o valor de
\PName{indentação interna} é adicionado à indentação interna normal. Para
páginas da direita esta é a margem esquerda; para páginas da esquerda, a margem
direita. Então o valor de \PName{indentação} determina a largura da
margem oposta.

Ambas as versões deste ambiente permitem valores negativos para todos os parâmetros.
\IfThisCommonLabelBase{scrextend}{%
  O ambiente então se estende para a margem adequadamente.%
}{%
  Isso pode ser feito de modo que o ambiente se estenda para a margem.%
}%
\IfThisCommonFirstRun{\iftrue}{\csname iffalse\endcsname}
  \begin{Example}
    \phantomsection\xmpllabel{env.addmargin}%
\begin{lstcode}
  \newenvironment{SourceCodeFrame}{%
    \begin{addmargin*}[1em]{-1em}%
      \begin{minipage}{\linewidth}%
        \rule{\linewidth}{2pt}%
  }{%
      \rule[.25\baselineskip]{\linewidth}{2pt}%
      \end{minipage}%
    \end{addmargin*}%
  }
\end{lstcode}
    Se você agora colocar seu código fonte em tal ambiente, ele aparecerá
    como:
    \begin{ShowOutput}
      \newenvironment{SourceCodeFrame}{%
        \begin{addmargin*}[1em]{-1em}%
          \begin{minipage}{\linewidth}%
            \rule{\linewidth}{2pt}%
          }{%
            \rule[.25\baselineskip]{\linewidth}{2pt}%
          \end{minipage}%
        \end{addmargin*}%
      }
      Você define o seguinte ambiente:
      \begin{SourceCodeFrame}
\begin{lstcode}
\newenvironment{\SourceCodeFrame}{%
  \begin{addmargin*}[1em]{-1em}%
    \begin{minipage}{\linewidth}%
      \rule{\linewidth}{2pt}%
}{%
    \rule[.25\baselineskip]{\linewidth}{2pt}%
    \end{minipage}%
  \end{addmargin*}%
}
\end{lstcode}
      \end{SourceCodeFrame}
      Isso pode ser viável ou não. De qualquer forma, mostra o uso deste
      ambiente.
    \end{ShowOutput}
    O argumento opcional do ambiente \Environment{addmargin*}
    garante que a margem interna seja estendida em 1\Unit{em}. Por sua vez,
    a margem externa é diminuída em 1\Unit{em}. O resultado é um deslocamento
    de 1\Unit{em} para fora. Em vez de \PValue{1em}, você pode
    usar um comprimento, por exemplo, \PValue{2\Macro{parindent}}.
  \end{Example}
\fi%

Se uma página vai estar no lado esquerdo ou direito do livro não pode ser
determinado de forma confiável na primeira execução do {\LaTeX}. Para detalhes, consulte
a explicação dos comandos
\DescRef{\ThisCommonLabelBase.cmd.Ifthispageodd}
(\autoref{sec:\ThisCommonLabelBase.oddOrEven},
\DescPageRef{\ThisCommonLabelBase.cmd.Ifthispageodd}) e
\iffree{\Macro{ifthispagewasodd}}{%
  \DescRef{maincls-experts.cmd.ifthispagewasodd}}
(\autoref{sec:maincls-experts.addInfos}\iffree{}{,
\DescPageRef{maincls-experts.cmd.Ifthispageodd}}).
\IfThisCommonLabelBase{scrlttr2}{}{%
\begin{Explain}
  A interação de ambientes como listas e parágrafos gera
  perguntas frequentes. Portanto, você pode encontrar explicações adicionais na
  descrição da opção \Option{parskip} em
  \autoref{sec:maincls-experts.addInfos}\iffree{}{,
  \DescPageRef{maincls-experts.option.parskip}. Também na seção de especialistas, em
  \autoref{sec:maincls-experts.addInfos},
  \DescPageRef{maincls-experts.env.addmargin*}, você pode encontrar informações
  adicionais sobre quebras de página dentro de \Environment{addmargin*}}.%
\end{Explain}}%
%
\EndIndexGroup
%
\EndIndexGroup

%%% Local Variables:
%%% mode: latex
%%% TeX-master: "scrguide-en.tex"
%%% coding: utf-8
%%% ispell-local-dictionary: "en_GB"
%%% eval: (flyspell-mode 1)
%%% End:
