% ======================================================================
% scraddr-en.tex
% Copyright (c) Markus Kohm, 2001-2022
%
% This file is part of the LaTeX2e KOMA-Script bundle.
%
% This work may be distributed and/or modified under the conditions of
% the LaTeX Project Public License, version 1.3c of the license.
% The latest version of this license is in
%   http://www.latex-project.org/lppl.txt
% and version 1.3c or later is part of all distributions of LaTeX 
% version 2005/12/01 or later and of this work.
%
% This work has the LPPL maintenance status "author-maintained".
%
% The Current Maintainer and author of this work is Markus Kohm.
%
% This work consists of all files listed in MANIFEST.md.
% ======================================================================
%
% Capítulo sobre scraddr do guia KOMA-Script
% Mantido por Jens-Uwe Morawski (com ajuda de Markus Kohm)
%
% ======================================================================

\KOMAProvidesFile{scraddr-en.tex}
                 [$Date: 2022-06-05 12:40:11 +0200 (So, 05. Jun 2022) $
                  KOMA-Script guide (chapter: scraddr)]
\translator{Jens-Uwe Morawski\and Gernot Hassenpflug\and Markus Kohm\and Karl
  Hagen}

\chapter{Acessando Arquivos de Endereços com \Package{scraddr}}%
\labelbase{scraddr}%
\BeginIndexGroup
\BeginIndex{Package}{scraddr}

O pacote \Package{scraddr} é uma pequena extensão da classe de carta e do
pacote de carta do \KOMAScript{}. Seu objetivo é tornar o acesso aos dados em
arquivos de endereços mais fácil e flexível.

\section{Visão Geral}\seclabel{overview}
Basicamente, o pacote fornece um novo mecanismo de carregamento para arquivos de
endereços consistindo de entradas nos formatos \DescRef{\LabelBase.cmd.adrentry}
e o mais novo \DescRef{\LabelBase.cmd.addrentry}, conforme descrito em
\autoref{cha:scrlttr2} a partir de \DescPageRef{scrlttr2.cmd.adrentry}.

\begin{Declaration}
\Macro{InputAddressFile}\Parameter{nome do arquivo}
\end{Declaration}%
O comando \Macro{InputAddressFile} é o comando principal do \Package{scraddr}.
Ele lê o conteúdo do arquivo de endereços\Index{endereço>arquivo} fornecido como
seu parâmetro. Se o arquivo não for encontrado, uma mensagem de erro é emitida.

Para cada entrada neste arquivo de endereços, o comando gera um conjunto de
macros para acessar os dados. Para arquivos de endereços grandes, isto exigirá
muita memória do \TeX{}.
%
\EndIndexGroup

\begin{Declaration}%
  \Macro{adrentry}\Parameter{Lastname}\Parameter{Firstname}%
  \Parameter{Address}\Parameter{Phone}\Parameter{F1}\Parameter{F2}%
  \Parameter{Comment}\Parameter{Key}%
  %
  \Macro{addrentry}\Parameter{Lastname}\Parameter{Firstname}%
  \Parameter{Address}\Parameter{Phone}\Parameter{F1}\Parameter{F2}%
  \Parameter{F3}\Parameter{F4}\Parameter{Key}%
  %
  \Macro{adrchar}\Parameter{initial}%
  \Macro{addrchar}\Parameter{initial}%
\end{Declaration}%
A estrutura das entradas de endereço no arquivo de endereços foi discutida em
detalhes em \autoref{sec:scrlttr2.addressFile}, começando em
\DescPageRef{scrlttr2.cmd.adrentry}. A subdivisão do arquivo de endereços com
a ajuda de \Macro{adrchar} ou \Macro{addrchar}, também discutida lá, não tem
significado para \Package{scraddr} e é simplesmente ignorada pelo pacote.%
\EndIndexGroup

\begin{Declaration}
  \Macro{Name}\Parameter{key}%
  \Macro{FirstName}\Parameter{key}%
  \Macro{LastName}\Parameter{key}%
  \Macro{Address}\Parameter{key}%
  \Macro{Telephone}\Parameter{key}%
  \Macro{FreeI}\Parameter{key}%
  \Macro{FreeII}\Parameter{key}%
  \Macro{Comment}\Parameter{key}%
  \Macro{FreeIII}\Parameter{key}%
  \Macro{FreeIV}\Parameter{key}
\end{Declaration}%
Estes comandos fornecem acesso aos dados do seu arquivo de endereços. O último
parâmetro, ou seja, o parâmetro 8 para a entrada \DescRef{\LabelBase.cmd.adrentry}
e o parâmetro 9 para a entrada \DescRef{\LabelBase.cmd.addrentry}, é o
identificador de uma entrada, assim a \PName{chave} tem que ser única e não vazia.
Para garantir o funcionamento seguro, você deve usar apenas letras ASCII na
\PName{chave}.

Além disso, se o arquivo contiver mais de uma entrada com o mesmo
nome \PName{chave}, a última ocorrência será usada.%
%
\EndIndexGroup


\section{Uso}\seclabel{usage}
\BeginIndexGroup
\BeginIndex[indexother]{Cmd}{addrentry}%
\BeginIndex[indexother]{Cmd}{adrentry}%
Para usar o pacote, precisamos de um arquivo de endereços válido. Por exemplo, o arquivo
\File{lotr.adr} contém as seguintes entradas:
\begin{lstcode}
  \addrentry{Baggins}{Frodo}%
            {The Hill\\ Bag End/Hobbiton in the Shire}{}%
            {Bilbo Baggins}{pipe-weed}%
            {the Ring-bearer}{Bilbo's heir}{FRODO}
  \adrentry{Gamgee}{Samwise}%
            {3 Bagshot Row\\Hobbiton in the Shire}{}%
            {Rosie Cotton}{taters}%
            {the Ring-bearer's faithful servant}{SAM}
  \adrentry{Bombadil}{Tom}%
            {The Old Forest}{}%
            {Goldberry}{trill queer songs}%
            {The Master of Wood, Water and Hill}{TOM}
\end{lstcode}

O quarto parâmetro, o número de telefone, foi deixado em branco, pois não há
telefones na Terra-Média. E como você pode ver, campos em branco são possíveis.
Por outro lado, você não pode simplesmente omitir um argumento completamente.

\BeginIndexGroup
\BeginIndex[indexother]{Cmd}{InputAddressFile}
Com o comando \Macro{InputAddressFile} descrito acima, lemos o arquivo de
endereços para o nosso documento de carta:
\begin{lstcode}
  \InputAddressFile{lotr}
\end{lstcode}
\EndIndexGroup

\BeginIndexGroup
\BeginIndex[indexother]{Cmd}{Name}%
\BeginIndex[indexother]{Cmd}{Address}%
\BeginIndex[indexother]{Cmd}{FirstName}%
\BeginIndex[indexother]{Cmd}{LastName}%
\BeginIndex[indexother]{Cmd}{FreeI}%
\BeginIndex[indexother]{Cmd}{FreeII}%
\BeginIndex[indexother]{Cmd}{FreeIII}%
\BeginIndex[indexother]{Cmd}{FreeIV}%
\BeginIndex[indexother]{Cmd}{Comment}%
Com a ajuda dos comandos introduzidos neste capítulo podemos agora escrever uma
carta para o velho \textsc{Tom Bombadil}, na qual pedimos a ele se ele consegue
lembrar de dois companheiros de tempos idos.
\begin{lstcode}
  \begin{letter}{\Name{TOM}\\\Address{TOM}}
     \opening{Dear \FirstName{TOM} \LastName{TOM},}
     
     Or \FreeIII{TOM}, as your beloved \FreeI{TOM} calls you. Do
     you still remember Mr \LastName{FRODO}, or more precisely
     \Name{FRODO}, since there was also Mr \FreeI{FRODO}. He was
     \Comment{FRODO} in the Third Age and \FreeIV{FRODO}. \Name{SAM},
     \Comment{SAM}, accompanied him.
      
      Their passions were very worldly. \FirstName{FRODO} enjoyed
      smoking \FreeII{FRODO}. His companion appreciated a good meal
      with \FreeII{SAM}.

      Do you remember? Certainly Mithrandir has told you much
      about their deeds and adventures.
    \closing{``O spring-time and summer-time
                and spring again after!\\
               O wind on the waterfall,
                and the leaves' laughter!''}
  \end{letter}
\end{lstcode}
Você também pode produzir a combinação de \Macro{FirstName}\Parameter{chave} e
\Macro{LastName}\Parameter{chave} usada em \DescRef{scrlttr2.cmd.opening} de
esta carta com \Macro{Name}\PParameter{chave}.

Você pode usar o quinto e sexto parâmetros de
\DescRef{\LabelBase.cmd.adrentry} ou \DescRef{\LabelBase.cmd.adrentry} para qualquer
propósito que desejar. Você pode acessá-los com os comandos \Macro{FreeI} e
\Macro{FreeII}. Neste exemplo, o quinto parâmetro contém o
nome da pessoa mais importante na vida da pessoa na entrada. O
sexto contém a coisa favorita da pessoa. O sétimo parâmetro é um
comentário ou em geral também um parâmetro livre. Você pode acessá-lo com os
comandos \Macro{Comment} ou \Macro{FreeIII}. \Macro{FreeIV} é válido apenas para
entradas \DescRef{\LabelBase.cmd.addrentry}. Para
entradas \DescRef{\LabelBase.cmd.adrentry}, resulta em um erro. Você pode
encontrar mais detalhes na próxima seção.
%
\EndIndexGroup
\EndIndexGroup


\section{Opções de Aviso do Pacote}

Como mencionado acima, você não pode usar o comando \Macro{FreeIV} com
entradas \DescRef{\LabelBase.cmd.adrentry}. No entanto, você pode configurar
como \Package{scraddr} reage em tal situação por opções do pacote.
Observe\textnote{Atenção!} que este pacote não suporta a interface de
opções estendida com \DescRef{maincls.cmd.KOMAoptions} e
\DescRef{maincls.cmd.KOMAoption}. Você deve especificar as opções
como opções globais em \DescRef{maincls.cmd.documentclass} ou como opções locais
em \DescRef{maincls.cmd.usepackage}


\begin{Declaration}
  \Option{adrFreeIVempty}%
  \Option{adrFreeIVshow}%
  \Option{adrFreeIVwarn}%
  \Option{adrFreeIVstop}%
\end{Declaration}%
Estas quatro opções permitem que você escolha entre quatro reações diferentes,
variando de \emph{ignorar} a \emph{abortar}, se \Macro{FreeIV} é usado
dentro de uma entrada \DescRef{\LabelBase.cmd.adrentry}.

\begin{labeling}[~--]{\Option{adrFreeIVempty}}
\item[\Option{adrFreeIVempty}]
        o comando \Macro{FreeIV} será ignorado
\item[\Option{adrFreeIVshow}]
        o aviso ``(entrada FreeIV não definida em \PName{chave})'' será
        escrito no texto
\item[\Option{adrFreeIVwarn}]
        um aviso é escrito no arquivo de log
\item[\Option{adrFreeIVstop}]
        a execução do \LaTeX{} será abortada com uma mensagem de erro
\end{labeling}
A configuração padrão é \Option{adrFreeIVshow}.%
\EndIndexGroup
%
\EndIndexGroup

\endinput

%%% Local Variables: 
%%% mode: latex
%%% TeX-master: "scrguide-en.tex"
%%% coding: utf-8
%%% ispell-local-dictionary: "en_GB"
%%% eval: (flyspell-mode 1)
%%% End: 
