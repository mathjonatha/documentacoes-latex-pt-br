% ======================================================================
% common-options-en.tex
% Copyright (c) Markus Kohm, 2001-2022
%
% This file is part of the LaTeX2e KOMA-Script bundle.
%
% This work may be distributed and/or modified under the conditions of
% the LaTeX Project Public License, version 1.3c of the license.
% The latest version of this license is in
%   http://www.latex-project.org/lppl.txt
% and version 1.3c or later is part of all distributions of LaTeX
% version 2005/12/01 or later and of this work.
%
% This work has the LPPL maintenance status "author-maintained".
%
% The Current Maintainer and author of this work is Markus Kohm.
%
% This work consists of all files listed in MANIFEST.md.
% ======================================================================
%
% Paragraphs that are common for several chapters of the KOMA-Script guide
% Maintained by Markus Kohm
%
% ======================================================================

\KOMAProvidesFile{common-options-en.tex}
                 [$Date: 2023-07-31 15:38:10 +0200 (Mo, 31. Jul 2023) $
                  KOMA-Script guide (common paragraphs)]
\translator{Gernot Hassenpflug\and Krickette Murabayashi\and Karl Hagen}

\section{Seleção Antecipada ou Tardia de Opções}
\seclabel{options}
\BeginIndexGroup
\BeginIndex{}{options}%

\IfThisCommonFirstRun{%
  Esta seção apresenta um recurso especial do \KOMAScript{} que, além de
  \IfThisCommonLabelBase{typearea}{\Package{typearea}}{%
    \IfThisCommonLabelBase{maincls}{\Class{scrbook}, \Class{scrreprt} e
      \Class{scrartcl}}{%
      \IfThisCommonLabelBase{scrlttr2}{a classe \Class{scrlttr2}}{%
        \IfThisCommonLabelBase{scrextend}{as classes e
          \Package{scrextend}}{%
            \IfThisCommonLabelBase{scrjura}{\Package{scrjura}}{%
              \IfThisCommonLabelBase{scrlayer}{\Package{scrlayer}}{%
                \IfThisCommonLabelBase{scrlayer-scrpage}{%
                  \Package{scrlayer-scrpage}}{%
                  \IfThisCommonLabelBase{scrlayer-notecolumn}{%
                    \Package{scrlayer-notecolumn}}{%
                    \InternalCommonFileUsageError}}}}}}}}%
  , também é relevante para outros pacotes e classes do \KOMAScript{}. %
  \IfThisCommonLabelBase{scrlttr2}{Para que cada capítulo possa ser
    autossuficiente, esta seção aparece de forma quase idêntica em vários
    capítulos, mas se você está estudando todo o \KOMAScript{}, é claro que
    precisa lê-la apenas uma vez.%
  }{Esta seção aparece de forma quase idêntica em vários capítulos, para que
    você possa encontrar todas as informações sobre um único pacote ou classe
    no capítulo relevante. Usuários que estão interessados não apenas em um
    pacote ou classe específico, mas em obter uma visão geral do
    \KOMAScript{} como um todo, precisam ler esta seção em apenas um dos
    capítulos e podem
    \IfThisCommonLabelBase{maincls}{pular o restante do capítulo.}{%
      então pulá-la enquanto estudam o guia.}%
  }%
}{%
  As informações em \autoref{sec:\ThisCommonFirstLabelBase.options} aplicam-se
  igualmente a este capítulo. Portanto, se você já leu e compreendeu
  \autoref{sec:\ThisCommonFirstLabelBase.options}, pode pular para
  \autoref{sec:\ThisCommonLabelBase.options.next},
  \autopageref{sec:\ThisCommonLabelBase.options.next}.%
}



\begin{Declaration}
  \Macro{documentclass}\OParameter{lista de opções}%
                       \Parameter{classe \KOMAScript}
  \Macro{usepackage}\OParameter{lista de opções}%
                    \Parameter{lista de pacotes}
\end{Declaration}
O \LaTeX{} permite que os usuários passem opções de classe\textnote{opções de
  classe} como uma lista de palavras-chave separadas por vírgulas no argumento
opcional de \Macro{documentclass}. Além de serem passadas para a classe, essas
opções também são repassadas para todos os pacotes\textnote{opções globais}
que possam compreendê-las. Os usuários também podem passar uma lista similar
de palavras-chave separadas por vírgulas no argumento opcional de
\Macro{usepackage}\textnote{opções de pacote}. O \KOMAScript{}
estende\ChangedAt{v3.00}{\Class{scrbook}\and \Class{scrreprt}\and
  \Class{scrartcl}\and \Package{scrextend}\and \Package{typearea}} o mecanismo
de opções para
\IfThisCommonLabelBaseOneOf{scrextend,scrjura}{}{as classes \KOMAScript{} e
}alguns pacotes com opções adicionais. Assim, a maioria das opções do
\KOMAScript{} também pode receber um valor, de modo que uma opção não
necessariamente assume a forma \PName{opção}, mas também pode assumir a forma
\PName{opção}\texttt{=}\PName{valor}%
\important{\PName{opção}\texttt{=}\PName{valor}}. Exceto por essa diferença,
\Macro{documentclass} e \Macro{usepackage} no \KOMAScript{} funcionam como
descrito em \cite{latex:usrguide} ou em qualquer introdução ao \LaTeX, por
exemplo \cite{lshort}.

\IfThisCommonLabelBaseNotOneOf{%
  scrjura,scrlayer,scrlayer-scrpage,scrlayer-notecolumn,scrextend%
}{%
  Ao usar uma classe \KOMAScript{}\textnote{Atenção!}, você não deve
  especificar opções ao carregar os pacotes \Package{typearea} ou
  \Package{scrbase}. A razão para essa restrição é que a classe já carrega
  esses pacotes sem opções, e o \LaTeX{} recusa-se a carregar um pacote
  várias vezes com configurações de opções diferentes.%
  \IfThisCommonLabelBaseOneOf{maincls,scrlttr2}{ Em geral, não é necessário
    carregar nenhum desses pacotes explicitamente ao usar qualquer classe
    \KOMAScript{}.}{}%
  \par
}{}

% The alternative text for wrapping optimization was redundant and, as
% it's not required for the English version, has been deleted.
Definir as opções com \Macro{documentclass} tem uma\textnote{Atenção!} grande
desvantagem: ao contrário da interface descrita abaixo, as opções em
\Macro{documentclass} não são robustas. Portanto, comandos, comprimentos,
contadores e construções semelhantes podem falhar dentro do argumento opcional
deste comando. Por exemplo, com muitas classes que não são do \KOMAScript{},
usar um comprimento \LaTeX{} no valor de uma opção resulta em um erro%
\IfThisCommonLabelBaseNotOneOf{maincls,scrlttr2}{ antes que o valor seja
  passado para um pacote \KOMAScript{} e este possa assumir o controle da
  execução da opção}{}%
. Portanto, se você quiser usar um comprimento, contador ou comando \LaTeX{}
como parte do valor de uma opção, deve usar
\DescRef{\LabelBase.cmd.KOMAoptions} ou \DescRef{\LabelBase.cmd.KOMAoption}.
Esses comandos serão descritos a seguir.%
%
\EndIndexGroup


\begin{Declaration}
  \Macro{KOMAoptions}\Parameter{lista de opções}
  \Macro{KOMAoption}\Parameter{opção}\Parameter{lista de valores}
\end{Declaration}
O \KOMAScript{}\ChangedAt{v3.00}{\Class{scrbook}\and \Class{scrreprt}\and
  \Class{scrartcl}\and \Package{scrextend}\and \Package{typearea}} também
fornece a capacidade de alterar os valores da maioria das opções de
\IfThisCommonLabelBaseOneOf{scrextend,scrjura}{}{classe e }pacote mesmo após
carregar o \IfThisCommonLabelBaseOneOf{scrextend,scrjura}{}{pacote ou a}%
classe. Você pode usar o comando \Macro{KOMAoptions} para alterar os valores
de uma lista de opções, como em \DescRef{\ThisCommonLabelBase.cmd.documentclass}
ou \DescRef{\ThisCommonLabelBase.cmd.usepackage}. Cada opção na \PName{lista
  de opções} tem a forma \PName{opção}\texttt{=}\PName{valor}%
\important{\PName{opção}\texttt{=}\PName{valor},\dots}.

Algumas opções também têm um valor padrão. Se você não especificar um valor,
ou seja, se você fornecer a opção simplesmente como \PName{opção}, então esse
valor padrão será usado.

Algumas opções podem ter vários valores simultaneamente. Para tais opções, é
possível, com a ajuda de \Macro{KOMAoption}, passar uma lista de valores para
uma única \PName{opção}. Os valores individuais são fornecidos como uma
\PName{lista de valores}\important{\PName{valor},\dots} separada por vírgulas.

\begin{Explain}
  O \KOMAScript{} usa os comandos \DescRef{scrbase.cmd.FamilyOptions} e
  \DescRef{scrbase.cmd.FamilyOption} com a família ``\PValue{KOMA}'' para
  implementar essa funcionalidade.
  \IfThisCommonLabelBaseOneOf{typearea}{% Umbruchkorrektur
    Usuários avançados encontrarão mais informações sobre essas instruções em
    \autoref{sec:scrbase.keyvalue}, \DescPageRef{scrbase.cmd.FamilyOptions}.
  }{%
    Veja \autoref{part:forExperts}, \autoref{sec:scrbase.keyvalue},
    \DescPageRef{scrbase.cmd.FamilyOptions}.  }%
\end{Explain}

Opções definidas com \Macro{KOMAoptions} ou \Macro{KOMAoption} alcançarão
\IfThisCommonLabelBase{scrextend}{}{tanto a classe \KOMAScript{} quanto }%
quaisquer pacotes \KOMAScript{} carregados anteriormente que reconheçam essas
opções. Se uma opção ou um valor for desconhecido, o
\hyperref[cha:scrbase]{\Package{scrbase}}%
\IndexPackage{scrbase}\important{\hyperref[cha:scrbase]{\Package{scrbase}}}
reportará isso como um erro.%
%
\iffalse% Umbruchkorrekturtext
\iffree{}{\IfThisCommonLabelBase{scrlayer-scrpage}{\par
  A propósito, o \Package{scrpage2}\IndexPackage{scrpage2}%
  \important{\Package{scrpage2}}, que é considerado obsoleto, não possui essa
  extensão. As opções só podem ser definidas quando o pacote é carregado com a
  opção explicada anteriormente.}{}}%
\fi%
%
\EndIndexGroup
%
\EndIndexGroup

\phantomsection
\label{sec:\ThisCommonLabelBase.options.end}
\endinput

%%% Local Variables:
%%% mode: latex
%%% TeX-master: "scrguide-en.tex"
%%% coding: utf-8
%%% ispell-local-dictionary: "en_GB"
%%% eval: (flyspell-mode 1)
%%% End:
