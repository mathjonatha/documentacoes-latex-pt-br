% ======================================================================
% scrextend-en.tex
% Copyright (c) Markus Kohm, 2002-2022
%
% This file is part of the LaTeX2e KOMA-Script bundle.
%
% This work may be distributed and/or modified under the conditions of
% the LaTeX Project Public License, version 1.3c of the license.
% The latest version of this license is in
%   http://www.latex-project.org/lppl.txt
% and version 1.3c or later is part of all distributions of LaTeX 
% version 2005/12/01 or later and of this work.
%
% This work has the LPPL maintenance status "author-maintained".
%
% The Current Maintainer and author of this work is Markus Kohm.
%
% This work consists of all files listed in MANIFEST.md.
% ======================================================================
%
% Pacote scrextend para Autores de Documentos
% Mantido por Markus Kohm
%
% ======================================================================

\KOMAProvidesFile{scrextend-en.tex}
                 [$Date: 2023-09-18 09:02:03 +0200 (Mo, 18. Sep 2023) $
                  KOMA-Script package scrextend]
\translator{Markus Kohm\and Karl Hagen}

\chapter[{\KOMAScript{} Features for Other Classes with \Package{scrextend}}]
  {Usando Funcionalidades Básicas das Classes \KOMAScript{} em Outras Classes com o
    Pacote \Package{scrextend}}
\labelbase{scrextend}
\BeginIndexGroup%
\BeginIndex{Package}{scrextend}%

Existem algumas funcionalidades que são comuns a todas as classes \KOMAScript{}. Isto
se aplica não apenas às classes \Class{scrbook}, \Class{scrreprt} e
\Class{scrartcl}, que se destinam a substituir as classes padrão
\Class{book}, \Class{report} e \Class{article}, mas também, em grande medida,
à classe \KOMAScript{} \Class{scrlttr2}, sucessora de \Class{scrlettr},
que se destina a cartas. Estas funcionalidades básicas, que podem ser encontradas nas
classes mencionadas acima, também são fornecidas pelo pacote \Package{scrextend} a partir
da versão~3.00 do \KOMAScript{}. Este\textnote{Atenção!} pacote
não deve ser usado com classes \KOMAScript{}. Ele foi projetado apenas para uso com outras
classes. Se você tentar carregar o pacote com uma classe \KOMAScript{},
\Package{scrextend} detectará isto e rejeitará o carregamento com uma
mensagem de aviso.

O fato de que \hyperref[cha:scrlttr2]{\Package{scrletter}}%
\IndexPackage{scrletter} possa ser usado não apenas com classes \KOMAScript{} mas
também com as classes padrão é em parte devido ao \Package{scrextend}. Se
\hyperref[cha:scrlttr2]{\Package{scrletter}} detecta que não está sendo usado
com uma classe \KOMAScript{}, ele carrega automaticamente \Package{scrextend}. Ao
fazer isso, todas as classes \KOMAScript{} necessárias ficam disponíveis.

Naturalmente, não há garantia de que \Package{scrextend} funcionará com todas as
classes. O pacote foi projetado principalmente para estender as classes padrão
e aquelas derivadas delas. Em qualquer caso, antes de usar
\Package{scrextend}, você deve primeiro certificar-se de que a classe que está usando
não já fornece a funcionalidade que você precisa.

Além das funcionalidades descritas neste capítulo, existem mais algumas
que se destinam principalmente aos autores de classes e pacotes. Estas podem ser
encontradas em \autoref{cha:scrbase}, a partir de \autopageref{cha:scrbase}. O
pacote \hyperref[cha:scrbase]{\Package{scrbase}}%
\important{\hyperref[cha:scrbase]{\Package{scrbase}}}\IndexPackage{scrbase}
documentado lá é usado por todas as classes \KOMAScript{} e pelo
pacote \Package{scrextend}.

Todas as classes \KOMAScript{} e o pacote \Package{scrextend} também carregam o
pacote \hyperref[cha:scrlfile]{\Package{scrlfile}}%
\important{\hyperref[cha:scrlfile]{\Package{scrlfile}}}\IndexPackage{scrlfile}
descrito em \autoref{cha:scrlfile} a partir de
\autopageref{cha:scrlfile}. Portanto, as funcionalidades deste pacote também estão
disponíveis ao usar \Package{scrextend}.

\iftrue % Umbruchkorrekturtext
Em contraste, apenas as classes \KOMAScript{} \Class{scrbook}, \Class{scrreprt}
e \Class{scrartcl} carregam o pacote \hyperref[cha:tocbasic]{\Package{tocbasic}}
(veja \autoref{cha:tocbasic} a partir de \autopageref{cha:tocbasic}),
que foi projetado para autores de classes e pacotes. Por esta razão, as
funcionalidades definidas lá são encontradas apenas nessas classes e não em
\Package{scrextend}. Naturalmente, você ainda pode usar
\hyperref[cha:tocbasic]{\Package{tocbasic}} junto com
\Package{scrextend}.%
\fi

\LoadCommonFile{options}% \section{Early or late Selection of Options}

\LoadCommonFile{compatibility}% \section{Compatibility with Earlier Versions of \KOMAScript}


\section{Funcionalidades Opcionais e Estendidas}
\seclabel{optionalFeatures}

O pacote \Package{scrextend} fornece algumas funcionalidades opcionais e estendidas.
Estas funcionalidades não estão disponíveis por padrão, mas podem ser ativadas. Estas
funcionalidades são opcionais porque, por exemplo, podem estar em conflito com funcionalidades
das classes padrão ou de outros pacotes comumente usados.

\begin{Declaration}
  \OptionVName{extendedfeature}{feature}
\end{Declaration}
Com esta opção, você pode ativar uma \PName{funcionalidade} estendida do
\Package{scrextend}. Esta opção está disponível apenas durante o carregamento
do \Package{scrextend}. Você deve, portanto, especificar esta opção no argumento
opcional de \DescRef{\LabelBase.cmd.usepackage}\PParameter{scrextend}. %
\iffree{%
  Uma visão geral de todas as funcionalidades disponíveis é mostrada em
  \autoref{tab:scrextend.optionalFeatures}.

  \begin{table}
    \caption[{Funcionalidades estendidas disponíveis de
      \Package{scrextend}}]{Visão geral das
      funcionalidades estendidas opcionais do \Package{scrextend}}
    \label{tab:scrextend.optionalFeatures}
    \begin{desctabular}
      \entry{\PName{title}}{%
        páginas de título têm as funcionalidades adicionais das classes \KOMAScript{};
        isto se aplica não apenas aos comandos para a página de título, mas também à
        opção \DescRef{\LabelBase.option.titlepage} (veja
        \autoref{sec:scrextend.titlepage}, de
        \autopageref{sec:scrextend.titlepage})%
      }%
    \end{desctabular}
  \end{table}
}{%
  \par%
  Atualmente, a única \PName{funcionalidade} estendida disponível é
  \PValue{title}\IndexOption[indexmain]{extendedfeature~=\textKValue{title}}%
    \important{\OptionValue{extendedfeature}{title}}.
  Esta \PName{funcionalidade} oferece às páginas de título as funcionalidades das classes \KOMAScript{},
  conforme descrito em \autoref{sec:scrextend.titlepage} a partir de
  \autopageref{sec:scrextend.titlepage}.%
}%
%
\EndIndexGroup


\LoadCommonFile{draftmode}% \section{Modo Rascunho}

\LoadCommonFile{fontsize}%

\LoadCommonFile{textmarkup}% \section{Marcação de Texto}

\LoadCommonFile{titles}% \section{Páginas de Título do Documento}

\LoadCommonFile{oddorevenpage}% \section{Detecção de Páginas Pares e Ímpares}

\section{Escolhendo um Estilo de Página Predefinido}
\seclabel{pagestyle}

Uma das funcionalidades básicas de um documento é o estilo
de página\Index[indexmain]{página>estilo}. Em \LaTeX{}, o estilo de página determina principalmente
o conteúdo de cabeçalhos e rodapés. O pacote \Package{scrextend}
não define nenhum estilo de página por si mesmo. Em vez disso, usa os estilos de página do
núcleo \LaTeX{}.


\begin{Declaration}
  \Macro{titlepagestyle}
\end{Declaration}%
\Index{título>estilo de página}%
Em algumas páginas \DescRef{maincls.cmd.thispagestyle}\IndexCmd{thispagestyle}
seleciona automaticamente um estilo de página diferente. Atualmente, \Package{scrextend}
faz isto apenas para páginas de título, e somente se
\OptionValueRef{\LabelBase}{extendedfeature}{title} tenha sido usado (veja
\autoref{sec:scrextend.optionalFeatures},
\DescPageRef{scrextend.option.extendedfeature}). Neste caso, o estilo de página
especificado em \DescRef{maincls.cmd.thispagestyle} será utilizado. O padrão para
\DescRef{maincls.cmd.thispagestyle} é
\PageStyle{plain}\IndexPagestyle{plain}. Este estilo de página é definido pelo
núcleo \LaTeX{}, portanto deve estar sempre disponível.%
\EndIndexGroup

\LoadCommonFile{interleafpage}% \section{Páginas Intercaladas}

\LoadCommonFile{footnotes}% \section{Notas de Rodapé}

\LoadCommonFile{dictum}% \section{Máximas}

\LoadCommonFile{lists}% \section{Listas}

\LoadCommonFile{marginpar}% \section{Notas Marginais}
%
\EndIndexGroup

\endinput

%%% Local Variables: 
%%% mode: latex
%%% TeX-master: "scrguide-en.tex"
%%% coding: utf-8
%%% ispell-local-dictionary: "en_GB"
%%% eval: (flyspell-mode 1)
%%% End: 
