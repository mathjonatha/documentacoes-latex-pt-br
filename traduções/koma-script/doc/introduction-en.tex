% ======================================================================
% introduction-en.tex
% Copyright (c) Markus Kohm, 2001-2022
%
% This file is part of the LaTeX2e KOMA-Script bundle.
%
% This work may be distributed and/or modified under the conditions of
% the LaTeX Project Public License, version 1.3c of the license.
% The latest version of this license is in
%   http://www.latex-project.org/lppl.txt
% and version 1.3c or later is part of all distributions of LaTeX 
% version 2005/12/01 or later and of this work.
%
% This work has the LPPL maintenance status "author-maintained".
%
% The Current Maintainer and author of this work is Markus Kohm.
%
% This work consists of all files listed in MANIFEST.md.
% ======================================================================
%
% Introduction of the KOMA-Script guide
% Maintained by Markus Kohm
%
% ======================================================================

\KOMAProvidesFile{introduction-en.tex}
                 [$Date: 2022-06-05 12:40:11 +0200 (So, 05. Jun 2022) $
                  KOMA-Script guide introduction]
\translator{Kevin Pfeiffer\and Gernot Hassenpflug\and 
  Krickette Murabayashi\and Markus Kohm\and Karl Hagen}

\chapter{Introdução}
\labelbase{introduction}

Este capítulo contém, entre outras coisas, informações importantes sobre a
estrutura do manual e a história do {\KOMAScript}, que começa
anos antes da primeira versão. Você também encontrará informações sobre como
instalar o {\KOMAScript} e o que fazer se encontrar erros.

\section{Nota Preliminar}\seclabel{preface}

O {\KOMAScript} é muito complexo. Isso se deve ao fato de que ele consiste não
apenas de uma única classe ou de um único pacote, mas de um conjunto de muitas classes e
pacotes. Embora as classes sejam projetadas como contrapartes às classes
padrão, isso não significa que elas forneçam apenas os comandos, ambientes e
configurações das classes padrão, ou que imitem sua aparência. As
capacidades do {\KOMAScript} às vezes superam em muito as das classes
padrão. Algumas delas devem ser consideradas extensões às capacidades básicas
do kernel do \LaTeX{}.

O que foi dito anteriormente significa que a documentação do {\KOMAScript} tem que ser
extensa. Além disso, o {\KOMAScript} normalmente não é ensinado. Isso significa que não há
professores que conhecem seus alunos e que, portanto, podem escolher os materiais de ensino
e adaptá-los de acordo. Seria fácil escrever documentação
para um público específico. A dificuldade que o autor enfrenta, no entanto, é que
o manual deve servir a todos os públicos potenciais. Tentei criar um guia
que seja igualmente adequado para o cientista da computação e para a
secretária do peixeiro. Tentei, embora esta seja de fato uma tarefa impossível. O
resultado são numerosos compromissos, e eu gostaria de pedir que você leve esta questão em
conta se tiver quaisquer reclamações ou sugestões para ajudar a melhorar a
situação atual.

Apesar do comprimento deste manual, eu gostaria de pedir que você consulte a
documentação primeiro em caso de problemas. Você deve começar consultando
o índice em várias partes no final deste documento. Além deste
manual, a documentação inclui todos os documentos de texto que fazem parte do
conjunto. Veja \File{manifest.tex} para uma lista completa.

\section{Estrutura do Guia}\seclabel{structure}

Este manual está dividido em várias partes: Há uma seção para usuários
médios, uma para usuários avançados e especialistas, e um apêndice com
informações adicionais e exemplos para aqueles que desejam compreender o {\KOMAScript}
profundamente.

O \autoref{part:forAuthors} é destinado a todos os usuários do {\KOMAScript}. Isso significa
que algumas informações nesta seção são direcionadas a iniciantes em \LaTeX. Em
particular, esta parte contém muitos exemplos que visam esclarecer as
explicações. Não hesite em experimentar esses exemplos você mesmo e descobrir como
o {\KOMAScript} funciona modificando-os. Dito isso, o guia do usuário do {\KOMAScript}
não se destina a ser uma introdução ao {\LaTeX}. Aqueles que são novos no {\LaTeX} devem consultar
\emph{The Not So Short Introduction to {\LaTeXe}} \cite{lshort} ou
\emph{{\LaTeXe} for Authors} \cite{latex:usrguide} ou um livro de referência
sobre {\LaTeX}. Você também encontrará informações úteis nas muitas FAQs do {\LaTeX},
incluindo as \emph{{\TeX} Frequently Asked Questions on the Web}
\cite{UK:FAQ}. Embora o comprimento das \emph{{\TeX} Frequently Asked
	Questions on the Web} seja considerável, você deve obter pelo menos uma
visão geral aproximada e consultá-las em caso de problemas, assim como este
\iffree{guia}{livro}.

O \autoref{part:forExperts} é destinado a usuários avançados do {\KOMAScript}, aqueles
que já estão familiarizados com \LaTeX{} ou que têm trabalhado com
{\KOMAScript} há algum tempo e desejam entender mais sobre como o {\KOMAScript}
funciona, como ele interage com outros pacotes, e como realizar tarefas mais
especializadas com ele. Para este propósito, retornamos a alguns aspectos das
descrições de classes do \autoref{part:forAuthors} e os explicamos em mais
detalhes. Além disso, documentamos alguns comandos que são particularmente destinados
a usuários avançados e especialistas. Isso é complementado pela documentação de
pacotes que normalmente ficam ocultos do usuário, na medida em que fazem seu trabalho
sob a superfície das classes e pacotes de usuário. Esses pacotes são
especificamente projetados para serem usados por autores de classes e pacotes.

O apêndice\iffree{, que só pode ser encontrado na versão alemã em livro,}{}
contém informações além daquelas que são cobertas no \autoref{part:forAuthors}
e no \autoref{part:forExperts}. Usuários avançados encontrarão informações de fundo
sobre questões de tipografia para lhes dar uma base para suas próprias decisões. Além
disso, o apêndice fornece exemplos para aspirantes a autores de pacotes. Esses
exemplos não se destinam simplesmente a serem copiados. Em vez disso, eles fornecem
informações sobre planejamento e implementação de projetos, bem como alguns
comandos básicos do \LaTeX{} para autores de pacotes.

O layout do guia deve ajudá-lo a ler apenas aquelas partes que são realmente de
interesse. Cada classe e pacote normalmente tem seu próprio capítulo.
Referências cruzadas para outro capítulo são, portanto, geralmente também referências a
outra parte do pacote geral. No entanto, como as três classes principais
(\Class{scrbook}, \Class{scrrprt} e \Class{scrartcl}) concordam amplamente, elas
são apresentadas juntas no \autoref{cha:maincls}. Diferenças entre as
classes, por exemplo, para algo que afeta apenas a classe
\Class{scrartcl}\OnlyAt{\Class{scrartcl}}, são claramente destacadas na
margem, conforme mostrado aqui com \Class{scrartcl}.

\begin{Explain}
  A documentação primária do {\KOMAScript} está em alemão e foi
  traduzida para sua conveniência; como a maior parte do mundo {\LaTeX}, seus
  comandos, ambientes, opções, etc., estão em inglês. Em alguns casos, o
  nome de um comando pode soar um pouco estranho, mas mesmo assim, esperamos e
  acreditamos que, com a ajuda deste guia, o {\KOMAScript} será utilizável
  e útil para você.
\end{Explain}

Neste ponto você deve saber o suficiente para entender o guia.
No entanto, ainda pode valer a pena ler o resto deste capítulo.

\section{História do {\KOMAScript}}\seclabel{history}

%\begin{Explain}
No início da década de 1990, Frank Neukam precisava de um método para publicar as
anotações de aula de um instrutor. Naquela época, o {\LaTeX} era {\LaTeX}2.09 e não havia
distinção entre classes e pacotes\,---\,havia apenas \emph{estilos}.
Frank sentiu que os estilos de documento padrão não eram bons o suficiente para seu
trabalho; ele queria comandos e ambientes adicionais. Ao mesmo tempo ele estava
interessado em tipografia e, após ler \emph{Ausgew\"ahlte
  Aufs\"atze \"uber Fragen der Gestalt des Buches und der Typographie} de Tschichold
(Artigos Selecionados sobre os Problemas do Design de Livros e da Tipografia)
\cite{JTsch87}, ele decidiu escrever seu próprio estilo de documento\,---\,e não apenas
uma solução pontual para suas anotações de aula, mas toda uma família de estilos,
especificamente projetada para a tipografia europeia e alemã. Assim nasceu o {\Script}.

Markus Kohm, o desenvolvedor do {\KOMAScript}, descobriu o {\Script} em dezembro de
1992 e adicionou uma opção para usar o formato de papel A5. Naquela época, nem o
estilo padrão nem o {\Script} forneciam suporte para papel A5. Portanto, não
demorou muito até que Markus fizesse as primeiras mudanças no {\Script}. Essa e outras
mudanças foram então incorporadas ao {\ScriptII}, lançado por Frank em dezembro de
1993.

Em meados de 1994, o {\LaTeXe} tornou-se disponível e trouxe consigo muitas mudanças.
Os usuários do {\ScriptII} foram confrontados com a opção de limitar seu uso ao
modo de compatibilidade do {\LaTeXe} ou abandonar completamente o {\Script}. Esta
situação levou Markus a criar um novo pacote {\LaTeXe}, lançado em
7 de julho de 1994 como {\KOMAScript}. Alguns meses depois, Frank declarou o {\KOMAScript}
como o sucessor oficial do {\Script}. O {\KOMAScript} originalmente não fornecia
uma classe \emph{letter} (carta), mas essa deficiência foi logo corrigida por Axel
Kielhorn, e o resultado tornou-se parte do {\KOMAScript} em dezembro de 1994. Axel
também escreveu o primeiro guia do usuário verdadeiramente em língua alemã, que foi seguido por um
guia em língua inglesa de Werner Lemberg.

Desde então, muito tempo passou. O {\LaTeX} mudou apenas de maneiras menores, mas
a paisagem do {\LaTeX} mudou bastante; muitos novos pacotes e classes
estão agora disponíveis e o próprio {\KOMAScript} cresceu muito além do que era em
1994. O objetivo inicial era fornecer boas classes {\LaTeX} para
autores de língua alemã, mas hoje seu propósito principal é fornecer
alternativas mais flexíveis às classes padrão. O sucesso do {\KOMAScript}
levou a e-mails de usuários de todo o mundo, e isso levou a muitos novos
macros\,---\,todos precisando de documentação; daí este ``pequeno guia''.
%\end{Explain}


\section{Agradecimentos Especiais}
\seclabel{thanks}

Agradecimentos na introdução? Não, os agradecimentos apropriados podem ser
encontrados no adendo. Meus comentários aqui não são destinados aos autores deste
guia\,---\,e esses agradecimentos devem vir justamente de você, o leitor,
de qualquer forma. Eu, o autor do {\KOMAScript}, gostaria de estender meus
agradecimentos pessoais a Frank Neukam. Sem sua família {\Script}, o {\KOMAScript} não
teria surgido. Sou grato às muitas pessoas que contribuíram para o
{\KOMAScript}, mas com sua indulgência, gostaria de mencionar especificamente
Jens-Uwe Morawski e Torsten Kr\"uger. A tradução inglesa do guia
é, entre muitas outras coisas, devida ao compromisso incansável de Jens. Torsten foi
o melhor testador beta que já tive. Seu trabalho melhorou particularmente a
usabilidade de \Class{scrlttr2} e \Class{scrpage2}. Muitos agradecimentos a todos que
me encorajaram a continuar, a tornar as coisas melhores e menos propensas a erros, ou a
implementar recursos adicionais.

Agradecimentos especiais vão também aos fundadores e membros da DANTE,
Deutschsprachige Anwendervereinigung {\TeX}~e.V\kern-.18em, (o
Grupo de Usuários {\TeX} de Língua Alemã). Sem o servidor DANTE, o {\KOMAScript}
não poderia ter sido lançado e distribuído. Agradecimentos também a todos nos
grupos de notícias e listas de discussão do {\TeX} que respondem perguntas e me ajudaram
a fornecer suporte para o {\KOMAScript}.

Meus agradecimentos também vão a todos aqueles que sempre me encorajaram a ir mais longe e
a implementar este ou aquele recurso melhor, com menos falhas, ou simplesmente como uma
extensão. Eu também gostaria de agradecer ao doador muito generoso que me deu
a quantia mais significativa de dinheiro que já recebi pelo trabalho feito
até agora no \KOMAScript{}.

\section{Notas Legais}
\seclabel{legal}

O {\KOMAScript} é lançado sob a {\LaTeX} Project Public License. Você
encontrará isso no arquivo \File{lppl.txt}. Uma tradução não oficial em alemão
também está disponível em \File{lppl-de.txt} e é válida para todos os países de língua alemã.

\iffree{Este documento e o conjunto {\KOMAScript} são fornecidos ``como estão'' e
  sem garantia de qualquer tipo.}%
{Por favor note: a versão impressa deste guia não é livre sob as
  condições da {\LaTeX} Project Public Licence. Se você precisar de uma versão
  gratuita deste guia, use a versão que é fornecida como parte do
  conjunto {\KOMAScript}.}


\section{Instalação}
\seclabel{installation}

As três distribuições \TeX{} mais importantes, Mac\TeX, MiK\TeX, e
\TeX{}~Live, disponibilizam o {\KOMAScript} através de seu software de gerenciamento
de pacotes. Você deve instalar e atualizar o {\KOMAScript} usando essas ferramentas, se
possível. A instalação manual sem usar os gerenciadores de pacotes é descrita
no arquivo \File{INSTALL.txt}, que faz parte de toda distribuição {\KOMAScript}.
Você também deve ler a documentação que vem com a
distribuição {\TeX} que você está usando.


\section{Relatórios de Bugs e Outras Solicitações}
\seclabel{errors}

Se você acha que encontrou um erro na documentação ou um bug em uma das
classes, pacotes ou outra parte do {\KOMAScript}, por favor
faça o seguinte:
\begin{itemize}
\item O problema também ocorre se uma classe padrão for usada em vez de uma
  classe {\KOMAScript}? Neste caso, o erro provavelmente não está no
  {\KOMAScript}, e faz mais sentido fazer sua pergunta em um fórum
  público, uma lista de discussão ou Usenet.
\item Qual versão do {\KOMAScript} você usa? Para informações relacionadas, veja o
  arquivo \File{log} da execução do \LaTeX{} de qualquer documento que usa uma
  classe {\KOMAScript}.
\item Se você não usa uma versão atualizada do \KOMAScript{}, por favor considere
  instalar uma nova versão do \KOMAScript{}. Se o problema não ocorrer com
  um \KOMAScript{} atualizado, você já encontrou uma solução.
\item Qual sistema operacional e qual distribuição \TeX{} você usa? Esta
  informação pode parecer bastante supérflua para um pacote independente de sistema
  como {\KOMAScript} ou {\LaTeX}, mas repetidamente eles certamente têm
  se mostrado importantes.
\item Qual é exatamente o problema ou o erro? Descreva o problema. É
  melhor ser muito detalhado do que muito breve. Muitas vezes faz sentido explicar
  o contexto.
\item Como é um exemplo mínimo funcional? Você pode criar um facilmente
  comentando conteúdo e pacotes do documento passo a passo. O
  resultado é um documento que contém apenas os pacotes e partes necessários para
  reproduzir o problema. Além disso, todas as imagens carregadas devem ser substituídas por
  instruções \Macro{rule} do tamanho apropriado ou por uma imagem de exemplo do
  pacote \Package{mwe} \cite{package:mwe}. Antes de enviar seu exemplo mínimo
  funcional, remova as partes comentadas, insira o comando
  \Macro{listfiles} no preâmbulo e execute novamente o {\LaTeX}. No
  final do arquivo \File{log}, você verá uma visão geral dos pacotes
  usados. Adicione o exemplo mínimo funcional e o arquivo log ao final de sua
  descrição do problema.
\end{itemize}

Não envie pacotes, arquivos PDF, PS ou DVI. Se toda a questão ou descrição do bug,
incluindo o exemplo mínimo e o arquivo \File{log} for maior
que algumas dezenas de kilobytes, você provavelmente está fazendo algo errado.

Se você seguiu todas essas etapas, por favor crie um novo ticket no
sistema de tickets do \KOMAScript{} em \url{https://sf.net/p/koma-script/tickets}. Se
você não puder fazer isso, você pode alternativamente enviar seu relatório de bug do {\KOMAScript}
(apenas) para \href{mailto:komascript@gmx.info}{komascript@gmx.info}.

Se você quiser fazer sua pergunta em um grupo Usenet, lista de discussão ou fórum de
Internet, você deve seguir os procedimentos mencionados acima e incluir um exemplo mínimo
funcional como parte de sua pergunta, mas geralmente você não precisa
fornecer o arquivo \File{log}. Em vez disso, apenas adicione a lista de pacotes e
versões de pacotes do arquivo \File{log} e, se seu exemplo mínimo funcional
compilar com erros, você deve citar essas mensagens do arquivo \File{log}.

Por favor note que configurações padrão que não são tipograficamente ótimas não
representam erros. Por razões de compatibilidade, os padrões são preservados
sempre que possível em novas versões do {\KOMAScript}. Além disso, as melhores práticas tipográficas
são em parte uma questão de idioma e cultura, e assim as configurações
padrão do {\KOMAScript} são necessariamente um compromisso.

\section{Informações Adicionais}
\seclabel{moreinfos}

Uma vez que você se familiarize com o {\KOMAScript}, você pode querer exemplos que mostrem
como realizar tarefas mais difíceis. Tais exemplos vão além do escopo
instrucional básico deste manual e, portanto, não são incluídos. No entanto, você
encontrará mais exemplos no site do Projeto de Documentação do {\KOMAScript}
\cite{homepage}. Esses exemplos são projetados para usuários avançados de {\LaTeX} e
não são particularmente adequados para iniciantes. O idioma principal do site
é o alemão, mas o inglês também é bem-vindo.

\endinput
%%% Local Variables:
%%% mode: latex
%%% TeX-master: "scrguide-en.tex"
%%% coding: utf-8
%%% ispell-local-dictionary: "en_GB"
%%% eval: (flyspell-mode 1)
%%% End: 
