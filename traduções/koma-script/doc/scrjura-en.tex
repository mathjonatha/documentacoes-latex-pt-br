% ======================================================================
% scrjura-en.tex
% Copyright (c) Markus Kohm, 2011-2024
%
% This file is part of the LaTeX2e KOMA-Script bundle.
%
% This work may be distributed and/or modified under the conditions of
% the LaTeX Project Public License, version 1.3c of the license.
% The latest version of this license is in
%   http://www.latex-project.org/lppl.txt
% and version 1.3c or later is part of all distributions of LaTeX 
% version 2005/12/01 or later and of this work.
%
% This work has the LPPL maintenance status "author-maintained".
%
% The Current Maintainer and author of this work is Markus Kohm.
%
% This work consists of all files listed in MANIFEST.md.
% ======================================================================
%
% Chapter about scrjura of the KOMA-Script guide
% Maintained by Markus Kohm
%
% ======================================================================

\KOMAProvidesFile{scrjura-en.tex}%
                 [$Date: 2024-10-15 10:33:25 +0200 (Di, 15. Okt 2024) $
                  KOMA-Script guide (chapter: scrjura)]

\translator{Markus Kohm}

\chapter{Suporte para Documentos Jurídicos com \Package{scrjura}}
\labelbase{scrjura}
\BeginIndexGroup
\BeginIndex{Package}{scrjura}
\BeginIndex{Package}{contract}

Até e incluindo a versão~3.41, \KOMAScript{} fornecia oficialmente o pacote
\Package{scrjura} para suportar documentos jurídicos. Estes eram principalmente
estatutos, leis, comentários sobre elas ou, no sentido mais amplo, contratos de
todos os tipos. No curso da reestruturação de \KOMAScript{}, o pacote foi
separado. Como o contrato é o elemento central do pacote, recebeu o novo nome
\Package{contract}. Sob este nome pode não apenas ser encontrado no CTAN
(veja \cite{package:contract}). Também foi integrado em distribuições comuns
de \TeX{} e pode ser instalado através do gerenciador de pacotes delas.

Por razões de compatibilidade, ainda haverá um pacote \Package{scrjura} em
\KOMAScript{} por enquanto. No entanto, esta é meramente uma embalagem
diferente do novo \Package{contract}, no qual algumas das incompatibilidades
entre o novo pacote e o anterior \Package{scrjura} foram corrigidas. Portanto,
em geral deve ser possível continuar editando documentos existentes baseados em
\Package{scrjura}. Para novos documentos, é fortemente recomendado mudar para
\Package{contract} em vez disso. A mudança também é recomendada ao revisar
documentos antigos. Consulte o manual do usuário de \Package{contract} para
as alterações a serem levadas em consideração.

\begin{Declaration}
  \Macro{Clause}\Parameter{options}%
  \Macro{SubClause}\Parameter{options}
\end{Declaration}
A diferença mais importante ao usar \Package{scrjura} em comparação com
\Package{contract} é que as \PName{options} para as duas instruções
\Macro{Clause} e \Macro{SubClause} dentro de um ambiente \Environment{contract}
com o pacote \Package{contract} representam um argumento opcional, ou seja,
devem ser especificadas entre colchetes. No pacote \Package{scrjura}, no
entanto, as \PName{options} foram sempre um argumento obrigatório, ou seja,
para ser especificado entre chaves. Isto ainda é o caso com \Package{scrjura}.%
%
\EndIndexGroup
%
\EndIndexGroup

\endinput

%%% Local Variables: 
%%% mode: latex
%%% TeX-master: "scrguide-en.tex"
%%% coding: utf-8
%%% ispell-local-dictionary: "en_GB"
%%% eval: (flyspell-mode 1)
%%% End: 
