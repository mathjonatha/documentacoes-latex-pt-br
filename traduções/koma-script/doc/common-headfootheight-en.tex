% ======================================================================
% common-headfootheight-en.tex
% Copyright (c) Markus Kohm, 2013-2023
%
% This file is part of the LaTeX2e KOMA-Script bundle.
%
% This work may be distributed and/or modified under the conditions of
% the LaTeX Project Public License, version 1.3c of the license.
% The latest version of this license is in
%   http://www.latex-project.org/lppl.txt
% and version 1.3c or later is part of all distributions of LaTeX
% version 2005/12/01 or later and of this work.
%
% This work has the LPPL maintenance status "author-maintained".
%
% The Current Maintainer and author of this work is Markus Kohm.
%
% This work consists of all files listed in MANIFEST.md.
% ======================================================================
%
% Text that is common for several chapters of the KOMA-Script guide
% Maintained by Markus Kohm
%
% ============================================================================

\KOMAProvidesFile{common-headfootheight-en.tex}
                 [$Date: 2023-04-06 09:05:58 +0200 (Do, 06. Apr 2023) $
                  KOMA-Script guide (common paragraph: Head and Foot Height)]
\translator{Markus Kohm\and Jana Schubert\and Jens Hühne\and Karl Hagen}

\section{Altura de Cabeçalho e Rodapé}
\seclabel{height}
\BeginIndexGroup
\BeginIndex{}{header>height}%
\BeginIndex{}{footer>height}%
\IfThisCommonLabelBase{scrlayer-scrpage}{%
  \begin{Explain}%
    As classes padrão do \LaTeX{} não usam muito o rodapé e, quando o usam,
    colocam o conteúdo dentro de um \Macro{mbox}, o que resulta no rodapé
    sendo uma única linha de texto. Esta é provavelmente a razão pela qual o
    próprio \LaTeX{} não tem uma altura de rodapé bem definida. Embora a
    distância entre a última linha de base da área de texto e a linha de base
    do rodapé seja definida com \Length{footskip}\IndexLength[indexmain]{footskip},
    se o rodapé consistir em mais de uma linha de texto, não há uma declaração
    definitiva se este comprimento deve ser a distância até a primeira ou a
    última linha de base do rodapé.

    Embora o cabeçalho de página das classes padrão também seja colocado em
    uma caixa horizontal e, portanto, também seja uma única linha de texto, o
    \LaTeX{} de fato fornece um comprimento para definir a altura do
    cabeçalho. A razão para isso pode ser que esta altura é necessária para
    determinar o início da área de texto.
  \end{Explain}%
}{%
  O cabeçalho e o rodapé de uma página são elementos centrais não apenas de
  um estilo de página. Eles também podem servir para restringir camadas
  usando as opções apropriadas (veja \autoref{tab:scrlayer.layerkeys},
  \autopageref{tab:scrlayer.layerkeys}). Portanto, as alturas desses
  elementos devem ser definidas.%
}

\IfThisCommonFirstRun{}{%
  As informações em \autoref{sec:\ThisCommonFirstLabelBase.height} aplicam-se
  igualmente a este capítulo. Portanto, se você já leu e entendeu
  \autoref{sec:\ThisCommonFirstLabelBase.height}, você pode pular para
  \autoref{sec:\ThisCommonLabelBase.height.next},
  \autopageref{sec:\ThisCommonLabelBase.height.next}.%
}

\begin{Declaration}
  \Length{footheight}
  \Length{headheight}
  \IfThisCommonLabelBase{scrlayer-scrpage}{%
    \OptionVName{autoenlargeheadfoot}{simple switch}%
  }{}%
\end{Declaration}
O pacote \Package{scrlayer} introduz um novo comprimento, \Length{footheight},
análogo a \Length{headheight}%
\IfThisCommonLabelBase{scrlayer-scrpage}{}{do kernel do \LaTeX{}}.
Adicionalmente,
\Package{scrlayer\IfThisCommonLabelBase{scrlayer-scrpage}{-scrpage}{}}
interpreta \Length{footskip} como sendo a distância da última linha de base da
área de texto até a primeira linha de base normal do rodapé. O pacote
\hyperref[cha:typearea]{\Package{typearea}}\IndexPackage{typearea}%
\important{\hyperref[cha:typearea]{\Package{typearea}}} interpreta
\Length{footheight} da mesma maneira, portanto as opções do
\Package{typearea} para a altura do rodapé também podem ser usadas para
definir os valores para o pacote \Package{scrlayer}. Veja as opções
\DescRef{typearea.option.footheight} e \DescRef{typearea.option.footlines}
em \autoref{sec:typearea.typearea}, \DescPageRef{typearea.option.footheight})
e a opção \DescRef{typearea.option.footinclude} em
\DescPageRef{typearea.option.footinclude} da mesma seção.

Se você não usar o pacote \hyperref[cha:typearea]{\Package{typearea}}%
\important{\hyperref[cha:typearea]{\Package{typearea}}}, você deve ajustar
as alturas de cabeçalho e rodapé usando valores apropriados para os
comprimentos quando necessário. Para o cabeçalho, pelo menos, o pacote
\Package{geometry}, por exemplo, fornece configurações semelhantes.
\IfThisCommonLabelBase{scrlayer-scrpage}{\par%
  Se você escolher uma altura de cabeçalho ou rodapé que seja muito pequena
  para o conteúdo real, \Package{scrlayer-scrpage} tenta, por padrão,
  ajustar os comprimentos apropriadamente. Ao mesmo tempo, ele emitirá um
  aviso contendo sugestões para configurações adequadas. Essas alterações
  automáticas entram em vigor imediatamente após a necessidade delas ter sido
  detectada e não são revertidas automaticamente, por exemplo, quando o
  conteúdo do cabeçalho ou rodapé se tornar menor posteriormente.
  No entanto,\ChangedAt{v3.25}{\Package{scrlayer-scrpage}}, este
  comportamento pode ser alterado usando a opção \Option{autoenlargeheadfoot}.
  Esta opção reconhece os valores para interruptores simples em
  \autoref{tab:truefalseswitch}, \autopageref{tab:truefalseswitch}. A opção
  é ativada por padrão. Se ela for desativada, o cabeçalho e o rodapé não
  são mais ampliados automaticamente. Apenas um aviso com dicas para
  configurações adequadas é emitido.%
}{%
  \IfThisCommonLabelBase{scrlayer}{\par%
    Se você escolher uma altura de cabeçalho ou rodapé que seja muito pequena
    para o conteúdo real, \Package{scrlayer} geralmente aceita isso sem
    emitir uma mensagem de erro ou um aviso. O cabeçalho então se expande de
    acordo com sua altura, geralmente para cima, com o rodapé movido para
    baixo de acordo. Informações sobre essa alteração não são obtidas
    automaticamente. No entanto, pacotes como
    \hyperref[cha:scrlayer-scrpage]{\Package{scrlayer-scrpage}}%
    \important{\hyperref[cha:scrlayer-scrpage]{\Package{scrlayer-scrpage}}}%
    \IndexPackage{scrlayer-scrpage} que se baseiam em \Package{scrlayer} podem
    conter seus próprios testes que podem levar às suas próprias ações (veja
    \DescRef{scrlayer-scrpage.length.headheight} e
    \DescRef{scrlayer-scrpage.length.footheight} em
    \DescPageRef{scrlayer-scrpage.length.headheight}).%
  }{}%
}%
\EndIndexGroup
%
\EndIndexGroup

%%% Local Variables:
%%% mode: latex
%%% TeX-master: "scrguide-en.tex"
%%% coding: utf-8
%%% ispell-local-dictionary: "en_GB"
%%% eval: (flyspell-mode 1)
%%% End:

