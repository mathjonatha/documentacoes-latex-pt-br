% ======================================================================
% common-parmarkup-en.tex
% Copyright (c) Markus Kohm, 2001-2022
%
% This file is part of the LaTeX2e KOMA-Script bundle.
%
% This work may be distributed and/or modified under the conditions of
% the LaTeX Project Public License, version 1.3c of the license.
% The latest version of this license is in
%   http://www.latex-project.org/lppl.txt
% and version 1.3c or later is part of all distributions of LaTeX
% version 2005/12/01 or later and of this work.
%
% This work has the LPPL maintenance status "author-maintained".
%
% The Current Maintainer and author of this work is Markus Kohm.
%
% This work consists of all files listed in MANIFEST.md.
% ======================================================================
%
% Paragraphs that are common for several chapters of the KOMA-Script guide
% Maintained by Markus Kohm
%
% ======================================================================

\KOMAProvidesFile{common-parmarkup-en.tex}
                 [$Date: 2022-06-05 12:40:11 +0200 (So, 05. Jun 2022) $
                  KOMA-Script guide (common paragraph: Paragraph Markup)]
\translator{Gernot Hassenpflug\and Markus Kohm\and Krickette Murabayashi\and
	Karl Hagen}

\section{Marcação de Parágrafos}
\seclabel{parmarkup}%
\BeginIndexGroup
\BeginIndex{}{paragraph>marking}%

\IfThisCommonLabelBase{maincls}{%
  As\textnote{recuo de parágrafo vs. espaçamento de parágrafo} classes
  padrão normalmente configuram parágrafos\Index[indexmain]{paragraph}
  recuados e sem qualquer espaçamento vertical entre parágrafos. Esta é a
  melhor solução quando se usa um layout de página regular como aqueles
  produzidos com o pacote \Package{typearea}. Se nem recuo nem espaço
  vertical forem usados, apenas o comprimento da última linha forneceria ao
  leitor uma indicação da quebra de parágrafo. Em casos extremos, é muito
  difícil dizer se uma linha está completa ou não. Além disso, tipógrafos
  consideram que um sinal dado no final do parágrafo é facilmente esquecido
  no início da próxima linha. Um sinal no início do parágrafo é mais
  facilmente lembrado. O espaçamento entre parágrafos tem a desvantagem de
  desaparecer em alguns contextos. Por exemplo, após uma fórmula destacada,
  seria impossível detectar se o parágrafo anterior continua ou se um novo
  começa. Além disso, no topo de uma nova página, pode ser necessário olhar
  para a página anterior para determinar se um novo parágrafo foi iniciado ou
  não. Todos esses problemas desaparecem ao usar recuo. Uma combinação de
  recuo e espaçamento vertical entre parágrafos é redundante e, portanto,
  deve ser evitada. O recuo\Index[indexmain]{indentation} sozinho é
  suficiente. A única desvantagem do recuo é que ele encurta o comprimento da
  linha. O uso de espaçamento entre parágrafos\Index{paragraph>spacing} é,
  portanto, justificado ao usar linhas curtas, como em um jornal.%
}{%
  \IfThisCommonLabelBase{scrlttr2}{%
    Os preliminares de \autoref{sec:maincls.parmarkup},
    \autopageref{sec:maincls.parmarkup} explicam por que o recuo de parágrafo
    é preferível ao espaçamento de parágrafo. Mas os elementos aos quais esta
    explicação se refere, por exemplo, figuras, tabelas, listas, equações e
    até novas páginas, são raros em cartas normais. Cartas geralmente não são
    tão longas que um parágrafo não reconhecido terá consequências sérias para
    a legibilidade do documento. Os argumentos para o recuo, portanto, são
    menos consequentes para cartas padrão. Esta pode ser uma razão pela qual
    você frequentemente encontra parágrafos em cartas marcados com
    espaçamento vertical. Mas duas vantagens do recuo de parágrafo permanecem.
    Uma é que tal carta se destaca da multidão. Outra é que mantém a
    \emph{identidade da marca}, isto é, a aparência uniforme de todos os
    documentos de uma única fonte.%
  }{\InternalCommonFileUsageError}%
} %
\IfThisCommonFirstRun{}{%
  Além dessas sugestões, a informação descrita em
  \autoref{sec:\ThisCommonFirstLabelBase.parmarkup} para as outras classes
  \KOMAScript{} também é válida para cartas. Portanto, se você já leu e
  entendeu \autoref{sec:\ThisCommonFirstLabelBase.parmarkup}, você pode pular
  para \autoref{sec:\ThisCommonLabelBase.parmarkup.next} na
  \autopageref{sec:\ThisCommonLabelBase.parmarkup.next}.%
  \IfThisCommonLabelBase{scrlttr2}{ %
    Isso também se aplica se você trabalha não com a classe
    \Class{scrlttr2}\OnlyAt{scrlttr2}, mas com o pacote \Package{scrletter}. O
    pacote não fornece suas próprias configurações para formatação de
    parágrafo, mas depende inteiramente da classe sendo usada.%
  }{}%
}


\begin{Declaration}
  \OptionVName{parskip}{method}
\end{Declaration}
\IfThisCommonLabelBase{maincls}{%
  De vez em quando você pode precisar de um layout de documento com
  espaçamento vertical entre parágrafos em vez de recuo. As classes
  \KOMAScript{} fornecem várias maneiras de realizar isso com a opção
  \Option{parskip}\ChangedAt{v3.00}{\Class{scrbook}\and \Class{scrreprt}\and
    \Class{scrartcl}}.%
}{%
  \IfThisCommonLabelBase{scrlttr2}{%
    Em cartas, você frequentemente encontra parágrafos marcados não por recuo
    da primeira linha, mas por um espaço vertical entre eles. A classe
    \KOMAScript{} \Class{scrlttr2}\OnlyAt{scrlttr2} fornece maneiras de
    realizar isso com a opção \Option{parskip}.%
  }{\InternalCommonFileUsageError}%
} %
O \PName{method} consiste em dois elementos. O primeiro elemento é
\PValue{full}\important{\OptionValue{parskip}{full}\\
  \OptionValue{parskip}{half}} ou \PValue{half}, onde \PValue{full}
representa um espaçamento de parágrafo de uma linha e \PValue{half}
representa um espaçamento de parágrafo de meia linha. O segundo elemento
consiste em um dos caracteres ``\PValue{*}'', ``\PValue{+}'' ou
``\PValue{-}'' e pode ser omitido. Sem o segundo
elemento\important{\OptionVName{parskip}{distance}}, a linha final de um
parágrafo terminará com um espaço em branco de pelo menos 1\Unit{em}. Com o
caractere de adição como segundo
elemento\important{\OptionValue{parskip}{\PName{distance}+}}, o espaço em
branco será de pelo menos um terço\,---\,e com o
asterisco\important{\OptionValue{parskip}{\PName{distance}*}} um
quarto\,---\,da largura de uma linha normal. Com a variante de
menos\important{\OptionValue{parskip}{\PName{Abstand}-}}, nenhuma provisão é
feita para espaço em branco na última linha de um parágrafo.

Você pode mudar a configuração a qualquer momento. Se você alterá-la dentro
do documento, o comando \Macro{selectfont}\IndexCmd{selectfont}%
\IfThisCommonLabelBase{maincls}{%
  \ChangedAt{v3.08}{\Class{scrbook}\and \Class{scrreprt}\and
    \Class{scrartcl}}%
}{%
  \IfThisCommonLabelBase{scrlttr2}{%
    \ChangedAt{v3.08}{\Class{scrlttr2}}%
  }{%
    \InternalComonFileUsageError%
  }%
} %
será chamado implicitamente. Mudanças no espaçamento de parágrafo dentro de
um parágrafo não serão visíveis até o final do parágrafo.

Além das oito combinações resultantes para \PName{method}, você pode usar os
valores para chaves simples mostrados em \autoref{tab:truefalseswitch},
\autopageref{tab:truefalseswitch}. Ativar a
opção\important{\Option{parskip}\\\OptionValue{parskip}{true}} corresponde a
usar \PValue{full} sem segundo elemento e, portanto, resulta em espaçamento
entre parágrafos de uma linha com pelo menos 1\Unit{em} de espaço em branco
no final da última linha de cada parágrafo.
Desativar\important{\OptionValue{parskip}{false}} a opção reativa o recuo
padrão de 1\Unit{em} na primeira linha do parágrafo em vez de espaçamento de
parágrafo. Um resumo de todos os valores possíveis para \PName{method} é
mostrado em \autoref{tab:\ThisCommonFirstLabelBase.parskip}%
\IfThisCommonFirstRun{.%
  \begin{desclist}
%  \begin{table}
  \desccaption
%    \caption
  [{Valores disponíveis da opção \Option{parskip}}]{%
    Valores disponíveis da opção \Option{parskip} para selecionar como os
    parágrafos são distinguidos\label{tab:\ThisCommonFirstLabelBase.parskip}%
  }%
  {%
    Valores disponíveis da opção \Option{parskip} (\emph{continuação})%
  }%
  % \begin{desctabular}
  \entry{\PValue{false}, \PValue{off}, \PValue{no}%
    \IndexOption{parskip~=\textKValue{false}}}{%
    Parágrafos são identificados por recuo da primeira linha em 1em. Não há
    requisito de espaçamento no final da última linha de um parágrafo.}%
  \entry{\PValue{full}, \PValue{true}, \PValue{on}, \PValue{yes}%
    \IndexOption{parskip~=\textKValue{full}}%
  }{%
    Parágrafos são identificados por um espaço vertical de uma linha entre
    parágrafos. Deve haver pelo menos 1\Unit{em} de espaço livre no final da
    última linha do parágrafo.}%
  \pventry{full-}{%
    Parágrafos são identificados por um espaço vertical de uma linha entre
    parágrafos. Não há requisito de espaçamento no final da última linha de
    um parágrafo.\IndexOption{parskip~=\textKValue{full-}}}%
  \pventry{full+}{%
    Parágrafos são identificados por um espaço vertical de uma linha entre
    parágrafos. Deve haver pelo menos um terço de uma linha de espaço livre
    no final de um parágrafo.\IndexOption{parskip~=\textKValue{full+}}}%
  \pventry{full*}{%
    Parágrafos são identificados por um espaço vertical de uma linha entre
    parágrafos. Deve haver pelo menos um quarto de uma linha de espaço livre
    no final de um parágrafo.\IndexOption{parskip~=\textKValue{full*}}}%
  \pventry{half}{%
    Parágrafos são identificados por um espaço vertical de meia linha entre
    parágrafos. Deve haver pelo menos 1\Unit{em} de espaço livre no final da
    última linha de um parágrafo.\IndexOption{parskip~=\textKValue{half}}}%
  \pventry{half-}{%
    Parágrafos são identificados por um espaço vertical de meia linha entre
    parágrafos. Não há requisito de espaçamento no final da última linha de
    um parágrafo.\IndexOption{parskip~=\textKValue{half-}}}%
  \pventry{half+}{%
    Parágrafos são identificados por um espaço vertical de meia linha entre
    parágrafos. Deve haver pelo menos um terço de uma linha de espaço livre
    no final de um parágrafo.\IndexOption{parskip~=\textKValue{half+}}}%
  \pventry{half*}{%
    Parágrafos são identificados por um espaço vertical de meia linha entre
    parágrafos. Deve haver pelo menos um quarto de uma linha de espaço livre
    no final de um parágrafo.\IndexOption{parskip~=\textKValue{half*}}}%
  \pventry{never}{%
    Nenhum%
    \IfThisCommonLabelBase{maincls}{%
      \ChangedAt{v3.08}{\Class{scrbook}\and \Class{scrreprt}\and
        \Class{scrartcl}}%
    }{%
      \IfThisCommonLabelBase{scrlttr2}{%
        \ChangedAt{v3.08}{\Class{scrlttr2}}%
      }{}%
    } %
    espaçamento entre parágrafos será inserido mesmo se espaçamento vertical
    adicional for necessário para ajuste vertical com
    \Macro{flushbottom}.\IndexCmd{flushbottom}%
    \IndexOption{parskip~=\textKValue{never}}}%
%  \end{desctabular}
%  \end{table}%
  \end{desclist}%
}{ na \autopageref{tab:\ThisCommonFirstLabelBase.parskip}.}

Todos\textnote{Atenção!} os oito valores de opção \PValue{full} e
\PValue{half} também alteram o espaçamento antes, depois e dentro de
ambientes de lista. Isso evita que esses ambientes ou os parágrafos dentro
deles tenham uma separação maior do que aquela entre os parágrafos de texto
normal.%
\IfThisCommonLabelBase{maincls}{ %
  Adicionalmente, essas opções garantem que o sumário e as listas de figuras
  e tabelas sejam configurados sem qualquer espaçamento adicional.%
}{ %
  Vários elementos do cabeçalho da carta são sempre configurados sem
  espaçamento entre parágrafos.%
}%

O comportamento\textnote{padrão} padrão do \KOMAScript{} é
\OptionValue{parskip}{false}. Neste caso, não há espaçamento entre
parágrafos, apenas um recuo da primeira linha em 1\Unit{em}.%
%
\EndIndexGroup
%
\EndIndexGroup

%%% Local Variables:
%%% mode: latex
%%% TeX-master: "scrguide-en.tex"
%%% coding: utf-8
%%% ispell-local-dictionary: "en_GB"
%%% eval: (flyspell-mode 1)
%%% End:
