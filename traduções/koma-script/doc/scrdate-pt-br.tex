% ======================================================================
% scrdate-pt-br.tex
% Copyright (c) Markus Kohm, 2001-2022
%
% This file is part of the LaTeX2e KOMA-Script bundle.
%
% This work may be distributed and/or modified under the conditions of
% the LaTeX Project Public License, version 1.3c of the license.
% The latest version of this license is in
%   http://www.latex-project.org/lppl.txt
% and version 1.3c or later is part of all distributions of LaTeX
% version 2005/12/01 or later and of this work.
%
% This work has the LPPL maintenance status "author-maintained".
%
% The Current Maintainer and author of this work is Markus Kohm.
%
% This work consists of all files listed in MANIFEST.md.
% ======================================================================
%
% Chapter about scrdate of the KOMA-Script guide
% Maintained by Markus Kohm
%
% ============================================================================

\KOMAProvidesFile{scrdate-pt-br.tex}
                 [$Date: 2022-06-05 12:40:11 +0200 (So, 05. Jun 2022) $
                  KOMA-Script guide (chapter: scrdate)]
\translator{Markus Kohm\and Gernot Hassenpflug\and Karl Hagen}

\chapter{O Dia da Semana com o Pacote \Package{scrdate}}
\labelbase{scrdate}
\BeginIndexGroup
\BeginIndex{Package}{scrdate}

Originalmente, o pacote \Package{scrdate} podia apenas fornecer o dia da semana
para a data atual. Atualmente, oferece isso e mais para qualquer data do
calendário gregoriano.

\begin{Declaration}
  \Macro{CenturyPart}\Parameter{year}\\%
  \Macro{DecadePart}\Parameter{year}%
\end{Declaration}%
O comando \Macro{CenturyPart}\ChangedAt{v3.05a}{\Package{scrdate}} retorna
o valor dos dígitos do século\,---\,milhares e centenas\,---\,de um
\PName{year}. O comando \Macro{DecadePart}, por outro lado, fornece o
valor dos dígitos restantes, ou seja, as dezenas e as unidades. O
\PName{year} pode ter qualquer número de dígitos. Você pode atribuir o valor
diretamente a um contador ou usá-lo para cálculos com
\Macro{numexpr}\IndexCmd{numexpr}. Para exibir\textnote{Atenção!} como um
número árabe, você deve precedê-lo com \Macro{the}\IndexCmd{the}.

\begin{Example}
  Você quer calcular e imprimir o século do ano atual.
\begin{lstcode}
  The year \the\year\ is year \the\DecadePart{\year}
  of the \engord{\numexpr\CenturyPart{\year}+1\relax} century.
\end{lstcode}
  O resultado seria:
  \begin{ShowOutput}
    The year \the\year\ is year \the\DecadePart{\year}
    of the \engordnumber{\numexpr\CenturyPart{\year}+1\relax} century.
  \end{ShowOutput}
  Este exemplo utiliza o pacote \Package{engord}\IndexPackage{engord}.
  Veja \cite{package:engord} para mais informações.
\end{Example}

Observe\textnote{Atenção!} que o método de contagem usado aqui trata o ano
2000 como ano~0\,---\,e portanto o primeiro ano\,---\,do século~21.
Se necessário, porém, você pode fazer uma correção com \Macro{numexpr}, como
mostrado para o número ordinal no exemplo.%
\EndIndexGroup


\begin{Declaration}
  \Macro{DayNumber}\Parameter{year}\Parameter{month}\Parameter{day}\\%
  \Macro{ISODayNumber}\Parameter{ISO-date}%
\end{Declaration}%
Estes\ChangedAt{v3.05a}{\Package{scrdate}} dois comandos retornam o valor do
número do dia da semana\Index{dia>da semana} para qualquer data. Diferem
apenas na forma de especificar a data. Enquanto o comando \Macro{DayNumber}
requer \PName{year}, \PName{month} e \PName{day} como parâmetros separados,
o comando \Macro{ISODayNumber} espera uma \PName{ISO-date} como um
único argumento, \PName{ISO-date}, usando a notação ISO
\PName{year}\texttt{-}\PName{month}\texttt{-}\PName{day}. Não importa
se o \PName{month} ou \PName{day} têm um ou dois dígitos. Você pode usar o
resultado de ambos os comandos para atribuir diretamente a um contador ou
para cálculos usando \Macro{numexpr}\IndexCmd{numexpr}. Para imprimir\textnote{Atenção!}
como um número árabe, você deve precedê-lo com \Macro{the}\IndexCmd{the}.

\begin{Example}
  Você quer saber o número do dia da semana de 1º de maio de 2027.
\begin{lstcode}
  The 1st~May~2027 has \the\ISODayNumber{2027-5-1}
  as the number of the day of the week.
\end{lstcode}
  O resultado será:
  \begin{ShowOutput}
    The 1st~May~2027 has \the\ISODayNumber{2027-5-1}
    as the number of the day of the week.
  \end{ShowOutput}
\end{Example}

Vale ressaltar particularmente que você pode até mesmo avançar um número
especificado de dias no futuro ou no passado a partir de uma data determinada.
\begin{Example}
  Você quer saber o número do dia da semana 12~dias a partir de agora
  e 24~dias antes de 24 de dezembro de 2027.
\begin{lstcode}
  In 12~days, the number of the day of the week
  will be \the\DayNumber{\year}{\month}{\day+12}, and
  24~days before the 24th~December~2027 it will be
  \the\ISODayNumber{2027-12-24-24}.
\end{lstcode}
  O resultado poderia ser, por exemplo:
  \begin{ShowOutput}
    In 12~days, the number of the day of the week
    will be \the\DayNumber{\year}{\month}{\day+12}, and
    24~days before the 24th~December~2027 it will be
    \the\ISODayNumber{2027-12-24-24}.
  \end{ShowOutput}
\end{Example}

Os dias da semana são numerados da seguinte forma: domingo\,=\,0, segunda-feira\,=\,1,
terça-feira\,=\,2, quarta-feira\,=\,3, quinta-feira\,=\,4, sexta-feira\,=\,5, e
sábado\,=\,6.%
%
\EndIndexGroup


\begin{Declaration}
  \Macro{DayNameByNumber}\Parameter{number of the day of the week}\\%
  \Macro{DayName}\Parameter{year}\Parameter{month}\Parameter{day}\\%
  \Macro{ISODayName}\Parameter{ISO-date}%
\end{Declaration}%
Normalmente\ChangedAt{v3.05a}{\Package{scrdate}} você está menos interessado no
número do dia da semana do que em seu nome. Portanto, o
comando \Macro{DayNameByNumber} retorna o nome do dia da semana
correspondente a um número de dia da semana. Este número pode ser o resultado,
por exemplo, de \Macro{DayNumber} ou \Macro{ISODayNumber}. Os dois comandos
\Macro{DayName} e \Macro{ISODayName} retornam diretamente o nome do dia da
semana de uma data determinada.

\begin{Example}
  Você quer saber o nome do dia da semana de 24 de dezembro de 2027.
\begin{lstcode}
  Please pay by \ISODayName{2027-12-24},
  24th~December~2027 the amount of \dots.
\end{lstcode}
  O resultado será:
  \begin{ShowOutput}
    Please pay by \ISODayName{2027-12-24},
    24th~December~2027 the amount of \dots.
  \end{ShowOutput}
\end{Example}

Novamente, vale ressaltar particularmente que você pode realizar cálculos,
em certa medida:
\begin{Example}
  Você quer saber os nomes do dia da semana 12~dias a partir de agora
  e 24~dias antes de 24 de dezembro de 2027.
\begin{lstcode}
  In 12~days, the name of the day of the week
  will be \DayName{\year}{\month}{\day+12}, and
  24~days before the 24th~December~2027 it will be
  \ISODayName{2027-12-24-24}, while two weeks
  and three days after a Wednesday will be a
  \DayNameByNumber{3+2*7+3}.
\end{lstcode}
  O resultado poderia ser, por exemplo:
  \begin{ShowOutput}
    In 12~days, the name of the day of the week
    will be \DayName{\year}{\month}{\day+12}, and
    24~days before the 24th~December~2027 it will be
    \ISODayName{2027-12-24-24}, while two weeks
    and three days after a Wednesday will be a
    \DayNameByNumber{3+2*7+3}.
  \end{ShowOutput}
\end{Example}%
%
\EndIndexGroup


\begin{Declaration}
  \Macro{ISOToday}%
  \Macro{IsoToday}%
  \Macro{todaysname}%
  \Macro{todaysnumber}%
\end{Declaration}%
Nos exemplos anteriores, a data atual era sempre especificada de forma
incômoda usando os registros \TeX{} \Macro{year}\IndexCmd{year},
\Macro{month}\IndexCmd{month} e \Macro{day}\IndexCmd{day}. Os
comandos \Macro{ISOToday}\ChangedAt{v3.05a}{\Package{scrdate}} e \Macro{IsoToday}
retornam diretamente a data atual em notação ISO. Estes comandos
diferem apenas pelo fato de que \Macro{ISOToday} sempre exibe um mês
e dia com dois dígitos, enquanto \Macro{IsoToday} exibe números com um dígito
para valores menores que 10. O comando \Macro{todaysname} retorna diretamente
o nome do dia da semana atual, enquanto \Macro{todaysnumber} retorna o número do
dia da semana atual. Você pode encontrar mais informações sobre como usar este
valor nas explicações de \DescRef{scrdate.cmd.DayNumber} e
\DescRef{scrdate.cmd.ISODayNumber}.

\begin{Example}
  Quero mostrar a você em que dia da semana este documento foi digitado:
\begin{lstlisting}
  This document was created on a \todaysname.
\end{lstlisting}
  Isto resultará, por exemplo, em:
  \begin{ShowOutput}
    This document was created on a \todaysname.
  \end{ShowOutput}
\end{Example}

Para idiomas que têm um sistema de casos para nomes, note que o pacote não pode
declinar palavras. Os termos são dados na forma apropriada para exibir uma
data em uma carta, que é o nominativo singular para os idiomas atualmente
suportados. Dada esta limitação, o exemplo acima não funcionará corretamente
se traduzido diretamente para alguns outros idiomas.

\begin{Explain}
  Os\textnote{Dica!} nomes dos dias da semana no \Package{scrdate} todos têm
  letras maiúsculas iniciais. Se você precisar dos nomes completamente em minúsculas,
  por exemplo porque essa é a convenção no idioma relevante, simplesmente envolva
  o comando com o comando \LaTeX{} \Macro{MakeLowercase}\IndexCmd{MakeLowercase}%
  \important{\Macro{MakeLowercase}}:
  % Umbruchkorrektur: listings
\begin{lstcode}
  \MakeLowercase{\todaysname}
\end{lstcode}
  Isto converte todo o argumento em letras minúsculas. Claro, você também pode
  fazer isso para os comandos
  \DescRef{scrdate.cmd.DayNameByNumber}\IndexCmd{DayNameByNumber},
  \DescRef{scrdate.cmd.DayName}\IndexCmd{DayName} e
  \DescRef{scrdate.cmd.ISODayName}\IndexCmd{ISODayName} descritos
  acima.%
\end{Explain}%
\EndIndexGroup


\begin{Declaration}
  \Macro{nameday}\Parameter{name}
\end{Declaration}%
Assim como você pode modificar diretamente a saída de \Macro{today} com
\DescRef{maincls.cmd.date}\IndexCmd{date}, você pode alterar a saída de
\DescRef{scrdate.cmd.todaysname} para \PName{name} com \Macro{nameday}.
\begin{Example}
  Você altera a data atual para um valor fixo usando
  \DescRef{maincls.cmd.date}. Você não está interessado no nome real do
  dia, mas quer apenas mostrar que é um dia de trabalho. Então você escreve:
\begin{lstlisting}
  \nameday{workday}
\end{lstlisting}
  Depois disso, o exemplo anterior resultará em:
  \begin{ShowOutput}\nameday{workday}
    This document was created on a \todaysname.
  \end{ShowOutput}
\end{Example}
Não há comando correspondente para alterar o resultado de
\DescRef{scrdate.cmd.ISOToday}\IndexCmd{ISOToday} ou
\DescRef{scrdate.cmd.IsoToday}\IndexCmd{IsoToday}.%
\EndIndexGroup


\begin{Declaration}
  \Macro{newdaylanguage}\Parameter{language}%
                        \Parameter{Monday}\Parameter{Tuesday}%
                        \Parameter{Wednesday}\Parameter{Thursday}%
                        \Parameter{Friday}\Parameter{Saturday}
                        \Parameter{Sunday}%
\end{Declaration}
Atualmente, o pacote \Package{scrdate} reconhece os seguintes idiomas:
\begin{itemize}\setlength{\itemsep}{.5\itemsep}
\item Croata (\PValue{croatian}),
\item Tcheco (\PValue{czech}\ChangedAt{v3.13}{\Package{scrdate}}),
\item Dinamarquês (\PValue{danish}),
\item Holandês (\PValue{dutch}),
\item Inglês (\PValue{american}\ChangedAt{v3.13}{\Package{scrdate}},
  \PValue{australian}, \PValue{british}, \PValue{canadian}, \PValue{english},
  \PValue{UKenglish} e USenglish),
\item Finlandês (\PValue{finnish}),
\item Francês (\PValue{acadian}, \PValue{canadien},
  \PValue{francais}\ChangedAt{v3.13}{\Package{scrdate}} e \PValue{french}),
\item Alemão (\PValue{austrian}\ChangedAt{v3.08b}{\Package{scrdate}},
  \PValue{german}, \PValue{naustrian}, \PValue{ngerman},
  \PValue{nswissgerman} e
  \PValue{swissgerman}\ChangedAt{v3.13}{\Package{scrdate}}),
\item Italiano (\PValue{italian}),
\item Norueguês (\PValue{norsk}),
\item Polonês (\PValue{polish}\ChangedAt{v3.13}{\Package{scrdate}}),
\item Eslovaco (\PValue{slovak}),
\item Espanhol (\PValue{spanish}),
\item Sueco (\PValue{swedish}).
\end{itemize}
Você também pode configurá-lo para idiomas adicionais. Para fazer isso, o primeiro
argumento de \Macro{newdaylanguage} é o nome do idioma, e os outros
argumentos são os nomes dos dias correspondentes da semana.

Na implementação atual, não importa se você carrega
\Package{scrdate} antes ou depois de \Package{ngerman}\IndexPackage{ngerman},
\Package{babel}\IndexPackage{babel} ou pacotes semelhantes. Em qualquer caso, o
idioma correto será usado, desde que seja suportado.

\begin{Explain}
  Ser mais preciso, desde que você selecione um idioma de forma que seja
  compatível com \Package{babel}\IndexPackage{babel}, \Package{scrdate} irá
  usar o idioma correto. Se este não for o caso, você obterá nomes em
  inglês (dos EUA).
\end{Explain}

Claro, se você criar definições para um idioma que era previamente
não suportado, envie-as por e-mail ao autor do \KOMAScript{}. Há uma boa
chance de que futuras versões do \KOMAScript{} adicionem suporte para esse
idioma.%
\EndIndexGroup
%
\EndIndexGroup

\endinput

%%% Local Variables:
%%% mode: latex
%%% TeX-master: "scrguide-en.tex"
%%% coding: utf-8
%%% ispell-local-dictionary: "pt_BR"
%%% eval: (flyspell-mode 1)
%%% End:
