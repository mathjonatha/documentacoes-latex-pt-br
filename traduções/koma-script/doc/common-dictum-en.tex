% ======================================================================
% common-dictum-en.tex
% Copyright (c) Markus Kohm, 2001-2023
%
% This file is part of the LaTeX2e KOMA-Script bundle.
%
% This work may be distributed and/or modified under the conditions of
% the LaTeX Project Public License, version 1.3c of the license.
% The latest version of this license is in
%   http://www.latex-project.org/lppl.txt
% and version 1.3c or later is part of all distributions of LaTeX
% version 2005/12/01 or later and of this work.
%
% This work has the LPPL maintenance status "author-maintained".
%
% The Current Maintainer and author of this work is Markus Kohm.
%
% This work consists of all files listed in MANIFEST.md.
% ======================================================================
%
% Paragraphs that are common for several chapters of the KOMA-Script guide
% Maintained by Markus Kohm
%
% ======================================================================

\KOMAProvidesFile{common-dictum-en.tex}
                 [$Date: 2023-04-20 10:04:17 +0200 (Do, 20. Apr 2023) $
                  KOMA-Script guide (common paragraphs)]
\translator{Gernot Hassenpflug\and Markus Kohm\and Krickette Murabayashi\and
	Karl Hagen}

\section{Ditos}
\seclabel{dictum}%
\BeginIndexGroup
\BeginIndex{}{saying}%
\BeginIndex{}{dictum}%
\BeginIndex{}{epigraph}%

\IfThisCommonFirstRun{}{%
  As informações em \autoref{sec:\ThisCommonFirstLabelBase.dictum} aplicam-se
  igualmente a este capítulo. %
  \IfThisCommonLabelBase{scrextend}{%
    No entanto, \Package{scrextend} não suporta os comandos
    \DescRef{maincls.cmd.setchapterpreamble} e
    \DescRef{maincls.cmd.setpartpreamble}.
	\iftrue% Umbruchvariante
	  Se a classe que você está usando oferece uma instrução equivalente
	  pode ser encontrado na documentação da respectiva classe.%
    \fi%
  }{}%
  Portanto, se você já leu e entendeu
  \autoref{sec:\ThisCommonFirstLabelBase.dictum}, pode pular para
  \autoref{sec:\ThisCommonLabelBase.dictum.next},
  \autopageref{sec:\ThisCommonLabelBase.dictum.next}.%
}{}%

\IfThisCommonLabelBase{scrextend}{% Umbruchkorrekturvarianten
  Um elemento comum em um documento é uma epígrafe ou citação que é colocada
  acima ou abaixo de um título de capítulo ou seção, tipicamente alinhada à
  direita. A epígrafe e sua fonte são geralmente formatadas de maneira
  especial.%
}{%
  Um elemento comum em um documento é uma epígrafe ou citação que é colocada
  acima ou abaixo de um título de capítulo ou seção, juntamente com uma
  referência à fonte e sua própria formatação.%
} %
\KOMAScript{} refere-se a tal epígrafe como um \emph{dictum}.

\begin{Declaration}
  \Macro{dictum}\OParameter{autor}\Parameter{texto}%
  \Macro{dictumwidth}%
  \Macro{dictumauthorformat}\Parameter{autor}%
  \Macro{dictumrule}%
  \Macro{raggeddictum}%
  \Macro{raggeddictumtext}%
  \Macro{raggeddictumauthor}
\end{Declaration}%
Você pode definir tal ditado com a ajuda do comando \Macro{dictum}. %
\IfThisCommonLabelBase{maincls}{%
  Este\textnote{Dica!} macro pode ser incluído no argumento obrigatório de
  tanto o comando \DescRef{maincls.cmd.setchapterpreamble} quanto o
  \DescRef{maincls.cmd.setpartpreamble}. No entanto, isso não é
  obrigatório.\par%
}{}%
O dictum, juntamente com um \PName{autor} opcional, é inserido em um
\Macro{parbox}\IndexCmd{parbox} (veja \cite{latex:usrguide}) de largura
\Macro{dictumwidth}. No entanto, \Macro{dictumwidth} não é um comprimento que
possa ser definido com \Macro{setlength}. É um macro que pode ser redefinido
usando \Macro{renewcommand}. O padrão é \PValue{0.3333\Length{textwidth}}, que
é um terço da largura do texto. A própria caixa é alinhada com o comando
\Macro{raggeddictum}. O padrão é \Macro{raggedleft}\IndexCmd{raggedleft},
ou seja, alinhado à direita. \Macro{raggeddictum} pode ser redefinido
com \IfThisCommonLabelBase{scrextend}{% Umbruchoptimierung
}{a ajuda de }\Macro{renewcommand}.

Você pode alinhar o \PName{dictum} dentro da caixa usando
\Macro{raggeddictumtext}\important{\Macro{raggeddictumtext}}.
O padrão é \Macro{raggedright}\IndexCmd{raggedright}, ou seja, alinhado à
esquerda. Você também pode redefinir este macro com \Macro{renewcommand}.%
\BeginIndexGroup
\BeginIndex{FontElement}{dictum}\LabelFontElement{dictum}%
\LabelFontElement{dictumtext}%
A saída usa a configuração de fonte padrão para o elemento
\FontElement{dictum}\important{\FontElement{dictum}}, que pode ser alterada
com os comandos \DescRef{\ThisCommonLabelBase.cmd.setkomafont} e
\DescRef{\ThisCommonLabelBase.cmd.addtokomafont} (veja
\autoref{sec:\ThisCommonLabelBase.textmarkup},
\DescPageRef{\ThisCommonLabelBase.cmd.setkomafont}). As configurações padrão
estão listadas em \autoref{tab:maincls.dictumfont}%
\IfThisCommonFirstRun{.%
  \begin{table}
%  \centering%
    \KOMAoptions{captions=topbeside}%
    \setcapindent{0pt}%
%  \caption
    \begin{captionbeside}{Configurações padrão para os elementos de um dictum}
      [l]
      \begin{tabular}[t]{ll}
        \toprule
        Elemento & Padrão \\
        \midrule
        \IfThisCommonLabelBase{maincls}{%
          \ChangedAt{v3.39}{\Class{scrbook}\and \Class{scrreprt}\and
            \Class{scrartcl}}%
        }{%
          \IfThisCommonLabelBase{scrextend}{%
            \ChangedAt{v3.39}{\Package{screxend}}%
          }{%
            \IfThisCommonLabelBase{scrlttr2}{%
              \ChangedAt{v3.39}{\Class{scrlttr2}}%
            }{}%
          }%
        }%
        \DescRef{\LabelBase.fontelement.dictum}
                & \Macro{normalfont}\Macro{normalcolor}%
                  \DescRef{\LabelBase.cmd.maybesffamily}\IndexCmd{maybesffamily}%
                  \Macro{small}\\
        \DescRef{\LabelBase.fontelement.dictumauthor} &
          \Macro{itshape}\\
        \bottomrule
      \end{tabular}
    \end{captionbeside}
    \label{tab:\ThisCommonLabelBase.dictumfont}
  \end{table}
}{%
  , \autopageref{tab:\ThisCommonFirstLabelBase.dictumfont}.%
}
\EndIndexGroup

Se um \PName{autor} for definido, ele é separado do \PName{dictum} por
uma régua horizontal que se estende por toda a largura do \Macro{parbox}.
Esta%
\IfThisCommonLabelBase{maincls}{%
  \ChangedAt{v3.10}{\Class{scrbook}\and \Class{scrreprt}\and
    \Class{scrartcl}}%
}{%
  \IfThisCommonLabelBase{scrextend}{%
    \ChangedAt{v3.10}{\Package{scrextend}}%
  }{\InternalCommonFileUsageError}%
} %
régua é definida em \Macro{dictumrule}\important{\Macro{dictumrule}} como um
objeto vertical com
\begin{lstcode}
  \newcommand*{\dictumrule}{\vskip-1ex\hrulefill\par}
\end{lstcode}

O comando \Macro{raggeddictumauthor}\important{\Macro{raggeddictumauthor}}
define o alinhamento para a régua e o \PName{autor}. O padrão é
\Macro{raggedleft}. Este comando também pode ser redefinido usando
\Macro{renewcommand}. O formato é definido com
\Macro{dictumauthorformat}\important{\Macro{dictumauthorformat}}. Este macro
espera o texto do \PName{autor} como seu argumento. Por padrão
\Macro{dictumauthorformat} é definido como
\begin{lstcode}
  \newcommand*{\dictumauthorformat}[1]{(#1)}
\end{lstcode}
Assim, o \PName{autor} é definido entre parênteses arredondados.
\BeginIndexGroup
\BeginIndex{FontElement}{dictumauthor}\LabelFontElement{dictumauthor}%
Para o elemento \FontElement{dictumauthor}\important{\FontElement{dictumauthor}},
você pode definir uma fonte diferente daquela usada para o elemento
\DescRef{\LabelBase.fontelement.dictum}. As configurações padrão estão
listadas em \autoref{tab:maincls.dictumfont}. Alterações podem ser feitas
usando os comandos \DescRef{\ThisCommonLabelBase.cmd.setkomafont} e
\DescRef{\ThisCommonLabelBase.cmd.addtokomafont} (veja
\autoref{sec:\ThisCommonLabelBase.textmarkup},
\DescPageRef{\ThisCommonLabelBase.cmd.setkomafont}).%
\EndIndexGroup

\IfThisCommonLabelBase{maincls}{%
  Se você usar \Macro{dictum} dentro do macro
  \DescRef{maincls.cmd.setchapterpreamble} ou
  \DescRef{maincls.cmd.setpartpreamble}, \Length{textwidth} não é a
  largura de todo o corpo de texto, mas a largura do texto atual. Se
  \Macro{dictumwidth} for definido como \PValue{.5\Length{textwidth}} e
  \DescRef{maincls.cmd.setchapterpreamble} tiver uma largura opcional de
  \PValue{.5\Length{textwidth}} também, você obterá uma caixa com uma largura
  de um quarto da largura do texto. O posicionamento horizontal do dictum
  dentro da caixa é sempre feito com \Macro{raggeddictum}. O argumento
  opcional para posicionamento horizontal que é implementado para esses dois
  comandos não tem efeito sobre o \Macro{text}. Se você usar \Macro{dictum},
  deve evitar definir a largura opcional para
  \DescRef{maincls.cmd.setchapterpreamble} ou
  \DescRef{maincls.cmd.setpartpreamble}.

  Se\textnote{Dica!} você tiver mais de um dictum, um abaixo do outro, você
  deve separá-los por um espaço vertical adicional, o que é facilmente
  realizado usando o comando \Macro{bigskip}\IndexCmd{bigskip}.%
  \iftrue%
}{\csname iffalse\endcsname}

 \begin{Example}
   Você está escrevendo um capítulo sobre casamento moderno e deseja colocar
   um dictum no preâmbulo do título do capítulo. Você escreve:
\begin{lstcode}
  \setchapterpreamble[u]{%
    \dictum[Schiller]{Então pause você que ligaria seus destinos~\dots}}
  \chapter{Casamento Moderno}
\end{lstcode}
  A saída ficaria assim:
  \begin{ShowOutput}
    {\usekomafont{disposition}\usekomafont{chapter}%
      17\enskip Casamento Moderno\par} \vspace{\baselineskip}
    \dictum[Schiller]{Então pause você que ligaria
      seus destinos~\dots}
  \end{ShowOutput}

  Se você quiser que o dictum ocupe apenas um quarto da largura do texto
  em vez de um terço, pode redefinir \Macro{dictumwidth} desta forma:
\begin{lstcode}
  \renewcommand*{\dictumwidth}{.25\textwidth}
\end{lstcode}
\end{Example}

\IfThisCommonLabelBase{maincls}{}{% Umbruchkorrekturtext
  Neste ponto, note o pacote \Package{ragged2e}\important{\Package{ragged2e}}%
  \IndexPackage{ragged2e}, com o qual você pode produzir texto não justificado
  que usa hifenização (veja \cite{package:ragged2e}).%
}%
\fi
%
\EndIndexGroup
%
\EndIndexGroup

%%% Local Variables:
%%% mode: latex
%%% TeX-master: "scrguide-en.tex"
%%% coding: utf-8
%%% ispell-local-dictionary: "en_GB"
%%% eval: (flyspell-mode 1)
%%% End:
