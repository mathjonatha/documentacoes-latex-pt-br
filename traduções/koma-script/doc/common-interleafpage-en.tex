% ======================================================================
% common-interleafpage-en.tex
% Copyright (c) Markus Kohm, 2001-2022
%
% This file is part of the LaTeX2e KOMA-Script bundle.
%
% This work may be distributed and/or modified under the conditions of
% the LaTeX Project Public License, version 1.3c of the license.
% The latest version of this license is in
%   http://www.latex-project.org/lppl.txt
% and version 1.3c or later is part of all distributions of LaTeX
% version 2005/12/01 or later and of this work.
%
% This work has the LPPL maintenance status "author-maintained".
%
% The Current Maintainer and author of this work is Markus Kohm.
%
% This work consists of all files listed in MANIFEST.md.
% ======================================================================
%
% Paragraphs that are common for several chapters of the KOMA-Script guide
% Maintained by Markus Kohm
%
% ======================================================================

\KOMAProvidesFile{common-interleafpage-en.tex}%
                 [$Date: 2022-06-05 12:40:11 +0200 (So, 05. Jun 2022) $
                  KOMA-Script guide (common paragraphs: Interleaf Pages)]
\translator{Markus Kohm\and Gernot Hassenpflug\and Krickette Murabayashi\and
	Karl Hagen}

\section{Páginas Intermediárias}
\seclabel{emptypage}%
\BeginIndexGroup
\BeginIndex{}{interleaf page}%%
\BeginIndex{}{page>style}%

\IfThisCommonFirstRun{}{%
  As informações em \autoref{sec:\ThisCommonFirstLabelBase.emptypage} aplicam-se
  igualmente a este capítulo. Portanto, se você já leu e compreendeu
  \autoref{sec:\ThisCommonFirstLabelBase.emptypage}, você pode pular para
  \autoref{sec:\ThisCommonLabelBase.emptypage.next},
  \autopageref{sec:\ThisCommonLabelBase.emptypage.next}.%
}

\IfThisCommonLabelBase{scrextend}{}{%
  Páginas intermediárias são páginas que são inseridas entre partes de um documento.
  Tradicionalmente, essas páginas são completamente em branco. O \LaTeX{}, no entanto, as define
  por padrão com o estilo de página atual. O \KOMAScript{} fornece várias
  extensões a essa funcionalidade.

  Páginas intermediárias são encontradas principalmente em livros. Como os capítulos de livros normalmente
  começam na página direita (recto) de uma dupla de páginas, uma página esquerda (verso)
  vazia deve ser inserida se o capítulo anterior terminar em uma página recto. Por essa
  razão, as páginas intermediárias realmente existem apenas para impressão em frente e verso.
%
  \iffalse % Umbruchkorrektur
  Os versos em branco na impressão de um lado não são verdadeiras páginas intermediárias,
  embora possam aparecer como tal na contagem das folhas impressas.%
  \fi%

  \IfThisCommonLabelBase{scrlttr2}{%
    Páginas intermediárias são incomuns em cartas. Isso se deve não apenas ao fato de que
    cartas em frente e verso são raras, já que as cartas geralmente não são
    encadernadas. No entanto, o \KOMAScript{} também suporta páginas intermediárias para
    cartas em frente e verso. Entretanto, como os comandos descritos aqui são raramente
    usados em cartas, você não encontrará nenhum exemplo aqui. Se necessário, consulte
    os exemplos em \autoref{sec:maincls.emptypage}, começando em
    \autopageref{sec:maincls.emptypage}.%
  }{}%
}%

\begin{Declaration}
  \OptionVName{cleardoublepage}{page style}
  \OptionValue{cleardoublepage}{current}
\end{Declaration}%
Com esta opção,\IfThisCommonLabelBase{maincls}{%
  \ChangedAt{v3.00}{\Class{scrbook}\and \Class{scrreprt}\and
    \Class{scrartcl}}%
}{%
  \IfThisCommonLabelBase{scrlttr2}{%
    \ChangedAt{v3.00}{\Class{scrlttr2}}%
  }{}%
} você pode definir o estilo de página das páginas intermediárias criadas pelos comandos
\DescRef{\LabelBase.cmd.cleardoublepage},
\DescRef{\LabelBase.cmd.cleardoubleoddpage}, ou
\DescRef{\LabelBase.cmd.cleardoubleevenpage} para avançar para a página desejada.
Você pode usar qualquer \PName{page style} previamente definido (veja
\autoref{sec:\ThisCommonLabelBase.pagestyle} de
\autopageref{sec:\ThisCommonLabelBase.pagestyle} e
\autoref{cha:scrlayer-scrpage} de \autopageref{cha:scrlayer-scrpage}).
Além disso, \OptionValue{cleardoublepage}{current} também é possível.
Este caso corresponde ao padrão anterior ao \KOMAScript~2.98c e cria uma
página intermediária sem alterar o estilo de página. A partir do
\KOMAScript~3.00\IfThisCommonLabelBase{maincls}{%
  \ChangedAt{v3.00}{\Class{scrbook}\and \Class{scrreprt}\and
    \Class{scrartcl}}%
}{%
  \IfThisCommonLabelBase{scrlttr2}{%
    \ChangedAt{v3.00}{\Class{scrlttr2}}%
  }{}%
}, o padrão\textnote{default} segue a recomendação da maioria dos
tipógrafos e cria páginas intermediárias com o
\IfThisCommonLabelBase{scrextend}{%
  \DescRef{maincls.pagestyle.empty}}{%
  \DescRef{\ThisCommonLabelBase.pagestyle.empty}}\IndexPagestyle{empty}
estilo de página, a menos que você mude a compatibilidade para versões anteriores do \KOMAScript{}
(veja a opção \DescRef{\ThisCommonLabelBase.option.version}%
\important{\OptionValueRef{\LabelBase}{version}{2.98c}},
\autoref{sec:\ThisCommonLabelBase.compatibilityOptions},
\DescPageRef{\ThisCommonLabelBase.option.version}).
\IfThisCommonLabelBase{maincls}{\iftrue}{\csname iffalse\endcsname}
  \begin{Example}
    \phantomsection\xmpllabel{option.cleardoublepage}%
    Suponha que você queira páginas intermediárias que estejam vazias exceto pela paginação%
    \iffree{, de modo que sejam criadas com \IfThisCommonLabelBase{scrextend}{%
        \DescRef{maincls.pagestyle.plain}}{%
        \DescRef{\LabelBase.pagestyle.plain}}}{}. Você pode conseguir isso,
    por exemplo, com:
\begin{lstcode}
  \KOMAoptions{cleardoublepage=plain}
\end{lstcode}
    Você pode encontrar mais informações sobre o
    \IfThisCommonLabelBase{scrextend}{%
      \DescRef{maincls.pagestyle.plain}}{\DescRef{\LabelBase.pagestyle.plain}}
    estilo de página em \IfThisCommonLabelBase{scrextend}{%
      \autoref{sec:maincls.pagestyle}}{%
      \autoref{sec:\LabelBase.pagestyle}},
    \IfThisCommonLabelBase{scrextend}{%
      \DescPageRef{maincls.pagestyle.plain}}{%
      \DescPageRef{\LabelBase.pagestyle.plain}}.
  \end{Example}
\else
  \IfThisCommonLabelBase{scrextend}{%
    Você pode encontrar um exemplo para configurar o estilo de página das páginas intermediárias em
    \autoref{sec:\ThisCommonFirstLabelBase.emptypage},
    \PageRefxmpl{\ThisCommonFirstLabelBase.option.cleardoublepage}.%
    \iffalse% Umbruchvariante ohne Beispiel
  }{\csname iffalse\endcsname}
    \begin{Example}
      \phantomsection\xmpllabel{option.cleardoublepage}%
      Suponha que você queira páginas intermediárias que estejam vazias exceto pela paginação,
      de modo que sejam criadas com o \IfThisCommonLabelBase{scrextend}{%
        \DescRef{maincls.pagestyle.plain}}{\DescRef{\LabelBase.pagestyle.plain}}
      estilo de página. Você pode conseguir isso com
\begin{lstcode}
  \KOMAoptions{cleardoublepage=plain}
\end{lstcode}
      Você pode encontrar mais informações sobre o
      \DescRef{maincls.pagestyle.plain} estilo de página em
      \autoref{sec:maincls.pagestyle}, \DescPageRef{maincls.pagestyle.plain}.
    \end{Example}%
  \fi%
\fi%
\EndIndexGroup


\begin{Declaration}
  \Macro{clearpage}%
  \Macro{cleardoublepage}%
  \Macro{cleardoublepageusingstyle}\Parameter{page style}%
  \Macro{cleardoubleemptypage}%
  \Macro{cleardoubleplainpage}%
  \Macro{cleardoublestandardpage}%
  \Macro{cleardoubleoddpage}%
  \Macro{cleardoubleoddpageusingstyle}\Parameter{page style}%
  \Macro{cleardoubleoddemptypage}%
  \Macro{cleardoubleoddplainpage}%
  \Macro{cleardoubleoddstandardpage}%
  \Macro{cleardoubleevenpage}%
  \Macro{cleardoubleevenpageusingstyle}\Parameter{page style}%
  \Macro{cleardoubleevenemptypage}%
  \Macro{cleardoubleevenplainpage}%
  \Macro{cleardoubleevenstandardpage}
\end{Declaration}%
O\textnote{standard classes} kernel do {\LaTeX} fornece o comando \Macro{clearpage},
que garante que todos os floats pendentes sejam produzidos e então inicia uma
nova página. Há também o comando \Macro{cleardoublepage}, que funciona como
\Macro{clearpage}, mas que inicia uma nova página da direita na impressão em frente e verso
(veja a opção de layout \Option{twoside} em \autoref{sec:typearea.options},
\DescPageRef{typearea.option.twoside}). Uma página esquerda vazia no estilo de
página atual é produzida se necessário.

Com\IfThisCommonLabelBase{maincls}{%
  \ChangedAt{v3.00}{\Class{scrbook}\and \Class{scrreprt}\and
    \Class{scrartcl}}%
}{%
  \IfThisCommonLabelBase{scrlttr2}{%
    \ChangedAt{v3.00}{\Class{scrlttr2}}%
  }{}%
} \Macro{cleardoubleoddstandardpage}, {\KOMAScript}\textnote{\KOMAScript}
funciona exatamente da maneira descrita para as classes padrão. O comando
\Macro{cleardoubleoddplainpage}%
\important{\IfThisCommonLabelBase{scrextend}{%
    \DescRef{maincls.pagestyle.plain}}{\DescRef{\LabelBase.pagestyle.plain}}},
por outro lado, adicionalmente altera o estilo de página da página
esquerda vazia para \IfThisCommonLabelBase{scrextend}{%
  \DescRef{maincls.pagestyle.plain}}{\DescRef{\LabelBase.pagestyle.plain}}%
\IndexPagestyle{plain} a fim de suprimir o
\IfThisCommonLabelBase{scrlttr2}{cabeçalho da página}{título corrente}. Da mesma forma, o
comando \Macro{cleardoubleoddemptypage}\important{%
  \IfThisCommonLabelBase{scrextend}{\DescRef{maincls.pagestyle.empty}}{%
    \DescRef{\LabelBase.pagestyle.empty}}} usa o
\IfThisCommonLabelBase{scrextend}{\DescRef{maincls.pagestyle.empty}}{%
  \DescRef{\LabelBase.pagestyle.empty}}\IndexPagestyle{empty} estilo de página para
suprimir tanto o \IfThisCommonLabelBase{scrlttr2}{cabeçalho quanto o rodapé da página}%
{título corrente e o número da página} no lado esquerdo vazio. A página fica assim
completamente vazia. Se você quiser especificar seu próprio \PName{page style} para a
página intermediária, este deve ser fornecido como um argumento de
\Macro{cleardoubleoddusingpagestyle}. Você pode usar qualquer \PName{page style}
previamente definido (veja \autoref{cha:scrlayer-scrpage}).

\IfThisCommonLabelBase{scrlttr2}{}{%
  Às vezes\textnote{odd-side interleaf pages} você quer que os capítulos comecem não
  na página da direita, mas na página da esquerda. Embora esse layout contradiga
  a tipografia clássica, pode ser apropriado se a dupla de páginas no
  início do capítulo tiver conteúdos muito específicos. Por essa razão,
  o \KOMAScript{} fornece o comando \Macro{cleardoubleevenstandardpage},
  que é equivalente ao comando \Macro{cleardoubleoddstandardpage}
  exceto que a próxima página é uma página esquerda. O mesmo se aplica aos
  comandos \Macro{cleardoubleevenplainpage}, \Macro{cleardoubleevenemptypage}, e
  \Macro{cleardoubleevenpageusingstyle}%
  \IfThisCommonLabelBase{maincls}{% Umbruchoptimierungsalternative
  , sendo que o último espera um argumento}{}.%
}

Os comandos \Macro{cleardoublestandardpage}, \Macro{cleardoubleemptypage}, e
\Macro{cleardoubleplainpage}, e o comando com um único argumento
\Macro{cleardoublepageusingstyle}, assim como o comando padrão
\Macro{cleardoublepage}, %
\IfThisCommonLabelBase{maincls}{%
  dependem da opção \DescRef{maincls.option.open}\IndexOption{open}%
  \important{\DescRef{maincls.option.open}} explicada em
  \autoref{sec:maincls.structure}, \DescPageRef{maincls.option.open} e,
  dependendo dessa configuração, correspondem a um dos comandos explicados nos
  parágrafos anteriores.  }{%
  correspondem aos comandos anteriormente explicados para a
  \IfThisCommonLabelBase{scrlttr2}{classe \Class{scrlttr2}}{%
    \IfThisCommonLabelBase{scrextend}{pacote \Package{scrextend}}{%
      \InternalCommonFileUsageError}%
  }%
  \IfThisCommonLabelBase{scrlttr2}{. %
    Os comandos restantes são definidos na \Class{scrlttr2} apenas por completude.
    Você pode encontrar mais informações sobre eles em
    \autoref{sec:maincls.emptypage}, \DescPageRef{maincls.cmd.cleardoublepage}
    se necessário%
  }{%
    \ para transição para a próxima página ímpar%
  }.%
}%
\IfThisCommonLabelBaseOneOf{scrlttr2,scrextend}{\iffalse}{\csname
  iftrue\endcsname}
  \begin{Example}
    \phantomsection\xmpllabel{cmd.cleardoublepage}%
    Suponha que você queira inserir uma dupla de páginas em seu documento com uma
    imagem na página esquerda e um novo capítulo começando na página direita.
    Se o capítulo anterior termina com uma página esquerda, uma página intermediária
    tem que ser adicionada, que deve estar completamente vazia. A imagem deve
    ter o mesmo tamanho que a área de texto sem nenhum cabeçalho ou rodapé.
\iffalse% Umbruchkorrekturtext
      Primeiramente,
\begin{lstcode}
  \KOMAoptions{cleardoublepage=empty}
\end{lstcode}
      garante que as páginas intermediárias usem o
      \IfThisCommonLabelBase{scrextend}{\DescRef{maincls.pagestyle.empty}}{%
    	\DescRef{\LabelBase.pagestyle.empty}} estilo de página. Você pode colocar esta
      configuração no preâmbulo do documento, ou você pode defini-la como um argumento
      opcional de \DescRef{\ThisCommonLabelBase.cmd.documentclass}.
\fi

    No local relevante em seu documento, escreva:
\begin{lstcode}
  \cleardoubleevenemptypage
  \thispagestyle{empty}
  \includegraphics[width=\textwidth,%
                   height=\textheight,%
                   keepaspectratio]%
                  {picture}
  \chapter{Chapter Heading}
\end{lstcode}
    A primeira dessas linhas muda para a próxima página esquerda. Se necessário,
    também adiciona uma página direita completamente em branco. A segunda linha garante
    que a página esquerda seguinte também será definida usando o
    \IfThisCommonLabelBase{scrextend}{\DescRef{maincls.pagestyle.empty}}{%
    	\DescRef{\LabelBase.pagestyle.empty}} estilo de página. A terceira até a
    sexta linhas carregam um arquivo de imagem chamado \File{picture} e o dimensionam para o
    tamanho desejado sem distorcê-lo. Este comando requer o
    pacote \Package{graphicx}\IndexPackage{graphicx} (veja
    \cite{package:graphics}). A última linha inicia
    um novo capítulo na próxima página, que será uma página direita.
  \end{Example}%
\fi%

Na impressão em frente e verso, \Macro{cleardoubleoddpage} sempre move para a próxima
página esquerda e \Macro{cleardoubleevenpage} para a próxima página direita.
O estilo da página intermediária a ser inserida se necessário é definido
com a opção \DescRef{\LabelBase.option.cleardoublepage}.%
\IfThisCommonLabelBase{scrextend}{\par%
  Para um exemplo que usa \Macro{cleardoubleevenemptypage}, veja
  \autoref{sec:maincls.emptypage},
  \PageRefxmpl{\ThisCommonFirstLabelBase.cmd.cleardoublepage}.%
}{}%
\EndIndexGroup
%
\EndIndexGroup

%%% Local Variables:
%%% mode: latex
%%% TeX-master: "scrguide-en.tex"
%%% coding: utf-8
%%% ispell-local-dictionary: "en_GB"
%%% eval: (flyspell-mode 1)
%%% End:

%  LocalWords:  mutatis mutandis Interleaf interleaf
