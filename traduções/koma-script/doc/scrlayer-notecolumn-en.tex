% ======================================================================
% scrlayer-notecolumn-en.tex
% Copyright (c) Markus Kohm, 2013-2024
%
% This file is part of the LaTeX2e KOMA-Script bundle.
%
% This work may be distributed and/or modified under the conditions of
% the LaTeX Project Public License, version 1.3c of the license.
% The latest version of this license is in
%   http://www.latex-project.org/lppl.txt
% and version 1.3c or later is part of all distributions of LaTeX 
% version 2005/12/01 or later and of this work.
%
% This work has the LPPL maintenance status "author-maintained".
%
% The Current Maintainer and author of this work is Markus Kohm.
%
% This work consists of all files listed in MANIFEST.md.
% ======================================================================
%
% Chapter about scrlayer-notecolumn of the KOMA-Script guide
% Maintained by Markus Kohm
%
% ============================================================================

\KOMAProvidesFile{scrlayer-notecolumn-en.tex}
                 [$Date: 2018-02-05 01:50:56 -0800 (Mon, 05 Feb 2018) $
                  KOMA-Script guide (chapter:scrlayer-notecolumn)]

\translator{Markus Kohm\and Arndt Schubert\and Karl Hagen}

\chapter{Colunas de Notas com \Package{scrlayer-notecolumn}}
\labelbase{scrlayer-notecolumn}

\BeginIndexGroup
\BeginIndex{Package}{scrlayer-notecolumn}%
Até a versão~3.11b, \KOMAScript{} suportava colunas de notas apenas na forma
de notas marginais que recebem seu conteúdo de \DescRef{maincls.cmd.marginpar}
e \DescRef{maincls.cmd.marginline} (veja \autoref{sec:maincls.marginNotes},
\DescPageRef{maincls.cmd.marginline}). Este tipo de coluna de notas tem várias
desvantagens:
\begin{itemize}
\item As notas marginais devem ser definidas completamente em uma única página. Quebras
  de página\textnote{page break} dentro de notas marginais não são possíveis. Isto
  às vezes faz com que as notas marginais se estendam para a margem inferior.
\item As notas marginais próximas a quebras de página às vezes fluem para a próxima página e,
  no caso de impressão dupla, causam colunas marginais alternadas que
  aparecem na margem errada.\textnote{assignment to the correct margin}. Este
  problema pode ser resolvido com o pacote adicional
  \Package{mparhack}\IndexPackage{mparhack} ou usando \Macro{marginnote}
  do pacote \Package{marginnote}\IndexPackage{marginnote} (veja
    \cite{package:marginnote}).
\item Notas marginais dentro de ambientes flutuantes\textnote{floating
    environments} ou notas de rodapé\textnote{footnotes} não são possíveis. Este
  problema também pode ser resolvido com \Macro{marginnote} do
  pacote \Package{marginnote}\IndexPackage{marginnote}.
\item Existe apenas uma coluna de notas marginais\textnote{several note columns},
  ou no máximo duas se você usar \Macro{reversemarginpar} e
  \Macro{normalmarginpar}. Note que \Macro{reversemarginpar} é de menor
  utilidade com documentos dupla-face.
\end{itemize}
Usar \Package{marginnote}\IndexPackage{marginnote} leva a mais um problema.
Como o pacote não tem detecção de colisão, notas marginais que
são definidas próximas uma da outra podem se sobrepor parcial ou totalmente. Além disso,
dependendo das configurações usadas, \Macro{marginnote} às vezes altera o
espaçamento de linha do texto normal.

O pacote \Package{scrlayer-notecolumn} deve resolver todos esses problemas. Para
fazer isso, ele se baseia na funcionalidade básica de
\hyperref[cha:scrlayer]{\Package{scrlayer}}\IndexPackage{scrlayer}%
\important{\hyperref[cha:scrlayer]{\Package{scrlayer}}}. No entanto, usar este
pacote tem uma desvantagem:\textnote{Attention!} você pode apenas exibir notas em
páginas que usam um estilo de página baseado em
\hyperref[cha:scrlayer]{\Package{scrlayer}}. Esta desvantagem, porém, pode
ser facilmente resolvida, ou até mesmo transformada em uma vantagem, com a ajuda de
\hyperref[cha:scrlayer-scrpage]{\Package{scrlayer-scrpage}}%
\important{\hyperref[cha:scrlayer-scrpage]{\Package{scrlayer-scrpage}}}%
\IndexPackage{scrlayer-scrpage}.

\section{Observação sobre o Estado do Desenvolvimento}
\seclabel{draft}

Este pacote foi originalmente desenvolvido como um assim chamado \emph{prova de
  conceito}\textnote{proof of concept} para demonstrar o potencial de
\hyperref[cha:scrlayer]{\Package{scrlayer}}%
\important{\hyperref[cha:scrlayer]{\Package{scrlayer}}}. Embora ainda esteja
em seus estágios iniciais de desenvolvimento, a maior parte de sua estabilidade é menos uma questão
de \Package{scrlayer-notecolumn} do que de
\hyperref[cha:scrlayer]{\Package{scrlayer}}. No entanto, você pode assumir que
ainda há bugs em \Package{scrlayer-notecolumn}. Por favor, reporte tais bugs
sempre que encontrá-los. Algumas das deficiências do pacote são causadas pela
tentativa de minimizar a complexidade. Por exemplo, embora colunas de notas possam se quebrar
em várias páginas, não há quebra de parágrafo nova. O próprio \TeX{} não
oferece isso.

Como o pacote é bastante experimental\textnote{experimental}, suas
instruções são encontradas aqui na segunda parte de \iffree{do manual \KOMAScript{}
}{deste livro}. Portanto, é principalmente direcionado a usuários
experientes. Se você é um iniciante ou um usuário a caminho de se tornar um
especialista\textnote{for experts}, algumas das seguintes explicações podem ser
pouco claras ou até mesmo incompreensíveis.
\iffree{Por favor, entenda que quero manter o esforço gasto no manual
  a algo razoavelmente suportável quando se trata de pacotes experimentais.}{%
  No entanto, isto deve ser suficiente para resolver tarefas simples com
  \Package{scrlayer-notecolumn}. Ao mesmo tempo, esperadamente desenvolvedores
  encontrarão estímulo útil para suas próprias ideias.}

\iffalse% Umbruchoptimierung
Note também\textnote{Attention!} que o autor de \KOMAScript{} não
garante o desenvolvimento posterior deste pacote e oferece apenas suporte
limitado para ele. Esta é a natureza de um pacote escrito unicamente para
fins de demonstração.%
\fi

\LoadCommonFile{options}% \section{Early or late Selection of Options}

\LoadCommonFile{textmarkup}% \section{Text Markup}

\section{Declarando Novas Colunas de Notas}
\seclabel{declaration}

Carregar o pacote declara automaticamente uma coluna de notas nomeada
\PValue{marginpar}. Como o nome sugere, essa coluna de notas é colocada na
área da coluna marginal normal usada por \DescRef{maincls.cmd.marginpar} e
\DescRef{maincls.cmd.marginline}. As configurações \Macro{reversemarginpar} e
\Macro{normalmarginpar} também são levadas em conta, mas apenas para todas
as notas em uma página em vez de nota por nota. A configuração relevante é aquela
que se aplica no final da página, nomeadamente durante a saída da coluna de
notas. Se você deseja ter notas em ambas as margens esquerda e direita da
mesma página, você deve definir uma segunda coluna de notas.

As configurações padrão para todas as colunas de notas recém-declaradas são as mesmas que os
padrões para \PValue{marginpar}. %
\iftrue% Umbruchoptimierung
Mas você pode facilmente alterá-las durante sua inicialização.%
\fi

Note\textnote{Attention!} que colunas de notas podem ser exibidas apenas em páginas que
usam um estilo de página baseado no pacote
\hyperref[cha:scrlayer]{\Package{scrlayer}}\IndexPackage{scrlayer}%
\important{\hyperref[cha:scrlayer]{\Package{scrlayer}}}. O
pacote \Package{scrlayer-notecolumn} carrega automaticamente
\hyperref[cha:scrlayer]{\Package{scrlayer}}, que por padrão fornece apenas
o estilo de página \PageStyle{empty}\IndexPagestyle{empty}. Se você precisa de estilos de
página adicionais, \hyperref[cha:scrlayer-scrpage]{\Package{scrlayer-scrpage}}%
\IndexPackage{scrlayer-scrpage}%
\important{\hyperref[cha:scrlayer-scrpage]{\Package{scrlayer-scrpage}}} é
recomendado.

\begin{Declaration}
  \Macro{DeclareNoteColumn}%
  \OParameter{option~list}\Parameter{note~column~name}%
  \Macro{DeclareNewNoteColumn}%
  \OParameter{option~list}\Parameter{note~column~name}%
  \Macro{ProvideNoteColumn}%
  \OParameter{option~list}\Parameter{note~column~name}%
  \Macro{RedeclareNoteColumn}%
  \OParameter{option~list}\Parameter{note~column~name}%
\end{Declaration}
Você pode usar estes comandos para criar colunas de notas. \Macro{DeclareNoteColumn}
cria a coluna de notas independentemente de ela já existir ou não.
\Macro{DeclareNewNoteColumn} gera um erro se o \PName{note column name}
já foi usado para outra coluna de notas. \Macro{ProvideNoteColumn}
simplesmente não faz nada nesse caso. Você pode usar \Macro{RedeclareNoteColumn} apenas
para reconfigurar uma coluna de notas existente.

Aliás, ao reconfigurar colunas de notas existentes com
\Macro{DeclareNoteColumn} ou \Macro{RedeclareNoteColumn}, as notas que já
foram geradas para essa coluna são retidas.

\BeginIndex{FontElement}{notecolumn.\PName{note column name}}%
\BeginIndex{FontElement}{notecolumn.marginpar}%
Declarar uma nova coluna de notas sempre define um novo elemento para alterar seus atributos de fonte
com \DescRef{\LabelBase.cmd.setkomafont} e
\DescRef{\LabelBase.cmd.addtokomafont}, se tal elemento ainda não existir.
O nome do elemento é \PValue{notecolumn.}\PName{note column name}. Por
esta razão, a coluna de notas padrão \PValue{marginnote} tem o
elemento\textnote{element name} \FontElement{notecolumn.marginpar}. Você pode
especificar diretamente a configuração inicial da fonte do elemento ao declarar uma
coluna de notas usando a opção \Option{font} dentro de \PName{option list}.%
\EndIndex{FontElement}{notecolumn.marginpar}%
\EndIndex{FontElement}{notecolumn.\PName{note column name}}%

A \PName{option list} é uma lista de chaves separadas por vírgulas com ou sem
valores, também conhecida como opções. As opções disponíveis são listadas em
\autoref{tab:scrlayer-notecolumn.note.column.options},
\autopageref{tab:scrlayer-notecolumn.note.column.options}.
A\textnote{default: option \Option{marginpar}} opção \Option{marginpar} é
definida por padrão, mas você pode sobrescrever este padrão com suas configurações
individuais.

Como as colunas de notas são implementadas usando \Package{scrlayer}, uma
camada\Index{layer} é criada para cada coluna de notas. O nome da
camada\textnote{layer name} é o mesmo que o nome do elemento,
\PValue{notecolumn.}\PName{note column name}. Para mais informações sobre
camadas, veja \autoref{sec:scrlayer.layers}, começando em
\autopageref{sec:scrlayer.layers}.
%
\begin{Example}
  Suponha que você seja um professor de direito da comédia e queira escrever um tratado sobre
  o novo ``Estatuto Sobre a Exibição Tumultuada do Humor Comum'', SCRACH
  para abreviar. A maior parte do trabalho consistirá de comentários sobre
  parágrafos individuais do estatuto. Você decide por um layout de duas colunas,
  com os comentários na coluna principal e os parágrafos colocados em uma coluna de
  notas menor à direita da coluna principal usando uma fonte que é
  menor\iffree{ e em uma cor diferente}{}.
\begin{lstcode}
  \documentclass{scrartcl}
  \usepackage{lmodern}
  \usepackage{xcolor}

  \usepackage{contract}
  \setkomafont{contract.Clause}{\bfseries}
  \setkeys{contract}{preskip=-\dp\strutbox}

  \usepackage{scrlayer-scrpage}
  \usepackage{scrlayer-notecolumn}

  \newlength{\paragraphscolwidth}
  \AfterCalculatingTypearea{%
    \setlength{\paragraphscolwidth}{%
      .333\textwidth}%
    \addtolength{\paragraphscolwidth}{%
      -\marginparsep}%
  }
  \recalctypearea
  \DeclareNewNoteColumn[%
    position=\oddsidemargin+1in
             +.667\textwidth
             +\marginparsep,
    width=\paragraphscolwidth,
    font=\raggedright\footnotesize
         \color{blue}
  ]{paragraphs}
\end{lstcode}
  O tratado deve ser um artigo de uma única página. A fonte é Latin Modern, e
  a seleção de cores usa o pacote \Package{xcolor}\IndexPackage{xcolor}.

  Para formatar textos legais\textnote{legal texts with
  \href{https://www.ctan.org/pkg/contract}{\Package{contract}}} com o
  pacote \href{https://www.ctan.org/pkg/contract}{\Package{contract}}\IndexPackage{contract},
  veja o manual do usuário do pacote.

  Como este documento usa um estilo de página\textnote{page style with
    \hyperref[cha:scrlayer-scrpage]{\Package{scrlayer-scrpage}}} com um
  número de página, o
  pacote \hyperref[cha:scrlayer-scrpage]{\Package{scrlayer-scrpage}}%
  \IndexPackage{scrlayer-scrpage} é carregado. Assim, colunas de notas podem ser
  exibidas em todas as páginas.

  Em seguida, o pacote \Package{scrlayer-notecolumn}\textnote{note columns with
  \Package{scrlayer-notecolumn}} é carregado. A largura necessária da
  coluna de notas é calculada com
  \DescRef{typearea-experts.cmd.AfterCalculatingTypearea}%
  \IndexPackage{typearea}\IndexCmd{AfterCalculatingTypearea} após qualquer
  recalculação\textnote{type area with
  \hyperref[cha:typearea]{\Package{typearea}}}%
  \IndexPackage{typearea} da área de tipo. Ela deve ser um terço da área de
  tipo menos a distância entre o texto principal e a coluna de notas. %

  Com esta informação, definimos a nova coluna de notas. Para as posições, usamos
  uma expressão de dimensão simples. Note que \Length{oddsidemargin} não é
  a margem esquerda total mas, por razões históricas, a margem esquerda menos
  1\Unit{inch}. Então temos que adicionar este valor.

  Isto conclui a definição. Note que a coluna de notas seria colocada atualmente
  dentro da área de tipo. Isto significa que a coluna de notas sobrescreveria
  o texto.

\begin{lstcode}
  \begin{document}

  \title{Comentário sobre o SCRACH}
  \author{Professor R. O. Tenase}
  \date{11/11/2011}
  \maketitle
  \tableofcontents

  \section{Preâmbulo}
  O SCRACH é sem dúvida a lei mais importante
  sobre as maneiras do humor que foi aprovada
  nos últimos mil anos. A primeira leitura ocorreu
  em 11/11/1111 no Supremo Congresso de Diversão
  Maníaca, mas a lei foi rejeitada pelo Vizir
  da Diversão. Somente depois que a absurda
  monarquia de Diversão Maníaca foi transformada em uma monarquia
  representativa e engenhosa por W. Itzbold, em 9/9/1999 foi
  que o caminho finalmente se clareou para esta lei.
\end{lstcode}
  Como\textnote{Attention!} a área de texto não foi reduzida, o preâmbulo é
  exibido estendendo-se por toda a largura da área de tipo. Para testar isso,
  você pode temporariamente adicionar:
\begin{lstcode}
  \end{document}
\end{lstcode}
\end{Example}
%
No exemplo, a questão de como o texto do comentário pode ser definido em
uma coluna mais estreita permanece sem solução. Você descobrirá como fazer isso
continuando o exemplo abaixo.%
%
%\begin{table}% Umbruchoptimierung: Tabelle hinter das Beispiel verschoben
\begin{desclist}
  \desccaption{%
%  \caption[Available settings for declaring note columns]{%
    Configurações disponíveis para declaração de colunas de notas%
%  }%
    \label{tab:scrlayer-notecolumn.note.column.options}%
  }{%
    Configurações disponíveis para declaração de colunas de notas
    (\emph{continuado})%
  }%
%  \begin{desctabular}
    \entry{\OptionVName{font}{font attribute}}{%
      Os atributos de fonte da coluna de notas definidos com
      \DescRef{\LabelBase.cmd.setkomafont}. Para valores possíveis, consulte
      \autoref{sec:maincls.textmarkup},
      \DescPageRef{maincls.cmd.setkomafont}.\par%
      Padrão: \emph{vazio}%
    }%
    \entry{\Option{marginpar}}{%
      Define \Option{position} e \Option{width} para corresponder à coluna
      de notas marginal de \DescRef{maincls.cmd.marginpar}. A mudança entre
      \Macro{reversemarginpar}\IndexCmd{reversemarginpar} e
      \Macro{normalmarginpar}\IndexCmd{normalmarginpar} é considerada apenas no
      final da página quando a coluna de notas é exibida. Note que esta
      opção não espera ou permite nenhum valor.\par%
      Padrão: \emph{sim}%
    }%
    \entry{\Option{normalmarginpar}}{%
      Define \Option{position} e \Option{width} para usar a coluna
      normal de notas marginais e ignorar \Macro{reversemarginpar} e
      \Macro{normalmarginpar}. Note que esta opção não espera ou permite
      um valor.\par%
      Padrão: \emph{não}%
    }%
    \entry{\OptionVName{position}{offset}}{%
      Define o deslocamento horizontal da coluna de notas a partir da borda esquerda do
      papel. O \PName{offset} pode ser uma expressão complexa desde que seja
      totalmente expansível e se expanda para um comprimento ou uma expressão dimensional no
      momento em que a coluna de notas é exibida. Veja \cite[section~3.5]{manual:eTeX}
      para mais informações sobre expressões dimensionais.\par%
      Padrão: \emph{através da opção \Option{marginpar}}%
    }%
    \entry{\Option{reversemarginpar}}{%
      Define \Option{position} e \Option{width} para usar a coluna
      inversa de notas marginais de \DescRef{maincls.cmd.marginpar} com a
      configuração \Macro{reversemarginpar}. Note que esta opção não espera
      ou permite um valor.\par%
      Padrão: \emph{não}%
    }%
    \entry{\OptionVName{width}{length}}{%
      Define a largura da coluna de notas. O \PName{length} pode ser uma expressão
      complexa desde que seja totalmente expansível e se expanda para um comprimento ou
      uma expressão dimensional no momento em que a coluna de notas é exibida. Veja
      \cite[section~3.5]{manual:eTeX} para mais informações sobre expressões dimensionais.\par%
      Padrão: \emph{através da opção \Option{marginpar}}%
    }%
%  \end{desctabular}
%\end{table}
\end{desclist}
\EndIndexGroup


\section{Criando uma Nota}
\seclabel{makenote}

Depois de declarar uma coluna de notas, você pode criar notas para essa coluna. Mas
essas notas não são exibidas imediatamente. Inicialmente, elas são escritas em um
arquivo auxiliar com extensão ``\File{.slnc}''. Especificamente, elas são primeiro
escritas no arquivo \File{aux} e, quando o arquivo \File{aux} é lido dentro
de \Macro{end}\PParameter{document}, elas são copiadas para o arquivo \File{slnc}. Se
necessário, a configuração \Macro{nofiles}\IndexCmd{nofiles} também é levada em
conta. Na próxima execução de \LaTeX{}, este arquivo auxiliar será lido pedaço por
pedaço, de acordo com o progresso do documento, e no final da página
as notas para essa página serão exibidas.

Note\textnote{Attention!}, porém, que colunas de notas são exibidas apenas em páginas
cujo estilo de página é baseado no pacote \Package{scrlayer}\IndexPackage{scrlayer}.
Este pacote é carregado automaticamente por \Package{scrlayer-notecolumn}
e por padrão fornece apenas o estilo de página \PageStyle{empty}\IndexPagestyle{empty}.
Se você precisar de estilos de página adicionais, o
pacote \Package{scrlayer-scrpage}\IndexPackage{scrlayer-scrpage} é
recomendado.

\begin{Declaration}
  \Macro{makenote}\OParameter{note-column name}\Parameter{note}\\
  \Macro{makenote*}\OParameter{note-column name}\Parameter{note}\\
\end{Declaration}
Você pode usar este comando para criar uma nova \PName{note}. A posição vertical
atual é usada como a posição vertical para o início da \PName{note}.
A posição horizontal da nota resulta da posição definida da coluna de
notas. Para funcionar corretamente, o pacote depende de
\Macro{pdfsavepos}\IndexCmd{pdfsavepos},
\Macro{pdflastypos}\IndexCmd{pdflastypos}, e
\Macro{pdfpageheight}\IndexLength{pdfpageheight} ou seus equivalentes em versões mais recentes
de \LuaTeX{}. Sem estes comandos, \Package{scrlayer-notecolumn} não
funcionará. Os primitivos devem agir exatamente como funcionariam usando pdf\TeX.

Porém, se o pacote detectar uma colisão\textnote{collision avoidance} com
uma \PName{note} anterior na mesma coluna de notas, a nova \PName{note} é movida
abaixo dessa \PName{note} anterior. Se a \PName{note} não couber na
página\textnote{page break}\Index{page>break}, ela será movida completamente ou
parcialmente para a próxima página.

O argumento opcional \PName{note column name} determina qual coluna de notas
deve ser usada para a \PName{note}. Se o argumento opcional for omitido, a
coluna de notas padrão \PValue{marginpar} é usada.%
\begin{Example}
  Vamos adicionar um parágrafo comentado ao exemplo da seção anterior. O
  próprio parágrafo deve ser colocado na coluna de notas recém-definida:
\begin{lstcode}
  \section{Análise}
  \begin{addmargin}[0pt]{.333\textwidth}
    \makenote[paragraphs]{%
      \protect\begin{contract}
        \protect\Clause[%
          title={Nenhuma Piada sem um Público}%
        ]
        Uma piada só pode ser engraçada se tiver um
        público.
      \protect\end{contract}%
    }
    Esta é uma das afirmações mais centrais
    da lei. É tão fundamental que é bastante
    apropriado se inclinar à sabedoria dos autores.
\end{lstcode}
  O ambiente \DescRef{maincls.env.addmargin}\IndexEnv{addmargin}%
  \textnote{column width with \DescRef{maincls.env.addmargin}},
  que é descrito em \autoref{sec:maincls.lists},
  \DescPageRef{maincls.env.addmargin}, é usado para reduzir a largura do texto
  principal pela largura da coluna para os parágrafos.

  Aqui você pode ver um dos poucos problemas de usar \Macro{makenote}. Como
  o argumento obrigatório é escrito em um arquivo auxiliar,
  comandos\textnote{fragile commands} dentro deste argumento podem, infelizmente,
  \emph{quebrar}. Para evitar isso, você deve usar \Macro{protect} na frente de todos
  os comandos que não devem se expandir quando escritos no arquivo auxiliar.
  Caso contrário, usar um comando dentro deste argumento pode resultar em mensagens de
  erro.

  Em princípio, você poderia agora terminar este exemplo com
\begin{lstcode}
  \end{addmargin}
  \end{document}
\end{lstcode}
  para ver um resultado preliminar.
\end{Example}
Se você testar este exemplo, verá que a coluna para o texto legal é
mais longa que a do comentário. Se\textnote{Attention!} você adicionar outra
seção com outro parágrafo, você pode encontrar o problema de que o
comentário não continuará abaixo do texto legal mas imediatamente após o
comentário anterior. Em seguida, você encontrará uma solução para este problema.

O\ChangedAt{v0.1.2583}{\Package{scrlayer-notecolumn}} problema com comandos frágeis
mencionado no exemplo acima não ocorre com a variante
com asterisco\important{\Macro{makenote*}}. Ela usa \Macro{detokenize} para evitar
a expansão de comandos. Mas isso também significa que você não deve usar
comandos na \PName{note} que alteram sua definição dentro do
documento.

Infelizmente, ambos os comandos têm dois outros
limitações conhecidas\textnote{Attention!}. O primeiro problema está relacionado a cores usando
\Package{color}\IndexPackage{color} ou \Package{xcolor}\IndexPackage{xcolor}
dentro das colunas de notas. Para tornar possíveis tais mudanças de cor, cada coluna de notas
requer seu próprio gerenciamento de cor usando os assim chamados \emph{pilhas de cores}.
Como o pacote foi projetado apenas como uma \emph{prova de conceito} e porque
\XeTeX{} não suporta pilhas de cores múltiplas, a mudança de cor do \XeTeX{} é
restrita aos atributos do elemento de fonte
\FontElement{notecolumn.\PName{note column name}}, uma limitação que
elimina o tempo e esforço necessários para implementar gerenciamento de cor personalizado.

O segundo problema é de natureza mais conceitual. O conteúdo do arquivo
auxiliar que contém as informações da coluna de notas é lido durante o processamento do
cabeçalho da página. Isto tem consequências em particular se a leitura ocorre enquanto
um ambiente como \Environment{verbatim} está ativo. Neste caso, as
configurações de \Macro{catcode} deste ambiente estariam ativas durante a leitura do
arquivo auxiliar. Isto inevitavelmente levará a erros no processamento e na saída.
Para atenuar este risco, os \Macro{catcodes} dos caracteres de
\Macro{dospecials}\IndexCmd{dospecials} são armazenados durante
\Macro{begin}\PParameter{document} e explicitamente restaurados quando lendo do
arquivo auxiliar.%
\EndIndexGroup


\begin{Declaration}
  \Macro{syncwithnotecolumn}\OParameter{note column name}
\end{Declaration}
Este comando adiciona um ponto de
sincronização\textnote{synchronisation}\Index{synchronisation} em uma
coluna de notas e no texto principal do documento. Sempre que um ponto de
sincronização é atingido durante a saída de uma coluna de notas ou do texto principal, uma marca
será gerada que consiste na página atual e na posição vertical
atual.

Em paralelo com a geração de pontos de sincronização,
\Package{scrlayer-notecolumn} determina se uma marca foi definida na
coluna de notas ou no texto principal durante a execução anterior de \LaTeX{}. Se sim,
ela compara seus valores. Se a marca da coluna de notas estiver mais abaixo na página
atual ou em uma página posterior, o texto principal será movido para a posição da
marca.

Como regra,\textnote{Hint!} você não deve colocar pontos de sincronização dentro
de parágrafos do texto principal mas apenas entre eles. Se você usar
\Macro{syncwithnotecolumn} dentro de um parágrafo, o ponto de sincronização será
atrasado até que a linha atual tenha sido exibida. Este comportamento é similar
ao de, p.ex., \Macro{vspace}\IndexCmd{vspace} neste aspecto.

Como os pontos de sincronização não são reconhecidos até a próxima execução de
\LaTeX{}\textnote{several \LaTeX{} runs}, este mecanismo requer pelo menos três
execuções de \LaTeX{}. Qualquer nova sincronização também pode resultar em mudanças de pontos
de sincronização posteriores, que por sua vez exigirão execuções adicionais de \LaTeX{}.
Tais mudanças geralmente são indicadas pela mensagem: ``\LaTeX{} Aviso: Rótulo(s)
pode ter mudado. Execute novamente para obter referências cruzadas corretas.'' Mas relatos
sobre rótulos indefinidos também podem indicar a necessidade de outra execução de \LaTeX{}.

Se você omitir o argumento opcional, a coluna de notas padrão \PValue{marginpar}
será usada. Note\textnote{Attention!} que um argumento opcional vazio não é
o mesmo que omitir o argumento opcional!

Você não pode usar\textnote{Attention!} \Macro{syncwithnotecolumn} dentro de uma nota
em si, isto é, dentro do argumento obrigatório de
\Macro{makenote}\IndexCmd{makenote}! Atualmente o pacote não consegue reconhecer
tal erro, e isso causa novas mudanças no ponto de sincronização com
cada execução de \LaTeX{}, então o processo nunca terminará. Para sincronizar duas ou
mais colunas de notas, você deve sincronizar cada uma delas com o texto principal.
O comando recomendado para isso é descrito abaixo.%
%
\begin{Example}
  Vamos estender o exemplo acima adicionando primeiro um ponto de sincronização
  e depois outro parágrafo com um comentário:
\begin{lstcode}
    \syncwithnotecolumn[paragraphs]\bigskip
    \makenote[paragraphs]{%
      \protect\begin{contract}
        \protect\Clause[title={Humor de uma Cultura}]
        \setcounter{par}{0}%
        O humor de uma piada pode ser determinado pelo
        ambiente cultural no qual ela é contada.

       O humor de uma piada pode ser determinado pelo
       ambiente cultural no qual ela atua.
      \protect\end{contract}
    }
    O componente cultural de uma piada não é, de fato,
    negligenciável. Embora a correção política de
    usar o ambiente cultural possa facilmente ser
    discutida, nem assim a precisão de tal comédia
    no ambiente apropriado é impressionante. Por outro
    lado, uma piada supostamente no ambiente cultural
    errado também pode ser um perigo real
    para o contador de piadas.
\end{lstcode}
  Além do ponto de sincronização, um espaço vertical foi adicionado
  com \Macro{bigskip} para melhor distinguir cada parágrafo e os
  comentários correspondentes.

  Além disso\textnote{Attention}, este exemplo ilustra outro problema
  potencial. Como as colunas de notas usam caixas que são montadas e
  desmontadas, contadores\textnote{counter in note column} dentro de colunas de notas
  às vezes podem oscilar. No exemplo, portanto, o primeiro parágrafo seria
  numerado como 2 em vez de 1. Isto, porém, pode ser facilmente corrigido por uma
  redefinição direta do contador correspondente.

  O exemplo está quase completo. Você só precisa fechar os ambientes:
\begin{lstcode}
  \end{addmargin}
  \end{document}
\end{lstcode}
  Na realidade, é claro, toda a seção restante da lei também deveria ser
  comentada. Mas vamos focar no propósito principal.
\end{Example}
Mas espera! E se, neste exemplo, os \PName{paragraphs} não cabessem
mais na página? Seria impresso na próxima página? Responderemos esta
pergunta na próxima seção.%
\EndIndexGroup


\begin{Declaration}
  \Macro{syncwithnotecolumns}\OParameter{list of note column names}
\end{Declaration}
Este comando sincroniza o texto principal com todas as colunas de notas da
\PName{list of note column names} separada por vírgulas. O texto principal
será sincronizado com a coluna de notas cuja marca está mais próxima do
fim do documento. Como efeito colateral, as colunas de notas serão sincronizadas
umas com as outras.

Se o argumento opcional for omitido ou vazio (ou começar com \Macro{relax}),
a sincronização será feita com todas as colunas de notas atualmente declaradas.%
\EndIndexGroup


\section{Saída Forçada de Colunas de Notas}
\seclabel{clearnotes}

Além da saída normal de colunas de notas descrita na seção anterior,
você às vezes pode precisar exibir todas as notas coletadas que
ainda não foram exibidas. Isto é especialmente útil quando notas grandes causam cada vez mais
notas para serem movidas para novas páginas. Um bom momento para forçar a
saída\textnote{force output} é, por exemplo, o fim de um capítulo ou o fim
do documento.

\begin{Declaration}
  \Macro{clearnotecolumn}\OParameter{note column name}
\end{Declaration}
Este comando imprime todas as notas\textnote{pending notes} de uma coluna de notas
particular que ainda não foram exibidas até o final da página atual mas que
foram criadas nessa página ou em uma página anterior. Páginas em branco são geradas conforme necessário
para exibir essas notas pendentes. Durante a saída das notas pendentes desta
coluna de notas, notas de outras colunas de notas também podem ser exibidas, mas apenas
conforme necessário para exibir as notas pendentes da coluna de notas especificada.

Durante a saída das notas pendentes, notas criadas na execução anterior de \LaTeX{}
nas páginas que agora são substituídas pelas páginas em branco inseridas podem ser
exibidas por erro. Isto será corrigido automaticamente em uma das
execuções subsequentes de \LaTeX{}. Tais mudanças geralmente são indicadas pela mensagem:
``\LaTeX{} Aviso: Rótulo(s) pode ter mudado. Execute novamente para obter referências cruzadas
corretas.''

A coluna de notas cujas notas pendentes devem ser exibidas é indicada pelo
argumento opcional \PName{note column name}. Se este argumento for omitido, as
notas da coluna de notas padrão \PValue{marginpar} serão exibidas.

O leitor atento\textnote{forced output vs. synchronisation} terá
notado que a saída forçada de uma coluna de notas não é diferente da sincronização.
Mas se a saída forçada realmente resultar em uma saída, ela será no
início de uma nova página e não apenas abaixo da saída anterior. Não obstante,
uma saída forçada geralmente resulta em menos execuções de \LaTeX{}.%
\EndIndexGroup


\begin{Declaration}
  \Macro{clearnotecolumns}\OParameter{list of note column names}
\end{Declaration}
This command is similar to \DescRef{\LabelBase.cmd.clearnotecolumn}, but the
the optional argument here can be not only the name of one note column but a
comma-separated \PName{list of note column names}. The pending notes of all
these note columns are then ouput.

If you omit the optional argument or leave it empty, all pending notes of all
note columns will be output.%
\EndIndexGroup


\begin{Declaration}
  \OptionVName{autoclearnotecolumns}{simple switch}
\end{Declaration}
As a rule, pending notes will always be output if a document
implicitly\,---\,e.\,g. because of a \DescRef{maincls.cmd.chapter}
command\,---\,or explicitly executes \DescRef{maincls.cmd.clearpage}. This is
also the case at the end of the document within
\Macro{end}\PParameter{document}. The \Option{autoclearnotecolumns} option can
be used to control whether \DescRef{\LabelBase.cmd.clearnotecolumns} should be
executed automatically without arguments when running
\DescRef{maincls.cmd.clearpage}.

Since this is generally the desired behaviour, the option is active by
default. But you can change this with the appropriate value for a simple
switch (see \autoref{tab:truefalseswitch}, \autopageref{tab:truefalseswitch})
at any time.

Note\textnote{Attention!} that disabling the automatic output of pending notes
can result in lost notes at the end of the document. So in this case you
should insert \DescRef{\LabelBase.cmd.clearnotecolumns} before
\Macro{end}\PParameter{document} fore safety's sake.

The question from the end of the last section should now be answered. If the
paragraph is to be completely output, it had to be wrapped to the next page.
This is, of course, the default setting. However, since this would happen
after the end of the \DescRef{maincls.env.addmargin} environment, the forced
output could still overlap with subsequent text. So in the example it would
make sense to add another synchronisation point after the
\DescRef{maincls.env.addmargin} environment.

The result of the example is shown in
\autoref{fig:scrlayer-notecolumn.example}.\iffree{}{ Only the colour has been
changed from blue to grey.}

\begin{figure}
  \KOMAoptions{captions=bottombeside}%
  \setcapindent{0pt}%
  \begin{captionbeside}[{A sample page for the example in
      \autoref{cha:scrlayer-notecolumn}}]{A sample page for the example in
      this chapter\label{fig:scrlayer-notecolumn.example}}[l]
  \frame{\includegraphics[width=.5\textwidth,keepaspectratio]%
    {scrlayer-notecolumn-example-en}}
  \end{captionbeside}
\end{figure}
\EndIndexGroup
%
\EndIndexGroup

%%% Local Variables: 
%%% mode: latex
%%% TeX-master: "scrguide-en.tex"
%%% coding: utf-8
%%% ispell-local-dictionary: "en_GB"
%%% eval: (flyspell-mode 1)
%%% End:
