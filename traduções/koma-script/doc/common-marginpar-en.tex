% ======================================================================
% common-marginpar-en.tex
% Copyright (c) Markus Kohm, 2001-2022
%
% This file is part of the LaTeX2e KOMA-Script bundle.
%
% This work may be distributed and/or modified under the conditions of
% the LaTeX Project Public License, version 1.3c of the license.
% The latest version of this license is in
%   http://www.latex-project.org/lppl.txt
% and version 1.3c or later is part of all distributions of LaTeX
% version 2005/12/01 or later and of this work.
%
% This work has the LPPL maintenance status "author-maintained".
%
% The Current Maintainer and author of this work is Markus Kohm.
%
% This work consists of all files listed in MANIFEST.md.
% ======================================================================
%
% Paragraphs that are common for several chapters of the KOMA-Script guide
% Maintained by Markus Kohm
%
% ======================================================================

\KOMAProvidesFile{common-marginpar-en.tex}
                 [$Date: 2022-06-05 12:40:11 +0200 (So, 05. Jun 2022) $
                  KOMA-Script guide (common paragraphs)]
\translator{Gernot Hassenpflug\and Markus Kohm\and Karl Hagen}

\section{Notas Marginais}
\seclabel{marginNotes}%
\BeginIndexGroup
\BeginIndex{}{notas marginais}%

\IfThisCommonFirstRun{}{%
  As informações em \autoref{sec:\ThisCommonFirstLabelBase.marginNotes}
  aplicam-se igualmente a este capítulo. Portanto, se você já leu e compreendeu
  \autoref{sec:\ThisCommonFirstLabelBase.marginNotes}, pode pular para
  \autoref{sec:\ThisCommonLabelBase.marginNotes.next},
  \autopageref{sec:\ThisCommonLabelBase.marginNotes.next}.%
}

Além da área de texto, que normalmente preenche a área de tipos, os documentos
frequentemente contêm uma coluna para marginália. Você pode colocar notas
marginais nesta área.
\IfThisCommonLabelBase{scrlttr2}{%
  Em cartas, no entanto, notas marginais são incomuns e devem ser usadas
  com moderação.%
}{%
  Este \iffree{guia}{livro} faz uso frequente delas.%
}%


\begin{Declaration}
  \Macro{marginpar}\OParameter{nota marginal esquerda}\Parameter{nota marginal}%
  \Macro{marginline}\Parameter{nota marginal}
\end{Declaration}%
Notas marginais\Index[indexmain]{notas marginais} em {\LaTeX} são normalmente
inseridas com o comando \Macro{marginpar}. Elas são colocadas na margem
externa. Documentos de um lado usam a borda direita. Embora você possa especificar
uma nota marginal diferente para \Macro{marginpar} caso ela termine na margem
esquerda, as notas marginais são sempre totalmente justificadas. No entanto, a experiência
mostrou que muitos usuários preferem notas marginais alinhadas à esquerda ou à direita.
Para este propósito, {\KOMAScript} oferece o comando \Macro{marginline}.

\IfThisCommonFirstRun{%
  \iftrue%
}{%
  Para um exemplo detalhado, veja
  \autoref{sec:\ThisCommonFirstLabelBase.marginNotes} em
  \PageRefxmpl{\ThisCommonFirstLabelBase.cmd.marginline}.%
  \csname iffalse\endcsname%
}%
  \begin{Example}
    \phantomsection\xmpllabel{cmd.marginline}%
    Em algumas partes deste \iffree{guia}{livro}, o nome da classe
    \Class{scrartcl} pode ser encontrado na margem. Você pode produzir isso com:%
    \iffalse% Umbruchkorrekturtext
      \footnote{Na verdade, em vez de \Macro{texttt}, uma marcação semântica
        foi usada. Para evitar confusão, isso foi substituído no exemplo.}%
    \fi
\begin{lstcode}
  \marginline{\texttt{scrartcl}}
\end{lstcode}

  Em vez de \Macro{marginline} você poderia ter usado \Macro{marginpar}. Na verdade,
  o primeiro comando é implementado internamente como:
\begin{lstcode}
  \marginpar[\raggedleft\texttt{scrartcl}]
    {\raggedright\texttt{scrartcl}}
\end{lstcode}
  Assim, \Macro{marginline} é realmente apenas uma notação abreviada para o
  código acima.%
\end{Example}%
\fi

Usuários avançados\textnote{Atenção!} encontrarão notas sobre dificuldades que
podem surgir ao usar \Macro{marginpar} em \autoref{sec:maincls-experts.addInfos}%
\iffree{}{, em \DescPageRef{maincls-experts.cmd.marginpar}}. Essas observações
também se aplicam a \Macro{marginline}. Além disso,
\autoref{cha:scrlayer-notecolumn} introduz um pacote que você pode usar para
criar colunas de notas com suas próprias quebras de página.%
\iffalse% Umbruchkorrektur
  \ No entanto, o pacote
  \hyperref[cha:scrlayer-notecolumn]{\Package{scrlayer-notecolumn}}%
  \IndexPackage{scrlayer-notecolumn} é mais uma prova de conceito do que um
  pacote finalizado.%
\fi%
%
\EndIndexGroup
%
\EndIndexGroup

%%% Local Variables:
%%% mode: latex
%%% TeX-master: "scrguide-en.tex"
%%% coding: utf-8
%%% ispell-local-dictionary: "en_GB"
%%% eval: (flyspell-mode 1)
%%% End: 
