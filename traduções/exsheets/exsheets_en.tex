% !arara: pdflatex
% !arara: biber
% arara: pdflatex
% arara: pdflatex
% arara: pdflatex
% --------------------------------------------------------------------------
% o pacote EXSHEETS
%
%   Mais um pacote para a criação de listas de exercícios
%
% --------------------------------------------------------------------------
% Clemens Niederberger
% Web:    https://bitbucket.org/cgnieder/exsheets/
% E-Mail: contact@mychemistry.eu
% --------------------------------------------------------------------------
% Copyright 2011-2019 Clemens Niederberger
%
% Este trabalho pode ser distribuído e/ou modificado sob as
% condições da Licença Pública do Projeto LaTeX, versão 1.3
% desta licença ou (a seu critério) qualquer versão posterior.
% A versão mais recente desta licença está em
%   http://www.latex-project.org/lppl.txt
% e a versão 1.3 ou posterior faz parte de todas as distribuições do LaTeX
% versão 2005/12/01 ou posterior.
%
% Este trabalho tem o status de manutenção LPPL 'mantido'.
%
% O Mantenedor Atual deste trabalho é Clemens Niederberger.
% --------------------------------------------------------------------------
% Se você tiver quaisquer ideias, perguntas, sugestões ou bugs para relatar,
% sinta-se à vontade para entrar em contato.
% --------------------------------------------------------------------------
\documentclass[
  a4paper,
  11pt,
  parskip=half,
  numbers=noenddot,
  bibliography=totoc,
  index=totoc
]{scrartcl}

\usepackage[utf8]{inputenc}
\usepackage[T1]{fontenc}
\usepackage[brazilian]{babel}
\usepackage{lmodern}

% Pacotes essenciais
\usepackage{exsheets}
\usepackage{bookmark}
\usepackage{multicol}
\usepackage{booktabs,array}
\usepackage{xcoffins,wasysym,enumitem,siunitx}
\usepackage{amssymb}
\usepackage{xcolor}
\usepackage{listings}
\usepackage{hyperref}

% Bibliografia
\usepackage[style=numeric,backend=biber]{biblatex}
\usepackage{csquotes}

% Microtype
\usepackage{microtype}
\microtypesetup{tracking=scshape}

% Embrac
\usepackage[biblatex]{embrac}[2012/06/29]
\ChangeEmph{[}[,.02em]{]}[.055em,-.08em]
\ChangeEmph{(}[-.01em,.04em]{)}[.04em,-.05em]

% Acronyms
\usepackage{acro}
\DeclareAcronym{faq}{
  short     = faq ,
  long      = Perguntas Frequentes ,
  format    = \scshape ,
  pdfstring = FAQ
}
\DeclareAcronym{id}{
  short     = id ,
  long      = Identificador ,
  format    = \scshape ,
  pdfstring = ID
}
\DeclareAcronym{ctan}{
  short     = ctan ,
  long      = Comprehensive \TeX\ Archive Network ,
  format    = \scshape ,
  pdfstring = CTAN
}

% Define custom commands to replace cnltx-doc functionality
\newcommand*\ExSheets{\textsf{ExSheets}}
\newcommand*\ExSheetslistings{\textsf{ExSheets-listings}}
\newcommand*\cntformats{\textsf{cntformats}}
\newcommand*\Tasks{\textsf{tasks}}

% Package/class name commands
\newcommand*\pkg[1]{\textsf{#1}}
\newcommand*\cls[1]{\textsf{#1}}
\newcommand*\needpackage[1]{\textsf{#1}}

% Code/command formatting
\newcommand*\code[1]{\texttt{#1}}
\makeatletter
\newcommand*\cs{\@ifstar\cs@star\cs@nostar}
\newcommand*\cs@star[1]{\texttt{\textbackslash#1}}
\newcommand*\cs@nostar[1]{\texttt{\textbackslash#1}}
\makeatother
\newcommand*\cmd[1]{\cs{#1}}
\newcommand*\env[1]{\texttt{#1}}
\newcommand*\option[1]{\texttt{#1}}
\newcommand*\key[1]{\texttt{#1}}
\newcommand*\keyis[2]{\texttt{#1}\,=\,\texttt{#2}}
\newcommand*\meta[1]{\ensuremath{\langle}\textit{#1}\ensuremath{\rangle}}
\newcommand*\Default[1]{\textit{Default: }\texttt{#1}}
\newcommand*\default[1]{\Default{#1}}

% Module
\newcommand*\module[1]{\texttt{#1}}

% CTAN URL
\newcommand*\CTANurl[2][]{%
  \ifx\relax#1\relax
    \url{https://ctan.org/pkg/#2}%
  \else
    \url{https://ctan.org/#1/#2}%
  \fi
}

% Version markers
\newcommand*\sinceversion[1]{\marginpar{\footnotesize\textit{desde v#1}}}
\newcommand*\changedversion[1]{\marginpar{\footnotesize\textit{mudado v#1}}}

% Expandable markers
\newcommand*\expandable{$^{\star}$}
\newcommand*\unexpandable{}

% Additional cnltx-doc commands for keys
\newcommand*\keyval[2]{\texttt{#1}\,=\,\meta{#2}}
\newcommand*\keybool[1]{\texttt{#1}\,=\,\meta{true|false}}
\newcommand*\keychoice[2]{\texttt{#1}\,=\,\meta{#2}}
\newcommand*\Module[1]{[\textit{#1}]}
\newcommand*\propis[2]{\texttt{#1}\,=\,\texttt{#2}}

% Define cnltx-doc environments
\usepackage{mdframed}
\newenvironment{commands}{\begin{description}}{\end{description}}
\newenvironment{options}{\begin{description}}{\end{description}}
\newenvironment{environments}{\begin{description}}{\end{description}}
\lstnewenvironment{sourcecode}{\lstset{backgroundcolor=\color{gray!10}}}{}
\newenvironment{example}{\begin{mdframed}[backgroundcolor=blue!5,linewidth=0pt]\small}{\end{mdframed}}
\newenvironment{bewareofthedog}{\begin{mdframed}[backgroundcolor=red!10,linecolor=red,linewidth=2pt]\textbf{Atenção:} }{\end{mdframed}}
\lstnewenvironment{cnltxcode}{\lstset{backgroundcolor=\color{gray!10}}}{}
\newenvironment{implementation}{\section{Implementação}\small}{}

% Command documentation
\newcommand*\command[2][]{\item[\cs{#2}#1]}
\newcommand*\environment[2][]{\item[\texttt{#2}#1]}
\newcommand*\opt[1]{[\meta{#1}]}
\newcommand*\Marg[1]{\{\meta{#1}\}}
\newcommand*\marg[1]{\{\meta{#1}\}}
\newcommand*\oarg[1]{[\meta{#1}]}

% License
\newcommand*\license{%
  Copyright \textcopyright\ 2011--2019 Clemens Niederberger\\[1ex]
  Este trabalho pode ser distribuído e/ou modificado sob as condições da
  Licença Pública do Projeto LaTeX, versão 1.3 ou posterior.
}

% Listings setup
\lstset{
  basicstyle=\ttfamily\small,
  columns=fullflexible,
  keepspaces=true,
  breaklines=true,
  breakatwhitespace=true,
  xleftmargin=2em,
  frame=single,
  framesep=5pt,
  backgroundcolor=\color{gray!10}
}

% Hyperref setup
\hypersetup{
  colorlinks=true,
  linkcolor=blue!50!black,
  urlcolor=blue!50!black,
  citecolor=blue!50!black,
  pdfauthor={Clemens Niederberger},
  pdftitle={exsheets - exercícios e soluções},
  pdfsubject={Documentação do pacote exsheets}
}

% Bibliography
\usepackage{filecontents}
\addbibresource{biblatex-examples.bib}
\addbibresource{\jobname.bib}

\begin{filecontents*}{\jobname.bib}
@online{tex.sx:131546,
  title   = {Fixing lstlisting inside ExSheets question and solution environments} ,
  author  = {Stefano Bragaglia} ,
  url     = {http://tex.stackexchange.com/q/131546/5049} ,
  date    = {2013-09-04} ,
  urldate = {2013-09-22}
}
@online{tex.sx:133907,
  title   = {How do I extract repeated functionality (ExSheets workaround) to make it reusable?} ,
  author  = {Forkrul Assail} ,
  url     = {http://tex.stackexchange.com/q/133907/5049} ,
  date    = {2013-09-18} ,
  urldate = {2013-09-22}
}
\end{filecontents*}

% ExSheets configuration
\usepackage{exsheets-listings}

\defbibheading{bibliography}[\bibname]{\section{#1}}

% Question classes and properties
\DeclareQuestionClass{difficulty}{difficulties}
\DeclareQuestionProperty{notes}
\DeclareQuestionProperty{reference}
\DeclareQuestionProperty{topic}

\DeclareRelGrades{
  1     = 1 ,
  {1,5} = .9167 ,
  2     = .8333 ,
  {2,5} = .75 ,
  3     = .6667 ,
  {3,5} = .5833 ,
  4     = .5
}

\let\checkmark\relax
\usepackage{dingbat}

\DeclareRobustCommand*\questionstar{\texorpdfstring{\bonusquestionsign}{* }}
\DeclareRobustCommand*\bonusquestionsign{\llap{$\bigstar$\space}}

\NewQuSolPair
  {question*}[name=\questionstar Questão Bônus]
  {solution*}[name=\questionstar Solução]

\NewTasks[style=multiplechoice]{multiplechoice}[\choice](3)
\newcommand*\correct{\PrintSolutionsTF{\checkedchoicebox}{\choicebox}}

\usepackage{alphalph}
\NewPatternFormat{aa}{\alphalph}
\NewCounterPattern{testa}{ta}

% Title setup
\title{O pacote \ExSheets}
\subtitle{Exercícios e Soluções}
\author{Clemens Niederberger}
\date{Versão 0.21k (2019/09/30)}

\begin{document}

\maketitle

\tableofcontents

\part{Preliminares}
\section{Licença e Requisitos}
\license

\ExSheets\ carrega e necessita dos seguintes pacotes:
\needpackage{l3kernel}~\cite{bnd:l3kernel}, \pkg{xparse}, \pkg{xtemplate},
\pkg{l3keys2e}\footnote{todos os três \CTANurl{l3packages}}~\cite{bnd:l3packages},
\pkg{l3sort}\footnote{\CTANurl{l3experimental}}~\cite{bnd:l3experimental},
\needpackage{xcolor}~\cite{pkg:xcolor}, \needpackage{ulem}~\cite{pkg:ulem},
\needpackage{etoolbox}~\cite{pkg:etoolbox},
\needpackage{environ}~\cite{pkg:environ}, e
\pkg{pgfcore}\footnote{\CTANurl[graphics]{pgf}}~\cite{pkg:pgf}.  \ExSheets\
chama \cs*{normalem} (do pacote \pkg{ulem}).

\section{Motivação}
Já existem vários pacotes que permitem a criação de
listas de exercícios ou provas escritas. Apenas para citar os mais comuns:
\pkg{eqexam}~\cite{pkg:eqexam}, \pkg{exam}~\cite{cls:exam},
\pkg{examdesign}~\cite{pkg:examdesign}, \pkg{exercise}~\cite{pkg:exercise},
\pkg{probsoln}~\cite{pkg:probsoln}, \pkg{answers}~\cite{pkg:answers},
\pkg{esami}~\cite{pkg:esami}, \pkg{exsol}~\cite{pkg:exsol} (e muitos
outros \ldots).

Uma coisa que senti falta em todos os pacotes que experimentei\footnote{Bem, provavelmente
  não me esforcei o suficiente\ldots} foi uma alta flexibilidade em escolher quais
questões e soluções devem ser impressas, onde quais soluções devem ser
impressas e assim por diante, combinado com a possibilidade de atribuir questões a
classes diferentes para que se possa, por exemplo, criar duas versões de uma prova
sem esforço. E --~não me canso~-- também quero poder usar/projetar
layouts diferentes para questões além de um formato padrão tipo seção.
Todos esses pontos são realizados no \ExSheets.

Adicionalmente, deve-se ser capaz de atribuir algum tipo de metadados a questões
que, é claro, devem ser facilmente reutilizáveis. Como isso pode ser feito é explicado
na seção~\ref{sec:additional_info}.

Então há --~pelo menos na Alemanha~-- o hábito de ter listas de exercícios
alinhadas em colunas mas contando da esquerda para a direita em vez de cima para
baixo. É por isso que o pacote \pkg{tasks} foi desenvolvido como parte do \ExSheets{}
e foi distribuído como parte do bundle\changedversion{0.15}. Agora é um
pacote próprio mas é carregado pelo \ExSheets{} automaticamente com a
configuração necessária para fazê-los funcionar bem juntos.

\ExSheets{} não tem suporte nativo para testes de múltipla escolha mas isso não
significa que você não possa criá-los com \ExSheets. Isso apenas significa que eles podem
dar um pouco mais de trabalho com \ExSheets{} do que com outros pacotes.

Tive a ideia para este pacote em 2008. Naquela época, minhas habilidades em \TeX{} estavam
longe de serem boas o suficiente para escrevê-lo. Na verdade, mesmo hoje eu não teria sido
capaz de realizá-lo sem todos os pacotes l3 como \pkg{l3kernel} e
\pkg{l3packages}. Comecei ativamente a desenvolver o \ExSheets\ na primavera de~2011 mas
não foi até agora (setembro de~2012) que o considero estável o suficiente para
uso mais amplo. No momento da escrita (\today) ainda há provavelmente muitas
arestas ásperas sem falar em bugs, então estou muito interessado em todo tipo de feedback.

\section{Pacotes Adicionais}
\ExSheets\ na verdade agrupa dois pacotes: \ExSheets, \ExSheetslistings.
\ExSheetslistings\ é um complemento do \ExSheets\ que oferece alguma funcionalidade
para usar \pkg{listings} com \ExSheets. É apresentado na
parte~\ref{part:listings}.

\ExSheets\ costumava agrupar o pacote \pkg{translations}
também\changedversion{0.9i}, mas não mais. Você pode encontrar o
pacote \pkg{translations} como um pacote próprio no \ac{ctan}. Ele também
costumava agrupar os pacotes \pkg{tasks} e
\pkg{cntformats}\changedversion{0.15}. Eles estão disponíveis agora como pacotes
próprios também.

% \section{Installation and Documentation}
% If you install \ExSheets\ manually beware to put the files
% \begin{itemize}
%   \item[]\verb+exsheets_headings.def+
%   \item[]\verb+exsheets_headings.cfg+
% \end{itemize}
% in the same directory as the \code{exsheets.sty} file\footnote{That is, a
%   directory like \code{texmf-local/tex/latex/exsheets}, probably}.

% As with every manual package installation you need to make sure to put the
% files in a directory where \TeX\ can find them and afterwards update the
% database.

% \subsection{The \pkg*{tasks} Package}
% The \pkg{tasks} package~\cite{pkg:tasks} used to be part of the \ExSheets\
% bundle but is a package of its own now\changedversion{0.15} and released
% independently.  You can find it as every other package on \ctan\ and in a full
% \TeX~Live or \hologo{MiKTeX} installation.

% \subsection{The \pkg*{cntformats} Package}
% The \pkg{cntformats} package~\cite{pkg:cntformats} used to be part of the
% \ExSheets\ bundle but is a package of its own now\changedversion{0.15} and
% released independently.  You can find it as every other package on \ctan\ and
% in a full \TeX~Live or \hologo{MiKTeX} installation.

% \subsection{The \pkg*{translations} Package}
% The \pkg{translations} package~\cite{pkg:translations} used to be part of the
% \ExSheets\ bundle but is a package of its own now\changedversion{0.9i} and
% released independently.  You can find it as every other package on \ctan\ and
% in a full \TeX~Live or \hologo{MiKTeX} installation.

% \section{News}
% \begin{description}
% \item[Version 0.7]
%   With version~0.7 there has been a potentially breaking change: the
%   \code{tasks} environment previously provided by \ExSheets\ has been
%   extracted into a package of its own.  This does not change any syntax
%   \emph{per se}. However, if you used custom settings then you'll probably run
%   into some problems.  The options for the environment are no longer set with
%   \cs{SetupExSheets} but with \cs{settasks}.  Also the object that is used for
%   the list template and its instances has been renamed from
%   \code{exsheets-tasks} into \code{tasks}.

%   What's probably even more of a breaking change is a syntax difference of the
%   \code{tasks} environment: the optional argument for the number of columns is
%   \emph{no longer set in braces but parentheses}.  This is deliberate as it
%   reflects the optional nature of the argument better and is consistent with
%   the syntax of \cs{NewTasks}, too.

%   Additionally the labels of the list got an additional offset of \code{1ex}
%   from the items which will lead to slightly different output.  In some cases
%   this might actually lead to the most annoying changes.  In this case say
%   \cs{settasks}\Marg{label-offset=0pt} which should cure things again.

%   I am very sorry for any inconvenience!  I am trying to keep such changes as
%   minimal and rare as possibly.  However, it is not always avoidable when a
%   package is new and still a child. It will grow up eventually.

%   \ExSheets' other packages -- \href{tasks_en.pdf}{\Tasks} and
%   \href{cntformats_en.pdf}{\cntformats} -- have gotten their own documentation
%   which are essentially extracted from this very document you're reading now.
%   This manual contains links to the respective manuals.

% \item[Version v0.9i]
%   The \pkg{translations} package~\cite{pkg:translations} is no longer part of
%   the \ExSheets\ bundle.  From now on (July~17.\@ 2013) it is provided as a
%   package of its own.

% \item[Version 0.10]
%   The \ExSheets\ family has got a new member: \ExSheetslistings.  This package
%   proposes a solution for the problem of using verbatim material in \ExSheets'
%   \env{question} and \env{solution} environments.  It is presented in
%   part~\ref{part:listings}.

%   Question now can get subtitles that are printed if the heading instance
%   supports it, see section~\ref{sec:subtitles-questions}.

% \item[Version 0.11]
%   The commands \cs{GetQuestionClass} and \cs{PrintQuestionClassTF} have been
%   added.  They're explained in section~\ref{sec:retr-class-value}.
 
% \item[Version 0.12]
%   The \option{auto-label} is now more flexible to allow the use together with
%   packages \pkg{cleveref}.

%   Question properties can now be retrieved before the question is printed (by
%   writing the properties to the \code{aux} file).

% \item[Version 0.13]
%   New options:
%   \begin{itemize}
%     \item \option{chapter-hook} allows to add code to the list of solutions
%       when the solutions of a new chapter are printed, see
%       section~\ref{sec:solutions-print-all}.
%     \item \option{section-hook} allows to add code to the list of solutions
%       when the solutions of a new section are printed, see
%       section~\ref{sec:solutions-print-all}.
%   \end{itemize}

% \item[Version 0.14]
%   New options:
%   \begin{itemize}
%     \item New option \option{pre-hook} to the \env{question} environment that
%       allows to add code directly before the question body, see
%       section~\ref{sec:opti-ques-envir}. 
%     \item New option \option{post-hook} to the \env{question} environment that
%       allows to add code directly after the question body, see
%       section~\ref{sec:opti-ques-envir}.
%     \item New command \cs{ExSheetsHeading}, see
%       section~\ref{sec:using-an-exsheets}.
%     \item New pre-defined question properties \code{question-body},
%       \code{bonus-points} and \code{counter}, see
%       section~\ref{sec:additional_info}.
%     \item New option \option{save-to-aux}, see
%       section~\ref{sec:additional_info}.
%   \end{itemize}

% \item[Version 0.15]
%   \begin{itemize}
%     \item The packages \pkg{tasks} and \pkg{cntformats} have been removed from
%       the bundle and are now distributed as packages of their own.
%      \item The options \option*{load-headings} and \option*{load-tasks} have
%        been dropped.  The optional functionality they provided is now provided
%        all the time.
%     \item New command \cs{IfQuestionPropertyTF}, see
%       section~\ref{sec:additional_info}.
%   \end{itemize}

% \item[Version 0.16]
%   New options/changes:
%   \begin{itemize}
%     \item The option \option{pre-hook} to the \env{question} environment now
%       places its contents before the question heading, see
%       section~\ref{sec:opti-ques-envir}.
%     \item New option \option{pre-body-hook} to the \env{question} environment
%       which adds its contents before the question body, see
%       section~\ref{sec:opti-ques-envir}.
%     \item New option \option{post-body-hook} to the \env{question} environment
%       which adds its contents after the question body, see
%       section~\ref{sec:opti-ques-envir}.
%     \item New option \option{pre-hook} to the \env{solution} environment which
%       adds code before a solution, see section~\ref{sec:opti-soli-envir}.
%     \item New option \option{post-hook} to the \env{solution} environment which
%       adds code after a solution, see section~\ref{sec:opti-soli-envir}.
%     \item New option \option{pre-body-hook} to the \env{solution} environment
%       which adds its contents before the solution body, see
%       section~\ref{sec:opti-soli-envir}.
%     \item New option \option{post-body-hook} to the \env{solution} environment
%       which adds its contents after the solution body, see
%       section~\ref{sec:opti-soli-envir}.
%   \end{itemize}

% \item[Version 0.17]
%   New option:
%   \begin{itemize}
%     \item The option \option{use-saved-counter-format} has been introduced. It
%       is described in section~\ref{sec:solutions} on
%       page~\pageref{option:use-saved-counter-format}.
%   \end{itemize}

% \item[Version 0.18]
%   The package now provides the correct Danish translations, thanks to Jonas
%   Nyrup.

%   The macro \cs{exsheetsprintsolution} is introduced, see
%   page~\pageref{exsheetsprintsolution} for a little bit of an explanation.

%   The option \option{no-skip-below} is introduced which disables the insertion
%   of vertical space after the question and solution environments.

% \item[Version 0.20]
%   New command \cs{DeclareExSheetsHeadingContainer}.

% \item[Version 0.21] Changes:
%   \begin{itemize}
%     \item \cs{includequestions} issues an error if it can't find the file to
%       include.
%     \item question properties are now also accessable when the corresponding
%       question isn't printed.
%     \item The variables \verbcode+\l_exsheets_counter_qu_int+ and \\
%       \verbcode+\g_exsheets_question_identification_prop+ are now public.
%   \end{itemize}
% \end{description}

\section{Agradecimentos}
Preciso agradecer aos muitos usuários que me deram feedback até agora! Por um lado
isso me mostra que \ExSheets\ é útil para as pessoas. Também levou a muitas
melhorias como novos recursos e inúmeras correções de bugs.

\part{O pacote \ExSheets}\label{part:exsheets}
\section{Configuração}
O pacote \ExSheets\ tem três tipos diferentes de opções, mais ou menos. O
primeiro tipo são as opções clássicas de pacote que são usadas quando você carrega
\ExSheets:
\begin{sourcecode}
  \usepackage[<options>]{exsheets}
\end{sourcecode}
Todas as opções gerais podem ser usadas desta forma e a maioria delas é descrita na
seção~\ref{sec:options}. Todas essas opções também podem ser definidas via o comando de configuração:
\begin{commands}
  \command{SetupExSheets}[\oarg{module}\marg{options}]
\end{commands}

O segundo tipo são opções que pertencem a um ambiente ou comando específico.
Essas opções são usadas diretamente com o ambiente/comando
\begin{sourcecode}
  \begin{env}[<options>]
   ...
  \end{env}
\end{sourcecode}
ou também podem ser definidas com o comando de configuração. No primeiro caso, elas apenas agem
sobre o ambiente ou comando onde são usadas. No segundo caso, elas
são definidas para todos os usos seguintes do ambiente ou comando correspondente.

As opções do segundo tipo todas pertencem a \module*{módulos}. Digamos que você
queira especificar algumas opções do ambiente \env{question}. Você pode então
dizer o seguinte:
\begin{sourcecode}
  \SetupExSheets[question]{option1,option2=value2}
  % ou:
  \SetupExSheets{question/option1,question/option2=value2}
\end{sourcecode}
O \module*{módulo} ao qual uma opção pertence é escrito na margem esquerda ao
lado quando a opção é descrita.

O terceiro tipo não são opções de verdade, na verdade. No entanto, graças ao excelente
pacote \pkg{xtemplate} você é capaz de definir suas próprias instâncias de alguns dos
objetos usados por \ExSheets. Isso é explicado com um pouco mais de detalhe na
parte~\ref{part:style} na página~\pageref{part:style}\,ff. Este terceiro tipo,
no entanto, traz uma possível instabilidade: o pacote \pkg{xtemplate} está em
um estado experimental e em desenvolvimento. Isso significa que a sintaxe do
pacote pode e possivelmente mudará em algum momento no futuro. Não posso prever
quaisquer consequências disso para \ExSheets.

\section{Opções Gerais}\label{sec:options}
O pacote \ExSheets\ tem algumas opções, a saber as seguintes:
\begin{options}
  %% counter-format
  \keyval{counter-format}{counter-format}\Default{qu.}
    Formatação do contador das questões. Esta opção recebe um tipo especial
    de string que é descrita na seção~\ref{ssec:counter}.
  \keyval{counter-within}{counter}\Default
    Reinicia o contador \code{question} a cada passo de \meta{counter}.
  %% auto-label
  \keybool{auto-label}\Default{false}
    Se definido como \code{true} \ExSheets\ irá automaticamente colocar um
    \cs*{label}\Marg{qu:\meta{id}} para cada questão. Veja
    seção~\ref{sec:auto-label-opti} para formas de personalizar isso. Também
    criará as propriedades de questão \code{ref} e \code{pageref}, veja
    seção~\ref{sec:additional_info} para mais sobre isso.
  %% headings
  \keyval{headings}{instance}\Default{block}
    Escolhe o estilo dos cabeçalhos das questões e soluções. Há dois
    estilos pré-definidos: \code{block} e \code{runin}.
  %% headings-format
  \keyval{headings-format}{code}\Default{\cs*{normalsize}\cs*{bfseries}}
    Este código é colocado imediatamente antes dos cabeçalhos das questões e
    soluções.
  \keyval{subtitle-format}{code}\Default{\cs*{normalsize}\cs*{itshape}}
    Este código é colocado imediatamente antes do subtítulo das questões e
    soluções. Só tem efeito com uma instância de título que usa o
    coffin de subtítulo, veja seção~\ref{sec:exsheets-headings}.
  % skip-below
  \keyval{skip-below}{dim}\Default{.5\cs*{baselineskip}}
    \sinceversion{0.18}Define o espaço vertical que é inserido após os
    ambientes question e solution.
  % no-skip-below
  \keybool{no-skip-below}\Default{false}
    \sinceversion{0.18}Desabilita a inserção de espaço vertical após os
    ambientes question e solution.
  %% totoc
  \keybool{totoc}\Default{false}
    Esta opção adiciona as questões e soluções com seus nomes e números
    ao sumário.
  %% questions-totoc
  \keybool{questions-totoc}\Default{false}
    Esta opção adiciona as questões com seus nomes e números ao
    sumário.
  %% solutions-totoc
  \keybool{solutions-totoc}\Default{false}
    Esta opção adiciona as soluções com seus nomes e números ao
    sumário.
  %% toc-level
  \keyval{toc-level}{toc level}\Default{subsection}
    Esta opção define o nível no qual questões e soluções devem aparecer
    no sumário.
  %% questions-toc-level
  \keyval{questions-toc-level}{toc level}\Default{subsection}
    Esta opção define o nível no qual questões devem aparecer no
    sumário.
  %% solutions-toc-level
  \keyval{solutions-toc-level}{toc level}\Default{subsection}
    Esta opção define o nível no qual soluções devem aparecer no
    sumário.
  %% use-ref
  \keybool{use-ref}\Default{false}
    habilita referenciamento a seções e capítulos de forma que as referências
    possam ser usadas com \cs{printsolutions}, veja
    seção~\ref{sssec:print_specific_section} para detalhes.
\end{options}
As opções \code{toc} são demonstradas com a seção~\ref{sec:solutions:list}
e as soluções impressas lá sendo listadas no sumário.


\section{Criar Questões/Exercícios e suas Soluções}
Agora, vamos começar com a parte mais importante: as questões e (possivelmente)
suas respectivas soluções.
\subsection{O Ambiente \env*{question}}\label{ssec:questions}
Questões são escritas dentro do ambiente \env{question}:
\begin{environments}
  \environment{question}[\oarg{options}\marg{points}]
    O ambiente principal: cria um novo exercício/questão. Ambos os argumentos são
    opcionais!
\end{environments}
\begin{example}
  \begin{question}
    Esta é nossa primeira questão muito difícil de resolver!
  \end{question}
\end{example}
Como você pode ver, um cabeçalho é criado automaticamente e a questão é
numerada. Você pode, é claro, mudar tanto a numeração quanto o nome, mas
mais sobre isso depois.

O ambiente \env{question} recebe um argumento opcional \marg{points} que
pode ser usado para atribuir pontos à questão (como é comum em provas escritas):
\begin{example}
  \begin{question}{3}
    Esta é nossa primeira questão difícil que vale 3 pontos!
  \end{question}
\end{example}
Esses pontos são salvos internamente (veja seção~\ref{sec:points} para os motivos
disso) e são escritos na margem direita ao lado do cabeçalho da questão na
configuração padrão.

Você também pode atribuir pontos bônus inserindo \code{\meta{point}+\meta{bonus
    points}} como argumento.
\begin{example}
  \begin{question}{1+1}
    Esta questão vale 1 ponto e 1 ponto bônus.
  \end{question}
  \begin{question}{+3}
    Esta questão é uma questão bônus. Ela vale 3 pontos bônus.
  \end{question}
\end{example}

Os pontos são contados e adicionados à soma total de pontos, veja
seção~\ref{sec:points} para detalhes sobre isso. \sinceversion{0.12}Caso você
queira que os pontos de uma questão específica \emph{não sejam adicionados} à
soma total, então preceda-os com uma exclamação \code{!}:
\begin{example}
  \begin{question}{!3}
    Os pontos desta questão não serão adicionados à soma total.
  \end{question}
\end{example}
Cuidado que isso também impede pontos bônus. Os pontos simplesmente serão
escritos onde a instância do cabeçalho os coloca.

\sinceversion{0.3}Uma coisa adicional: você pode querer definir comandos
personalizados que devem se comportar diferentemente se estiverem dentro ou fora do
ambiente \env{question}. Neste caso você pode usar estes comandos:
\begin{commands}
  \expandable\command{IfInsideQuestionTF}[\marg{true code}\marg{false code}]
    Verifica se está dentro de uma questão e deixa ou \meta{true code} ou
    \meta{false code} no fluxo de entrada.
  \expandable\command{IfInsideQuestionT}[\marg{true code}]
    Verifica se está dentro de uma questão e deixa \meta{true code} no
    fluxo de entrada se verdadeiro.
  \expandable\command{IfInsideQuestionF}[\marg{false code}]
    Verifica se está dentro de uma questão e deixa \meta{false code} no
    fluxo de entrada se não.
\end{commands}

\subsection{Opções para o Ambiente \env*{question}}\label{sec:opti-ques-envir}
O ambiente \env{question} recebe uma ou mais das seguintes opções:
\begin{options}
  \keychoice{type}{exam,exercise}\Module{question}\Default{exercise}
    Determina o tipo de questão e muda o nome padrão de uma questão
    de ``Exercise'' para ``Question''. Esses nomes padrão dependem do
    idioma.\par
    Se você usar \cs*{usepackage}\oarg{ngerman}\marg{babel}, por exemplo, então
    os nomes são ``Übung'' e ``Aufgabe''.
  \keyval{name}{name}\Module{question}\Default
    Define um nome personalizado. Todos os nomes pré-definidos são descartados.
  \keyval{subtitle}{subtitle}\Module{question}\Default
    Adiciona um subtítulo \meta{subtitle} para a questão que é usado por
    instâncias de cabeçalhos que fazem uso do coffin de subtítulo, veja
    seção~\ref{sec:exsheets-headings}.
  \keyval{skip-below}{dim}\Module{question}\Default{.5\cs*{baselineskip}}
    \sinceversion{0.18}Define o espaço vertical que é inserido após o
    ambiente question.
  \keybool{no-skip-below}\Module{question}\Default{false}
    \sinceversion{0.18}Desabilita a inserção de espaço vertical após o
    ambiente question.
  \keybool{print}\Module{question}\Default{true}
    Imprime ou oculta a questão.
  \keyval{ID}{id}\Module{question}\Default
    Atribui um \acs{id} personalizado à questão. Veja seção~\ref{ssec:ids} para
    mais informações.
  \keyval{label}{label}\Module{question}\Default
    Coloca um \cs*{label}\marg{label} para a questão. Isto sobrescreverá
    qualquer rótulo que seja colocado pela opção \option{auto-label}.
  \keyval{class}{class}\Module{question}\Default
    Atribui uma classe \meta{class} à questão. Veja
    seção~\ref{sec:classes} para mais informações.
  \keyval{topic}{topic}\Module{question}\Default
    Atribui um tópico \meta{topic} à questão. Veja
    seção~\ref{sec:topics} para mais informações.
  \keybool{use}\Module{question}\Default{true}
    Descarta a questão. Ou não.
  \keyval{pre-hook}{code}\Module{question}\Default
    \changedversion{0.16}Adiciona \meta{code} diretamente antes do título da questão.
  \keyval{post-hook}{code}\Module{question}\Default
    \changedversion{0.16}Adiciona \meta{code} diretamente após a questão.
  \keyval{pre-body-hook}{code}\Module{question}\Default
    \sinceversion{0.16}Adiciona \meta{code} diretamente antes do corpo da questão.
  \keyval{post-body-hook}{code}\Module{question}\Default
    \sinceversion{0.16}Adiciona \meta{code} diretamente após o corpo da questão.
\end{options}

\begin{example}
  \begin{question}[type=exam]
    Esta questão tem o tipo \keyis{type}{exam}. O nome padrão mudou
    de ``Exercise'' para ``Question''.
  \end{question}
  \begin{question}[name=Nome elegante]
    Esta questão tem um nome personalizado.
  \end{question}
  \begin{question}[print=false]
    Esta questão não é impressa.
  \end{question}
\end{example}

A diferença entre \option{print} e \option{use} está nos bastidores:
com \keyis{print}{false} a questão não é impressa, mas ela ainda recebe um
\ac{id} individual, é numerada, e uma possível solução é salva. Isto é
por exemplo útil quando você quer imprimir uma solução de amostra para uma prova. Com
\keyis{use}{false} ela é completamente descartada o que significa que não é acessível
através de um \acs{id} e uma possível solução não será salva.

\subsection{Subtítulos para Questões}\label{sec:subtitles-questions}
A opção \option{subtitle} mencionada na seção~\ref{sec:opti-ques-envir}
pode ser usada para adicionar um subtítulo a uma questão. No entanto, a menos que você escolha um
cabeçalho adequado (veja seção~\ref{sec:exsheets-headings}) ele não será
impresso. Atualmente há \emph{uma} instância de cabeçalho que usa os
subtítulos mas deveria ser fácil criar um cabeçalho personalizado usando um dos
existentes como exemplo inicial. Ao criar tal cabeçalho você pode querer
distinguir entre os casos quando um subtítulo foi dado e quando nenhum
subtítulo está presente. Isto pode ser feito com os seguintes comandos:
\begin{commands}
  \expandable\command{IfQuestionSubtitleTF}[\marg{true code}\marg{false code}]
    Testa se a questão atual tem um subtítulo. Deixa ou \meta{true
      code} ou \meta{false code} no fluxo de entrada.
  \expandable\command{IfQuestionSubtitleT}[\marg{true code}]
    Testa se a questão atual tem um subtítulo. Deixa \meta{true code} no
    fluxo de entrada se tiver.
  \expandable\command{IfQuestionSubtitleF}[\marg{false code}]
    Testa se a questão atual tem um subtítulo. Deixa \meta{false code} no
    fluxo de entrada se não tiver.
\end{commands}

Um subtítulo também é uma propriedade de uma questão no sentido da
seção~\ref{sec:additional_info}. Isso significa que se um subtítulo é dado ele pode
ser recuperado com \cs{GetQuestionProperty}.

Como exemplo você poderia definir sua própria instância de cabeçalho que imprime o
\acs{id} de uma questão e (se dado) o subtítulo:

\begin{sourcecode}
  \DeclareInstance{exsheets-heading}{QE}{default}{
    join = {
      title[r,B]number[l,B](.333em,0pt) ;
      title[r,B]subtitle[l,B](1em,0pt)
    } ,
    attach = {
      main[l,vc]title[l,vc](0pt,0pt) ;
      main[r,vc]points[l,vc](\marginparsep,0pt)
    } ,
    subtitle-post-code = {ID: \CurrentQuestionID} ,
    number-post-code   = {\IfQuestionSubtitleF{ID: \CurrentQuestionID}}
  }
\end{sourcecode}

Por favor, veja a seção~\ref{sec:exsheets-headings} para mais detalhes sobre instâncias de
cabeçalho.

\subsection{O Ambiente \env*{solution}}
Se você deseja salvar/imprimir (mais sobre o uso exato na
seção~\ref{sec:solutions}) uma solução, você deve usar o ambiente \env{solution}
\emph{após} a questão a qual ela pertence e \emph{antes} da próxima
questão.
\begin{environments}
  \environment{solution}[\oarg{options}]
    O ambiente principal para adicionar soluções a exercícios/questões.
\end{environments}
\begin{example}
  \begin{question}[ID=first]\label{qu:question_with_solution}
    Esta é nossa primeira questão que recebe uma solução!
  \end{question}
  \begin{solution}
    Esta é a solução para o exercício~\ref{qu:question_with_solution}!
  \end{solution}
\end{example}
Você pode ver que nas configurações padrão a solução \emph{não} é escrita no
documento. Ela foi salva, porém, para possível uso posterior. Veremos
a solução mais tarde!

\subsection{Opções para o Ambiente \env*{solution}}\label{sec:opti-soli-envir}
O ambiente \env{solutions} também possui opções, a saber:
\begin{options}
  \keyval{name}{name}\Module{solution}\Default
    Define um nome personalizado.
  \keybool{print}\Module{solution}\Default{false}
    Imprime ou oculta a solução.
  \keyval{skip-below}{dim}\Module{solution}\Default{.5\cs*{baselineskip}}
    \sinceversion{0.18}Define o espaço vertical que é inserido após o
    ambiente solution.
  \keybool{no-skip-below}\Module{solution}\Default{false}
    \sinceversion{0.18}Desabilita a inserção de espaço vertical após o
    ambiente solution.
  \keyval{pre-hook}{code}\Module{solution}\Default
    \sinceversion{0.16}Adiciona \meta{code} diretamente antes do título da solução.
  \keyval{post-hook}{code}\Module{solution}\Default
    \sinceversion{0.16}Adiciona \meta{code} diretamente após a solução.
  \keyval{pre-body-hook}{code}\Module{solution}\Default
    \sinceversion{0.16}Adiciona \meta{code} diretamente antes do corpo da solução.
  \keyval{post-body-hook}{code}\Module{solution}\Default
    \sinceversion{0.16}Adiciona \meta{code} diretamente após o corpo da solução.
\end{options}
Seus significados são os mesmos que aqueles para o ambiente \code{question}.
\begin{example}
  \begin{question}{5}
    A solução para esta questão é impressa onde está.
  \end{question}
  \begin{solution}[print]
    Viu? Esta solução é impressa onde você a colocou no código do
    seu documento.
  \end{solution}
  \begin{question}{2.5}
    A solução para esta questão é impressa onde está \emph{e}
    tem um nome elegante. Você notou que pode atribuir pontos
    parciais?
  \end{question}
  \begin{solution}[print,name=Nome elegante]
    Viu? Esta solução é impressa onde você a colocou e tem um nome
    elegante!
  \end{solution}
\end{example}

\subsection{Configurando o Contador}\label{ssec:counter}
A opção de pacote \option{counter-format} permite especificar como o contador de questões
(um contador com o nome nada surpreendente \code{question}) é formatado.

A entrada é uma string arbitrária, o que significa que você pode ter qualquer coisa como número
de contador. No entanto, as combinações de letras \code{ch}, \code{se}, \code{qu} e
\code{tsk} são substituídas pelos contadores de capítulo, seção, questão
ou tarefas (veja o pacote \Tasks), respectivamente. Embora o último não seja
realmente útil neste caso, os outros permitem uma numeração combinada. Cada uma
dessas combinações de letras pode ter um argumento opcional que especifica o
formato do respectivo contador. \code{1}: \cs*{arabic}, \code{a}:
\cs*{alph}, \code{A}: \cs*{Alph}, \code{r}: \cs*{roman} e \code{R}:
\cs*{Roman}.
\begin{example}
  \SetupExSheets{counter-format=Nr~se~(qu[a])}
  \begin{question}
    Uma questão com um número formatado diferentemente.
  \end{question}
\end{example}
Como as strings associadas aos contadores são substituídas, é necessário ocultá-las
se elas realmente forem desejadas no formato do contador. A maneira mais fácil seria
ocultá-las entre chaves.
\begin{example}
  \SetupExSheets{counter-format={section}\,se~{question}\,(qu[a])}
  \begin{question}
    Uma questão com um número ainda diferentemente formatado.
  \end{question}
\end{example}

\subsection{Configurações de Idioma}
Os nomes das questões e soluções são dependentes do idioma. Se você usar
\pkg{babel} ou \pkg{polyglossia}, \ExSheets\ se adaptará ao idioma do documento.
\ExSheets\ tem várias traduções, mas certamente não todas! Se você sentir falta de um
idioma, por favor me envie uma linha em um
email\footnote{\href{mailto:contact@mychemistry.eu}{contact@mychemistry.eu}}
contendo o nome do idioma do \pkg{babel} e as traduções corretas para
questões (possivelmente distinguindo entre exercícios e questões de prova) e
soluções.

Até que eu implemente isso, você pode adicionar algo assim ao seu preâmbulo (exemplo
para Dinamarquês) e tentar se funciona:
\begin{sourcecode}
  \DeclareTranslation{Danish}{exsheets-exercise-name}{\O{}velse}
  \DeclareTranslation{Danish}{exsheets-question-name}{Opgave}
  \DeclareTranslation{Danish}{exsheets-solution-name}{Opl\o{}sning}
\end{sourcecode}
Se isto não funcionar, significa que o idioma que você está usando é desconhecido para
o pacote \pkg{translations}. Neste caso, por favor me notifique também. Você ainda pode
usar as opções \option{name}.

\section{Contagem de Pontos}\label{sec:points}
\subsection{Os Comandos}
Você viu na seção~\ref{ssec:questions} que pode atribuir pontos a uma
questão. Se você fizer isso, esses pontos são impressos na
margem\footnote{Bem, não necessariamente. Depende do estilo de cabeçalho que você
  escolheu.} e são contados internamente. Mas existem comandos adicionais
para atribuir pontos ou pontos bônus e vários comandos para recuperar a soma
de pontos e/ou pontos bônus.
\begin{commands}
  \command{addpoints}[\sarg\marg{num}]
    Este comando pode ser usado para adicionar pontos atribuídos a subquestões.
    \cs{addpoints} imprimirá os pontos (com ``unidade'') \emph{e} os adicionará
    à soma de todos os pontos, \cs{addpoints}\sarg\ apenas os adicionará mas não imprimirá
    nada.
  \command{points}[\sarg\marg{num}]
    Este comando apenas imprimirá os pontos (com ``unidade'') mas não os adicionará
    à soma de pontos.
  \command{addbonus}[\sarg\marg{num}]
    Este comando pode ser usado para adicionar pontos bônus atribuídos a subquestões.
    \cs{addbonus} imprimirá os pontos (com ``unidade'') \emph{e} os adicionará
    à soma de todos os pontos bônus, \cs{addbonus}\sarg\ apenas os adicionará mas
    não imprimirá nada.
  \command{bonus}[\sarg\marg{num}]
    Este comando apenas imprimirá os pontos bônus (com ``unidade'') mas não os
    adicionará à soma de pontos bônus.
  \command{pointssum}[\sarg]
    Imprime a soma de todos os pontos com ou sem (versão estrelada) ``unidade'':
    \pointssum
  \command{currentpointssum}[\sarg]
    Imprime a soma atual de pontos com ou sem (versão estrelada)
    ``unidade'': \currentpointssum
  \command{bonussum}[\sarg]
    Imprime a soma de todos os pontos bônus com ou sem (versão estrelada)
    ``unidade'': \bonussum
  \command{currentbonussum}[\sarg]
    Imprime a soma atual de pontos bônus com ou sem (versão estrelada)
    ``unidade'': \currentbonussum
  \command{totalpoints}[\sarg]
    imprime a soma dos pontos \emph{e} a soma dos pontos bônus com
    ``unidade'': \totalpoints\space A versão estrelada imprime a soma dos
    pontos sem ``unidade'': \totalpoints*.
\end{commands}
Os comandos \cs{pointssum}, \cs{bonussum} e \cs{totalpoints} precisam de pelo
menos \emph{duas} execuções do \LaTeX\ para obter a soma correta.

Suponha que você tenha um exercício valendo \points{4} que consiste em quatro questões
listadas com um ambiente \env{enumerate} que valem todas \points{1}
cada. Você tem duas possibilidades para exibi-las e contá-las:
\begin{example}
  % usa pacote `enumitem'
  \begin{question}{4}
    \begin{enumerate}[label=\alph*)]
      \item blah (\points{1})
      \item blah (\points{1})
      \item blah (\points{1})
      \item blah (\points{1})
    \end{enumerate}
  \end{question}
  \begin{question}
    \begin{enumerate}[label=\alph*)]
      \item blah (\addpoints{1})
      \item blah (\addpoints{1})
      \item blah (\addpoints{1})
      \item blah (\addpoints{1})
    \end{enumerate}
  \end{question}
\end{example}

\subsection{Opções}
\begin{options}
  \keyval{name}{name}\Module{points}\Default{P.}
    Escolhe a ``unidade'' para os pontos. Se você deseja diferenciar entre
    um único ponto e mais de um ponto, pode fornecer uma terminação plural
    separada com uma barra: \keyis{name}{point/s}. Isso também define o nome dos
    pontos bônus.
  \keyval{name-plural}{plural form of name}\Module{points}\Default
    Em vez de formar o plural com uma terminação para o singular,
    esta opção permite definir uma palavra extra para isso. Isso também define a forma
    plural para os pontos bônus.
  \keyval{bonus-name}{name}\Module{points}\Default{P.}
    Escolhe a ``unidade'' para os pontos bônus. Se você deseja diferenciar
    entre um único ponto e mais de um ponto, pode fornecer uma terminação
    plural separada com uma barra: \key{bonus-name}{point/s}.
  \keyval{bonus-plural}{plural form of name}\Module{points}\Default
    Em vez de formar o plural com uma terminação para o singular,
    esta opção permite definir uma palavra extra para isso.
  \keybool{use-name}\Module{points}\Default{true}
    Não exibe o nome de forma alguma. Ou exibe.
  \keyval{format}{code}\Module{points}\Default{\cs*{@firtsofone}}
    \sinceversion{0.9d}Formata número mais nome como um todo. Idealmente
    \meta{code} terminaria com um comando que recebe um argumento. Caso contrário,
    número mais nome serão colocados entre chaves.
  \keyval{number-format}{any code}\Module{points}\Default
    Esta opção permite formatação do número, \eg, itálico:
    \keyis{number-format}{\cs*{textit}}.
  \keyval{bonus-format}{any code}\Module{points}\Default
    Esta opção permite formatação do número dos pontos bônus, \eg,
    itálico: \keyis{bonus-format}{\cs*{textit}}.
  \keybool{parse}\Module{points}\Default{true}
    Se definido como \code{false}, os pontos não são contados e os
    comandos \cs{totalpoints}, \cs{pointssum} e \cs{bonussum} não saberão
    seu valor.
  \keybool{separate-bonus}\Module{points}\Default{false}
    Esta opção determina se pontos e pontos bônus recebem cada um sua própria
    unidade quando aparecem juntos (na margem ou com \cs{totalpoints}).
  \keyval{pre-bonus}{tokens}\Module{points}\Default{\cs*{space}(+}
    Código a ser inserido antes dos pontos bônus quando eles seguem pontos
    normais.
  \keyval{post-bonus}{tokens}\Module{points}\Default{)}
    Código a ser inserido após os pontos bônus quando eles seguem pontos
    normais.
\end{options}

\begin{example}[add-sourcecode-options={literate=}]
  \SetupExSheets[points]{name=point/s,number-format=\color{red}}
  \begin{question}{1}
    Esta é fácil, então apenas 1 ponto pode ser ganho.
  \end{question}
  \begin{question}{7.5}
    Mas esta é difícil! 7.5 pontos estão aí para você!
  \end{question}
\end{example}

\section{Imprimindo Soluções}\label{sec:solutions}
Você já viu que pode imprimir soluções onde elas estão usando a
opção \option{print}. Mas \ExSheets\ oferece muito mais
possibilidades.

Nas próximas subseções o uso do seguinte comando é discutido.
\begin{commands}
  \command{printsolutions}[\oarg{setting}]
    Imprime soluções de questões/exercícios.
\end{commands}

Antes de fazermos isso, uma dica: lembre-se de que você pode definir a opção \option{print}
globalmente:
\begin{sourcecode}
  % no preâmbulo
  \SetupExSheets{solution/print=true}
\end{sourcecode}

Agora, se você deseja compor algum texto dependendo de a opção ser verdadeira ou não,
pode usar os seguintes comandos:
\begin{commands}
  \expandable\command{PrintSolutionsTF}[\marg{true code}\marg{false code}]
    Deixa ou \meta{true code} ou \meta{false code} no fluxo de entrada
    dependendo de as soluções serem impressas ou não, \ie, do valor da
    opção \option{print} da solução. Dentro de um ambiente \env{solution}
    isso sempre imprime \meta{true code}.
  \expandable\command{PrintSolutionsT}[\marg{true code}]
    Deixa ou \meta{true code} ou nada no fluxo de entrada dependendo
    de as soluções serem impressas ou não, \ie, do valor da opção
    \option{print} da solução. Dentro de um ambiente \env{solution} isso sempre
    imprime \meta{true code}.
  \expandable\command{PrintSolutionsF}[\marg{false code}]
    Deixa nada ou \meta{false code} no fluxo de entrada dependendo
    de as soluções serem impressas ou não, \ie, do valor da
    opção \option{print} da solução. Dentro de um ambiente \env{solution} isso
    sempre não imprime nada.
\end{commands}
Eles podem ser úteis se você quiser duas versões de uma folha de exercícios, uma
com os exercícios e uma com as soluções, e você quiser adicionar títulos diferentes
a essas versões, por exemplo.

% When solutions are saved a lot of information is saved. One of them is the
% current counter format. The following option determines wether the saved
% counter format or the currently active one is used when \cs{printsolutions} is
% called:
% \begin{options}
%   \keybool{use-saved-counter-format}\Default{true}
%     \changedversion{0.21}When set to true the counter format of solutions
%     printed by \cs{printsolutions}\label{option:use-saved-counter-format} are
%     independent from the setting of \option{counter-format}. The saved format
%     is used instead.
% \end{options}

\subsection{Imprimir todas}\label{sec:solutions-print-all}
O primeiro e mais fácil uso de \cs{printsolutions} é o seguinte:
\begin{sourcecode}
  \printsolutions
\end{sourcecode}
Não há mais nada a dizer, realmente. Ele imprime todas as soluções que você
especificou, exceto aquelas pertencentes a uma questão com opção \keyis{use}{false}.
Sim, há mais um ponto: \cs{printsolutions} só conhece as soluções
que foram definidas \emph{antes} de seu uso! Isso também é verdade para todo uso
explicado nas próximas seções.

\begin{example}
  \printsolutions
\end{example}

Duas opções permitem adicionar código à lista de soluções quando usadas com
\cs{printsolutions}\Oarg{all} (que é o mesmo que usá-lo sem opção):

\begin{options}
  \keyval{chapter-hook}{code}
    \sinceversion{0.13}Adiciona \meta{code} à lista de soluções toda vez que
    soluções de um novo capítulo são impressas (antes das soluções do
    capítulo correspondente serem impressas).
  \keyval{section-hook}{code}
    \sinceversion{0.13}Adiciona \meta{code} à lista de soluções toda vez que
    soluções de uma nova seção são impressas (antes das soluções da
    seção correspondente serem impressas).
\end{options}

\subsection{Imprimir por capítulo/seção}
\minisec{Capítulo/seção atual}
Se você não está criando uma folha de exercícios ou uma prova, mas está escrevendo um
livro-texto, você talvez queira uma seção no final de cada capítulo mostrando a
solução para os exercícios apresentados naquele capítulo. Neste caso, use o
comando da seguinte forma:
\begin{sourcecode}
  \printsolutions[section]
  % ou
  \printsolutions[chapter]
\end{sourcecode}
Novamente, isso é praticamente auto-explicativo. As soluções para as questões do
capítulo atual\footnote{Somente se a classe de documento que você está usando
  \emph{tiver} capítulos, é claro!} ou seção são impressas.
\begin{example}
  \begin{question}
    Esta é a primeira e única questão nesta seção.
  \end{question}
  \begin{solution}
    Esta será uma das poucas soluções impressas pela seguinte chamada de
    \cs{printsolutions}.
  \end{solution}
  E agora:
  \printsolutions[section]
\end{example}

\minisec{Capítulo/seção específica}\label{sssec:print_specific_section}
Você também pode imprimir apenas as soluções de capítulos ou seções diferentes dos
atuais. A sintaxe é bastante fácil:
\begin{example}
  \printsolutions[section={1-7,10}]
  % o mesmo para capítulos:
  % \printsolutions[chapter={1-7,10}]
\end{example}
Não se esqueça de que \cs{printsolutions} ainda não pode conhecer as soluções da
seção~10. Ele é usado apenas para demonstrar a sintaxe. Você também pode usar
um intervalo aberto, \eg, algo como
\begin{sourcecode}
  \printsolutions[section={-4,10-}]
\end{sourcecode}
Isso imprimiria as soluções das seções~1--4 e de todas as seções com
número 10\footnote{Ou melhor, onde \cs*{value}\Marg{section} é 10 ou maior --
  a formatação real do contador é irrelevante.} e maior.

Há uma desvantagem óbvia: você precisa saber os números das seções! Mas
há uma solução: use a opção de pacote \keyis{use-ref}{true}. Então você
pode fazer algo como
\begin{sourcecode}
  % no preâmbulo:
  \usepackage[use-ref]{exsheets}
  % em algum lugar do seu código após \section{Um título de seção realmente legal}:
  \label{sec:ReallyCool}
  % em algum lugar mais adiante no seu código:
  \printsolutions[section={-\S{sec:ReallyCool}}]
  % que imprimirá todas as soluções de questões até e
  % incluindo a seção realmente legal
\end{sourcecode}
Com a opção de pacote \keyis{use-ref}{true}, cada uso de \cs*{label} criará
rótulos adicionais (um precedido por \code{exse:} e outro por
\code{exch:}) que armazenam o número da seção e o número do capítulo,
respectivamente. Estes são usados internamente por dois comandos \cs{S} e \cs{C}
que se referem ao número da seção e ao número do capítulo em que o rótulo foi criado.
\emph{Estes comandos estão disponíveis apenas como argumentos de}
\cs{printsolutions}.

Como alguns pacotes como o bem conhecido \pkg{hyperref}, por exemplo, redefinem
\cs*{label}, a opção \option{use-ref} não funcionará em conjunto com ele. Neste caso
não use \option{use-ref} e defina \cs{exlabel}\marg{label} em vez disso para
lembrar o número da seção/do capítulo. Seu uso é exatamente como \cs*{label}.
Então, a maneira mais segura é a seguinte:
\begin{sourcecode}
  % no preâmbulo:
  \usepackage{exsheets}
  % em algum lugar do seu código após \section{Um título de seção realmente legal}:
  \exlabel{sec:ReallyCool}
  % em algum lugar mais adiante no seu código:
  \printsolutions[section={-\S{sec:ReallyCool}}]
  % que imprimirá todas as soluções de questões até e
  % incluindo a seção realmente legal
\end{sourcecode}
Por favor, esteja ciente de que os rótulos devem ser processados em uma execução anterior do \LaTeX\
antes que \cs{S} e \cs{C} possam passá-los para \cs{printsolutions}.

\subsection{Imprimir por \acs{id}}\label{ssec:ids}
Agora vem a melhor parte: você também pode imprimir soluções selecionadas! Toda
questão tem um \acs{id}. Para ver qual \acs{id} uma questão tem, você pode chamar
o seguinte comando:
\begin{commands}
  % \command{DebugExSheets}[\Marg{\choices{true,false}}]
  %   Enable or disable visual \ExSheets' debugging.
  \expandable\command{CurrentQuestionID}
    \sinceversion{0.4a}Expande para o \acs{id} da questão atual.
\end{commands}
\begin{options}
  \keybool{debug}
    Habilita ou desabilita a depuração visual do \ExSheets.
\end{options}
Vamos criar mais algumas questões e dar uma olhada no que este comando faz:
\begin{example}
  \SetupExSheets{debug=true}
  \begin{question}[ID=nice!]
    Uma questão com um \acs{id} legal!
  \end{question}
  \begin{solution}
    A solução para a questão com o \acs{id} legal.
  \end{solution}
  \begin{question}{3.75}
    Ainda outra questão. Mas desta vez com um quarto de ponto!
  \end{question}
  \begin{solution}
    Ainda outra solução.
  \end{solution}
\end{example}

Então agora podemos chamar algumas soluções específicas:
\begin{example}
  \printsolutions[byID={first,nice!,10,14}]
\end{example}
Isso faz uso do pacote \pkg{l3sort} que no momento da escrita ainda é
considerado experimental. Caso você esteja se perguntando onde está a solução~14:
a questão~14 não tem solução fornecida.

Se você não quiser que as soluções sejam ordenadas automaticamente mas apareçam na
ordem fornecida, você pode usar a opção
\begin{options}
  \keybool{sorted}\Module{solution}\Default{true}
    Ordena soluções fornecidas por \acs{id} ou não.
\end{options}

\section{Impressão Condicional de Questões}\label{sec:cond-print-quest}
\subsection{Usando Classes}\label{sec:classes}
Para criar diferentes variantes de uma prova escrita ou diferentes níveis de dificuldade
de uma folha de exercícios, é útil se for possível atribuir certas
classes a questões e então dizer ao \ExSheets\ para usar apenas uma ou mais
classes específicas.
\begin{options}
  \keyval{use-classes}{list of classes}\Default
    Quando esta opção é usada, apenas as questões pertencentes às classes
    especificadas são impressas e têm suas soluções salvas.
\end{options}
\begin{example}
  \SetupExSheets{use-classes={A,C}}
  \begin{question}[class=A]
    Pertencendo à classe A.
  \end{question}
  \begin{question}[class=B]
    Pertencendo à classe B.
  \end{question}
  \begin{question}[class=C]
    Pertencendo à classe C!
  \end{question}
\end{example}
Questões de classes que não são usadas são totalmente descartadas. \emph{Isso também
  significa que questões que não têm uma classe atribuída são descartadas.}

\ExplSyntaxOn
 \bool_set_false:N \g__exsheets_use_classes_bool
\ExplSyntaxOff

\subsection{Usando Tópicos}\label{sec:topics}
Similarmente a classes, pode-se atribuir tópicos a questões. O uso é
praticamente idêntico, o significado semântico é diferente.
\begin{options}
  \keyval{use-topics}{list of topics}\Default
    Quando esta opção é usada, apenas as questões pertencentes aos tópicos
    especificados são impressas e têm suas soluções salvas.
\end{options}
\begin{example}
  \SetupExSheets{use-topics={trigonometry}}
  \begin{question}[topic=trigonometry]
    Uma questão de trigonometria.
  \end{question}
  \begin{question}[topic=arithmetics]
    Uma questão de aritmética
  \end{question}
\end{example}
Questões de tópicos que não são usados são totalmente descartadas. \emph{Isso também
  significa que questões que não têm um tópico atribuído são descartadas.}

Se você definir tanto \option{use-classes} quanto \option{use-topics}, então apenas
questões que \emph{correspondem a ambas as categorias} serão usadas.

Idealmente poder-se-ia atribuir mais de um tópico a uma questão, mas isso \emph{não} é
suportado ainda.

\ExplSyntaxOn
 \bool_set_false:N \g__exsheets_use_topics_bool
\ExplSyntaxOff

\subsection{Conceitos de Divisão Próprios}
Na verdade\sinceversion{0.8} tanto classes quanto tópicos são introduzidos no
\ExSheets\ internamente desta forma:
\begin{sourcecode}
  \DeclareQuestionClass{class}{classes}
  \DeclareQuestionClass{topic}{topics}
\end{sourcecode}
o que significa que você pode fazer o mesmo introduzindo seus próprios conceitos de divisão.
\begin{commands}
  \command{DeclareQuestionClass}[\marg{singular name}\marg{plural name}]
    Introduz um novo conceito de divisão e define tanto novas opções para o
    ambiente \env{question} quanto novas opções globais.
\end{commands}

Por exemplo, você poderia decidir que quer agrupar suas questões de acordo com
sua dificuldade. Você poderia colocar a seguinte linha em seu preâmbulo:
\begin{sourcecode}
  \DeclareQuestionClass{difficulty}{difficulties}
\end{sourcecode}
Isso definiria uma opção \option*{use-difficulties} análoga a
\option{use-classes} e \option{use-topics}. Também definiria uma opção
\option{difficulty} para o ambiente \env{question}. Isso significa que você poderia
agora fazer algo como o seguinte:
\begin{example}
  \SetupExSheets{use-difficulties={easy,hard}}
  \begin{question}[difficulty=easy]
    Uma questão fácil.
  \end{question}
  \begin{question}[difficulty=medium]
    Esta é um pouco mais difícil.
  \end{question}
  \begin{question}[difficulty=hard]
    Agora vamos ver se você consegue resolver esta.
  \end{question}
\end{example}

\subsection{Recuperando o Valor de Classe em uma Questão}\label{sec:retr-class-value}
Às vezes pode ser desejável recuperar o valor de uma classe definida por
\cs{DeclareQuestionClass} que uma questão tem para poder imprimir,
digamos. Isso é possível com os seguintes comandos:
\begin{commands}
  \expandable\command{GetQuestionClass}[\marg{class}]
    Imprime o valor de \meta{class} que uma questão tem. O comando é
    expansível. Se a classe não existe ou o valor está vazio, o comando
    expande para nada.
  \command{PrintQuestionClassTF}[\marg{class}\marg{true}\marg{false}]
    Testa se uma questão tem um valor não-vazio para a classe \meta{class} e ou
    deixa \meta{true} ou \meta{false} no fluxo de entrada. No
    argumento \meta{true} você pode se referir ao valor com \code{\#1} onde você
    quer que ele seja impresso.
  \command{PrintQuestionClassT}[\marg{class}\marg{true}]
    Como \cs{PrintQuestionClassTF} mas tem apenas o ramo \meta{true}.
  \command{PrintQuestionClassF}[\marg{class}\marg{false}]
    Como \cs{PrintQuestionClassTF} mas tem apenas o ramo \meta{false}.
\end{commands}

\begin{example}
  \begin{question}[difficulty=hard]
    Esta questão tem o nível de dificuldade
    ``\PrintQuestionClassTF{difficulty}{#1}{??}''.
  \end{question}
\end{example}

\ExplSyntaxOn
 \bool_set_false:N \g__exsheets_use_difficulties_bool
\ExplSyntaxOff

\subsection{Marcando Questões com Tags}
Há\sinceversion{0.20} outra maneira de dividir questões: você pode atribuir
tags a questões:
\begin{sourcecode}
  \begin{question}[tags={foo,bar,baz}]
    ...
  \end{question}
\end{sourcecode}
Você pode então decidir imprimir apenas questões com certas tags usando a
seguinte opção:
\begin{options}
  \keyval{use-tags}{csv list of tags to include}
    Seleciona tags. Quando usada, apenas questões marcadas com pelo menos uma das
    tags em \meta{csv list of tags to include} são impressas.
\end{options}

\section{Adicionando e Usando Informações Adicionais a
  Questões}\label{sec:additional_info}
\subsection{Propriedades de Questões -- o Básico}

Para gerenciar muitas questões e soluções correspondentes, pode ser muito
útil poder salvar e recuperar informações adicionais para as questões.
Isso é possível com os seguintes comandos. Primeiro os para salvar:
\begin{commands}
  \command{DeclareQuestionProperty}[\marg{name}]
    Este comando define uma propriedade de questão \meta{name}. Pode apenas ser
    usado no preâmbulo do documento.
  \command{SetQuestionProperties}[\Marg{\meta{name}=\meta{value},...}]
    Define as propriedades para uma questão específica. Este comando pode apenas ser usado
    dentro do ambiente \env{question}.
\end{commands}
Agora os comandos para recuperar as propriedades:
\begin{commands}
  \command{QuestionNumber}[\marg{id}]
    Recupera o número da questão com o \acs{id} \meta{id}. O
    número é exibido de acordo com o formato definido com
    \option{counter-format}.
  \expandable\command{GetQuestionProperty}[\marg{name}\marg{id}]
    Recupera a propriedade \meta{name} da questão com o \acs{id}
    \meta{id}. É claro que a propriedade deve ter sido declarada antes. O
    comando é expansível. Desde\changedversion{0.12} que as propriedades de uma
    questão são escritas no arquivo \code{aux}, é possível recuperá-las
    antes que o ambiente \env{question} correspondente tenha sido usado.
  \expandable\command{IfQuestionPropertyTF}[\marg{name}\marg{id}\marg{true}\marg{false}]
    Um comando\sinceversion{0.15} que retorna \meta{true} se a questão com
    o \acs{id} \meta{id} tem a propriedade \meta{name} e \meta{false}
    caso contrário. As variantes \cs{IfQuestionPropertyT} e
    \cs{IfQuestionPropertyF} também existem, que têm apenas o ramo \meta{true} ou o
    ramo \meta{false}.
\end{commands}

Digamos que declaramos as propriedades \code{notes}, \code{reference} e
\code{topic}. Por padrão, a propriedade \code{points} está disponível e recebe o
valor do argumento opcional do ambiente \code{question}.

Agora podemos fazer o seguinte:
\begin{example}
  % usa `biblatex'
  \begin{question}[ID=center,topic=LaTeX]{3}
    Explique como você poderia centralizar texto em um documento \LaTeX.
    \SetQuestionProperties{
       topic     = \TeX/\LaTeX ,
       notes     = {Como centralizar texto.},
       reference = {\textcite{companion}}}
  \end{question}
  \begin{solution}
    Para centralizar uma pequena parte do corpo do texto pode-se usar o ambiente
    \env*{center} (\points{1}). Dentro de um ambiente como \env*{table} deve-se
    usar \cs*{centering} (\points{1}). Para linhas únicas há também
    o comando \cs*{centerline} (\points{1}).
  \end{solution}
  \begin{question}[ID=knuthbooks,topic=LaTeX]{2}
    Nomeie dois livros de D.\,E.\,Knuth.
    \SetQuestionProperties{
       topic     = \TeX/\LaTeX ,
       notes     = {Livros de Knuth.},
       reference = {\textcite{knuth:ct:a,knuth:ct:b,knuth:ct:c,knuth:ct:d,knuth:ct:e}}}
  \end{question}
  \begin{solution}
    Por exemplo dois volumes de \citetitle{knuth:ct}:
    \citetitle{knuth:ct:a,knuth:ct:b,knuth:ct:c,knuth:ct:d,knuth:ct:e}. Cada
    resposta válida vale \points{1}
  \end{solution}
\end{example}

Agora é possível recuperar estes valores mais tarde:
\begin{example}
  % usa `booktabs'
  \begin{center}
    \begin{tabular}{lll}
      \toprule
        Questão & Propriedade & \\
      \midrule
      \QuestionNumber{center}
        & Pontos     & \GetQuestionProperty{points}{center} \\
        & Tópico      & \GetQuestionProperty{topic}{center} \\
        & Referências & \GetQuestionProperty{reference}{center} \\
        & Nota       & \GetQuestionProperty{notes}{center} \\
      \midrule
      \QuestionNumber{knuthbooks}
        & Pontos     & \GetQuestionProperty{points}{knuthbooks} \\
        & Tópico      & \GetQuestionProperty{topic}{knuthbooks} \\
        & Referências & \GetQuestionProperty{reference}{knuthbooks} \\
        & Nota       & \GetQuestionProperty{notes}{knuthbooks} \\
      \bottomrule
    \end{tabular}
  \end{center}
\end{example}

Por favor, note que propriedades \emph{não são a mesma coisa} que os conceitos de divisão
explicados na seção~\ref{sec:cond-print-quest}, embora possam parecer
similares em significado ou mesmo ter o mesmo nome.

Quando propriedades são definidas, elas também são escritas no arquivo \code{aux}, o que
significa que podem ser recuperadas \emph{antes} da questão correspondente. É
claro que isso significa que duas execuções de compilação são necessárias.

\subsection{Propriedades Pré-definidas}

Algumas propriedades já são definidas pelo \ExSheets:
\begin{itemize}
  \item \code{counter}:\sinceversion{0.14} esta propriedade contém o
    número real da questão formatado de acordo com a formatação definida com a opção
    \option{counter-format}.
  \item \code{subtitle}:\sinceversion{0.12} esta propriedade contém o subtítulo
    da questão se fornecido.
  \item \code{question-body}:\sinceversion{0.14} esta propriedade contém o corpo
    do ambiente \env{question} correspondente. Ao contrário das outras
    propriedades, por padrão \emph{não} é escrita no arquivo \code{aux}.
  \item \code{points}: esta propriedade contém a soma de pontos atribuídos a uma
    questão.
  \item \code{bonus-points}:\sinceversion{0.14} esta propriedade contém a soma de
    pontos bônus atribuídos a uma questão.
  \item \code{ref}:\sinceversion{0.7f} quando a opção \option{auto-label} é
    usada, esta propriedade é definida e expande para o \cs*{ref} correspondente.
    Veja também a seção~\ref{sec:auto-label-opti}.
  \item \code{page-ref}:\sinceversion{0.7f} quando a opção
    \option{auto-label} é usada, esta propriedade é definida e expande para o
    \cs*{pageref} correspondente. Veja também a seção~\ref{sec:auto-label-opti}.
\end{itemize}

Há uma opção que afeta a propriedade \code{question-body}:
\begin{options}
  \keybool{save-to-aux}\Module{question}\Default{false}
    Quando definida como \code{true}, a propriedade \code{question-body} também é escrita
    no arquivo \code{aux}.
\end{options}

\subsection{Uso Avançado}

Há comandos adicionais\sinceversion{0.3} que podem ser úteis. Eles
permitem uso avançado de propriedades definidas. Abaixo um exemplo é mostrado de como
eles podem ser usados para gerar uma tabela de notas.
\begin{commands}
  \command{ForEachQuestion}[\marg{code to be executed for each used question}]
    \changedversion{0.14}Dentro do argumento pode-se referir ao \ac{id} de uma
    questão com \code{\#1}. Você também pode se referir ao número da
    questão com \code{\#2}. \emph{Número} significa que se você \emph{usar}
    sete questões, então essas questões têm números~1 a~7.
  \expandable\command{numberofquestions}
    \changedversion{0.14} retorna o número completo de questões usadas.
  \expandable\command{iflastquestion}[\marg{true code}\marg{false code}]
    Embora este comando esteja disponível em todo o documento, é útil apenas
    dentro de \cs{ForEachQuestion}. Ele informa se o fim do loop é
    alcançado ou não.
\end{commands}

Poder-se-ia usar esses comandos para criar uma tabela de notas, por exemplo:
\begin{sourcecode}
  \begin{tabular}{|l|*{\numberofquestions}{c|}c|}\hline
    Questão &
      \ForEachQuestion{\QuestionNumber{#1}\iflastquestion{}{&}} &
      Total \\ \hline
    Pontos   &
      \ForEachQuestion{\GetQuestionProperty{points}{#1}\iflastquestion{}{&}} &
      \pointssum* \\ \hline
    Obtidos  &
      \ForEachQuestion{\iflastquestion{}{&}} & \\ \hline
  \end{tabular}
\end{sourcecode}
Para quatro questões, a tabela agora pareceria similar à
figura~\ref{fig:grading-table}.

\begin{figure}[ht]
  \centering
  \begin{tabular}{|l|*{4}{c|}c|}\hline
    Questão & 1. & 2. & 3. & 4. & Total \\ \hline
    Pontos   &  3 &  5 & 10 &  8 & 26 \\ \hline
    Obtidos  &    &    &    &    &    \\ \hline
  \end{tabular}
  \caption{Um exemplo de tabela de notas. (Na verdade isso é uma simulação. Veja o
    arquivo \code{grading-table.tex} fornecido com exsheets para o caso de uso real.)}
  \label{fig:grading-table}
\end{figure}


\section{Variações de uma Prova}

É uma tarefa bastante comum\sinceversion{0.6} projetar uma prova em duas
variantes diferentes. Isso é claro que é possível com as classes do \ExSheets (veja
seção~\ref{sec:classes}). No entanto, frequentemente não é a questão inteira que deve ser
diferente mas apenas pequenos detalhes, os números em uma prova de matemática, digamos. Para este
propósito, \ExSheets\ fornece os seguintes comandos:
\begin{commands}
  \command{SetVariations}[\marg{num}]
    Define o número de variantes diferentes. Isso determinará quantos
    argumentos o comando \cs{vary} receberá. \meta{num} deve ser pelo menos
    \code{2} e é inicialmente definido como \code{2}.
  \command{variant}[\marg{num}]
    Escolhe a variante ativa. O argumento deve ser um número entre \code{1}
    e o número definido com \cs{SetVariations}. Inicialmente definido como \code{1}.
  \command{vary}[\marg{variant 1}\marg{variant 2}]
    Este comando é o que realmente é usado no documento. Ele tem um número de
    argumentos obrigatórios igual ao número definido com \cs{SetVariations}. Todos
    os seus argumentos são descartados exceto o especificado com
    \cs{variant}.
  \command{lastvariant}
    \sinceversion{0.7b}Cada vez que \cs{vary} é chamado, ele armazena o valor que
    escolheu em \cs{lastversion}. Isso pode ser conveniente para usar se
    de outra forma teria que escrever repetidamente o mesmo \cs{vary}.
\end{commands}

\begin{example}
  \SetVariations{6}%
  \variant{6}\vary{A}{B}{C}{D}{E}{F}
  (última variante: \lastvariant)
  \variant{1}\vary{A}{B}{C}{D}{E}{F}
  (última variante: \lastvariant)
  \variant{5}\vary{A}{B}{C}{D}{E}{F}
  (última variante: \lastvariant)
  \variant{2}\vary{A}{B}{C}{D}{E}{F}
  (última variante: \lastvariant)
  \variant{4}\vary{A}{B}{C}{D}{E}{F}
  (última variante: \lastvariant)
  \variant{3}\vary{A}{B}{C}{D}{E}{F}
  (última variante: \lastvariant)
\end{example}

\section{Uma Distribuição de Notas}
Provavelmente este é um recurso bastante esotérico, mas poderia ser útil em alguns
casos. Suponha que você é um professor de matemática alemão e quer dar notas exatamente
correspondentes ao número de pontos relativos à soma dos pontos totais,
independentemente de quão grande isso possa ser. Você poderia fazer algo assim para
apresentar suas decisões de avaliação para a prova:
\begin{example}
  % preâmbulo:
  % \DeclareRelGrades{
  %   1     = 1 ,
  %   {1,5} = .9167 ,
  %   2     = .8333 ,
  %   {2,5} = .75 ,
  %   3     = .6667 ,
  %   {3,5} = .5833 ,
  %   4     = .5
  % }
  \small\setlength\tabcolsep{2pt}
  \begin{tabular}{r|*8c}
    Pontos
    & $\grade*{1}$      & $\le\grade*{1}$ & $\le\grade*{1,5}$ & $\le\grade*{2}$
    & $\le\grade*{2,5}$ & $\le\grade*{3}$ & $\le\grade*{3,5}$ & $<\grade*{4}$ \\
    Nota
    & 1 & 1--2 & 2 & 2--3 & 3 & 3--4 & 4 & 5
  \end{tabular}
\end{example}

Estes são os comandos e opções disponíveis:
\begin{commands}
  \command{DeclareRelGrades}[\Marg{\meta{grade}=\meta{num},...}]
    Este comando é usado para definir notas e atribuir a porcentagem dos pontos
    totais a elas.
  \command{grade}[\sarg\marg{grade}]
    Fornece o número de pontos correspondentes a uma nota dependendo do valor
    de \cs{pointssum} com ou sem (versão estrelada) ``unidade''.
\end{commands}
\begin{options}
  \keyval{round}{num}\Module{grades}\Default{0}
    O número de decimais para os quais os pontos de uma nota são arredondados. Isso não se
    aplica ao número máximo de pontos se o número arredondado seria
    maior que a soma real.
  \keybool{half}\Module{grades}\Default{false}
    Se definido como \code{true}, os pontos são arredondados para pontos inteiros ou metade de
    pontos.
\end{options}

\section{Incluir Seletivamente Questões de Arquivos Externos}\label{sec:include}
\subsection{Advertência}
Preciso fazer algumas palavras de cautela: o \cs{includequestions} que será
apresentado em breve é provavelmente o mais experimental do \ExSheets na época da
escrita (\today). Graças ao feedback dos usuários, é constantemente melhorado e
bugs são corrigidos. Não é uma maneira muito eficiente de inserir questões no que diz respeito ao
desempenho e você não deve se surpreender se a compilação ficar mais lenta quando você o usa.
Provavelmente precisa ser reescrito por completo, mas por um lado isso
introduziria novos bugs e por outro lado por enquanto eu não tenho
as capacidades, de qualquer forma, então você terá que conviver com isso, receio.

\subsection{Como funciona}
Suponha que você tenha um ou mais arquivos com questões preparadas para usá-los como uma
espécie de banco de dados. Um para a classe A, digamos, um para a classe B, um para a classe C e
assim por diante, algo assim:
\begin{sourcecode}
  % este é o arquivo classA.tex
  \begin{question}[class=A,ID=A1,topic=X]
    Primeira questão da classe A, tópico X.
  \end{question}
  \begin{solution}
    Primeira solução da classe A.
  \end{solution}
  \begin{question}[class=A,ID=A2,topic=Y]
    Segunda questão da classe A, tópico Y.
  \end{question}
  \begin{solution}
    Segunda solução da classe A.
  \end{solution}
  ...
  % fim do arquivo classA.tex
  \endinput
\end{sourcecode}
Você pode é claro apenas \cs*{input} ou \cs*{include} isto, mas isso
claro incluiria o arquivo inteiro em seu documento. Mas não seria bom
apenas incluir questões selecionadas? Ou talvez cinco questões aleatórias do
arquivo? Isso é possível com o seguinte comando:
\begin{commands}
  \command{includequestions}[\oarg{options}\marg{list of filenames}]
    Inclui questões de arquivos externos.
\end{commands}
Se você usá-lo sem opções, terá o mesmo efeito que \cs*{input}.
Existem, no entanto, as seguintes opções:
\begin{options}
  \keybool{all}\Module{include}
  \keyval{IDs}{list of IDs}\Module{include}\Default
    Inclui apenas as questões especificadas.
  \keyval{random}{num}\Module{include}\Default
    Inclui \meta{num} questões selecionadas aleatoriamente. Esta opção usa o
    pacote \pkg{pgfcore} para criar os números pseudo-aleatórios.
  \keyval{exclude}{list of IDs}\Module{include}\Default
    Questões cujos \acp{id} são especificados aqui \emph{não} são incluídos. Esta
    opção pode ser combinada com a opção \option{random}.
\end{options}

O uso deve ser auto-explicável:
\begin{sourcecode}
  % incluir questões A1, A3 e A4:
  \includequestions[IDs={A1,A3,A4}]{classA.tex}
  % ou incluir 3 questões aleatórias:
  \includequestions[random=3]{classA}
\end{sourcecode}
Para poder selecionar as questões, \ExSheets\ precisa fazer \cs*{input} do
arquivo duas vezes. A primeira vez as questões disponíveis são determinadas, a
segunda vez as questões selecionadas são usadas. Isso infelizmente significa que
qualquer coisa que \emph{não} seja parte de uma questão ou solução também é lida
duas vezes. Ou não coloque mais nada no arquivo ou use um dos
seguintes comandos para controle:
\begin{commands}
  \command{PrintIfIncludeActiveTF}[\marg{true code}\marg{false code}]
    Verifica se as questões são ativamente incluídas ou não e coloca \meta{true
      code} ou \meta{false code} no fluxo de entrada dependendo da resposta.
  \command{PrintIfIncludeActiveT}[\marg{true code}]
    Verifica se as questões são ativamente incluídas ou não e coloca \meta{true
      code} no fluxo de entrada se a resposta for sim.
  \command{PrintIfIncludeActiveF}[\marg{false code}]
    Verifica se as questões são ativamente incluídas ou não e coloca \meta{false
      code} no fluxo de entrada se a resposta for não.
\end{commands}

A seleção pode ser refinada ainda mais selecionando questões pertencentes a uma
classe específica de questões (veja seção~\ref{sec:cond-print-quest}) antes de
usar \cs{includequestions}.

\sinceversion{0.8}Depois que você usou \cs{includequestions}, os \acp{id} das
questões incluídas estão disponíveis como uma lista separada por vírgulas não ordenada na
seguinte macro:
\begin{commands}
  \command{questionsincludedlast}
    Lista separada por vírgulas não ordenada de \acp{id} de questões incluídos com o último
    uso de \cs{includequestions}.
\end{commands}

\section{A Opção \option*{auto-label}}\label{sec:auto-label-opti}
A\sinceversion{0.12} opção de pacote \option{auto-label} define um
\cs*{label}\Marg{qu:\meta{id}} toda vez que o ambiente question é usado.
Tanto o comando usado quanto o rótulo automatizado podem ser personalizados usando as
seguintes opções:

\begin{options}
  \keyval{label-format}{code}\Default{qu:\#1}
    O padrão para gerar o rótulo automático. \code{\#1} é substituído
    pelo \ac{id} da questão correspondente.
  \keyval{label-cmd}{macro}\Default{\cs*{label}}
    O comando usado para gerar o rótulo. Um comando que deve receber um
    argumento obrigatório.
  \keyval{ref-cmd}{macro}\Default{\cs*{ref}}
    O comando usado na propriedade \code{ref} criada pela
    opção \option{auto-label}, veja também seção \ref{sec:additional_info}.
    O comando deve receber um argumento obrigatório.
  \keyval{pageref-cmd}{macro}\Default{\cs*{pageref}}
    O comando usado na propriedade \code{pageref} criada pela
    opção \option{auto-label}, veja também seção \ref{sec:additional_info}.
    O comando deve receber um argumento obrigatório.
\end{options}

\section{Pares Questão/Solução Próprios}
\ExSheets\changedversion{0.9} fornece a possibilidade de criar novos
ambientes que se comportam como os ambientes \env{question} e \env{solution}.
Isso permitiria, por exemplo, definir um
par de ambientes \env*{question*}/\env*{solution*} para questões bônus. Os
seguintes comandos podem ser usados no preâmbulo do documento:
\begin{commands}
  \command{NewQuSolPair}[\marg{question}\oarg{question options}\oarg{general
    options}\marg{solution}\oarg{solution options}\oarg{general options}]
    Define um novo par de ambientes de questão e solução.
  \command{RenewQuSolPair}[\marg{question}\oarg{question options}\oarg{general
    options}\marg{solution}\oarg{solution options}\oarg{general options}]
    Redefine um par existente de ambientes de questão e solução.
\end{commands}
Os ambientes padrão são definidos da seguinte forma:
\begin{sourcecode}
  \NewQuSolPair{question}{solution}
\end{sourcecode}

Digamos que queremos a possibilidade de adicionar questões bônus. Uma maneira simples seria
definir variantes estreladas que adicionam uma estrela na margem à esquerda do título:
\begin{example}
  % preâmbulo:
  % - \texorpdfstring é fornecido por `hyperref'
  % - \bigstar é fornecido por `amssymb'
  % \DeclareRobustCommand*\questionstar{\texorpdfstring{\bonusquestionsign}{* }}
  % \DeclareRobustCommand*\bonusquestionsign{\llap{$\bigstar$\space}}
  %
  % \NewQuSolPair
  %   {question*}[name=\questionstar Questão Bônus]
  %   {solution*}[name=\questionstar Solução]
  \begin{question*}
    Esta é uma questão bônus.
  \end{question*}
  \begin{solution*}[print]
    Isto é como a solução se parece.
  \end{solution*}
\end{example}
Como você pode ver, os ambientes recebem as mesmas opções que são descritas para os
ambientes padrão \env{question} e \env{solution}.

\section{Preenchendo Espaços em Branco}
\subsection{Cloze}
Tanto\changedversion{0.4} em folhas de exercícios quanto em provas, às vezes é
desejável poder criar \blank{espaços em branco} que precisam ser preenchidos. Ou
talvez mais algumas linhas: \blank[width=5\linewidth]{}

\begin{commands}
  \command{blank}[\sarg\oarg{options}\marg{text to be filled in}]
    cria um espaço em branco em texto normal ou em uma questão, mas preenche o texto de seu
    argumento se dentro de uma solução. Se usado no \emph{início de um parágrafo}
    \cs{blank} fará duas coisas: definirá o espaçamento de linha de acordo com uma
    opção explicada abaixo e inserirá \cs*{par} após as linhas. Se você
    não quiser isso, use a versão estrelada.
\end{commands}

As opções são estas:
\begin{options}
  \keychoice{style}{line,wave,dline,dotted,dashed}\Module{blank}\Default{line}
    O estilo da linha. Isso usa o comando correspondente do
    pacote \pkg{ulem} e é toda a razão pela qual \ExSheets\ o carrega em
    primeiro lugar.
  \keyval{scale}{num}\Module{blank}\Default{1}
    Escala a largura do espaço em branco pelo fator \meta{num}, a menos que a largura seja
    explicitamente definida.
  \keyval{width}{dim}\Module{blank}\Default
    A largura da linha. Se não for usada, a largura do texto preenchido
    é usada.
  \keyval{linespread}{num}\Module{blank}\Default{1}
    Define o espaçamento de linha para as linhas em branco. Isso só tem efeito se
    \cs{blank} for usado no início de um parágrafo.
  \keyval{line-increment}{dim}\Module{blank}\Default{1pt}
    \sinceversion{0.21h}Quando a linha em branco é construída, ela é construída em múltiplos
    deste valor. Se o valor for muito grande, você pode acabar com linhas
    irregulares. Se o valor for muito pequeno, você pode acabar com uma
    compilação sem fim.
  \keyval{line-minimum-length}{dim}\Module{blank}\Default{2em}
    \sinceversion{0.21h}O comprimento mínimo que uma linha deve ter antes de ser construída
    passo a passo.
\end{options}
\begin{example}
  \begin{question}
    Tente preencher \blank[width=4cm]{estes} espaços em branco. Todos eles
    \blank[style=dotted]{são criados} usando o comando \cs{blank}
    \blank[style=dashed]{}.
  \end{question}
  \begin{solution}[print]
    Tente preencher \blank[width=4cm]{estes} espaços em branco. Todos eles
    \blank[style=dotted]{são criados} usando o comando \cs{blank}
    \blank[style=dashed]{}.
  \end{solution}
\end{example}
Várias linhas vazias são facilmente criadas definindo a opção width:
\begin{example}
  \blank[width=4.8\linewidth,linespread=1.5]{}
\end{example}

\subsection{Espaço Vertical para respostas}
Quando\sinceversion{0.3} você está criando uma prova, talvez queira adicionar algum
espaço vertical onde os alunos possam escrever suas respostas. Embora você sempre possa
usar \cs*{vspace}, isso nem sempre é prático quando o espaço restante na
página é menor do que você deseja. Neste caso, seria bom se a) não houvesse
aviso e b) o resto do espaço fosse adicionado no topo da
próxima página. Isto é para o que o seguinte comando serve:
\begin{commands}
  \command{examspace}[\sarg\marg{dim}]
    Adiciona espaço conforme especificado em \meta{dim}. Se o espaço disponível na
    página atual não for suficiente, o resto do espaço será adicionado no topo
    da próxima página. A versão estrelada descartará silenciosamente qualquer espaço
    restante em vez de adicioná-lo à próxima página.
\end{commands}
\begin{example}[side-by-side]
  \begin{question}
   O que você acha deste recurso?
   \examspace{3cm}
  \end{question}
  Esta linha vem após o espaço.
\end{example}

\section{Estilizando suas Folhas de Exercícios/Provas}\label{part:style}
\subsection{Contexto}
O pacote \ExSheets\ faz uso extensivo dos coffins do \LaTeX3\footnote{Veja
  a documentação do pacote \pkg{xcoffins} para mais informações sobre
  isso.} assim como de seu conceito de templates\footnote{Dê uma olhada na
  documentação do pacote \pkg{xtemplate}.}. Este último permite uma
extensão e personalização relativamente fácil de alguns ambientes do \ExSheets.
Para ser mais preciso: você pode definir suas próprias instâncias para os cabeçalhos usados
para questões e soluções.

O que este pacote não fornece é alterar o plano de fundo de questões ou
emoldurá-las. Mas isso é facilmente possível usando o pacote \pkg{mdframed}
e seu comando \cs*{surroundwithmdframed}.

\ExSheets{} também fornece as opções \option{pre-hook}, \option{post-hook},
\option{pre-body-hook} e \option{post-body-hook} tanto para o ambiente question quanto para
o ambiente solution. Com elas, é bastante direto adicionar uma
moldura \pkg{mdframed}, por exemplo:
\begin{sourcecode}
  \SetupExSheets{
    solution/pre-hook = \mdframed ,
    solution/post-hook = \endmdframed
  }
\end{sourcecode}

Então\sinceversion{0.18} há a macro
\cs{exsheetsprintsolution}\marg{heading}\marg{body}\label{exsheetsprintsolution}
que pode ser redefinida para atender às suas necessidades. A definição padrão é
equivalente a
\begin{sourcecode}
  \newcommand\exsheetsprintsolution[2]{#1#2}
\end{sourcecode}

\subsection{O Objeto \code{exsheets-headings}}\label{sec:exsheets-headings}
\ExSheets\ define o objeto \code{exsheets-headings} e um template para ele,
o template `default'. O pacote também define duas instâncias deste
template, a instância `block' e a instância `runin'.

\begin{example}
  \SetupExSheets{headings=block}
  \begin{question}{1}
    um cabeçalho `block'
  \end{question}
  \SetupExSheets{headings=runin}
  \begin{question}{1}
    um cabeçalho `runin'
  \end{question}
\end{example}

\subsubsection{Opções Disponíveis}
Esta seção apenas lista as opções que podem ser usadas ao definir uma instância
do template `default'. As subseções seguintes darão muitos
exemplos de seu uso. As opções estão listadas na definição para a
interface do template:

\begin{sourcecode}
  \DeclareTemplateInterface{exsheets-heading}{default}{3}{
    % option           : type      = default
    inline             : boolean   = false ,
    runin              : boolean   = false ,
    indent-first       : boolean   = false ,
    toc-reversed       : boolean   = false ,
    vscale             : real      = 1     ,
    above              : length    = 2pt   ,
    below              : length    = 2pt   ,
    main               : tokenlist =       ,
    pre-code           : tokenlist =       ,
    post-code          : tokenlist =       ,
    title-format       : tokenlist =       ,
    title-pre-code     : tokenlist =       ,
    title-post-code    : tokenlist =       ,
    number-format      : tokenlist =       ,
    number-pre-code    : tokenlist =       ,
    number-post-code   : tokenlist =       ,
    subtitle-format    : tokenlist =       ,
    subtitle-pre-code  : tokenlist =       ,
    subtitle-post-code : tokenlist =       ,
    points-format      : tokenlist =       ,
    points-pre-code    : tokenlist =       ,
    points-post-code   : tokenlist =       ,
    join               : tokenlist =       ,
    attach             : tokenlist =
  }
\end{sourcecode}

Cada cabeçalho é construído com no máximo cinco coffins disponíveis com os nomes
`main', `title', `subtitle', `number' e `points'. Esses coffins colocam
possivelmente o cabeçalho inteiro, o título, o subtítulo, o número da questão e
os pontos atribuídos. O único coffin que sempre é composto é o coffin `main',
que é vazio por padrão.

Coffins podem ser unidos (dois se tornam um, o primeiro estende sua caixa delimitadora para
conter o segundo) usando a seguinte sintaxe:
\begin{sourcecode}
  join = coffin1[handle11,handle12]coffin2[handle21,handle22](x-offset,y-offset)
\end{sourcecode}
A sintaxe para anexar (dois se tornam um, o primeiro \emph{não} estende sua
caixa delimitadora ao redor do segundo) é a mesma.

Mais sobre handles de coffin é descrito na documentação do
\pkg{xcoffins}. A Figura~\ref{fig:handles} demonstra brevemente os pares de
handles disponíveis.

\begin{figure}[ht]
 \centering
 \parbox{4.5cm}{%
   \NewCoffin\ExampleCoffin
   \SetHorizontalCoffin\ExampleCoffin{\color{gray!30}\rule{4cm}{4cm}}%
   \DisplayCoffinHandles\ExampleCoffin{blue}%
 }
 \caption{Handles disponíveis para um coffin horizontal.}\label{fig:handles}
\end{figure}

É possível\sinceversion{0.20} adicionar coffins estáticos próprios:
\begin{commands}
  \command{DeclareExSheetsHeadingContainer}[\marg{name}\marg{code}]
    Define um novo coffin \meta{name} contendo \meta{code}. Você pode se referir ao
    \ac{id} da questão atual com \cs{CurrentQuestionID}.
\end{commands}

As subseções seguintes mostrarão todas as definições das instâncias disponíveis
e como elas parecem. Isso esperançosamente lhe dará ideias suficientes para criar sua
própria instância se você quiser ter outro estilo de cabeçalho além dos
disponíveis. Cada uma das instâncias seguintes está disponível através da opção
\key{headings}{instance}.

Os exemplos seguintes usam um texto de amostra definido da seguinte forma:
\begin{sourcecode}
  \def\s{Este é um texto de amostra que usaremos para criar um texto
    um pouco mais longo que se estende por algumas linhas.}
  \def\sample{\s\ \s\par\s}
\end{sourcecode}
\def\s{Este é um texto de amostra que usaremos para criar um texto um pouco mais longo
 que se estende por algumas linhas.}
\def\sample{\s\ \s\par\s}

Todos os exemplos seguintes usam a mesma chamada de questão:
\begin{sourcecode}
  \SetupExSheets{headings=<name>}
  \begin{question}[subtitle=O subtítulo da questão]{1}
    Um cabeçalho `<name>'. \sample
  \end{question}
\end{sourcecode}

\subsubsection{The `block' Instance}
\begin{sourcecode}
  \DeclareInstance{exsheets-heading}{block}{default}{
    join             = { title[r,B]number[l,B](.333em,0pt) } ,
    attach           =
      {
        main[l,vc]title[l,vc](0pt,0pt) ;
        main[r,vc]points[l,vc](\marginparsep,0pt)
      }
  }
\end{sourcecode}
\SetupExSheets{headings=block}
\begin{question}[subtitle=The subtitle of the question]{1}
  A `block' heading. \sample
\end{question}

\subsubsection{The `runin' Instance}
\begin{sourcecode}
  \DeclareInstance{exsheets-heading}{runin}{default}{
    runin            = true ,
    number-post-code = \space ,
    attach           =
      { main[l,vc]points[l,vc](\linewidth+\marginparsep,0pt) } ,
    join             =
      {
        main[r,vc]title[r,vc](0pt,0pt) ;
        main[r,vc]number[l,vc](.333em,0pt)
      }
  }
\end{sourcecode}
\SetupExSheets{headings=runin}
\begin{question}[subtitle=The subtitle of the question]{1}
  A `runin' heading. \sample
\end{question}

\subsubsection{The `simple' Instance}
\begin{sourcecode}
  \DeclareInstance{exsheets-heading}{simple}{default}{
    title-format     = \normalsize ,
    points-pre-code  = ( ,
    points-post-code = ) ,
    attach           = { main[l,t]number[l,t](0pt,0pt) } ,
    join             =
      {
        number[r,b]title[l,b](.333em,0pt) ;
        main[l,b]points[l,t](1em,0pt)
      }
  }
\end{sourcecode}
\SetupExSheets{headings=simple}
\begin{question}[subtitle=The subtitle of the question]{1}
  A `simple' heading. \sample
\end{question}

\subsubsection{The `empty' Instance}
\sinceversion{0.9a}
\begin{sourcecode}
  \DeclareInstance{exsheets-heading}{empty}{default}{
    runin  = true ,
    above  = \parskip ,
    below  = \parskip ,
    attach = { main[l,vc]points[l,vc](\linewidth+\marginparsep,0pt) }
  }
\end{sourcecode}
\SetupExSheets{headings=empty}
\begin{question}[subtitle=The subtitle of the question]{1}
  An `empty' heading. \sample
\end{question}

\subsubsection{The `block-rev' Instance}
\begin{sourcecode}
  \DeclareInstance{exsheets-heading}{block-rev}{default}{
    toc-reversed     = true ,
    join             = { number[r,B]title[l,B](.333em,0pt) } ,
    attach           =
      {
        main[l,vc]number[l,vc](0pt,0pt) ;
        main[r,vc]points[l,vc](\marginparsep,0pt)
      }
  }
\end{sourcecode}
\SetupExSheets{headings=block-rev}
\begin{question}[subtitle=The subtitle of the question]{1}
  A `block-rev' heading. \sample
\end{question}

\subsubsection{The `block-subtitle' Instance}
\sinceversion{0.10}
\begin{sourcecode}
  \DeclareInstance{exsheets-heading}{block-subtitle}{default}{
    join = {
      title[r,B]number[l,B](.333em,0pt) ;
      title[r,B]subtitle[l,B](1em,0pt)
    } ,
    attach = {
      main[l,vc]title[l,vc](0pt,0pt) ;
      main[r,vc]points[l,vc](\marginparsep,0pt)
    }
  }
\end{sourcecode}
\SetupExSheets{headings=block-subtitle}
\begin{question}[subtitle=The subtitle of the question]{1}
  A `block-subtitle' heading. \sample
\end{question}
\subsubsection{The `block-wp' Instance}
\begin{sourcecode}
  \DeclareInstance{exsheets-heading}{block-wp}{default}{
    points-pre-code  = ( ,
    points-post-code = ) ,
    join             =
      {
        title[r,B]number[l,B](.333em,0pt) ;
        title[r,B]points[l,B](.333em,0pt)
      } ,
    attach           = { main[l,vc]title[l,vc](0pt,0pt) }
  }
\end{sourcecode}
\SetupExSheets{headings=block-wp}
\begin{question}[subtitle=The subtitle of the question]{1}
  A `block-wp' heading. \sample
\end{question}

\subsubsection{The `block-wp-rev' Instance}
\begin{sourcecode}
  \DeclareInstance{exsheets-heading}{block-wp-rev}{default}{
    toc-reversed     = true ,
    points-pre-code  = ( ,
    points-post-code = ) ,
    join             =
      {
        number[r,B]title[l,B](.333em,0pt) ;
        number[r,B]points[l,B](.333em,0pt)
      } ,
    attach           = { main[l,vc]number[l,vc](0pt,0pt) }
  }
\end{sourcecode}
\SetupExSheets{headings=block-wp-rev}
\begin{question}[subtitle=The subtitle of the question]{1}
  A `block-wp-rev' heading. \sample
\end{question}

\subsubsection{The `block-nr' Instance}
\begin{sourcecode}
  \DeclareInstance{exsheets-heading}{block-nr}{default}{
    attach           =
      {
        main[l,vc]number[l,vc](0pt,0pt) ;
        main[r,vc]points[l,vc](\marginparsep,0pt)
      }
  }
\end{sourcecode}
\SetupExSheets{headings=block-nr}
\begin{question}[subtitle=The subtitle of the question]{1}
  A `block-nr' heading. \sample
\end{question}

\subsubsection{The `block-nr-wp' Instance}
\begin{sourcecode}
  \DeclareInstance{exsheets-heading}{block-nr-wp}{default}{
    points-pre-code  = ( ,
    points-post-code = ) ,
    join             = { number[r,vc]points[l,vc](.333em,0pt) } ,
    attach           = { main[l,vc]number[l,vc](0pt,0pt) }
  }
\end{sourcecode}
\SetupExSheets{headings=block-nr-wp}
\begin{question}[subtitle=The subtitle of the question]{1}
  A `block-nr-wp' heading. \sample
\end{question}

\subsubsection{The `runin-rev' Instance}
\begin{sourcecode}
  \DeclareInstance{exsheets-heading}{runin-rev}{default}{
    toc-reversed     = true ,
    runin            = true ,
    title-post-code  = \space ,
    attach           =
      { main[l,vc]points[l,vc](\linewidth+\marginparsep,0pt) } ,
    join             =
      {
        main[r,vc]number[r,vc](0pt,0pt) ;
        main[r,vc]title[l,vc](.333em,0pt)
      }
  }
\end{sourcecode}
\SetupExSheets{headings=runin-rev}
\begin{question}[subtitle=The subtitle of the question]{1}
  A `runin-rev' heading. \sample
\end{question}

\subsubsection{The `runin-wp' Instance}
\begin{sourcecode}
  \DeclareInstance{exsheets-heading}{runin-wp}{default}{
    runin            = true ,
    points-pre-code  = ( ,
    points-post-code = )\space ,
    join             =
      {
        main[r,vc]title[r,vc](0pt,0pt) ;
        main[r,vc]number[l,vc](.333em,0pt) ;
        main[r,vc]points[l,vc](.333em,0pt)
      }
  }
\end{sourcecode}
\SetupExSheets{headings=runin-wp}
\begin{question}[subtitle=The subtitle of the question]{1}
  A `runin-wp' heading. \sample
\end{question}

\subsubsection{The `runin-wp-rev' Instance}
\begin{sourcecode}
  \DeclareInstance{exsheets-heading}{runin-wp-rev}{default}{
    toc-reversed     = true ,
    runin            = true ,
    points-pre-code  = ( ,
    points-post-code = )\space ,
    join             =
      {
        main[r,vc]number[r,vc](0pt,0pt) ;
        main[r,vc]title[l,vc](.333em,0pt) ;
        main[r,vc]points[l,vc](.333em,0pt)
      }
  }
\end{sourcecode}
\SetupExSheets{headings=runin-wp-rev}
\begin{question}[subtitle=The subtitle of the question]{1}
  A `runin-wp-rev' heading. \sample
\end{question}

\subsubsection{The `runin-nr' Instance}
\begin{sourcecode}
  \DeclareInstance{exsheets-heading}{runin-nr}{default}{
    runin            = true ,
    number-post-code = \space ,
    attach           =
      { main[l,vc]points[l,vc](\linewidth+\marginparsep,0pt) } ,
    join             = { main[r,vc]number[l,vc](0pt,0pt) }
  }
\end{sourcecode}
\SetupExSheets{headings=runin-nr}
\begin{question}[subtitle=The subtitle of the question]{1}
  A `runin-nr' heading. \sample
\end{question}

\subsubsection{The `runin-fixed-nr' Instance}
\begin{sourcecode}
  \DeclareInstance{exsheets-heading}{runin-fixed-nr}{default}{
    runin            = true ,
    number-pre-code  = \hbox to 2em \bgroup ,
    number-post-code = \hfil\egroup ,
    attach           =
      { main[l,vc]points[l,vc](\linewidth+\marginparsep,0pt) } ,
    join             = { main[r,vc]number[l,vc](0pt,0pt) }
  }
\end{sourcecode}
\SetupExSheets{headings=runin-fixed-nr}
\begin{question}[subtitle=The subtitle of the question]{1}
  A `runin-fixed-nr' heading. \sample
\end{question}

\subsubsection{The `runin-nr-wp' Instance}
\begin{sourcecode}
  \DeclareInstance{exsheets-heading}{runin-nr-wp}{default}{
    runin            = true ,
    points-pre-code  = ( ,
    points-post-code = )\space ,
    join             =
      {
        main[r,vc]number[l,vc](0pt,0pt) ;
        main[r,vc]points[l,vc](.333em,0pt)
      }
  }
\end{sourcecode}
\SetupExSheets{headings=runin-nr-wp}
\begin{question}[subtitle=The subtitle of the question]{1}
  A `runin-nr-wp' heading. \sample
\end{question}

\subsubsection{The `inline' Instance}
\sinceversion{0.5}
\begin{sourcecode}
  \DeclareInstance{exsheets-heading}{inline}{default}{
    inline           = true ,
    number-pre-code  = \space ,
    number-post-code = \space ,
    join             =
      {
        main[r,vc]title[r,vc](0pt,0pt) ;
        main[r,vc]number[l,vc](0pt,0pt)
      }
  }
\end{sourcecode}
\SetupExSheets{headings=inline}
Text before
\begin{question}[subtitle=The subtitle of the question]{1}
  An `inline' heading. \sample
\end{question}
 Text after

\subsubsection{The `inline-wp' Instance}
\sinceversion{0.5}
\begin{sourcecode}
  \DeclareInstance{exsheets-heading}{inline-wp}{default}{
    inline           = true ,
    number-pre-code  = \space ,
    number-post-code = \space ,
    points-pre-code  = ( ,
    points-post-code = )\space ,
    join             =
      {
        main[r,vc]title[r,vc](0pt,0pt) ;
        main[r,vc]number[l,vc](0pt,0pt) ;
        main[r,vc]points[l,vc](0pt,0pt)
      }
  }
\end{sourcecode}
\SetupExSheets{headings=inline-wp}
Text before
\begin{question}[subtitle=The subtitle of the question]{1}
  An `inline-wp' heading. \sample
\end{question}
 Text after

\subsubsection{The `inline-nr' Instance}
\sinceversion{0.5}
\begin{sourcecode}
  \DeclareInstance{exsheets-heading}{inline-nr}{default}{
    inline           = true ,
    number-post-code = \space ,
    join             = { main[r,vc]number[l,vc](0pt,0pt) }
  }
\end{sourcecode}
\SetupExSheets{headings=inline-nr}
Text before
\begin{question}[subtitle=The subtitle of the question]{1}
  An `inline-nr' heading. \sample
\end{question}
 Text after

\subsubsection{The `centered' Instance}
\begin{sourcecode}
  \DeclareInstance{exsheets-heading}{centered}{default}{
    join             = { title[r,B]number[l,B](.333em,0pt) } ,
    attach           =
      {
        main[hc,vc]title[hc,vc](0pt,0pt) ;
        main[r,vc]points[l,vc](\marginparsep,0pt)
      }
  }
\end{sourcecode}
\SetupExSheets{headings=centered}
\begin{question}[subtitle=The subtitle of the question]{1}
  A `centered' heading. \sample
\end{question}

\subsubsection{The `centered-wp' Instance}
\begin{sourcecode}
  \DeclareInstance{exsheets-heading}{centered-wp}{default}{
    points-pre-code  = ( ,
    points-post-code = ) ,
    join             =
      {
        title[r,B]number[l,B](.333em,0pt) ;
        title[r,B]points[l,B](.333em,0pt)
      } ,
    attach           = { main[hc,vc]title[hc,vc](0pt,0pt) }
  }
\end{sourcecode}
\SetupExSheets{headings=centered-wp}
\begin{question}[subtitle=The subtitle of the question]{1}
  A `centered-wp' heading. \sample
\end{question}

\subsubsection{The `margin' Instance}
\begin{sourcecode}
  \DeclareInstance{exsheets-heading}{margin}{default}{
    runin            = true ,
    number-post-code = \space ,
    points-pre-code  = ( ,
    points-post-code = )\space ,
    join             = { title[r,b]number[l,b](.333em,0pt) } ,
    attach           =
      {
        main[l,vc]title[r,vc](0pt,0pt) ;
        main[l,b]points[r,t](0pt,0pt)
      }
  }
\end{sourcecode}
\SetupExSheets{headings=margin}
\begin{question}[subtitle=The subtitle of the question]{1}
  A `margin' heading. \sample
\end{question}

\subsubsection{The `margin-nr' Instance}
\begin{sourcecode}
  \DeclareInstance{exsheets-heading}{margin-nr}{default}{
    runin  = true ,
    attach =
      {
        main[l,vc]number[r,vc](-.333em,0pt) ;
        main[r,vc]points[l,vc](\linewidth+\marginparsep,0pt)
      }
  }
\end{sourcecode}
\SetupExSheets{headings=margin-nr}
\begin{question}[subtitle=The subtitle of the question]{1}
  A `margin-nr' heading. \sample
\end{question}

\subsubsection{The `raggedleft' Instance}
\begin{sourcecode}
  \DeclareInstance{exsheets-heading}{raggedleft}{default}{
    join             = { title[r,B]number[l,B](.333em,0pt) } ,
    attach           =
      {
        main[r,vc]title[r,vc](0pt,0pt) ;
        main[r,vc]points[l,vc](\marginparsep,0pt)
      }
  }
\end{sourcecode}
\SetupExSheets{headings=raggedleft}
\begin{question}[subtitle=The subtitle of the question]{1}
  A `raggedleft' heading. \sample
\end{question}

\subsubsection{The `fancy' Instance}
\begin{sourcecode}
  \DeclareInstance{exsheets-heading}{fancy}{default}{
    toc-reversed     = true ,
    indent-first     = true ,
    vscale           = 2 ,
    pre-code         = \rule{\linewidth}{1pt} ,
    post-code        = \rule{\linewidth}{1pt} ,
    title-format     = \large\scshape\color{rgb:red,0.65;green,0.04;blue,0.07} ,
    number-format    = \large\bfseries\color{rgb:red,0.02;green,0.04;blue,0.48} ,
    points-format    = \itshape ,
    join             = { number[r,B]title[l,B](.333em,0pt) } ,
    attach           =
      {
        main[hc,vc]number[hc,vc](0pt,0pt) ;
        main[l,vc]points[r,vc](-\marginparsep,0pt)
      }
  }
\end{sourcecode}
\SetupExSheets{headings=fancy}
\begin{question}[subtitle=The subtitle of the question]{1}
  A `fancy' heading. \sample
\end{question}

\subsubsection{The `fancy-wp' Instance}
\begin{sourcecode}
  \DeclareInstance{exsheets-heading}{fancy-wp}{default}{
    toc-reversed     = true ,
    indent-first     = true ,
    vscale           = 2 ,
    pre-code         = \rule{\linewidth}{1pt} ,
    post-code        = \rule{\linewidth}{1pt} ,
    title-format     = \large\scshape\color{rgb:red,0.65;green,0.04;blue,0.07} ,
    number-format    = \large\bfseries\color{rgb:red,0.02;green,0.04;blue,0.48} ,
    points-format    = \itshape ,
    points-pre-code  = ( ,
    points-post-code = ) ,
    join             =
      {
        number[r,B]title[l,B](.333em,0pt) ;
        number[r,B]points[l,B](.333em,0pt)
      } ,
    attach           = { main[hc,vc]number[hc,vc](0pt,0pt) }
  }
\end{sourcecode}
\SetupExSheets{headings=fancy-wp}
\begin{question}[subtitle=The subtitle of the question]{1}
  A `fancy-wp' heading. \sample
\end{question}

\subsection{Usando um Cabeçalho \ExSheets{} em Código Personalizado}\label{sec:using-an-exsheets}

Pode ser útil ter acesso a cabeçalhos \ExSheets{} em código personalizado. Isso
é possível com o seguinte comando\sinceversion{0.14}:

\begin{commands}
  \command{ExSheetsHeading}[\marg{instance}\marg{title}\marg{number}%
    \marg{points}\marg{bonus}\marg{id}]
    O significado dos argumentos é o seguinte:
    \begin{itemize}
      \item \meta{instance}: o nome da instância de cabeçalho a ser usada.
      \item \meta{title}: o conteúdo do coffin \code{title}.
      \item \meta{number}: o conteúdo do coffin \code{number}.
      \item \meta{points}: O número de pontos dados à questão. Se
        não-zero, isso fará com que os pontos sejam impressos no coffin \code{points}.
      \item \meta{bonus}: o mesmo que \meta{points} mas para pontos bônus.
      \item \meta{id}: o \acs{id} da questão a qual este cabeçalho pertence.
    \end{itemize}
\end{commands}

Em combinação com \cs{ForEachQuestion}, o comando pode ser usado para construir uma
lista personalizada de questões. Um exemplo de seu uso pode ser visto no site alemão
de perguntas e respostas \TeX welt: \url{http://texwelt.de/wissen/fragen/6698#6738}.

\subsection{Carregar Configurações Personalizadas}
Se você tem configurações personalizadas que deseja que sejam carregadas automaticamente, então salve-as
em um arquivo \code{exsheets\_configurations.cfg}. Se este arquivo estiver presente, ele
será carregado \cs*{AtBeginDocument}.

\SetupExSheets{headings=block}

\part{O Pacote \ExSheetslistings}\label{part:listings}
\section{O Problema}
Eu sabia que o dia chegaria quando as pessoas perguntariam como incluir material
verbatim nos ambientes \env{question} e \env{solution}. Como eles são
definidos com o pacote \pkg{environ}, estão lendo os corpos de seus ambientes
como macros leem seus argumentos. Isso torna impossível usar
material verbatim dentro deles\footnote{Veja o \acs{faq} do \TeX\
  \url{http://www.tex.ac.uk/cgi-bin/texfaq2html?label=verbwithin} para as razões
  por quê.}. Agora o dia chegou~\cite{tex.sx:131546}. Logo após a primeira
pergunta aparecer, escrevi o primeiro rascunho para \ExSheetslistings\ para uma questão
no \TeX.sx~\cite{tex.sx:133907}.

\section{A Solução Proposta}

O pacote \ExSheetslistings\ define ambientes \pkg{listings} que colocam
seus conteúdos dentro de ambientes \env{question} e \env{solution}. Eles fazem
isso escrevendo a listagem em um arquivo auxiliar único --
questões em \code{\cs*{jobname}-ex\meta{num}.lst} e soluções em
\code{\cs*{jobname}-sol\meta{num}.lst} onde \meta{num} é um número
inteiro crescente que garante que cada listagem receba um nome de arquivo único. Esses arquivos
são então incluídos com \cs{lstinputlisting} se e quando a questão ou
solução é impressa.

\begin{environments}
  \environment{lstquestion}[\oarg{options}]
    Um ambiente \pkg{listings} colocado em uma \env{question}.
  \environment{lstsolution}[\oarg{options}]
    Um ambiente \pkg{listings} colocado em uma \env{solution}.
\end{environments}

Tudo o que você precisa fazer para usar o pacote é carregá-lo da maneira usual:
\begin{sourcecode}
  \usepackage{exsheets-listings}
\end{sourcecode}
Isso também carregará os pacotes \ExSheets\ e \pkg{listings} se eles ainda não estiverem
carregados.

\begin{example}
  % isso usa meu estilo de listings usado nesta documentação para todos os pedaços de
  % código:
  \begin{lstquestion}[%
      pre=Explique o que este pedaço de código \TeX\ faz:,
      listings={style=cnltx}]
    \begingroup\expandafter\expandafter\expandafter\endgroup
    \expandafter\ifx\csname foo\endcsname\relax
    ...
    \else
    ...
    \fi
  \end{lstquestion}
\end{example}

O exemplo já mostra duas opções desses ambientes. Aqui está a
lista completa:
\begin{options}
  \keyval{pre}{text}
    \meta{text} é colocado antes do código no ambiente \env{question} ou
    \env{solution}.
  \keyval{post}{text}
    \meta{text} é colocado após o código no ambiente \env{question} ou
    \env{solution}.
  \keyval{options}{options}
    Opções passadas para o ambiente \env{question} ou \env{solution}
    subjacente.
  \keyval{points}{points}
    Os pontos atribuídos ao ambiente \env{question} subjacente.
  \keyval{listings}{options}
    Opções passadas para o ambiente \pkg{listings} subjacente.
\end{options}

Há também duas novas opções para \ExSheets\ que podem ser definidas com
\cs{SetupExSheets}:
\begin{options}
  \keyval{listings}{options}\Module{question}
    Opções passadas para o ambiente \pkg{listings} subjacente de
    \env{lstquestion}.
  \keyval{listings}{options}\Module{solution}
    Opções passadas para o ambiente \pkg{listings} subjacente de
    \env{lstsolution}.
\end{options}

\section{Ambientes Próprios}

\begin{commands}
  \command{NewLstQuSolPair}[\oarg{options for both environments}\marg{lst question
    env}\marg{question env}\oarg{options for lst question env}\marg{lst
    solution env}\marg{solution env}\oarg{options for lst solution env}]
    Define dois novos ambientes \pkg{listings} que colocam a listagem em um
    ambiente de questão \meta{question env} ou um ambiente de solução
    \meta{solution env}. Esses ambientes subjacentes devem ser
    ambientes como definidos por \cs{NewQuSolPair}. As diferentes opções
    permitem predefinir opções para os ambientes recém-definidos.
\end{commands}

Os ambientes existentes foram definidos assim:
\begin{sourcecode}
  \NewLstQuSolPair{lstquestion}{question}{lstsolution}{solution}
\end{sourcecode}

\appendix
\part{Apêndice}
\section{Uma Lista de todas as Soluções usadas neste Manual}\label{sec:solutions:list}
\SetupExSheets{headings=block-wp,solutions-totoc}
\printsolutions

\end{document}
