\section{Definindo pontos usando um vetor}

\subsection{\tkzcname{tkzDefPointWith}}
Existem várias possibilidades para criar pontos que atendam a certas condições vetoriais.
Isso pode ser feito com

\tkzcname{tkzDefPointWith}. O princípio geral é o seguinte: dois pontos são passados como argumentos, ou seja, um vetor. As diferentes opções permitem obter um novo ponto formando com o primeiro ponto (com algumas exceções) um vetor colinear ou um vetor ortogonal ao primeiro vetor. Em seguida, o comprimento é ou proporcional ao do primeiro, ou proporcional à unidade. Como este ponto é usado apenas temporariamente, ele não precisa ser nomeado imediatamente. O resultado está em \tkzname{tkzPointResult}. A macro \tkzNameMacro{tkzGetPoint} permite recuperar o ponto e nomeá-lo de forma diferente.

 Existem opções para definir a distância entre o ponto dado e o ponto obtido.
No caso geral, esta distância é a distância entre os 2 pontos dados como argumentos; se a opção for do tipo "normed", então a distância entre o ponto dado e o ponto obtido é 1 cm. Então a opção $K$ permite obter múltiplos.

\begin{NewMacroBox}{tkzDefPointWith}{\parg{pt1,pt2}}%
 É na verdade a definição de um ponto atendendo a condições vetoriais.

\medskip

\begin{tabular}{lll}%
\toprule
argumentos             & definição & explicação                         \\
\midrule
\TAline{(pt1,pt2)} {par de pontos}{o resultado é um ponto em \tkzname{tkzPointResult} } \\

\bottomrule
\end{tabular}

\medskip
No que segue, assume-se que o ponto é recuperado por \tkzNameMacro{tkzGetPoint\{C\}}

\begin{tabular}{lll}%
\toprule
opções             & exemplo & explicação                         \\
\midrule
\TOline{orthogonal}{[orthogonal](A,B)}{$AC=AB$ e $\overrightarrow{AC} \perp \overrightarrow{AB}$}
\TOline{orthogonal normed}{[orthogonal normed](A,B)}{$AC=1$ e $\overrightarrow{AC} \perp \overrightarrow{AB}$}
\TOline{linear}{[linear](A,B)}{$\overrightarrow{AC}=K \times \overrightarrow{AB}$}
\TOline{linear normed}{[linear normed](A,B)}{$AC=K$ e $\overrightarrow{AC}=k\times \overrightarrow{AB}$ }
\TOline{colinear= at \#1}{[colinear= at C](A,B)}{$\overrightarrow{CD}= \overrightarrow{AB}$ }
\TOline{colinear normed= at \#1}{[colinear normed= at C](A,B)}{$\overrightarrow{CD}= \overrightarrow{AB}$ }
\TOline{K}{[linear](A,B),K=2}{$\overrightarrow{AC}=2\times \overrightarrow{AB}$}
\end{tabular}
\end{NewMacroBox}

\subsubsection{Opção \tkzname{colinear at}, exemplo simples}
 $(\overrightarrow{AB}=\overrightarrow{CD})$
\begin{tkzexample}[latex=6cm,small]
\begin{tikzpicture}[scale=1.2,
   vect/.style={->,shorten >=1pt,>=latex'}]
  \tkzDefPoint(2,3){A}   \tkzDefPoint(4,2){B}
  \tkzDefPoint(0,1){C}
  \tkzDefPointWith[colinear=at C](A,B)
  \tkzGetPoint{D}
  \tkzDrawPoints[new](A,B,C,D)
  \tkzLabelPoints[above right=3pt](A,B,C,D)
  \tkzDrawSegments[vect](A,B C,D)
\end{tikzpicture}
\end{tkzexample}

\subsubsection{Opção \tkzname{colinear at}, exemplo complexo}
\begin{tkzexample}[vbox,small]
\begin{tikzpicture}[scale=.75]
\tkzDefPoints{0/0/B,3.6/0/C,1.5/4/A}
\tkzDefSpcTriangle[ortho](A,B,C){Ha,Hb,Hc}
\tkzDefTriangleCenter[ortho](A,B,C) \tkzGetPoint{H}
\tkzDefSquare(A,C) \tkzGetPoints{R}{S}
\tkzDefSquare(B,A) \tkzGetPoints{M}{N}
\tkzDefSquare(C,B) \tkzGetPoints{P}{Q}
\tkzDefPointWith[colinear= at M](A,S) \tkzGetPoint{A'}
\tkzDefPointWith[colinear= at P](B,N) \tkzGetPoint{B'}
\tkzDefPointWith[colinear= at Q](C,R) \tkzGetPoint{C'}
\tkzDefPointBy[projection=onto P--Q](Ha) \tkzGetPoint{Pa}
\tkzDrawPolygon[teal,thick](A,C,R,S)\tkzDrawPolygon[teal,thick](A,B,N,M)
\tkzDrawPolygon[teal,thick](C,B,P,Q)
\tkzDrawPoints[teal,size=2](A,B,C,Ha,Hb,Hc,A',B',C')
\tkzDrawSegments[ultra thin,red](M,A' A',S P,B' B',N Q,C' C',R B,S C,M C,N B,R A,P A,Q)
\tkzDrawSegments[ultra thin,teal, dashed](A,Ha B,Hb C,Hc)
\tkzDefPointBy[rotation=center A angle 90](S) \tkzGetPoint{S'}
\tkzDrawSegments[ultra thin,teal,dashed](B,S' A,S' A,A' M,S' B',Q P,C' M,S Ha,Pa)
\tkzDrawArc(A,S)(S')
\end{tikzpicture}
\end{tkzexample}

\subsubsection{Opção \tkzname{colinear at}}
Como usar $K$
\begin{tkzexample}[latex=7cm,small]
\begin{tikzpicture}[vect/.style={->,
               shorten >=1pt,>=latex'}]
  \tkzDefPoints{0/0/A,5/0/B,1/2/C}
  \tkzDefPointWith[colinear=at C](A,B)
  \tkzGetPoint{G}
  \tkzDefPointWith[colinear=at C, K=0.5](A,B)
  \tkzGetPoint{H}
  \tkzLabelPoints(A,B,C,G,H)
  \tkzDrawPoints(A,B,C,G,H)
  \tkzDrawSegments[vect](A,B C,H)
\end{tikzpicture}
\end{tkzexample}

\subsubsection{Opção \tkzname{colinear at} }
Com $K=\frac{\sqrt{2}}{2}$

\begin{tkzexample}[latex=6cm,small]
\begin{tikzpicture}[vect/.style={->,
            shorten >=1pt,>=latex'}]
 \tkzDefPoints{1/1/A,4/2/B,2/2/C}
 \tkzDefPointWith[colinear=at C,K=sqrt(2)/2](A,B)
 \tkzGetPoint{D}
 \tkzDrawPoints[color=red](A,B,C,D)
 \tkzDrawSegments[vect](A,B C,D)
\end{tikzpicture}
\end{tkzexample}

\subsubsection{Opção \tkzname{orthogonal}}
AB=AC já que $K=1$.
\begin{tkzexample}[latex=6cm,small]
\begin{tikzpicture}[scale=1.2,
  vect/.style={->,shorten >=1pt,>=latex'}]
  \tkzDefPoints{2/3/A,4/2/B}
   \tkzDefPointWith[orthogonal,K=1](A,B)
     \tkzGetPoint{C}
   \tkzDrawPoints[color=red](A,B,C)
   \tkzLabelPoints[right=3pt](B,C)
   \tkzLabelPoints[below=3pt](A)
   \tkzDrawSegments[vect](A,B A,C)
   \tkzMarkRightAngle(B,A,C)
\end{tikzpicture}
\end{tkzexample}



\subsubsection{Opção \tkzname{orthogonal}}
 Com $K=-1$
OK=OI já que $\lvert K \rvert=1$ então OI=OJ=OK.

\begin{tkzexample}[latex=7cm,small]
\begin{tikzpicture}[scale=.75]
  \tkzDefPoints{1/2/O,2/5/I}
  \tkzDefPointWith[orthogonal](O,I)
  \tkzGetPoint{J}
  \tkzDefPointWith[orthogonal,K=-1](O,I)
  \tkzGetPoint{K}
  \tkzDrawSegment(O,I)
  \tkzDrawSegments[->](O,J O,K)
  \tkzMarkRightAngles(I,O,J I,O,K)
  \tkzDrawPoints(O,I,J,K)
  \tkzLabelPoints(O,I,J,K)
\end{tikzpicture}
\end{tkzexample}

\subsubsection{Opção \tkzname{orthogonal} exemplo mais complicado}
\begin{tkzexample}[latex=7cm,small]
\begin{tikzpicture}[scale=.75]
  \tkzDefPoints{0/0/A,6/0/B}
  \tkzDefMidPoint(A,B)
    \tkzGetPoint{I}
  \tkzDefPointWith[orthogonal,K=-.75](B,A)
  \tkzGetPoint{C}
  \tkzInterLC(B,C)(B,I)
     \tkzGetPoints{D}{F}
  \tkzDuplicateSegment(B,F)(A,F)
  \tkzGetPoint{E}
  \tkzDrawArc[delta=10](F,E)(B)
  \tkzInterLC(A,B)(A,E)
    \tkzGetPoints{N}{M}
  \tkzDrawArc[delta=10](A,M)(E)
  \tkzDrawLines(A,B B,C A,F)
  \tkzCompass(B,F)
  \tkzDrawPoints(A,B,C,F,M,E)
  \tkzLabelPoints(A,B,C,F,M)
  \tkzLabelPoints[above](E)
\end{tikzpicture}
\end{tkzexample}

\subsubsection{Opções \tkzname{colinear} e \tkzname{orthogonal}}
\begin{tkzexample}[latex=7cm,small]
\begin{tikzpicture}[scale=1.2,
  vect/.style={->,shorten >=1pt,>=latex'}]
  \tkzDefPoints{2/1/A,6/2/B}
  \tkzDefPointWith[orthogonal,K=.5](A,B)
  \tkzGetPoint{C}
  \tkzDefPointWith[colinear=at C,K=.5](A,B)
  \tkzGetPoint{D}
  \tkzMarkRightAngle[fill=gray!20](B,A,C)
  \tkzDrawSegments[vect](A,B A,C C,D)
  \tkzDrawPoints(A,...,D)
\end{tikzpicture}
\end{tkzexample}

\subsubsection{Opção  \tkzname{orthogonal normed}}
 $K=1$ $AC=1$.

\begin{tkzexample}[latex=7cm,small]
\begin{tikzpicture}[scale=1.2,
  vect/.style={->,shorten >=1pt,>=latex'}]
  \tkzDefPoints{2/3/A,4/2/B}
  \tkzDefPointWith[orthogonal normed](A,B)
  \tkzGetPoint{C}
  \tkzDrawPoints[color=red](A,B,C)
  \tkzDrawSegments[vect](A,B A,C)
  \tkzMarkRightAngle[fill=gray!20](B,A,C)
\end{tikzpicture}
\end{tkzexample}

\subsubsection{Opção \tkzname{orthogonal normed} e K=2}
$K=2$ portanto $AC=2$.

\begin{tkzexample}[latex=7cm,small]
\begin{tikzpicture}[scale=1.2,
   vect/.style={->,shorten >=1pt,>=latex'}]
  \tkzDefPoints{2/3/A,5/1/B}
  \tkzDefPointWith[orthogonal normed,K=2](A,B)
  \tkzGetPoint{C}
  \tkzDrawPoints[color=red](A,B,C)
  \tkzDefCircle[R](A,2) \tkzGetPoint{a}
  \tkzDrawCircle(A,a)
  \tkzDrawSegments[vect](A,B A,C)
  \tkzMarkRightAngle[fill=gray!20](B,A,C)
  \tkzLabelPoints[above=3pt](A,B,C)
\end{tikzpicture}
\end{tkzexample}

\subsubsection{Opção \tkzname{linear}}
Aqui $K=0.5$.

Isso equivale a aplicar uma homotetia ou uma multiplicação de um vetor por um real. Aqui está o ponto médio de $[AB]$.

\begin{tkzexample}[latex=7cm,small]
\begin{tikzpicture}[scale=1.2]
  \tkzDefPoints{1/3/A,4/2/B}
  \tkzDefPointWith[linear,K=0.5](A,B)
  \tkzGetPoint{C}
  \tkzDrawPoints[color=red](A,B,C)
  \tkzDrawSegment(A,B)
  \tkzLabelPoints[above right=3pt](A,B,C)
\end{tikzpicture}
\end{tkzexample}

\subsubsection{Opção \tkzname{linear normed}}
No exemplo seguinte $AC=1$ e $C$ pertence a $(AB)$.

\begin{tkzexample}[latex=7cm,small]
\begin{tikzpicture}[scale=1.2]
 \tkzDefPoints{1/3/A,4/2/B}
 \tkzDefPointWith[linear normed](A,B)
 \tkzGetPoint{C}
 \tkzDrawPoints[color=red](A,B,C)
 \tkzDrawSegment(A,B)
 \tkzLabelSegment(A,C){$1$}
 \tkzLabelPoints[above right=3pt](A,B,C)
\end{tikzpicture}
\end{tkzexample}
%<--------------------------------------------------------------------------–>
%         tkzGetVectxy
%<--------------------------------------------------------------------------–>
\subsection{\tkzcname{tkzGetVectxy} }
Recuperando as coordenadas de um vetor.

\begin{NewMacroBox}{tkzGetVectxy}{\parg{$A,B$}\var{text}}%
Permite obter as coordenadas de um vetor.

\medskip
\begin{tabular}{lll}%
\toprule
argumentos    & exemplo & explicação      \\

\midrule

\TAline{(ponto)\{nome da macro\}} {\tkzcname{tkzGetVectxy}(A,B)\{V\}}{\tkzcname{Vx},\tkzcname{Vy}: coordenadas de $\overrightarrow{AB}$}
\end{tabular}
\end{NewMacroBox}

\subsubsection{Transferência de coordenadas com \tkzcname{tkzGetVectxy}}

\begin{tkzexample}[latex=7cm,small]
\begin{tikzpicture}
 \tkzDefPoints{0/0/O,1/1/A,4/2/B}
 \tkzGetVectxy(A,B){v}
 \tkzDefPoint(\vx,\vy){V}
 \tkzDrawSegment[->,color=red](O,V)
 \tkzDrawSegment[->,color=blue](A,B)
 \tkzDrawPoints(A,B,O)
 \tkzLabelPoints(A,B,O,V)
\end{tikzpicture}
\end{tkzexample}
\endinput
