Agora que os pontos fixos estão definidos, podemos com suas referências usando macros do pacote ou macros que você criará obter novos pontos. Os cálculos podem não ser aparentes, mas geralmente são feitos pelo pacote.
Você pode precisar usar algumas constantes matemáticas, aqui está a lista de constantes definidas pelo pacote.


\section{Ferramentas auxiliares}
\subsection{Constantes}

\tkzname{\tkznameofpack} conhece algumas constantes, aqui está a lista:
\begin{tkzltxexample}[]
  \def\tkzPhi{1.618034}
  \def\tkzInvPhi{0.618034}
  \def\tkzSqrtPhi{1.27202}
  \def\tkzSqrTwo{1.414213}
  \def\tkzSqrThree{1.7320508}
  \def\tkzSqrFive{2.2360679}
  \def\tkzSqrTwobyTwo{0.7071065}
  \def\tkzPi{3.1415926}
  \def\tkzEuler{2.71828182}
\end{tkzltxexample}

\subsection{Novo ponto por cálculo}

Quando uma macro de \tkzname{tkznameofpack} cria um novo ponto, ele é armazenado internamente com a referência \tkzname{tkzPointResult}. Você pode atribuir sua própria referência a ele. Isso é feito com a macro \tkzcname{tkzGetPoint}. Uma nova referência é criada, sua escolha de referência deve ser colocada entre chaves.

\begin{NewMacroBox}{tkzGetPoint}{\marg{ref}}%
Se o resultado está em \tkzname{tkzPointResult}, você pode acessá-lo com \tkzcname{tkzGetPoint}.

 \medskip
\begin{tabular}{lll}%
\toprule
argumentos & padrão & exemplo \\
\midrule
\TAline{ref}{sem padrão}{ \tkzcname{tkzGetPoint\{M\} } veja o próximo exemplo}
\end{tabular}
\end{NewMacroBox}

Às vezes você precisa obter dois pontos. É possível com

\begin{NewMacroBox}{tkzGetPoints}{\marg{ref1}\marg{ref2}}%
O resultado está em \tkzname{tkzPointFirstResult} e \tkzname{tkzPointSecondResult}.

 \medskip
\begin{tabular}{lll}%
\toprule
argumentos & padrão & exemplo \\
\midrule
\TAline{\{ref1,ref2\}}{sem padrão}{ \tkzcname{tkzGetPoints\{M,N\} } É o caso com \tkzcname{tkzInterCC}}
\end{tabular}
\end{NewMacroBox}

Se você precisar apenas do primeiro ou do segundo ponto, você também pode usar:

\begin{NewMacroBox}{tkzGetFirstPoint}{\marg{ref1}}%

 \medskip
\begin{tabular}{lll}%
\toprule
argumentos & padrão & exemplo \\
\midrule
\TAline{ref1}{sem padrão}{ \tkzcname{tkzGetFirstPoint\{M\} }}
\end{tabular}
\end{NewMacroBox}

\begin{NewMacroBox}{tkzGetSecondPoint}{\marg{ref2}}%

 \medskip
\begin{tabular}{lll}%
\toprule
argumentos & padrão & exemplo \\
\midrule
\TAline{ref2}{sem padrão}{ \tkzcname{tkzGetSecondPoint\{M\} }}
\end{tabular}
\end{NewMacroBox}

Às vezes os resultados consistem em um ponto e uma dimensão. Você obtém o ponto com \tkzcname{tkzGetPoint} e a dimensão com \tkzcname{tkzGetLength}.

\begin{NewMacroBox}{tkzGetLength}{\marg{nome de uma macro}}%

 \medskip
\begin{tabular}{lll}%
\toprule
argumentos & padrão & exemplo \\
\midrule
\TAline{nome de uma macro}{sem padrão}{ \tkzcname{tkzGetLength\{rAB\} \tkzcname{rAB} fornece o comprimento em cm}}
\end{tabular}
\end{NewMacroBox}

%\tkzcname{tkzCalcLength}(A,B) Após \tkzcname{tkzGetLength\{dAB\}} \tkzcname{dAB} fornece $AB$ em cm}


\section{Pontos especiais}
Aqui estão alguns pontos especiais.
%<--------------------------------------------------------------------------->
\subsection{Ponto médio de um segmento \tkzcname{tkzDefMidPoint}}
É uma questão de determinar o ponto médio de um segmento.

\begin{NewMacroBox}{tkzDefMidPoint}{\parg{pt1,pt2}}%
O resultado está em \tkzname{tkzPointResult}. Podemos acessá-lo com \tkzcname{tkzGetPoint}.

 \medskip
\begin{tabular}{lll}%
\toprule
argumentos & padrão & definição \\
\midrule
\TAline{(pt1,pt2)}{sem padrão}{pt1 e pt2 são dois pontos}
\end{tabular}
\end{NewMacroBox}

\subsubsection{Uso de \tkzcname{tkzDefMidPoint}}
Revise o uso de \tkzcname{tkzDefPoint}.
\begin{tkzexample}[latex=7cm,small]
\begin{tikzpicture}[scale=1]
 \tkzDefPoint(2,3){A}
 \tkzDefPoint(6,2){B}
 \tkzDefMidPoint(A,B)
 \tkzGetPoint{M}
 \tkzDrawSegment(A,B)
 \tkzDrawPoints(A,B,M)
 \tkzLabelPoints[below](A,B,M)
\end{tikzpicture}
\end{tkzexample}

\subsection{\tkzname{Razão áurea} \tkzcname{tkzDefGoldenRatio}}
Da Wikipedia: Em matemática, duas quantidades estão na razão áurea se sua razão é a mesma que a razão de sua soma para a maior das duas quantidades. Expresso algebricamente, para quantidades $a$, $b$ tal que $a > b > 0$; $a+b$ está para $a$ assim como $a$ está para $b$.

$ \frac{a+b}{a} = \frac{a}{b} = \phi = \frac{1 + \sqrt{5}}{2}$


Uma das duas soluções para a equação $x^2 - x - 1 = 0$
é a razão áurea $\phi$, $\phi = \frac{1 + \sqrt{5}}{2}$.

\begin{NewMacroBox}{tkzDefGoldenRatio}{\parg{pt1,pt2}}%
\begin{tabular}{lll}%
argumentos & padrão & exemplo \\
\midrule
\TAline{(pt1,pt2)}{sem padrão}{\tkzcname{tkzDefGoldenRatio(A,C)} \tkzcname{tkzGetPoint}\{B\}}
\bottomrule
\end{tabular}

\medskip
$AB=a$, $BC=b$ e $\dfrac{AC}{AB} = \dfrac{AB}{BC} =\phi$
\end{NewMacroBox}

\subsubsection{Use a razão áurea para dividir um segmento de linha}
\begin{tkzexample}[latex=7cm,small]
\begin{tikzpicture}
 \tkzDefPoints{0/0/A,6/0/C}
 \tkzDefMidPoint(A,C) \tkzGetPoint{I}
 %\tkzDefPointWith[linear,K=\tkzInvPhi](A,C)
 \tkzDefGoldenRatio(A,C) \tkzGetPoint{B}
 \tkzDrawSegments(A,C)
 \tkzDrawPoints(A,B,C)
 \tkzLabelPoints(A,B,C)
\end{tikzpicture}
\end{tkzexample}

\subsubsection{Arbelos dourado}
\begin{tkzexample}[latex=7cm,small]
\begin{tikzpicture}[scale=.6]
\tkzDefPoints{0/0/A,10/0/B}
\tkzDefGoldenRatio(A,B)     \tkzGetPoint{C}
\tkzDefMidPoint(A,B)        \tkzGetPoint{O_1}
\tkzDefMidPoint(A,C)        \tkzGetPoint{O_2}
\tkzDefMidPoint(C,B)        \tkzGetPoint{O_3}
\tkzDrawSemiCircles[fill=purple!15](O_1,B)
\tkzDrawSemiCircles[fill=teal!15](O_2,C O_3,B)
\end{tikzpicture}
\end{tkzexample}

Também é possível usar a seguinte macro.
\subsection{\tkzname{Coordenadas baricêntricas} com \tkzcname{tkzDefBarycentricPoint}}

$pt_1$, $pt_2$, \dots, $pt_n$ sendo $n$ pontos, eles definem $n$ vetores $\overrightarrow{v_1}$, $\overrightarrow{v_2}$, \dots, $\overrightarrow{v_n}$ com a origem do referencial como o ponto final comum. $\alpha_1$, $\alpha_2$,
\dots $\alpha_n$ são $n$ números, o vetor obtido por:
\begin{align*}
  \frac{\alpha_1 \overrightarrow{v_1} + \alpha_2 \overrightarrow{v_2} + \cdots + \alpha_n \overrightarrow{v_n}}{\alpha_1
    + \alpha_2 + \cdots + \alpha_n}
\end{align*}
define um único ponto.

\begin{NewMacroBox}{tkzDefBarycentricPoint}{\parg{pt1=$\alpha_1$,pt2=$\alpha_2$,\dots}}%
\begin{tabular}{lll}%
argumentos & padrão & definição \\
\midrule
\TAline{(pt1=$\alpha_1$,pt2=$\alpha_2$,\dots)}{sem padrão}{Cada ponto tem um peso atribuído}
\bottomrule
\end{tabular}

\medskip
\emph{Você precisa de pelo menos dois pontos. Resultado em \tkzname{tkzPointResult}.}
\end{NewMacroBox}


\subsubsection{com dois pontos}
No exemplo seguinte, obtemos o baricentro dos pontos $A$ e $B$ com coeficientes $1$ e $2$, em outras palavras:
\[
  \overrightarrow{AI}= \frac{2}{3}\overrightarrow{AB}
\]

\begin{tkzexample}[latex=7cm,small]
\begin{tikzpicture}
  \tkzDefPoint(2,3){A}
  \tkzDefShiftPointCoord[2,3](30:4){B}
  \tkzDefBarycentricPoint(A=1,B=2)
  \tkzGetPoint{G}
  \tkzDrawLine(A,B)
  \tkzDrawPoints(A,B,G)
  \tkzLabelPoints(A,B,G)
\end{tikzpicture}
\end{tkzexample}

\subsubsection{com três pontos}
Desta vez $M$ é simplesmente o centro de gravidade do triângulo.

 Por razões de simplificação e homogeneidade, há também \tkzcname{tkzCentroid}.
\begin{tkzexample}[latex=7cm,small]
\begin{tikzpicture}[scale=.8]
  \tkzDefPoints{2/1/A,5/3/B,0/6/C}
  \tkzDefBarycentricPoint(A=1,B=1,C=1)
  \tkzGetPoint{G}
  \tkzDefMidPoint(A,B)  \tkzGetPoint{C'}
  \tkzDefMidPoint(A,C)  \tkzGetPoint{B'}
  \tkzDefMidPoint(C,B)  \tkzGetPoint{A'}
  \tkzDrawPolygon(A,B,C)
  \tkzDrawLines[add=0 and 1,new](A,G B,G C,G)
  \tkzDrawPoints[new](A',B',C',G)
  \tkzDrawPoints(A,B,C)
  \tkzLabelPoint[above right](G){$G$}
  \tkzAutoLabelPoints[center=G](A,B,C)
  \tkzLabelPoints[above right](A')
  \tkzLabelPoints[below](B',C')
\end{tikzpicture}
\end{tkzexample}


\subsection{\tkzname{Centro de similitude interno e externo}}
Os centros das duas homotetias em que dois círculos correspondem são chamados de centros de similitude externo e interno. Você pode usar \tkzcname{tkzDefIntSimilitudeCenter} e \tkzcname{tkzDefExtSimilitudeCenter}, mas a próxima macro é melhor.

\begin{NewMacroBox}{tkzDefSimilitudeCenter}{\oarg{opções}\parg{O,A}\parg{O',B}}%

\begin{tabular}{lll}%
argumentos           & exemplo & explicação                         \\
\midrule
\TAline{\parg{pt1,pt2}\parg{pt3,pt4}}{$(O,A)(O',B)$} {$r=OA,r'=O'B$}
\end{tabular}

\medskip
\begin{tabular}{lll}%
\toprule
opções             & padrão & definição                         \\
\midrule
\TOline{ext}{ext}{centro externo}
\TOline{int}{ext}{centro interno}
\end{tabular}
\end{NewMacroBox}

\subsubsection{Interno e externo com \tkzname{node}}
\begin{tkzexample}[latex=7.5cm,small]
\begin{tikzpicture}[scale=.7]
 \tkzDefPoints{0/0/O,4/-5/A,3/0/B,5/-5/C}
 \tkzDefSimilitudeCenter[int](O,B)(A,C)
 \tkzGetPoint{I}
 \tkzDefSimilitudeCenter[ext](O,B)(A,C)
 \tkzGetPoint{J}
 \tkzDefLine[tangent from = I](O,B)
 \tkzGetPoints{D}{E}
 \tkzDefLine[tangent from = I](A,C)
 \tkzGetPoints{D'}{E'}
 \tkzDefLine[tangent from = J](O,B)
 \tkzGetPoints{F}{G}
 \tkzDefLine[tangent from = J](A,C)
 \tkzGetPoints{F'}{G'}
 \tkzDrawCircles(O,B A,C)
 \tkzDrawSegments[add = .5 and .5,new](D,D' E,E')
 \tkzDrawSegments[add= 0 and 0.25,new](J,F J,G)
 \tkzDrawPoints(O,A,I,J,D,E,F,G,D',E',F',G')
\end{tikzpicture}
\end{tkzexample}

\subsubsection{Teorema de D'Alembert} % (fold)
\label{ssub:d_alembert_theorem}

\begin{tkzexample}[latex=7cm,small]
 \begin{tikzpicture}[scale=.6,rotate=90]
 \tkzDefPoints{0/0/A,3/0/a,7/-1/B,5.5/-1/b}
 \tkzDefPoints{5/-4/C,4.25/-4/c}
 \tkzDrawCircles(A,a B,b C,c)
 \tkzDefExtSimilitudeCenter(A,a)(B,b) \tkzGetPoint{I}
 \tkzDefExtSimilitudeCenter(A,a)(C,c) \tkzGetPoint{J}
 \tkzDefExtSimilitudeCenter(C,c)(B,b) \tkzGetPoint{K}
 \tkzDefIntSimilitudeCenter(A,a)(B,b) \tkzGetPoint{I'}
 \tkzDefIntSimilitudeCenter(A,a)(C,c) \tkzGetPoint{J'}
 \tkzDefIntSimilitudeCenter(C,c)(B,b) \tkzGetPoint{K'}
 \tkzDrawPoints(A,B,C,I,J,K,I',J',K')
 \tkzDrawSegments[new](I,K A,I A,J B,I B,K C,J C,K)
 \tkzDrawSegments[new](I,J' I',J I',K)
 \end{tikzpicture}
\end{tkzexample}

% subsubsection d_alembert_theorem (end)

Você pode usar \tkzcname{tkzDefBarycentricPoint} para encontrar um centro homotético

|\tkzDefBarycentricPoint(O=\r,A=\R)     \tkzGetPoint{I}| \\
|\tkzDefBarycentricPoint(O={-\r},A=\R)  \tkzGetPoint{J}|

\subsubsection{Exemplo com \tkzname{node}}
\begin{tkzexample}[latex=7cm,small]
\begin{tikzpicture}[rotate=60,scale=.5]
 \tkzDefPoints{0/0/A,5/0/C}
 \tkzDefGoldenRatio(A,C) \tkzGetPoint{B}
 \tkzDefSimilitudeCenter(A,B)(C,B)\tkzGetPoint{J}
 \tkzDefTangent[from = J](A,B)  \tkzGetPoints{F}{G}
 \tkzDefTangent[from = J](C,B)  \tkzGetPoints{F'}{G'}
 \tkzDrawCircles(A,B C,B)
 \tkzDrawSegments[add= 0 and 0.25,cyan](J,F J,G)
 \tkzDrawPoints(A,J,F,G,F',G')
\end{tikzpicture}
\end{tkzexample}
\newpage
%<---------------------------------------------------------------------->
\subsection{ \tkzname{Divisão harmônica} com \tkzcname{tkzDefHarmonic}}
%<---------------------------------------------------------------------->

\begin{NewMacroBox}{tkzDefHarmonic}{\oarg{opções}\parg{pt1,pt2,pt3} ou \parg{pt1,pt2,k}}%

\begin{tabular}{lll}%
opções             & padrão & definição                         \\
\midrule
\TOline{both}{both}{\parg{A,B,2} procuramos C e D tais que $(A,B;C,D) = -1$ e CA=2CB }
\TOline{ext}{both}{\parg{A,B,C} procuramos D tal que $(A,B;C,D) = -1$}
\TOline{int}{both}{\parg{A,B,D} procuramos C tal que $(A,B;C,D) = -1$}
\end{tabular}
\end{NewMacroBox}

\subsubsection{opções \tkzname{ext} e \tkzname{int}}
\begin{tkzexample}[vbox,small]
  \begin{tikzpicture}
  \tkzDefPoints{0/0/A,6/0/B,4/0/C}
  \tkzDefHarmonic[ext](A,B,C) \tkzGetPoint{J}
  \tkzDefHarmonic[int](A,B,J) \tkzGetPoint{I}
  \tkzDrawPoints(A,B,I,J)
  \tkzDrawLine[add=.5 and 1](A,B)
  \tkzLabelPoints(A,B,I,J)
  \end{tikzpicture}
\end{tkzexample}

\subsubsection{Bissetriz e divisão harmônica} % (fold)
\label{ssub:bisector_and_harmonic_division}

\begin{tkzexample}[vbox,small]
  \begin{tikzpicture}[scale=1.25]
  \tkzDefPoints{0/0/A,4/0/C,5/3/X}
  \tkzDefLine[bisector](A,X,C) \tkzGetPoint{x}
  \tkzInterLL(X,x)(A,C)        \tkzGetPoint{B}
  \tkzDefHarmonic[ext](A,C,B)  \tkzGetPoint{D}
  \tkzDrawPolygon(A,X,C)
  \tkzDrawSegments(X,B C,D D,X)
  \tkzDrawPoints(A,B,C,D,X)
  \tkzMarkAngles[mark=s|](A,X,B B,X,C)
  \tkzMarkRightAngle[size=.4,
                     fill=gray!20,
                     opacity=.3](B,X,D)
  \tkzLabelPoints(A,B,C,D)
  \tkzLabelPoints[above right](X)
  \end{tikzpicture}
\end{tkzexample}


% subsubsection bisector_and_harmonic_division (end)
\subsubsection{opção \tkzname{both} }
\tkzname{both} é a opção padrão
\begin{tkzexample}[vbox,small]
\begin{tikzpicture}
 \tkzDefPoints{0/0/A,6/0/B}
 \tkzDefHarmonic(A,B,{1/2})\tkzGetPoints{I}{J}
 \tkzDrawPoints(A,B,I,J)
 \tkzDrawLine[add=1 and .5](A,B)
 \tkzLabelPoints(A,B,I,J)
\end{tikzpicture}
\end{tkzexample}

%<---------------------------------------------------------------------->
\subsection{\tkzname{Pontos equidistantes} com \tkzcname{tkzDefEquiPoints} }
%<---------------------------------------------------------------------->

\begin{NewMacroBox}{tkzDefEquiPoints}{\oarg{opções locais}\parg{pt1,pt2}}%
\begin{tabular}{lll}%
argumentos &  padrão & definição \\
\midrule
\TAline{(pt1,pt2)}{sem padrão}{lista não ordenada de dois itens}
\end{tabular}

\begin{tabular}{lll}%
opções             & padrão & definição  \\
\midrule
\TOline{dist} {2 (cm)} {metade da distância entre os dois pontos}
\TOline{from=pt} {sem padrão} {ponto de referência}
\TOline{show} {false} {se verdadeiro exibe traços de compasso}
\TOline{/compass/delta} {0} {tamanho do traço do compasso }
\end{tabular}

\medskip
\emph{Esta macro torna possível obter dois pontos em uma linha reta equidistantes de um ponto dado.}
\end{NewMacroBox}


\subsubsection{Usando \tkzcname{tkzDefEquiPoints} com opções}
\begin{tkzexample}[latex=7cm,small]
\begin{tikzpicture}
  \tkzSetUpCompass[color=purple,line width=1pt]
  \tkzDefPoints{0/1/A,5/2/B,3/4/C}
  \tkzDefEquiPoints[from=C,dist=1,show,
      /tkzcompass/delta=20](A,B)
   \tkzGetPoints{E}{H}
   \tkzDrawLines[color=blue](C,E C,H A,B)
   \tkzDrawPoints[color=blue](A,B,C)
   \tkzDrawPoints[color=red](E,H)
   \tkzLabelPoints(E,H)
   \tkzLabelPoints[color=blue](A,B,C)
\end{tikzpicture}
\end{tkzexample}
%<---------------------------------------------------------------------->
%                          Middle of an arc                             >
%<---------------------------------------------------------------------->
\subsection{Ponto médio de um arco}
\begin{NewMacroBox}{tkzDefMidArc}{\parg{pt1,pt2,pt3}}%
\begin{tabular}{lll}%
argumentos &  padrão & definição \\
\midrule
\TAline{$pt1,pt2,pt3$}{sem padrão}{$pt1$ é o centro, $\widearc{pt2pt3}$ o arco}
\end{tabular}
\end{NewMacroBox}

\begin{tkzexample}[vbox,small]
  \begin{tikzpicture}[scale=1]
   \tkzDefPoints{0/0/A,10/0/B}
   \tkzDefGoldenRatio(A,B)                              \tkzGetPoint{C}
   \tkzDefMidPoint(A,B)                                 \tkzGetPoint{O_1}
   \tkzDefMidPoint(A,C)                                 \tkzGetPoint{O_2}
   \tkzDefMidPoint(C,B)                                 \tkzGetPoint{O_3}
   \tkzDefMidArc(O_3,B,C)                               \tkzGetPoint{P}
   \tkzDefMidArc(O_2,C,A)                               \tkzGetPoint{Q}
   \tkzDefMidArc(O_1,B,A)                               \tkzGetPoint{L}
   \tkzDefPointBy[rotation=center C angle 90](B)        \tkzGetPoint{c}
   \tkzInterCC[common=B](P,B)(O_1,B)                    \tkzGetFirstPoint{P_1}
   \tkzInterCC[common=C](P,C)(O_2,C)                    \tkzGetFirstPoint{P_2}
   \tkzInterCC[common=C](Q,C)(O_3,C)                    \tkzGetFirstPoint{P_3}
   \tkzInterLC[near](c,C)(O_1,A)                        \tkzGetFirstPoint{D}
   \tkzInterLL(A,P_1)(C,D)                              \tkzGetPoint{P_1'}
   \tkzDefPointBy[inversion = center A through D](P_2)  \tkzGetPoint{P_2'}
   \tkzDefCircle[circum](P_3,P_2,P_1)                   \tkzGetPoint{O_4}
   \tkzInterLL(B,Q)(A,P)                                \tkzGetPoint{S}
   \tkzDefMidPoint(P_2',P_1')                           \tkzGetPoint{o}
   \tkzDefPointBy[inversion = center A through D](S)    \tkzGetPoint{S'}
   \tkzDrawArc[cyan,delta=0](Q,A)(P_1)
   \tkzDrawArc[cyan,delta=0](P,P_1)(B)
   \tkzDrawSemiCircles[teal](O_1,B O_2,C O_3,B)
   \tkzDrawCircles[new](o,P O_4,P_1)
   \tkzDrawSegments(A,B)
   \tkzDrawSegments[cyan](A,P_1 A,S' A,P_2')
   \tkzDrawSegments[purple](B,L C,P_2' B,Q B,L S',P_1')
   \tkzDrawLines[add=0 and .8](B,P_2')
   \tkzDrawLines[add=0 and .4](C,D)
   \tkzDrawPoints(A,B,C,P,Q,P_3,P_2,P_1,P_1',D,P_2',L,S,S')
   \tkzLabelPoints(A,B,C,P_3)
   \tkzLabelPoints[above](P,Q,P_1)
   \tkzLabelPoints[above right](P_2,P_2',D,S')
   \tkzLabelPoints[above left](L,S)
    \tkzLabelPoints[below left](P_1')
  \end{tikzpicture}
\end{tkzexample}

%<---------------------------------------------------------------------->
%                          Point on a line                              >
%<---------------------------------------------------------------------->

\section{Ponto sobre linha ou círculo}
\subsection{Ponto sobre uma linha com \tkzcname{tkzDefPointOnLine}}

\begin{NewMacroBox}{tkzDefPointOnLine}{\oarg{opções locais}\parg{A,B}}%
\begin{tabular}{lll}%
argumentos &  padrão & definição                 \\
\midrule
\TAline{pt1,pt2} {sem padrão}  {Dois pontos para definir uma linha}
\bottomrule
\end{tabular}

\medskip
\begin{tabular}{lll}%
opções       & padrão & definição \\
\midrule
\TOline{pos=nb}  {}{nb é um decimal  }
\end{tabular}
\end{NewMacroBox}

\subsubsection{Uso da opção \tkzname{pos}}
\begin{tkzexample}[latex=7cm,small]
\begin{tikzpicture}
\tkzDefPoints{0/0/A,3/0/B}
\tkzDefPointOnLine[pos=1.2](A,B)\tkzGetPoint{P}
\tkzDefPointOnLine[pos=-0.2](A,B)\tkzGetPoint{R}
\tkzDefPointOnLine[pos=0.5](A,B) \tkzGetPoint{S}
\tkzDrawLine[new](A,B)
\tkzDrawPoints(A,B,P)
\tkzLabelPoints(A,B)
\tkzLabelPoint[above](P){pos=$1.2$}
\tkzLabelPoint[above](R){pos=$-.2$}
\tkzLabelPoint[above](S){pos=$.5$}
\tkzDrawPoints(A,B,P,R,S)
\tkzLabelPoints(A,B)
\end{tikzpicture}
\end{tkzexample}

\subsection{Ponto sobre um círculo com \tkzcname{tkzDefPointOnCircle}}
A ordem dos argumentos mudou: agora é centro, ângulo e ponto ou raio.
Adicionei duas opções para trabalhar com radianos que são \tkzname{through in rad} e \tkzname{R in rad}.

\begin{NewMacroBox}{tkzDefPointOnCircle}{\oarg{opções locais}}%
\begin{tabular}{lll}%
opções   & padrão & exemplos definição \\
\midrule
\TOline{through}  {}{through =  center K1 angle 30 point B]}
\TOline{R} {}{R =  center K1 angle 30 radius \tkzcname{rAp}}
\TOline{through in rad}  {}{through in rad=  center K1 angle pi/4 point B]}
\TOline{R in rad} {}{R in rad =  center K1 angle pi/6 radius \tkzcname{rAp}}
\end{tabular}

\medskip
\emph{A nova ordem para os argumentos são: centro, ângulo e ponto ou raio.}
\end{NewMacroBox}

\subsubsection{Teorema de Altshiller}
 As duas linhas que unem os pontos de interseção de dois círculos ortogonais a um ponto em um dos círculos encontram o outro círculo em dois pontos diametralmente opostos. Altshiller p 176

\begin{tkzexample}[latex=6cm,small]
\begin{tikzpicture}[scale=.4]
\tkzDefPoints{0/0/P,5/0/Q,3/2/I}
\tkzDefCircle[orthogonal from=P](Q,I)
\tkzGetFirstPoint{E}
\tkzDrawCircles(P,E Q,E)
\tkzInterCC[common=E](P,E)(Q,E) \tkzGetFirstPoint{F}
\tkzDefPointOnCircle[through =  center P angle 80 point E]
 \tkzGetPoint{A}
\tkzInterLC[common=E](A,E)(Q,E)  \tkzGetFirstPoint{C}
\tkzInterLL(A,F)(C,Q)  \tkzGetPoint{D}
\tkzDrawLines[add=0 and .75](P,Q)
\tkzDrawLines[add=0 and 2](A,E)
\tkzDrawSegments(P,E E,F F,C A,F C,D)
\tkzDrawPoints(P,Q,E,F,A,C,D)
\tkzLabelPoints(P,Q,F,C,D)
\tkzLabelPoints[above](E,A)
\end{tikzpicture}
\end{tkzexample}

\subsubsection{Uso de  \tkzcname{tkzDefPointOnCircle}}

\begin{tkzexample}[latex=6cm,small]
\begin{tikzpicture}
\tkzDefPoints{0/0/A,4/0/B,0.8/3/C}
\tkzDefPointOnCircle[R = center B  angle 90 radius 1]
\tkzGetPoint{I}
\tkzDefCircle[circum](A,B,C)
\tkzGetPoints{G}{g}
\tkzDefPointOnCircle[through = center G angle 30 point g]
\tkzGetPoint{J}
\tkzDefCircle[R](B,1) \tkzGetPoint{b}
\tkzDrawCircle[teal](B,b)
\tkzDrawCircle(G,J)
\tkzDrawPoints(A,B,C,G,I,J)
\tkzAutoLabelPoints[center=G](A,B,C,J)
\tkzLabelPoints[below](G,I)
\end{tikzpicture}
\end{tkzexample}

\newpage
\section{Pontos especiais relacionados a um triângulo}

\subsection{Centro do triângulo: \tkzcname{tkzDefTriangleCenter}}

\begin{NewMacroBox}{tkzDefTriangleCenter}{\oarg{opções locais}\parg{A,B,C}}%
\tkzHandBomb\ Esta macro permite definir o centro de um triângulo. Cuidado, os argumentos são listas de três pontos. Esta macro é usada em conjunto com \tkzcname{tkzGetPoint} para obter o centro que você está procurando.

 Você pode usar \tkzname{tkzPointResult} se não for necessário manter os resultados.

\medskip
\begin{tabular}{lll}%
\toprule
argumentos & padrão & exemplo \\
\midrule
\TAline{(pt1,pt2,pt3)}{sem padrão}{ \tkzcname{tkzDefTriangleCenter[ortho](B,C,A)}}
\midrule
opções             & padrão & definição                         \\
\midrule
\TOline{ortho}  {circum}{interseção das alturas}
\TOline{orthic}  {circum}{\dots}
\TOline{centroid} {circum}{interseção das medianas}
\TOline{median} {circum}{ \dots }
\TOline{circum}{circum}{centro do círculo circunscrito}
\TOline{in}    {circum}{centro do círculo inscrito em um triângulo }
\TOline{in}    {circum}{interseção das bissetrizes}
\TOline{ex}    {circum}{centro de um círculo ex-inscrito em um triângulo }
\TOline{euler}{circum}{centro do círculo de Euler }
\TOline{gergonne}{circum}{definido com o triângulo de contato}
\TOline{symmedian} {circum}{Ponto de Lemoine ou centro simediano ou Ponto de Grebe }
\TOline{lemoine} {circum}{ \dots}
\TOline{grebe} {circum}{ \dots}
\TOline{spieker} {circum}{centro do círculo de Spieker}
\TOline{nagel}{circum}{Centro de Nagel}
\TOline{mittenpunkt} {circum}{Ou middlespoint}
\TOline{feuerbach}{circum}{Ponto de Feuerbach}

\end{tabular}
\end{NewMacroBox}

\subsubsection{Opção \tkzname{ortho} ou \tkzname{orthic}}
 A interseção $H$ das três alturas de um triângulo é chamada de ortocentro.

\begin{tkzexample}[latex=6cm,small]
\begin{tikzpicture}
  \tkzDefPoint(0,0){A}
  \tkzDefPoint(5,1){B}
  \tkzDefPoint(1,4){C}
  \tkzDefTriangleCenter[ortho](B,C,A)
  \tkzGetPoint{H}
  \tkzDefSpcTriangle[orthic,name=H](A,B,C){a,b,c}
  \tkzDrawPolygon(A,B,C)
  \tkzDrawSegments[new](A,Ha B,Hb C,Hc)
  \tkzDrawPoints(A,B,C,H)
  \tkzLabelPoint(H){$H$}
  \tkzLabelPoints[below](A,B)
  \tkzLabelPoints[above](C)
 \tkzMarkRightAngles(A,Ha,B B,Hb,C C,Hc,A)
\end{tikzpicture}
\end{tkzexample}

\subsubsection{Opção \tkzname{centroid}}
\begin{tkzexample}[latex=6cm,small]
\begin{tikzpicture}[scale=.75]
  \tkzDefPoints{0/0/A,5/0/B,1/4/C}
  \tkzDefTriangleCenter[centroid](A,B,C)
  \tkzGetPoint{G}
  \tkzDrawPolygon(A,B,C)
  \tkzDrawLines[add = 0 and 2/3,new](A,G B,G C,G)
  \tkzDrawPoints(A,B,C,G)
  \tkzLabelPoint(G){$G$}
\end{tikzpicture}
\end{tkzexample}

\subsubsection{Opção \tkzname{circum}}
\begin{tkzexample}[latex=6cm,small]
 \begin{tikzpicture}
  \tkzDefPoints{0/1/A,3/2/B,1/4/C}
  \tkzDefTriangleCenter[circum](A,B,C)
  \tkzGetPoint{O}
  \tkzDrawPolygon(A,B,C)
  \tkzDrawCircle(O,A)
  \tkzDrawPoints(A,B,C,O)
  \tkzLabelPoint(O){$O$}
\end{tikzpicture}
\end{tkzexample}

\subsubsection{Opção \tkzname{in}}
Em geometria, o incírculo ou círculo inscrito de um triângulo é o maior círculo contido no triângulo; ele toca (é tangente a) os três lados. O centro do incírculo é um centro do triângulo chamado incentro do triângulo.
O centro do incírculo, chamado incentro, pode ser encontrado como a interseção das três bissetrizes internas dos ângulos. O centro de um excírculo é a interseção da bissetriz interna de um ângulo (no vértice $A$, por exemplo) e as bissetrizes externas dos outros dois. O centro deste excírculo é chamado de excentro relativo ao vértice $A$, ou excentro de $A$. Como a bissetriz interna de um ângulo é perpendicular à sua bissetriz externa, segue-se que o centro do incírculo juntamente com os três centros dos excírculos formam um sistema ortocêntrico.\\
(Artigo na \href{https://en.wikipedia.org/wiki/Incircle_and_excircles_of_a_triangle}{Wikipedia})

 \medskip
 Obtemos o centro do círculo inscrito do triângulo. O resultado é claro em \tkzname{tkzPointResult}. Podemos recuperá-lo com \tkzcname{tkzGetPoint}.

\begin{tkzexample}[latex=7cm,small]
\begin{tikzpicture}
\tkzDefPoints{0/0/A,6/0/B,0.8/4/C}
\tkzDefTriangleCenter[in](A,B,C)
   \tkzGetPoint{I}
\tkzDrawLines(A,B B,C C,A)
\tkzDefCircle[in](A,B,C) \tkzGetPoints{I}{i}
\tkzDrawCircle(I,i)
\tkzDrawPoint[red](I)
\tkzDrawPoints(A,B,C)
\tkzLabelPoint(I){$I$}
\end{tikzpicture}
\end{tkzexample}

\subsubsection{Opção \tkzname{ex}}
Um excírculo ou círculo ex-inscrito do triângulo é um círculo situado fora do triângulo, tangente a um de seus lados e tangente às extensões dos outros dois. Todo triângulo tem três excírculos distintos, cada um tangente a um dos lados do triângulo.\\
(Artigo na \href{https://en.wikipedia.org/wiki/Incircle_and_excircles_of_a_triangle}{Wikipedia})


 Obtemos o centro de um círculo inscrito do triângulo. O resultado é claro em \tkzname{tkzPointResult}. Podemos recuperá-lo com \tkzcname{tkzGetPoint}.

\begin{tkzexample}[latex=7cm,small]
\begin{tikzpicture}[scale=.5]
  \tkzDefPoints{0/1/A,3/2/B,1/4/C}
  \tkzDefTriangleCenter[ex](B,C,A)
  \tkzGetPoint{J_c}
  \tkzDefPointBy[projection=onto A--B](J_c)
  \tkzGetPoint{Tc}
  \tkzDrawPolygon(A,B,C)
  \tkzDrawCircle[new](J_c,Tc)
  \tkzDrawLines[add=1.5 and 0](A,C B,C)
  \tkzDrawPoints(A,B,C,J_c)
  \tkzLabelPoints(J_c)
\end{tikzpicture}
\end{tkzexample}

\subsubsection{Opção \tkzname{euler}}
Esta macro permite obter o centro do círculo dos nove pontos ou círculo de Euler ou círculo de Feuerbach. O círculo de nove pontos, também chamado círculo de Euler ou círculo de Feuerbach, é o círculo que passa pelos pés perpendiculares $H_A$, $H_B$ e $H_C$ baixados dos vértices de qualquer triângulo de referência $ABC$ sobre os lados opostos a eles. Euler mostrou em 1765 que ele também passa pelos pontos médios $M_A$, $M_B$, $M_C$ dos lados de $ABC$. Pelo teorema de Feuerbach, o círculo de nove pontos também passa pelos pontos médios $E_A$, $E_B$ e $E_C$ dos segmentos que unem os vértices e o ortocentro $H$. Esses pontos são comumente referidos como os pontos de Euler.\\ (\url{https://mathworld.wolfram.com/Nine-PointCircle.html})

\begin{tkzexample}[latex=6cm,small]
\begin{tikzpicture}[scale=1.2,rotate=90]
 \tkzDefPoints{0/0/A,6/0/B,0.8/4/C}
 \tkzDefSpcTriangle[medial,name=M](A,B,C){_A,_B,_C}
 \tkzDefTriangleCenter[euler](A,B,C)\tkzGetPoint{N}
 % I= N nine points
 \tkzDefTriangleCenter[ortho](A,B,C)\tkzGetPoint{H}
 \tkzDefMidPoint(A,H) \tkzGetPoint{E_A}
 \tkzDefMidPoint(C,H) \tkzGetPoint{E_C}
 \tkzDefMidPoint(B,H) \tkzGetPoint{E_B}
 \tkzDefSpcTriangle[ortho,name=H](A,B,C){_A,_B,_C}
 \tkzDrawPolygon(A,B,C)
 \tkzDrawCircle[new](N,E_A)
 \tkzDrawSegments[new](A,H_A B,H_B C,H_C)
 \tkzDrawPoints(A,B,C,N,H)
 \tkzDrawPoints[new](M_A,M_B,M_C)
 \tkzDrawPoints( H_A,H_B,H_C)
 \tkzDrawPoints[green](E_A,E_B,E_C)
 \tkzAutoLabelPoints[center=N,
 font=\scriptsize](A,B,C,M_A,M_B,M_C,H_A,H_B,H_C,%
   E_A,E_B,E_C)
 \tkzLabelPoints[font=\scriptsize](H,N)
 \tkzMarkSegments[mark=s|,size=3pt,
 color=blue,line width=1pt](B,E_B E_B,H)
\end{tikzpicture}
\end{tkzexample}


\subsubsection{Opção \tkzname{symmedian}}

O ponto de concorrência $K$ das simedianas, às vezes também chamado de ponto de Lemoine (na Inglaterra e França) ou ponto de Grebe (na Alemanha).\\
\href{https://mathworld.wolfram.com/SymmedianPoint.html}{Weisstein, Eric W. "Symmedian Point." From MathWorld--A Wolfram Web Resource.}

\begin{tkzexample}[latex=7cm,small]
\begin{tikzpicture}
  \tkzDefPoint(0,0){A}
  \tkzDefPoint(5,0){B}
  \tkzDefPoint(1,4){C}
  \tkzDefTriangleCenter[symmedian](A,B,C)
  \tkzGetPoint{K}
  \tkzDefTriangleCenter[median](A,B,C)
  \tkzGetPoint{G}
  \tkzDefTriangleCenter[in](A,B,C)\tkzGetPoint{I}
  \tkzDefSpcTriangle[centroid,name=M](A,B,C){a,b,c}
  \tkzDefSpcTriangle[incentral,name=I](A,B,C){a,b,c}
  \tkzDrawPolygon(A,B,C)
  \tkzDrawLines[add = 0 and 2/3,new](A,K B,K C,K)
  \tkzDrawSegments[color=cyan](A,Ma B,Mb C,Mc)
  \tkzDrawSegments[color=green](A,Ia B,Ib C,Ic)
  \tkzDrawPoints(A,B,C,K,G,I)
  \tkzLabelPoints[font=\scriptsize](A,B,K,G,I)
  \tkzLabelPoints[above,font=\scriptsize](C)
\end{tikzpicture}
\end{tkzexample}

\subsubsection{Opção \tkzname{spieker}}
O centro de Spieker é o centro $Sp$ do círculo de Spieker, ou seja, o incentro do triângulo medial de um triângulo de referência.\\
\href{https://mathworld.wolfram.com/SpiekerCenter.html}{Weisstein, Eric W. "Spieker Center." From MathWorld--A Wolfram Web Resource. }

\begin{tkzexample}[latex=8cm,small]
\begin{tikzpicture}
 \tkzDefPoints{0/0/A,6/0/B,5/5/C}
 \tkzDefSpcTriangle[medial](A,B,C){Ma,Mb,Mc}
 \tkzDefTriangleCenter[centroid](A,B,C)
 \tkzGetPoint{G}
 \tkzDefTriangleCenter[spieker](A,B,C)
 \tkzGetPoint{Sp}
 \tkzDrawPolygon[](A,B,C)
 \tkzDrawPolygon[new](Ma,Mb,Mc)
 \tkzDefCircle[in](Ma,Mb,Mc) \tkzGetPoints{I}{i}
 \tkzDrawCircle(I,i)
 \tkzDrawPoints(B,C,A,Sp,Ma,Mb,Mc)
 \tkzAutoLabelPoints[center=G,dist=.3](Ma,Mb)
 \tkzLabelPoints[right](Sp)
 \tkzLabelPoints[below](A,B,Mc)
 \tkzLabelPoints[above](C)
\end{tikzpicture}
\end{tkzexample}

\subsubsection{Opção \tkzname{gergonne}}

O Ponto de Gergonne é o ponto de concorrência que resulta da conexão dos vértices de um triângulo aos pontos opostos de tangência do incírculo do triângulo.
(Joseph Gergonne matemático francês)

\begin{tkzexample}[latex=8cm,small]
\begin{tikzpicture}
\tkzDefPoints{0/0/B,3.6/0/C,2.8/4/A}
\tkzDefTriangleCenter[gergonne](A,B,C)
\tkzGetPoint{Ge}
\tkzDefSpcTriangle[intouch](A,B,C){C_1,C_2,C_3}
\tkzDefCircle[in](A,B,C) \tkzGetPoints{I}{i}
\tkzDrawLines[add=.25 and .25,teal](A,B A,C B,C)
\tkzDrawSegments[new](A,C_1 B,C_2 C,C_3)
\tkzDrawPoints(A,...,C,C_1,C_2,C_3)
\tkzDrawPoints[red](Ge)
\tkzLabelPoints(B,C,C_1,Ge)
\tkzLabelPoints[above](A,C_2,C_3)
\end{tikzpicture}
\end{tkzexample}

\subsubsection{Opção \tkzname{nagel}}
Seja $Ta$ o ponto em que o excírculo com centro $Ja$ encontra o lado $BC$ de um triângulo $ABC$, e defina $Tb$ e $Tc$ de forma similar. Então as linhas $ATa$, $BTb$ e $CTc$ concorrem no ponto de Nagel $Na$.\\
\href{https://mathworld.wolfram.com/NagelPoint.html}{Weisstein, Eric W. "Nagel point." From MathWorld--A Wolfram Web Resource. }


\begin{tkzexample}[latex=7cm,small]
  \begin{tikzpicture}[scale=.5]
  \tkzDefPoints{0/0/A,6/0/B,4/6/C}
  \tkzDefSpcTriangle[ex](A,B,C){Ja,Jb,Jc}
  \tkzDefSpcTriangle[extouch](A,B,C){Ta,Tb,Tc}
  \tkzDefTriangleCenter[nagel](A,B,C)
  \tkzGetPoint{Na}
  \tkzDrawPolygon[blue](A,B,C)
  \tkzDrawLines[add=0 and 1](A,Ta B,Tb C,Tc)
  \tkzDrawPoints[new](Ja,Jb,Jc,Ta,Tb,Tc)
  \tkzClipBB
  \tkzDrawLines[add=1 and 1,dashed](A,B B,C C,A)
  \tkzDrawCircles[new](Ja,Ta Jb,Tb Jc,Tc)
  \tkzDrawSegments[new,dashed](Ja,Ta Jb,Tb Jc,Tc)
  \tkzDrawPoints(B,C,A)
  \tkzDrawPoints[new](Na)
  \tkzLabelPoints(B,C,A)
  \tkzLabelPoints[new](Na)
  \tkzLabelPoints[new](Ja,Jb,Jc,Ta,Tb,Tc)
  \tkzMarkRightAngles[fill=gray!20](Ja,Ta,C
              Jb,Tb,A Jc,Tc,B)
  \end{tikzpicture}
\end{tkzexample}


\subsubsection{Opção   \tkzname{mittenpunkt}}

O mittenpunkt (também chamado middlespoint) de um triângulo $ABC$ é o ponto simediano do triângulo excentral, ou seja, o ponto de concorrência M das linhas dos excentros através dos pontos médios dos lados do triângulo correspondentes.\\
\href{https://mathworld.wolfram.com/Mittenpunkt.html}{Weisstein, Eric W. "Mittenpunkt." From MathWorld--A Wolfram Web Resource.}


\begin{tkzexample}[latex=8cm,small]
\begin{tikzpicture}[scale=.5]
 \tkzDefPoints{0/0/A,6/0/B,4/6/C}
 \tkzDefSpcTriangle[centroid](A,B,C){Ma,Mb,Mc}
 \tkzDefSpcTriangle[ex](A,B,C){Ja,Jb,Jc}
 \tkzDefSpcTriangle[extouch](A,B,C){Ta,Tb,Tc}
 \tkzDefTriangleCenter[mittenpunkt](A,B,C)
 \tkzGetPoint{Mi}
 \tkzDrawPoints[new](Ma,Mb,Mc,Ja,Jb,Jc)
 \tkzClipBB
 \tkzDrawPolygon[blue](A,B,C)
 \tkzDrawLines[add=0 and 1](Ja,Ma
               Jb,Mb Jc,Mc)
 \tkzDrawLines[add=1 and 1](A,B A,C B,C)
 \tkzDrawCircles[new](Ja,Ta Jb,Tb Jc,Tc)
 \tkzDrawPoints(B,C,A)
 \tkzDrawPoints[new](Mi)
 \tkzLabelPoints(Mi)
 \tkzLabelPoints[left](Mb)
 \tkzLabelPoints[new](Ma,Mc,Jb,Jc)
 \tkzLabelPoints[above left](Ja,Jc)
\end{tikzpicture}
\end{tkzexample}

\subsubsection{Relação entre  \tkzname{gergonne}, \tkzname{centroid} e \tkzname{mittenpunkt}}

O ponto de Gergonne $Ge$, centroide do triângulo $G$ e mittenpunkt $M$ são colineares, com GeG/GM=2.

\begin{tkzexample}[vbox,small]
\begin{tikzpicture}
\tkzDefPoints{0/0/A,2/2/B,8/0/C}
\tkzDefTriangleCenter[gergonne](A,B,C) \tkzGetPoint{Ge}
\tkzDefTriangleCenter[centroid](A,B,C)
\tkzGetPoint{G}
\tkzDefTriangleCenter[mittenpunkt](A,B,C)
\tkzGetPoint{M}
\tkzDrawLines[add=.25 and .25,teal](A,B A,C B,C)
\tkzDrawLines[add=.25 and .25,new](Ge,M)
\tkzDrawPoints(A,...,C)
\tkzDrawPoints[red,size=2](G,M,Ge)
\tkzLabelPoints(A,...,C,M,G,Ge)
\tkzMarkSegment[mark=s||](Ge,G)
\tkzMarkSegment[mark=s|](G,M)
\end{tikzpicture}
\end{tkzexample}

\endinput
