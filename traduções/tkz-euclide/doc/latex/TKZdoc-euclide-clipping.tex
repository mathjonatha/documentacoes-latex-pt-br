\section{Controlando a Caixa Delimitadora}
Do \tkzimp{PgfManual}:

"Quando você adiciona a opção clip, o caminho atual é usado para recortar desenhos subsequentes. O recorte nunca aumenta a área de recorte. Assim, quando você recorta contra um determinado caminho e depois recorta novamente contra outro caminho, você recorta contra a interseção de ambos.
A única maneira de aumentar o caminho de recorte é terminar o {pgfscope} no qual o recorte foi feito. No final de um {pgfscope}, o caminho de recorte que estava em vigor no início do escopo é reinstalado."


Antes de tudo, você não precisa lidar com \TIKZ\ o tamanho da caixa delimitadora. Versões anteriores de \tkzNamePack{tkz-euclide} não controlavam o tamanho da caixa delimitadora, agora com \tkzNamePack{\tkznameofpack} 4 o tamanho da caixa delimitadora é limitado.

A caixa delimitadora inicial após usar a macro \tkzcname{tkzInit} é definida pelo retângulo baseado nos pontos $(0,0)$ e $(10,10)$. A macro \tkzcname{tkzInit} permite que esta caixa delimitadora inicial seja modificada usando os argumentos (\tkzname{xmin}, \tkzname{xmax}, \tkzname{ymin}, e \tkzname{ymax}). Claro que qualquer traço externo modifica a caixa delimitadora. \TIKZ\ mantém essa caixa delimitadora. É possível influenciar este comportamento diretamente com comandos ou opções em \TIKZ\ como um comando como \tkzcname{useasboundingbox} ou a opção \tkzname{use as bounding box}. Uma consequência possível é reservar uma caixa para uma figura, mas a figura pode transbordar a caixa e se espalhar sobre o texto principal.
O seguinte comando \tkzcname{pgfresetboundingbox} limpa uma caixa delimitadora e estabelece uma nova.

\subsection{Utilidade de \tkzcname{tkzInit}}
 No entanto, às vezes é necessário controlar o tamanho do que será exibido.
 Para fazer isso, você precisa ter preparado a caixa delimitadora em que vai trabalhar, este é o papel da macro \tkzNameMacro{tkzInit}. Para alguns desenhos, é interessante fixar os valores extremos (xmin, xmax, ymin e ymax) e \code{clip} o retângulo de definição para controlar o tamanho da figura da melhor maneira possível.

As duas macros que são úteis para controlar a caixa delimitadora:
\begin{itemize}
   \item \tkzcname{tkzInit}
   \item \tkzcname{tkzClip}
\end{itemize}
\vspace{20pt}

A isso, adicionei macros diretamente vinculadas à caixa delimitadora. Você agora pode visualizá-la, fazer backup dela, restaurá-la (veja a seção Bounding Box).

\subsection{\tkzcname{tkzInit}}

\begin{NewMacroBox}{tkzInit}{\oarg{local opções}}\hypertarget{init}{}%
\begin{tabular}{lll}%
opções  & padrão & definição             \\
\midrule
\TOline{xmin} {0} {valor mínimo das abscissas em cm}
\TOline{xmax} {10} {valor máximo das abscissas em cm}
\TOline{xstep}{1} {diferença entre duas graduações em $x$}
\TOline{ymin} {0} {valor mínimo do eixo y em cm}
\TOline{ymax} {10} {valor máximo do eixo y em cm}
\TOline{ystep}{1} {diferença entre duas graduações em $y$}
\bottomrule
\end{tabular}

\medskip

O papel de \tkzcname{tkzInit} é definir um sistema de coordenadas \textcolor{red}{ortogonal} e uma parte retangular do plano na qual você colocará seus desenhos usando coordenadas cartesianas.
Esta macro permite que você defina seu ambiente de trabalho como em uma calculadora. Com \tkzname{\tkznameofpack} 4 \tkzcname{xstep} e \tkzcname{ystep} são sempre 1. Logicamente não é mais útil usar \tkzcname{tkzInit}, exceto para uma ação como \code{Clipping Out}.
\end{NewMacroBox}


\subsection{\tkzcname{tkzClip}}

\begin{NewMacroBox}{tkzClip}{\oarg{local opções}}
O papel desta macro é tornar invisível o que está fora do retângulo definido por (xmin~;~ymin) e (xmax~;~ymax).

\medskip
\begin{tabular}{lll}
\hline
opções  & padrão & definição             \\
\midrule
\TOline{space} {1} {valor adicionado à direita, esquerda, inferior e superior do fundo}
\bottomrule
\end{tabular}

\medskip

O papel da opção \tkzname{space} é aumentar a parte visível do desenho. Esta parte torna-se o retângulo definido por (xmin-space~;~ymin-space) e (xmax+space~;~ymax+space). \tkzname{space} pode ser negativo! A unidade é cm e não deve ser especificada.
\end{NewMacroBox}



O papel desta macro é \code{clip} (recortar) o retângulo inicial para que apenas os caminhos contidos neste retângulo sejam desenhados.

\begin{tkzexample}[latex=8cm,small]
\begin{tikzpicture}
 \tkzInit[xmax=4, ymax=3]
 \tkzDefPoints{-1/-1/A,5/2/B}
 \tkzDrawX \tkzDrawY
 \tkzGrid
 \tkzClip
 \tkzDrawSegment(A,B)
\end{tikzpicture}
\end{tkzexample}

É possível adicionar um pouco de espaço
\begin{tkzltxexample}[]
  \tkzClip[space=1]
\end{tkzltxexample} 

\subsection{\tkzcname{tkzClip} e a opção \tkzname{space}}
Esta opção permite adicionar algum espaço ao redor do retângulo \code{clipped} (recortado).
\begin{tkzexample}[latex=8cm,small]
\begin{tikzpicture}
 \tkzInit[xmax=4, ymax=3]
 \tkzDefPoints{-1/-1/A,5/2/B}
 \tkzDrawX \tkzDrawY 
 \tkzGrid
 \tkzClip[space=1]
 \tkzDrawSegment(A,B)
\end{tikzpicture}
\end{tkzexample}   
As dimensões do retângulo \code{clipped} (recortado) são \tkzname{xmin-1}, \tkzname{ymin-1}, \tkzname{xmax+1} e \tkzname{ymax+1}.

%<--------------------------------------------------------------------------->
%              tkzShowBB
%<--------------------------------------------------------------------------->
\subsection{tkzShowBB}
A macro mais simples.
\begin{NewMacroBox}{tkzShowBB}{\oarg{local opções}}%
Esta macro exibe a caixa delimitadora. Um quadro retangular envolve a caixa delimitadora. Esta macro aceita opções \TIKZ.
\end{NewMacroBox} 


\subsubsection{Exemplo with \tkzcname{tkzShowBB}}
\begin{tkzexample}[latex=8cm,small]
\begin{tikzpicture}[scale=.5]
  \tkzInit[ymax=5,xmax=8]
  \tkzGrid  
  \tkzDefPoint(3,0){A}
   \begin{scope}
    \tkzClipBB
    \tkzDefCircle[R](A,5) \tkzGetPoint{a}
    \tkzDrawCircle(A,a)
    \tkzShowBB[line width = 4pt,fill=teal!10,%
              opacity=.4]
   \end{scope}
\tkzDefCircle[R](A,4) \tkzGetPoint{b}
\tkzDrawCircle[red](A,b)
\end{tikzpicture}
\end{tkzexample}
%<--------------------------------------------------------------------------->
%         tkzClipBB
%<--------------------------------------------------------------------------->
\subsection{tkzClipBB}
\begin{NewMacroBox}{tkzClipBB}{}%
A ideia é limitar construções futuras à caixa delimitadora atual.
\end{NewMacroBox}

\subsubsection{Exemplo com \tkzcname{tkzClipBB} e as bissetrizes}

\begin{tkzexample}[latex=6cm,small]
  \begin{tikzpicture}
  \tkzInit[xmin=-3,xmax=6, ymin=-1,ymax=6]
  \tkzDefPoint(0,0){O}\tkzDefPoint(3,1){I}
  \tkzDefPoint(1,4){J}
  \tkzDefLine[bisector](I,O,J) \tkzGetPoint{i}
  \tkzDefLine[bisector out](I,O,J) \tkzGetPoint{j}
  \tkzDrawPoints(O,I,J,i,j)
  \tkzClipBB
  \tkzDrawLines[add = 1 and 2,color=orange](O,I O,J)
  \tkzDrawLines[add = 1 and 2](O,i O,j)
  \tkzShowBB
  \end{tikzpicture}
\end{tkzexample}


\newpage

\section{Recortando diferentes objetos}

\subsection{Recortando um polígono}
 \begin{NewMacroBox}{tkzClipPolygon}{\oarg{local opções}\parg{lista de pontos}}%
Esta macro torna possível conter os diferentes gráficos no polígono designado.

\medskip
\begin{tabular}{lll}%
\toprule
argumentos       & exemplo & explicação     \\
\midrule
\TAline{\parg{pt1,pt2,pt3,\dots}}{\parg{A,B,C}}{}
\midrule
opções  & padrão & definição             \\
\midrule
\TOline{out} {} {permite recortar o exterior do objeto}
 \end{tabular}
\end{NewMacroBox}

\subsubsection{\tkzcname{tkzClipPolygon}}

\begin{tkzexample}[latex=7cm,small]
  \begin{tikzpicture}[scale=1.25] 
  \tkzDefPoint(0,0){A} 
  \tkzDefPoint(4,0){B} 
  \tkzDefPoint(1,3){C} 
  \tkzDrawPolygon(A,B,C) 
  \tkzDefPoint(0,2){D} 
  \tkzDefPoint(2,0){E} 
  \tkzDrawPoints(D,E) 
  \tkzLabelPoints(D,E) 
  \tkzClipPolygon(A,B,C) 
  \tkzDrawLine[new](D,E)
\end{tikzpicture}
\end{tkzexample}

\subsubsection{\tkzcname{tkzClipPolygon[out]}}

\begin{tkzexample}[latex=7cm,small]
  \begin{tikzpicture}[scale=1]
  \tkzDefPoint(0,0){P1}
  \tkzDefPoint(4,0){P2}
  \tkzDefPoint(4,4){P3}
  \tkzDefPoint(0,4){P4}
  \tkzDefPoint(1,1){Q1}
  \tkzDefPoint(3,1){Q2}
  \tkzDefPoint(3,3){Q3}
  \tkzDefPoint(1,3){Q4}
  \tkzDrawPolygon(P1,P2,P3,P4)
  \begin{scope}
     \tkzClipPolygon[out](Q1,Q2,Q3,Q4)
    \tkzFillPolygon[teal!20](P1,P2,P3,P4)
  \end{scope}
  \tkzDrawPolygon(Q1,Q2,Q3,Q4)
  \end{tikzpicture}
\end{tkzexample}

\subsubsection{Exemplo: uso de \code{Clip} para Sangaku em um quadrado} 
\begin{tkzexample}[latex=7cm, small]  
\begin{tikzpicture}[scale=.75]
 \tkzDefPoint(0,0){A} \tkzDefPoint(8,0){B}
 \tkzDefSquare(A,B)   \tkzGetPoints{C}{D}
 \tkzDefPoint(4,8){F}
 \tkzDefTriangle[equilateral](C,D) 
 \tkzGetPoint{I}
 \tkzDefPointBy[projection=onto B--C](I) 
 \tkzGetPoint{J}
 \tkzInterLL(D,B)(I,J)  \tkzGetPoint{K}
 \tkzDefPointBy[symmetry=center K](B)  
 \tkzGetPoint{M}
 \tkzClipPolygon(B,C,D,A)
 \tkzFillPolygon[color = orange](A,B,C,D)
 \tkzFillCircle[color = yellow](M,I)
 \tkzFillCircle[color = blue!50!black](F,D)
\end{tikzpicture}
\end{tkzexample}

\subsection{Recortando um disco}

\begin{NewMacroBox}{tkzClipCircle}{\oarg{local opções}\parg{A,B}}%
\begin{tabular}{lll}%
\toprule
argumentos           & exemplo & explicação                         \\
\midrule
\TAline{\parg{A,B}}{\parg{A,B}} {raio AB}
\bottomrule
\end{tabular}

\medskip
\begin{tabular}{lll}%
opções             & padrão & definição                         \\
\midrule
\TOline{out} {} {permite recortar o exterior do objeto}
 \bottomrule
\end{tabular}

\medskip
Não é necessário colocar \tkzname{radius} porque essa é a opção padrão.
\end{NewMacroBox}

 \subsubsection{Recorte simples} 
\begin{tkzexample}[latex=6cm,small] 
\begin{tikzpicture}[scale=.5]
  \tkzDefPoint(0,0){A} \tkzDefPoint(2,2){O}
  \tkzDefPoint(4,4){B} \tkzDefPoint(5,5){C}
  \tkzDrawPoints(O,A,B,C) 
  \tkzLabelPoints(O,A,B,C)
  \tkzDrawCircle(O,A) 
  \tkzClipCircle(O,A)
  \tkzDrawLine(A,C)
  \tkzDrawCircle[fill=teal!10,opacity=.5](C,O)
\end{tikzpicture} 
\end{tkzexample}

\subsection{Recorte externo}

\begin{tkzexample}[latex=6cm,small]
\begin{tikzpicture}
  \tkzInit[xmin=-3,ymin=-2,xmax=4,ymax=3]
   \tkzDefPoint(0,0){O}
   \tkzDefPoint(-4,-2){A}
   \tkzDefPoint(3,1){B}
   \tkzDefCircle[R](O,2) \tkzGetPoint{o}
   \tkzDrawPoints(A,B) % to have a good bounding box
   \begin{scope}
    \tkzClipCircle[out](O,o)
    \tkzDrawLines(A,B)
   \end{scope}
\end{tikzpicture}  
\end{tkzexample} 

\subsection{Interseção de discos}

\begin{tkzexample}[latex=6cm,small]
\begin{tikzpicture}
\tkzDefPoints{0/0/O,4/0/A,0/4/B}
\tkzDrawPolygon[fill=teal](O,A,B)
\tkzClipPolygon(O,A,B)
\tkzClipCircle(A,O)
\tkzClipCircle(B,O)
\tkzFillPolygon[white](O,A,B)
\end{tikzpicture}
\end{tkzexample}

veja um exemplo mais complexo sobre recorte de círculos aqui: \ref{About clipping circles}

\subsection{Recortando um setor}
\tkzHandBomb\  Atenção os argumentos variam de acordo com as opções. 
\begin{NewMacroBox}{tkzClipSector}{\oarg{local opções}\parg{O,\dots}\parg{\dots}}%
\begin{tabular}{lll}%
opções             & padrão & definição                         \\
\midrule
\TOline{towards}{towards}{$O$ é o centro e o setor começa de $A$ para $(OB)$}
\TOline{rotate} {towards}{O setor começa de $A$ e o ângulo determina sua amplitude.}
\TOline{R}{towards}{Fornecemos o raio e dois ângulos}
\bottomrule
\end{tabular}

\medskip
Você tem que adicionar, é claro, todos os estilos de \TIKZ\ para os traçados...

\medskip
\begin{tabular}{lll}%
\toprule
opções             & argumentos & exemplo                         \\
\midrule
\TOline{towards}{\parg{pt,pt}\parg{pt}}{\tkzcname{tkzClipSector(O,A)(B)}}
\TOline{rotate} {\parg{pt,pt}\parg{ângulo}}{\tkzcname{tkzClipSector[rotate](O,A)(90)}}
\TOline{R}{\parg{pt,$r$}\parg{ângulo 1,ângulo 2}}{\tkzcname{tkzClipSector[R](O,2)(30,90)}}
\end{tabular}
\end{NewMacroBox}

\subsubsection{Exemplo 1} 
\begin{tkzexample}[latex=7cm,small] 
\begin{tikzpicture}[scale=0.5]
\tkzDefPoint(0,0){a}
\tkzDefPoint(12,0){b}
\tkzDefPoint(4,10){c}
\tkzInterCC[R](a,6)(b,8) 
\tkzGetFirstPoint{AB1} \tkzGetSecondPoint{AB2}
\tkzInterCC[R](a,6)(c,6) 
\tkzGetFirstPoint{AC1} \tkzGetSecondPoint{AC2}
\tkzInterCC[R](b,8)(c,6) 
\tkzGetFirstPoint{BC1} \tkzGetSecondPoint{BC2}
\tkzDrawArc(a,AB2)(AB1)
\tkzDrawArc(b,AB1)(AB2)
\tkzDrawArc(a,AC2)(AC1)
\tkzDrawArc(c,AC1)(AC2)
\tkzDrawArc(b,BC2)(BC1)
\tkzDrawArc(c,BC1)(BC2)
\begin{scope}
\tkzClipSector(b,BC2)(BC1)
\tkzFillSector[teal!40!white](c,BC1)(BC2)
\end{scope}
\begin{scope}
\tkzClipSector(a,AB2)(AB1)
\tkzFillSector[teal!40!white](b,AB1)(AB2)
\end{scope}
\begin{scope}
\tkzClipSector(a,AC2)(AC1)
\tkzFillSector[teal!40!white](c,AC1)(AC2)
\end{scope}
\end{tikzpicture}
\end{tkzexample}

\subsubsection{Exemplo 2} 
\begin{tkzexample}[latex=7cm,small] 
\begin{tikzpicture}[scale=1.5] 
  \tkzDefPoint(0,0){O}
  \tkzDefPoint(2,-1){A}
  \tkzDefPoint(1,1){B} 
  \tkzDrawSector[new,dashed](O,A)(B)
  \tkzDrawSector[new](O,B)(A)
\begin{scope}
\tkzClipSector(O,B)(A)
\tkzDefSquare(O,B) \tkzGetPoints{B'}{O'}
\tkzDrawPolygon[color=teal,fill=teal!20](O,B,B',O')
\end{scope}
\tkzDrawPoints(A,B,O) 
\end{tikzpicture} 
\end{tkzexample}


\subsection{Opções do \TIKZ: trim left ou right}
Veja o \tkzimp{pgfmanual}

\subsection{Controles \TIKZ\ \tkzcname{pgfinterruptboundingbox} e \tkzcname{endpgfinterruptboundingbox}}
Este comando interrompe temporariamente o cálculo da caixa e configura uma nova caixa.
Veja o \tkzimp{pgfmanual}

\subsubsection{Exemplo sobre controle da caixa delimitadora} 
\begin{tkzexample}[latex=7cm,small] 
\begin{tikzpicture}
\tkzDefPoint(0,5){A}\tkzDefPoint(5,4){B}
\tkzDefPoint(0,0){C}\tkzDefPoint(5,1){D}
\tkzDrawSegments(A,B C,D A,C)
\pgfinterruptboundingbox
   \tkzInterLL(A,B)(C,D)\tkzGetPoint{I}
\endpgfinterruptboundingbox
\tkzClipBB
\tkzDrawCircle(I,B)
\end{tikzpicture}
\end{tkzexample}

\subsection{Recorte reverso: tkzreverseclip}

Para usar esta opção, uma caixa delimitadora deve ser definida. 

\begin{tkzltxexample}[]
  \tikzset{tkzreverseclip/.style={insert path={
     (current bounding box.south west) --(current bounding box.north west)
   --(current bounding box.north east) --  (current bounding box.south east)
   -- cycle} }}
\end{tkzltxexample}

\subsubsection{Exemplo with \tkzcname{tkzClipPolygon[out]}}
\tkzcname{tkzClipPolygon[out]}, \tkzcname{tkzClipCircle[out]} use this opção.
\begin{tkzexample}[vbox,small]
\begin{tikzpicture}[scale=1]
\tkzInit[xmin=-5,xmax=5,ymin=-4,ymax=6]
\tkzClip
\tkzDefPoints{-.5/0/P1,.5/0/P2}
\foreach \i [count=\j from 3] in {2,...,7}{%
    \tkzDefShiftPoint[P\i]({45*(\i-1)}:1){P\j}}  
\tkzClipPolygon[out](P1,P...,P8)
\tkzCalcLength(P1,P5)\tkzGetLength{r}
\begin{scope}[blend group=screen]
  \foreach \i in {1,...,8}{%
   \tkzDefCircle[R](P\i,\r) \tkzGetPoint{x}
   \tkzFillCircle[color=teal](P\i,x)}
  \end{scope}
\end{tikzpicture}
\end{tkzexample}

\endinput