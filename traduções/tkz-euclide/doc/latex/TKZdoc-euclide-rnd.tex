\section{Definição de ponto aleatório}
%<--------------------------------------------------------------------------->
%           points random
%<--------------------------------------------------------------------------->
No momento existem quatro possibilidades:
\begin{enumerate}
  \item ponto em um retângulo;
  \item em um segmento;
  \item em uma reta;
  \item em um círculo.
\end{enumerate}

\subsection{Obtendo pontos aleatórios}
Esta é a nova versão que substitui  \tkzcname{tkzGetRandPointOn}.
\begin{NewMacroBox}{tkzDefRandPointOn}{\oarg{opções locais}}%
{O resultado é um ponto com uma posição aleatória que pode ser nomeado com a macro \tkzcname{tkzGetPoint}. É possível usar \tkzname{tkzPointResult} se não for necessário reter os resultados.}

\medskip
\begin{tabular}{lll}%
\toprule
opções             & padrão & definição                         \\
\midrule
\TOline{rectangle=pt1 and pt2}  {}{[rectangle=A and B]}
\TOline{segment= pt1--pt2} {}{[segment=A--B]}
\TOline{line=pt1--pt2}{}{[line=A--B]}
\TOline{circle =center pt1 radius dim}{}{[circle = center A radius 2]}
\TOline{circle through=center pt1 through pt2}{}{[circle through= center A through B]}
\TOline{disk through=center pt1 through pt2}{}{[disk through=center A through B]}
\end{tabular}
\end{NewMacroBox}

\subsubsection{Ponto aleatório em um retângulo}

\begin{tkzexample}[latex=7cm,small]
\begin{tikzpicture}
  \tkzDefPoints{0/0/A,5/3/C}
  \tkzDefRandPointOn[rectangle = A and C]
  \tkzGetPoint{E}
  \tkzDefRectangle(A,C)\tkzGetPoints{B}{D}
  \tkzDrawPolygon[red](A,...,D)
  \tkzDrawPoints(A,...,E)
  \tkzLabelPoints(A,B)
  \tkzLabelPoints[above](C,D,E)
\end{tikzpicture}
\end{tkzexample}

\subsubsection{Ponto aleatório em um segmento ou uma reta}
\begin{tkzexample}[latex=7cm,small]
\begin{tikzpicture}
  \tkzDefPoints{0/0/A,5/2/C}
  \tkzDefRandPointOn[segment = A--C]\tkzGetPoint{B}
  \tkzDrawLine(A,C)
  \tkzDrawPoints(A,C) \tkzDrawPoint[red](B)
  \tkzLabelPoints(A,C) \tkzLabelPoints[red](B)
\end{tikzpicture}
\end{tkzexample}


\subsubsection{Ponto aleatório em um círculo ou um disco}
\begin{tkzexample}[latex=7cm,small]
\begin{tikzpicture}
\tkzDefPoints{3/2/A,1/1/B}
\tkzCalcLength(A,B) \tkzGetLength{rAB}
\tkzDefRandPointOn[circle = center A radius \rAB]
\tkzGetPoint{C}
\tkzDefRandPointOn[circle through= center A through B]
\tkzGetPoint{D}
\tkzDefRandPointOn[disk through=center A through B]
\tkzGetPoint{E}
\tkzDrawCircle(A,B)
\tkzDrawPoints(A,B)
\tkzLabelPoints(A,B)
\tkzDrawPoints[red](C,D,E)
\tkzLabelPoints[red,right](C,D,E)
\end{tikzpicture}
\end{tkzexample}

\endinput
