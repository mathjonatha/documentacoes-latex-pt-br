\section{\tkzname{Interseções}}

É possível determinar as coordenadas dos pontos de interseção entre duas retas, uma reta e um círculo, e dois círculos.

Os comandos associados não têm argumentos opcionais e o usuário deve determinar por si mesmo a existência dos pontos de interseção.

\subsection{Interseção de duas retas \tkzcname{tkzInterLL}}
\begin{NewMacroBox}{tkzInterLL}{\parg{$A,B$}\parg{$C,D$}}%
Define o ponto de interseção \tkzname{tkzPointResult} das duas retas $(AB)$ e $(CD)$. Os pontos conhecidos são fornecidos em pares (dois por reta) entre parênteses, e o ponto resultante pode ser recuperado com a macro \tkzcname{tkzDefPoint}.
\end{NewMacroBox}

\subsubsection{Exemplo de interseção entre duas retas}

\begin{tkzexample}[latex=7cm,small]
\begin{tikzpicture}[rotate=-45,scale=.75]
  \tkzDefPoint(2,1){A}   
  \tkzDefPoint(6,5){B}
  \tkzDefPoint(3,6){C}   
  \tkzDefPoint(5,2){D}
  \tkzDrawLines(A,B C,D)
  \tkzInterLL(A,B)(C,D)  
     \tkzGetPoint{I}
  \tkzDrawPoints[color=blue](A,B,C,D)
   \tkzDrawPoint[color=red](I)
\end{tikzpicture}
\end{tkzexample}

\subsection{Interseção de uma reta e um círculo \tkzcname{tkzInterLC}}

Como antes, a reta é definida por um par de pontos. O círculo
 também é definido por um par:
\begin{itemize}
\item $(O,C)$ que é um par de pontos, o primeiro é o centro e o segundo é qualquer ponto no círculo.
\item $(O,r)$ A medida $r$ é a medida do raio.
\end{itemize}

\begin{NewMacroBox}{tkzInterLC}{\oarg{opções}\parg{$A,B$}\parg{$O,C$} ou \parg{$O,r$} ou \parg{$O,C,D$}}%
Portanto, os argumentos são dois pares. 

\medskip
\begin{tabular}{lll}%
\toprule
opções            & padrão & definição                         \\ 
\midrule
\TOline{N}         {N}    {(O,C) determina o círculo}
\TOline{R}         {N}    {(O, 1 ) unidade 1 cm}
\TOline{with nodes}{N}    {(O,C,D) CD é um raio}
\TOline{common=pt} {}     {pt é ponto comum; tkzFirstPoint fornece o outro ponto}
\TOline{near}      {}     {tkzFirstPoint é o ponto mais próximo do primeiro ponto da reta}
\bottomrule
\end{tabular}

\medskip
A macro define os pontos de interseção $I$ e $J$ da reta $(AB)$ e do círculo de centro $O$ com raio $r$ se existirem; caso contrário, um erro será relatado no arquivo |.log|. \tkzname{with nodes} evita que você calcule o raio que é o comprimento de $[CD]$.
Se \tkzname{common} e \tkzname{near} não forem usados, então \tkzname{tkzFirstPoint} é o menor ângulo (ângulo com \tkzname{tkzSecondPoint} e o centro do círculo).
\end{NewMacroBox}

\begin{NewMacroBox}{tkzTestInterLC}{\parg{$O,A$}\parg{$O',B$}}%
Portanto, os argumentos são dois pares que definem uma reta e um círculo com um centro e um ponto no círculo. Se houver uma interseção não vazia entre a reta e o círculo, então o teste \tkzcname{iftkzFlagLC} retorna verdadeiro.
\end{NewMacroBox}

\subsubsection{Teste de interseção reta-círculo}

\begin{tkzexample}[latex=7cm,small]
  \begin{tikzpicture}[scale=1]
    \tkzDefPoints{% x   y   name
                    3    /4    /I,
                    3    /2    /P,
                    0    /2    /La,
                    8    /3    /Lb}
  \tkzDrawCircle(I,P)
  \foreach \i in {1,...,3}{%
     \coordinate  (Lb) at (8,\i);
     \tkzDrawLine(La,Lb)
     \tkzTestInterLC(La,Lb)(I,P)
      \iftkzFlagLC
      \tkzInterLC(La,Lb)(I,P)  
      \tkzGetPoints{a}{b}
      \tkzDrawPoints(a,b)
      \fi
     }
  \end{tikzpicture}
\end{tkzexample}


\subsubsection{Interseção reta-círculo}

No exemplo seguinte, o desenho do círculo usa dois pontos e a interseção da reta e do círculo usa dois pares de pontos. Vamos comparar os ângulos $\widehat{D,E,O}$ e $\widehat{E,D,O}$. Esses ângulos estão em direções opostas. \tkzname{tkzFirstPoint} é atribuído ao ponto que forma o ângulo com a menor medida (direção anti-horária). O ângulo anti-horário $\widehat{D,E,O}$ tem uma medida igual a $360\circ$ menos a medida de $\widehat{O,E,D}$.

\begin{tkzexample}[latex=7cm,small]
\begin{tikzpicture}[scale=.75]
 \tkzInit[xmax=5,ymax=4]
 \tkzDefPoint(1,1){O} 
 \tkzDefPoint(-2,4){La} 
 \tkzDefPoint(5,0){Lb} 
 \tkzDefPoint(3,3){C}
 \tkzInterLC(La,Lb)(O,C)  \tkzGetPoints{D}{E}  
 \tkzMarkAngle[->,size=1.5](E,D,O)
 \tkzDrawPolygons[new](O,D,E)
 \tkzMarkAngle[->,size=1.5](D,E,O)
 \tkzDrawCircle(O,C)
 \tkzDrawPoints[color=teal](O,La,Lb,C)
 \tkzDrawPoints[color=red](D,E)
 \tkzDrawLine(La,Lb)
 \tkzLabelPoints[above right](O,La,Lb,C,D,E)
\end{tikzpicture} 
\end{tkzexample}

\subsubsection{Reta passando pelo centro opção \tkzname{common}}
Este caso é especial. Você não pode comparar os ângulos. Neste caso, a opção \tkzname{near} deve ser usada. \tkzname{tkzFirstPoint} é atribuído ao ponto mais próximo do primeiro ponto dado para a reta. Aqui queremos que $A$ esteja mais próximo de $Lb$.

\begin{tkzexample}[latex=8cm,small]
\begin{tikzpicture}
\tkzDefPoints{% x   y   name
             0    /1    /D,
             6    /0    /B,
             3    /3    /O,
             2    /2    /La,
             5    /5    /Lb}
  \tkzDrawCircle(O,D)
  \tkzDrawLine(La,Lb)
  \tkzInterLC[near](Lb,La)(O,D)  
  \tkzGetFirstPoint{A}
  \tkzDrawSegments(O,A)
  \tkzDrawPoints(O,D,A,La,Lb)
  \tkzLabelPoints(O,D,A,La,Lb)
\end{tikzpicture}
\end{tkzexample}

\subsubsection{Interseção reta-círculo com opção \tkzname{common}}
Um caso especial que frequentemente encontramos, um ponto da reta está no círculo e estamos procurando o outro ponto comum.
\begin{tkzexample}[latex=7cm,small]
\begin{tikzpicture}[scale=.5]
 \tkzDefPoints{0/0/O,-5/0/A,2/-2/B,0/5/D}
 \tkzInterLC[common=A](B,A)(O,D)
 \tkzGetFirstPoint{C}
 \tkzDrawPoints(O,A,B)
 \tkzDrawCircle(O,A)
 \tkzDrawLine(A,C)
 \tkzDrawPoint(C)
 \tkzLabelPoints(A,B,C)
\end{tikzpicture}
\end{tkzexample}


\subsubsection{Ordem dos pontos na interseção reta-círculo}
A ideia é comparar os ângulos formados com o primeiro ponto definidor da reta, um ponto resultante e o centro do círculo. O primeiro ponto é aquele que corresponde ao menor ângulo.

Como você pode ver $\widehat{BCO} < \widehat{BEO}$. Para dizer a verdade, $\widehat{BEO}$ é anti-horário.

\begin{tkzexample}[latex=6cm,small]
\begin{tikzpicture}[scale=.5]
  \tkzDefPoints{0/0/O,5/1/A,2/2/B,3/1/D}
  \tkzInterLC[common=A](B,D)(O,A) \tkzGetPoints{C}{E}
  \tkzDrawPoints(O,A,B,D)
  \tkzDrawCircle(O,A) \tkzDrawLine(E,C)
  \tkzDrawSegments[dashed](B,O O,C)
  \tkzMarkAngle[->,size=1.5](B,C,O)
  \tkzDrawSegments[dashed](O,E)
  \tkzMarkAngle[->,size=1.5](B,E,O)
  \tkzDrawPoints(C,E)
  \tkzLabelPoints[above](O,E)
  \tkzLabelPoints[right](A,B,C,D)
\end{tikzpicture}
\end{tkzexample}

\subsubsection{Exemplo with \tkzcname{foreach}}
\begin{tkzexample}[latex=7cm,small]
\begin{tikzpicture}[scale=3,rotate=180]
\tkzDefPoint(0,1){J} 
\tkzDefPoint(0,0){O}
\foreach \i in {0,-5,-10,...,-90}{
 \tkzDefPoint({2.5*cos(\i*pi/180)},%
   {1+2.5*sin(\i*pi/180)}){P}
 \tkzInterLC[R](P,J)(O,1)\tkzGetPoints{N}{M}
 \tkzDrawSegment[color=orange](J,N)
 \tkzDrawPoints[red](N)} 
\foreach \i in {-90,-95,...,-175,-180}{
 \tkzDefPoint({2.5*cos(\i*pi/180)},%
   {1+2.5*sin(\i*pi/180)}){P} 
 \tkzInterLC[R](P,J)(O,1)\tkzGetPoints{N}{M}
 \tkzDrawSegment[color=orange](J,M)
 \tkzDrawPoints[red](M)}   
\end{tikzpicture} 
\end{tkzexample}

\subsubsection{Interseção reta-círculo com opção \tkzname{near}}
$D$ é o ponto mais próximo de $b$.

\begin{tkzexample}[vbox,small]
  \begin{tikzpicture}
    \tkzDefPoints{0/0/A,12/0/C}
    \tkzDefGoldenRatio(A,C)                          \tkzGetPoint{B}
    \tkzDefMidPoint(A,C)                             \tkzGetPoint{O}
    \tkzDefMidPoint(A,B)                             \tkzGetPoint{O_1}
    \tkzDefMidPoint(B,C)                             \tkzGetPoint{O_2}
    \tkzDefPointBy[rotation=center O_2 angle 90](C)  \tkzGetPoint{P}
    \tkzDefPointBy[rotation=center O_1 angle 90](B)  \tkzGetPoint{Q}
    \tkzDefPointBy[rotation=center B angle 90](C)    \tkzGetPoint{b}
    \tkzInterLC[near](b,B)(O,A)                      \tkzGetFirstPoint{D}
    \tkzInterCC(D,B)(O,C)                            \tkzGetPoints{V}{U}
    \tkzDefPointBy[projection=onto U--V](O_1)        \tkzGetPoint{M}
    \tkzDefPointBy[projection=onto U--V](O_2)        \tkzGetPoint{N}  
    \tkzDrawPoints(A,B,C,O,O_1,O_2,D,U,V,M,N,b)
    \tkzDrawSemiCircles[teal](O,C O_1,B O_2,C)
    \tkzDrawSegments(A,C B,D U,V A,D C,D M,B B,N)
    \tkzDrawArc(D,U)(V)
    \tkzLabelPoints(A,B,C,O,O_1,O_2)
    \tkzLabelPoints[above](D,U,V,M,N)
  \end{tikzpicture}
\end{tkzexample}


\subsubsection{Exemplo mais complexo de uma interseção reta-círculo}
Figura de  \url{http://gogeometry.com/problem/p190_tangent_circle}

\begin{tkzexample}[latex=6.5cm,small]
\begin{tikzpicture}[scale=.75]
 \tkzDefPoint(0,0){A}  
 \tkzDefPoint(8,0){B}
 \tkzDefMidPoint(A,B)             \tkzGetPoint{O}
 \tkzDefMidPoint(O,B)             \tkzGetPoint{O'}
 \tkzDefLine[tangent from=A](O',B)\tkzGetFirstPoint{E}
 \tkzInterLC(A,E)(O,B)            \tkzGetFirstPoint{D}
 \tkzDefPointBy[projection=onto A--B](D)  
 \tkzGetPoint{F}
 \tkzDrawCircles(O,B O',B)
 \tkzDrawSegments(A,D A,B D,F) 
 \tkzDrawSegments[color=red,line width=1pt,
     opacity=.4](A,O F,B)
 \tkzDrawPoints(A,B,O,O',E,D) 
 \tkzMarkRightAngle(D,F,B)
 \tkzLabelPoints[below right](A,B,O,O',E,D) 
\end{tikzpicture}
\end{tkzexample}

\subsubsection{Círculo definido por um centro e uma medida, e casos especiais}
Vejamos alguns casos especiais como retas tangentes ao círculo.

\begin{tkzexample}[latex=7cm,small]
\begin{tikzpicture}[scale=.5]
 \tkzDefPoint(0,8){A}      \tkzDefPoint(8,0){B}
 \tkzDefPoint(8,8){C}      \tkzDefPoint(4,4){D}
 \tkzDefPoint(2,4){E}      \tkzDefPoint(4,2){F}
 \tkzDefPoint(8,4){G}
 \tkzInterLC(A,C)(D,G)     \tkzGetPoints{I1}{I2}
 \tkzInterLC(B,C)(D,G)     \tkzGetPoints{J1}{J2}
 \tkzInterLC[near](A,B)(D,G)  \tkzGetPoints{K1}{K2}
 \tkzInterLC(E,F)(D,G)     \tkzGetPoints{E1}{E2}
 \tkzDrawCircle(D,G)
 \tkzDrawPoints[color=red](I1,J1,K1,K2,E1,E2)
 \tkzDrawLines(A,B B,C A,C I2,J2 E1,E2)
 \tkzDrawPoints[color=blue](A,...,F)
 \tkzDrawPoints[color=red](I2,J2)
 \tkzLabelPoints[left](B,D,E,F)
 \tkzLabelPoints[below left](A,C)
 \tkzLabelPoints[below=4pt](I1,K1,K2,E2)
 \tkzLabelPoints[left](J1,E1)
\end{tikzpicture}

\end{tkzexample}

\subsubsection{Cálculo do raio}
 Com \tkzname{pgfmath} e \tkzcname{pgfmathsetmacro}

A medida do raio pode ser o resultado de um cálculo que não é feito dentro da macro de interseção, mas antes.
Um comprimento pode ser calculado de várias maneiras. É possível, claro,
 usar o módulo \tkzname{pgfmath} e a macro \tkzcname{pgfmathsetmacro}. Em alguns casos, os resultados obtidos não são precisos o suficiente, portanto o seguinte cálculo $0.0002 \div 0.0001$ dá $1.98$ com pgfmath enquanto xfp dará $2$.

Com \tkzname{xfp} e \tkzcname{fpeval}:

\begin{tkzexample}[latex=7cm,small]
  \begin{tikzpicture}
  \tkzDefPoint(2,2){A}
  \tkzDefPoint(5,4){B}
  \tkzDefPoint(4,4){O}
  \pgfmathsetmacro\tkzLen{\fpeval{0.0002/0.0001}}
 % or \edef\tkzLen{\fpeval{0.0002/0.0001}}
  \tkzInterLC[R](A,B)(O, \tkzLen)
  \tkzGetPoints{I}{J}
  \tkzDrawCircle(O,I)
  \tkzDrawPoints[color=blue](A,B)
  \tkzDrawPoints[color=red](I,J)
  \tkzDrawLine(I,J)
\end{tikzpicture}
  \end{tkzexample}


\subsubsection{Opção \code{with nodes}}
\begin{tkzexample}[latex=8cm,small]
\begin{tikzpicture}[scale=.75]
\tkzDefPoints{0/0/A,4/0/B,1/1/D,2/0/E}
\tkzDefTriangle[equilateral](A,B)
\tkzGetPoint{C}
\tkzInterLC[with nodes](D,E)(C,A,B)
\tkzGetPoints{F}{G}
\tkzDrawCircle(C,A)
\tkzDrawPolygon(A,B,C)
\tkzDrawPoints(A,...,G)
\tkzDrawLine(F,G)
\end{tikzpicture}
\end{tkzexample}

\subsection{Interseção de dois círculos  \tkzcname{tkzInterCC}}

O caso mais frequente é aquele de dois círculos definidos por seu centro e um ponto, mas como antes a opção \tkzname{R} permite usar as medidas dos raios.

\begin{NewMacroBox}{tkzInterCC}{\oarg{opções}\parg{$O,A$}\parg{$O',A'$} ou \parg{$O,r$}\parg{$O',r'$} ou   \parg{$O,A,B$} \parg{$O',C,D$}}%
\begin{tabular}{lll}%
opções       & padrão & definição                         \\
\midrule
\TOline{N}   {N}    {$OA$ e $O'A'$ são raios, $O$ e $O'$ são os centros.}
\TOline{R}   {N}    {$r$ e $r'$ são dimensões e medem os raios.}
\TOline{with nodes} {N}  {em (A,A,C)(C,B,F) AC e BF fornecem os raios.}
\TOline{common=pt}  {}   {pt é ponto comum; tkzFirstPoint fornece o outro ponto.}
\bottomrule
\end{tabular}

\medskip
Esta macro define o(s) ponto(s) de interseção $I$ e $J$ dos dois círculos de centros $O$ e $O'$. Se os dois círculos não tiverem um ponto em comum, então a macro termina com um erro que não é tratado. Se os centros são $O$ e $O'$ e as interseções são $A$ e $B$, então os ângulos $\widehat{O,A,O'}$ e $\widehat{O,B,O'}$ estão em direções opostas. \tkzname{tkzFirstPoint} é atribuído ao ponto que forma o ângulo \code{clockwise} (horário).
\end{NewMacroBox}

\begin{NewMacroBox}{tkzTestInterCC}{\parg{$O,A$}\parg{$O',B$}}%
Portanto, os argumentos são dois pares que definem dois círculos com um centro e um ponto no círculo. Se houver uma interseção não vazia entre esses dois círculos, então o teste \tkzcname{iftkzFlagCC} retorna verdadeiro.
\end{NewMacroBox}

\subsubsection{Teste de interseção círculo-círculo}

\begin{tkzexample}[latex=7cm,small]
\begin{tikzpicture}[scale=.75]
  \tkzDefPoints{% x   y   name
                   0    /0    /A,
                   2    /0    /B,
                   4    /0    /I,
                   1    /0    /P}
\tkzDrawCircle(A,B)
\foreach \i in {1,...,3}{%
   \coordinate  (P) at (\i,0);
\tkzDrawCircle[new](I,P)
   \tkzTestInterCC(A,B)(I,P)
    \iftkzFlagCC
    \tkzInterCC(A,B)(I,P)  \tkzGetPoints{a}{b}
    \tkzDrawPoints(a,b)
    \fi}
  \end{tikzpicture}
\end{tkzexample}

\subsubsection{Interseção círculo-círculo com ponto \tkzname{common}.}

\begin{tkzexample}[latex=7cm,small]
\begin{tikzpicture}[scale=.5]
  \tkzDefPoints{0/0/O,5/-1/A,2/2/B}
  \tkzDrawPoints(O,A,B)
  \tkzDrawCircles(O,B A,B)
  \tkzInterCC[common=B](O,B)(A,B)
  \tkzGetFirstPoint{C}
  \tkzDrawPoint(C)
  \tkzLabelPoints[above](O,A,B,C)
\end{tikzpicture}
\end{tkzexample}

\subsubsection{Interseção círculo-círculo ordem dos pontos.}
A ideia é comparar os ângulos formados com o primeiro centro, um ponto resultante e o centro do segundo círculo. O primeiro ponto é aquele que corresponde ao menor ângulo.

Como você pode ver $\widehat{ODB} < \widehat{OBE} $

\begin{tkzexample}[latex=6cm,small]
\begin{tikzpicture}[scale=.5]
  \pgfkeys{/pgf/number format/.cd,fixed relative,
     precision=4}
  \tkzDefPoints{0/0/O,5/-1/A,2/2/B,2/-1/C}
  \tkzDrawPoints(O,A,B)
  \tkzDrawCircles(O,A B,C)
  \tkzInterCC(O,A)(B,C)\tkzGetPoints{D}{E}
  \tkzDrawPoints(C,D,E)
  \tkzLabelPoints(O,A,B,C)
  \tkzLabelPoints[above](D,E) 
  \tkzDrawSegments[cyan](D,O D,B)
  \tkzMarkAngle[red,->,size=1.5](O,D,B)
  \tkzFindAngle(O,D,B)   \tkzGetAngle{an}
  \tkzLabelAngle(O,D,B){$ \pgfmathprintnumber{\an}$}
  \tkzDrawSegments[cyan](E,O E,B)
  \tkzMarkAngle[red,->,size=1.5](O,E,B)  
  \tkzFindAngle(O,E,B)   \tkzGetAngle{an}
  \tkzLabelAngle(O,E,B){$ \pgfmathprintnumber{\an}$}  
\end{tikzpicture}
\end{tkzexample}

\subsubsection{Construção de um triângulo equilátero.}
$\widehat{A,C,B}$ é um ângulo horário
\begin{tkzexample}[latex=7cm,small]
\begin{tikzpicture}[trim left=-1cm,scale=.5]
 \tkzDefPoint(1,1){A}
 \tkzDefPoint(5,1){B}
 \tkzInterCC(A,B)(B,A)\tkzGetPoints{C}{D}
 \tkzDrawPoint[color=black](C)
 \tkzDrawCircles(A,B B,A)
 \tkzCompass[color=red](A,C)
 \tkzCompass[color=red](B,C)
 \tkzDrawPolygon(A,B,C)
 \tkzMarkSegments[mark=s|](A,C B,C)
 \tkzLabelPoints[](A,B)
 \tkzLabelPoint[above](C){$C$}
\end{tikzpicture}
\end{tkzexample}


\subsubsection{Trisseção de segmento}
 A ideia aqui é dividir um segmento com uma régua e um compasso em três segmentos de comprimento igual.

\begin{tkzexample}[latex=7cm,small]
\begin{tikzpicture}[scale=.6]
 \tkzDefPoint(0,0){A}
 \tkzDefPoint(3,2){B}
 \tkzInterCC(A,B)(B,A)          \tkzGetSecondPoint{D}
 \tkzInterCC(D,B)(B,A)          \tkzGetPoints{A}{C}
 \tkzInterCC(D,B)(A,B)          \tkzGetPoints{E}{B}
 \tkzInterLC[common=D](C,D)(E,D)\tkzGetFirstPoint{F}
 \tkzInterLL(A,F)(B,C)          \tkzGetPoint{O}
 \tkzInterLL(O,D)(A,B)          \tkzGetPoint{H}
 \tkzInterLL(O,E)(A,B)          \tkzGetPoint{G}
 \tkzDrawCircles(D,E A,B B,A E,A)
 \tkzDrawSegments[](O,F O,B O,D O,E)
 \tkzDrawPoints(A,...,H)
 \tkzDrawSegments(A,B B,D A,D A,E E,F C,F B,C)
 \tkzMarkSegments[mark=s|](A,G G,H H,B)
\end{tikzpicture}
\end{tkzexample}

\subsubsection{Com a opção \code{with nodes}}
\begin{tkzexample}[latex=6cm,small]
\begin{tikzpicture}[scale=.5]
 \tkzDefPoints{0/0/A,0/5/B,5/0/C}
 \tkzDefPoint(54:5){F}
 \tkzInterCC[with nodes](A,A,C)(C,B,F)
 \tkzGetPoints{a}{e}
 \tkzInterCC(A,C)(a,e) \tkzGetFirstPoint{b}
 \tkzInterCC(A,C)(b,a) \tkzGetFirstPoint{c}
 \tkzInterCC(A,C)(c,b) \tkzGetFirstPoint{d}
 \tkzDrawCircle[new](A,C)
 \tkzDrawPoints(a,b,c,d,e)
 \tkzDrawPolygon(a,b,c,d,e)
 \foreach \vertex/\num in {a/36,b/108,c/180,
                          d/252,e/324}{%
 \tkzDrawPoint(\vertex)
 \tkzLabelPoint[label=\num:$\vertex$](\vertex){}
 \tkzDrawSegment(A,\vertex)
 }
\end{tikzpicture}
\end{tkzexample}

\subsubsection{Combinação de interseções}
\begin{tkzexample}[latex=8cm,small]
\begin{tikzpicture}[scale = .7]
  \tkzDefPoint(2,2){A}
  \tkzDefPoint(0,0){B}
  \tkzDefPoint(-2,2){C}
  \tkzDefPoint(0,4){D}
  \tkzDefPoint(4,2){E}
  \tkzCircumCenter(A,B,C)\tkzGetPoint{O}
  \tkzInterCC[R](O,2)(D,2)\tkzGetPoints{M1}{M2}
  \tkzInterCC(O,A)(D,O) \tkzGetPoints{1}{2}
  \tkzInterLC(A,E)(B,M1)\tkzGetSecondPoint{M3}
  \tkzInterLC(O,C)(M3,D)\tkzGetSecondPoint{L}
  \tkzDrawSegments(C,L)
  \tkzDrawPoints(A,B,C,D,E,M1,M2,M3,O,L)
  \tkzDrawSegments(O,E)
  \tkzDrawSegments[new](C,A D,B)
  \tkzDrawPoint(O)
  \tkzDrawCircles[new](M3,D B,M2 D,O)
  \tkzDrawCircle(O,A)
  \tkzLabelPoints[below right](A,B,C,D,E,M1,M2,M3,O,L)
\end{tikzpicture}
\end{tkzexample}


\subsubsection{Teorema de Altshiller-Court}
  As duas retas que unem os pontos de interseção de dois círculos ortogonais a um ponto em um dos círculos encontram o outro círculo em dois pontos diametralmente opostos. Altshiller p 176


\begin{tkzexample}[vbox,small]
\begin{tikzpicture}
  \tkzDefPoints{0/0/P,5/0/Q,3/2/I}
  \tkzDefCircle[orthogonal from=P](Q,I) 
  \tkzGetFirstPoint{E}
  \tkzDrawCircles(P,E Q,E)
  \tkzInterCC[common=E](P,E)(Q,E) \tkzGetFirstPoint{F}
  \tkzDefPointOnCircle[through =  center P angle 80 point E] 
  \tkzGetPoint{A}
  \tkzInterLC[common=E](A,E)(Q,E)  \tkzGetFirstPoint{C}
  \tkzInterLL(A,F)(C,Q)  \tkzGetPoint{D}
  \tkzDrawLines[add=0 and 1](P,Q)
  \tkzDrawLines[add=0 and 2](A,E)
  \tkzDrawSegments(P,E E,F F,C A,F C,D)
  \tkzDrawPoints(P,Q,E,F,A,C,D)
  \tkzLabelPoints(P,Q,F)
  \tkzLabelPoints[above](E,A)
  \tkzLabelPoints[left](D)
  \tkzLabelPoints[above right](C)
\end{tikzpicture}
\end{tkzexample}


\endinput