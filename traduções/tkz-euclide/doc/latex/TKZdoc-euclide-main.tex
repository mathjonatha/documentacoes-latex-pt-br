% !TEX TS-program = lualatex
% encoding : utf8 
% Documentation of tkz-euclide v5
% Copyright 2025  Alain Matthes
% This work may be distributed and/or modified under the
% conditions of the LaTeX Project Public License, either version 1.3
% of this license or (at your option) any later version.
% The latest version of this license is in
% http://www.latex-project.org/lppl.txt
% and version 1.3 or later is part of all distributions of LaTeX
% version 2005/12/01 or later.
% This work has the LPPL maintenance status “maintained”.
% The Current Maintainer of this work is Alain Matthes.
\PassOptionsToPackage{unicode}{hyperref}

\documentclass[DIV         = 14,
               fontsize    = 10,
               index       = totoc,
               twoside,
               cadre,
               headings    = small,
               ]{tkz-doc}
\gdef\tkznameofpack{tkz-euclide}
\gdef\tkzversionofpack{5.11c}
\gdef\tkzdateofpack{\today}
\gdef\tkznameofdoc{doc-tkz-euclide}
\gdef\tkzversionofdoc{5.11c} 
\gdef\tkzdateofdoc{\today}
\gdef\tkzauthorofpack{Alain Matthes}
\gdef\tkzadressofauthor{}
\gdef\tkznamecollection{AlterMundus}
\gdef\tkzurlauthor{http://altermundus.fr}
\gdef\tkzengine{lualatex}
\gdef\tkzurlauthorcom{http://altermundus.fr}
\nameoffile{\tkznameofpack}
% -- Packages ---------------------------------------------------          
\usepackage[dvipsnames,svgnames]{xcolor}
\usepackage{calc}
\usepackage{tkz-base} 
\usepackage[lua]{tkz-euclide} 
\usepackage{pgfornament} 
\usetikzlibrary{backgrounds}
\usepackage[colorlinks,pdfencoding=auto, psdextra]{hyperref}
\hypersetup{
      linkcolor=Gray,
      citecolor=Green,
      filecolor=Mulberry,
      urlcolor=NavyBlue,
      menucolor=Gray,
      runcolor=Mulberry,
      linkbordercolor=Gray,
      citebordercolor=Green,
      filebordercolor=Mulberry,
      urlbordercolor=NavyBlue,
      menubordercolor=Gray,
      runbordercolor=Mulberry,
      pdfsubject={Euclidean Geometry},
      pdfauthor={\tkzauthorofpack},
      pdftitle={\tkznameofpack},
      pdfcreator={\tkzengine}
}
\usepackage{tkzexample}
\usepackage{fontspec}
\setmainfont{texgyrepagella}[
  Extension = .otf,
  UprightFont = *-regular ,
  ItalicFont  = *-italic  ,
  BoldFont    = *-bold    ,
  BoldItalicFont = *-bolditalic
]
\setsansfont{texgyreheros}[
  Extension = .otf,
  UprightFont = *-regular ,
  ItalicFont  = *-italic  ,
  BoldFont    = *-bold    ,
  BoldItalicFont = *-bolditalic ,
]

\setmonofont{lmmono10-regular.otf}[
  Numbers={Lining,SlashedZero},
  ItalicFont=lmmonoslant10-regular.otf,
  BoldFont=lmmonolt10-bold.otf,
  BoldItalicFont=lmmonolt10-boldoblique.otf,
]
\newfontfamily\ttcondensed{lmmonoltcond10-regular.otf}
%% (La)TeX font-related declarations:
\linespread{1.05}      % Pagella needs more space between lines
\usepackage[math-style=literal,bold-style=literal]{unicode-math}
\usepackage{fourier-otf}
\let\rmfamily\ttfamily
\usepackage{multicol,lscape}
\usepackage[english]{babel}
\usepackage[normalem]{ulem}
\usepackage{multirow,multido,booktabs,cellspace}
\usepackage{shortvrb,fancyvrb,bookmark} 
\usepackage{makeidx}
\makeindex 

%<---------------------------------------------------------------------------> 
% settings styles
\tkzSetUpColors[background=white,text=black]  
\tkzSetUpCompass[color=orange, line width=.2pt,delta=10]
\tkzSetUpArc[color=gray,line width=.2pt]
\tkzSetUpPoint[size=2,color=teal]
\tkzSetUpLine[line width=.2pt,color=teal]
\tkzSetUpStyle[color=orange,line width=.2pt]{new}
\tikzset{every picture/.style={line width=.2pt}}
\tikzset{label angle style/.append style={color=teal,font=\footnotesize}} 
\tikzset{label style/.append style={below,color=teal,font=\scriptsize}}
\tikzset{new/.style={color=orange,line width=.2pt}} 

\AtBeginDocument{\MakeShortVerb{\|}} % link to shortvrb
\def\code{\texttt}
\newcommand*{\ItkzPopt}[2]{\texttt{#2}\index{#1_3@\texttt{#1: options}!\texttt{#2}}}

\begin{document}

\parindent=0pt
\tkzTitleFrame{tkz-euclide v5\\Geometria Euclidiana}
\clearpage

\defoffile{\lefthand\ \tkzname{\tkznameofpack} é um conjunto de macros convenientes para desenhar em um plano (objeto bidimensional fundamental) com um sistema de coordenadas cartesianas. Ele lida com as situações mais clássicas em Geometria Euclidiana. \tkzname{\tkznameofpack} é construído sobre PGF e sua interface front-end associada \TIKZ\ e é um pacote de desenho amigável para (La)TeX. O objetivo é fornecer uma interface de usuário de alto nível para construir gráficos com relativo pouco esforço. O objetivo é guiar os usuários através da construção de diagramas passo a passo, espelhando o processo natural de construção manual o mais próximo possível.\\
A versão 5 do \tkzname{\tkznameofpack} inclui a opção de utilizar Lua para realizar certos cálculos, consulte as seções \code{news} e \code{lua}.\\
Observação: Esta é uma tradução para o português brasileiro da documentação original em inglês.
}

\presentation

\vspace*{1cm}
\lefthand\ Primeiramente, gostaria de agradecer a \textbf{Till Tantau} pelo belo pacote \LaTeX{}, chamado \href{http://sourceforge.net/projects/pgf/}{\TIKZ}.

\vspace*{12pt}
\lefthand\ Agradecimentos: Recebi muitos conselhos valiosos, observações, correções e exemplos de

\tkzimp{Jean-Côme Charpentier}, \tkzimp{Josselin Noirel}, \tkzimp{Manuel Pégourié-Gonnard}, \tkzimp{Franck Pastor}, \tkzimp{David Arnold},

\tkzimp{Ulrike Fischer}, \tkzimp{Stefan Kottwitz}, \tkzimp{Christian Tellechea}, \tkzimp{Nicolas Kisselhoff}, \tkzimp{David Arnold}, \tkzimp{Wolfgang Büchel},

\tkzimp{John Kitzmiller}, \tkzimp{Dimitri Kapetas}, \tkzimp{Gaétan Marris}, \tkzimp{Mark Wibrow}, \tkzimp{Yves Combe}, \tkzimp{Paul Gaborit}, \tkzimp{Laurent Van Deik} e \tkzimp{Muzimuzhi Z}.

\vspace*{12pt}
\lefthand\ Também gostaria de agradecer a Eric Weisstein, criador do MathWorld:
\href{http://mathworld.wolfram.com/about/author.html}{MathWorld}.

\vspace*{12pt}
\lefthand\ Você pode encontrar alguns exemplos no meu site:
\href{http://altermundus.fr}{altermundus.fr}. \hspace{2cm} em construção!

\vfill
Por favor, reporte erros de digitação ou quaisquer outros comentários sobre esta documentação para: \href{mailto:al.ma@mac.com}{\textcolor{blue}{Alain Matthes}}.

Este arquivo pode ser redistribuído e/ou modificado sob os termos da Licença Pública do Projeto \LaTeX{}
distribuída a partir dos arquivos \href{http://www.ctan.org/}{CTAN}.

\clearpage
\tableofcontents
\clearpage
\newpage

\part{Visão geral: uma revisão breve mas abrangente}
\input{TKZdoc-euclide-news.tex}
\newpage
\section{Trabalhando com lua} \label{calc_with_lua}

\subsubsection{Opção \code{lua}} % (fold)
\label{ssub:option_code_lua}

% subsubsection option_code_lua (end)
Você pode agora usar a opção \ItkzPopt{tkz-euclide}{lua} com \tkzname{\tkznameofpack} versão 5.
Você só precisa escrever no seu preâmbulo

 |usepackage[lua]{tkz-euclide}|.
 Obviamente, você precisará compilar com LuaLaTeX. Nada muda para a sintaxe.

Sem a opção, você pode usar \tkzname{\tkznameofpack} com o código proposto da versão 4.25.

Esta versão ainda não está finalizada, embora a documentação que você está lendo atualmente tenha sido compilada com esta opção.

Algumas informações sobre o método usado e os resultados obtidos. Quanto ao método, considerei duas possibilidades. A primeira era simplesmente substituir em todos os lugares que eu pudesse os cálculos feitos por \code{xfp} ou às vezes por \code{lua}. Foi assim que passei de \code{fp} para \code{xfp} e agora para \code{lua}. A segunda possibilidade, mais ambiciosa, teria sido associar a cada ponto um número complexo e fazer os cálculos sobre os complexos com \code{lua}. Infelizmente para isso eu teria que usar bibliotecas das quais não conheço a licença.

Caso contrário, os resultados são bons. Esta documentação com \code{LualaTeX} e \code{xfp} compila em 47s enquanto com \code{lua} leva apenas 30s para 236 páginas.

Outro documento de 61 páginas é compilado em 16s com \code{pdflaTeX} e \code{xfp} e em 13s com  \code{LualaTeX} e \code{xfp}.

Esta documentação compila com |\usepackage{tkz-base}|  e |\usepackage[lua]{tkz-euclide}|, mas não testei todas as interações minuciosamente.

\subsubsection{Opção \code{mini}} % (fold)
\label{ssub:option_code_mini}

Quando você usa \tkzNamePack{tkz-elements} apenas para determinar os pontos em suas figuras, não é necessário carregar todos os módulos do \tkzname{\tkznameofpack}. Neste caso, usando a opção \ItkzPopt{tkz-euclide}{mini} |\usepackage[mini]{tkz-euclide}|, você carregará apenas os módulos necessários para os desenhos.

% subsubsection option_code_mini (end)
\endinput

\section{Instalação}

\tkzname{\tkznameofpack} está no servidor da \tkzname{CTAN}\footnote{\tkzname{\tkznameofpack} faz parte do \NameDist{TeXLive} e \tkzname{tlmgr} permite instalá-los. Este pacote também faz parte do \NameDist{MiKTeX} no \NameSys{Windows}.}. Se você deseja testar uma versão beta, basta colocar os seguintes arquivos em uma pasta texmf que seu sistema possa encontrar.
Você terá que verificar vários pontos:

\begin{itemize}\setlength{\itemsep}{5pt}
\item  A pasta \tkzname{\tkznameofpack} deve estar localizada em um caminho reconhecido pelo \tkzname{latex}.
\item  O  \tkzname{\tkznameofpack} usa \tkzNamePack{xfp}.

\item Você precisa ter \PGF\ instalado no seu computador. \tkzname{\tkznameofpack} usa várias bibliotecas do \TIKZ

 \begin{tabular}{l}
    angles,                         \\
    arrows,                         \\
    arrows.meta,                    \\
    calc,                           \\
    decorations,                    \\
    decorations.markings,           \\
    decorations.pathreplacing,      \\
    decorations.shapes,             \\
    decorations.text,               \\
    decorations.pathmorphing,       \\
    intersections,                  \\
    math,                           \\
    plotmarks,                      \\
    positioning,                    \\
    quotes,                         \\
    shapes.misc,                    \\
    through
\end{tabular}

\item Esta documentação e todos os exemplos foram obtidos com \tkzname{lualatex}, mas \tkzname{pdflatex} ou \tkzname{xelatex} devem ser adequados.
\end{itemize}

\endinput

\section{Apresentação e Visão Geral}

\begin{tkzexample}[latex=5cm,small]
  \begin{tikzpicture}[scale=.25]
  \tkzDefPoints{0/0/A,12/0/B,6/12*sind(60)/C}
  \foreach \density in {20,30,...,240}{%
    \tkzDrawPolygon[fill=teal!\density](A,B,C)
    \pgfnodealias{X}{A}
    \tkzDefPointWith[linear,K=.15](A,B) \tkzGetPoint{A}
    \tkzDefPointWith[linear,K=.15](B,C) \tkzGetPoint{B}
    \tkzDefPointWith[linear,K=.15](C,X) \tkzGetPoint{C}}
  \end{tikzpicture}
\end{tkzexample}

\vspace*{12pt}

\subsection{Por que \tkzname{\tkznameofpack}? }
Meu objetivo inicial era fornecer a outros professores de matemática e a mim mesmo uma ferramenta para criar rapidamente figuras de geometria euclidiana sem investir muito esforço no aprendizado de uma nova linguagem de programação.
Obviamente, \tkzname{\tkznameofpack} é para professores de matemática que usam \LATEX\ e torna possível criar facilmente desenhos corretos por meio do \LATEX.

Pareceu-me que o método mais simples era reproduzir aquele usado para obter construções à mão.
Para descrever uma construção, você deve, é claro, definir os objetos, mas também as ações que você executa. Pareceu-me que uma sintaxe próxima à linguagem dos matemáticos e seus alunos seria mais facilmente compreensível; além disso, também me pareceu que essa sintaxe deveria estar próxima da do \LaTeX.
Os objetos, é claro, são pontos, segmentos, retas, triângulos, polígonos e círculos. Quanto às ações, considerei cinco como suficientes, a saber: definir, criar, desenhar, marcar e rotular.

A sintaxe talvez seja muito verbosa, mas é, acredito, facilmente acessível.
Como resultado, os alunos assim como os professores conseguiram acessar facilmente esta ferramenta.

\subsection{ \tkzname{\TIKZ } vs \tkzname{\tkznameofpack} }

Eu amo programar com \TIKZ, e sem \TIKZ\ nunca teria tido a ideia de criar \tkzname{\tkznameofpack}, mas nunca esqueça que por trás dele está o \TIKZ\ e que sempre é possível inserir código do \TIKZ. \tkzname{\tkznameofpack} não impede você de usar \TIKZ.
Dito isso, não acho que misturar sintaxes seja uma boa coisa.

Não há necessidade de comparar \TIKZ\ e \tkzname{\tkznameofpack}. Este último não é direcionado ao mesmo público que o \TIKZ. O primeiro permite fazer muitas coisas, o segundo faz apenas desenhos de geometria. O primeiro pode fazer tudo que o segundo faz, mas o segundo fará mais facilmente o que você deseja.

O objetivo principal é definir pontos para criar figuras geométricas. \tkzname{\tkznameofpack} permite desenhar os objetos essenciais da geometria euclidiana a partir desses pontos, mas pode ser insuficiente para algumas ações como colorir superfícies. Neste caso você terá que usar \TIKZ\ o que sempre é possível.

Aqui estão algumas comparações entre \tkzname{\TIKZ } e \tkzname{\tkznameofpack} 4. Para isso usarei os exemplos de geometria do PGFManual.
  As duas ferramentas euclidianas mais importantes usadas pelos gregos antigos para construir diferentes formas geométricas e ângulos eram um compasso e uma régua. Minha ideia é permitir que você siga passo a passo uma construção que seria feita à mão (com compasso e régua) da forma mais natural possível.

\subsubsection{Livro I, proposição I  \_Elementos de Euclides\_ }

\begin{tikzpicture}
\node [mybox,title={Livro I, proposição I  \_Elementos de Euclides\_}] (box){%
    \begin{minipage}{0.90\textwidth}
{\emph{Construir um triângulo equilátero sobre uma reta finita dada.}
}
    \end{minipage}
};
\end{tikzpicture}%


Explicação:

O quarto tutorial do \emph{PgfManual} é sobre construções geométricas. \emph{T. Tantau} propõe obter o desenho com sua bela ferramenta Ti\emph{k}Z. Aqui proponho a mesma construção com \emph{tkz-elements}. A cor do código Ti\emph{k}Z é green!50!black e a do \emph{tkz-elements} é red.

\medskip

\vbox{\color{green!50!black} |\usepackage{tikz}|\\
|\usetikzlibrary{calc,intersections,through,backgrounds}|}

\medskip
\vbox{\color{red} |\usepackage{tkz-euclide}|}

\medskip
Como obter a reta AB? Para obter esta reta, usamos dois pontos fixos.\\

\medskip
\vbox{\color{green!50!black} 
|\coordinate [label=left:$A$] (A) at (0,0);|\\
|\coordinate [label=right:$B$] (B) at (1.25,0.25);|\\
|\draw (A) -- (B);|}

\medskip
\vbox{\color{red}
|\tkzDefPoint(0,0){A}|\\
|\tkzDefPoint(1.25,0.25){B}|\\
|\tkzDrawSegment(A,B)|\\
|\tkzLabelPoint[left](A){$A$}|\\
|\tkzLabelPoint[right](B){$B$}|}

Queremos desenhar um círculo ao redor dos pontos $A$ e $B$ cujo raio é dado pelo comprimento da reta AB. 
\medskip

\vbox{\color{green!50!black}
|\draw let \p1 = ($ (B) - (A) $),|\\
|\n2 = {veclen(\x1,\y1)} in|\\
|          (A) circle (\n2)|\\
|          (B) circle (\n2);|}

\medskip
\vbox{\color{red} 
|\tkzDrawCircles(A,B B,A)|
}

A interseção dos círculos $\mathcal{D}$ e $\mathcal{E}$

\medskip

\vbox{\color{green!50!black} 
|draw [name path=A--B] (A) -- (B);|\\
|node (D) [name path=D,draw,circle through=(B),label=left:$D$] at (A) {}; |\\
|node (E) [name path=E,draw,circle through=(A),label=right:$E$] at (B) {};|\\
|path [name intersections={of=D and E, by={[label=above:$C$]C,[label=below:$C'$]C'}}]; |\\
|draw [name path=C--C',red] (C) -- (C');|\\
|path [name intersections={of=A--B and C--C',by=F}];|\\
|node [fill=red,inner sep=1pt,label=-45:$F$] at (F) {};|\\}

\medskip
\vbox{\color{red} |\tkzInterCC(A,B)(B,A) \tkzGetPoints{C}{X}|\\}


Como desenhar pontos:

\medskip
\vbox{\color{green!50!black} |\foreach \point in {A,B,C}|\\
|\fill [black,opacity=.5] (\point) circle (2pt);|\\}

\medskip
\vbox{\color{red}| \tkzDrawPoints[fill=gray,opacity=.5](A,B,C)|\\}

\subsubsection{Código completo com \pkg{tkz-euclide}}

Precisamos definir cores 

|\colorlet{input}{red!80!black} |\\
|\colorlet{output}{red!70!black}|\\
|\colorlet{triangle}{green!50!black!40}  |

\begin{tkzexample}[vbox,small]
  \colorlet{input}{red!80!black} 
  \colorlet{output}{red!70!black}
  \colorlet{triangle}{green!50!black!40}
  \begin{tikzpicture}[scale=1.25,thick,help lines/.style={thin,draw=black!50}]
  \tkzDefPoint(0,0){A}     
  \tkzDefPoint(1.25+rand(),0.25+rand()){B}      
  \tkzInterCC(A,B)(B,A) \tkzGetPoints{C}{X}

  \tkzFillPolygon[triangle,opacity=.5](A,B,C)
  \tkzDrawSegment[input](A,B) 
  \tkzDrawSegments[red](A,C B,C)  
  \tkzDrawCircles[help lines](A,B B,A)
  \tkzDrawPoints[fill=gray,opacity=.5](A,B,C)
  
  \tkzLabelPoints(A,B)
  \tkzLabelCircle[below=12pt](A,B)(180){$\mathcal{D}$}
  \tkzLabelCircle[above=12pt](B,A)(180){$\mathcal{E}$}
  \tkzLabelPoint[above,red](C){$C$}
      
  \end{tikzpicture}
\end{tkzexample}

\subsubsection{Livro I, Proposição II  \_Elementos de Euclides\_}

\begin{tikzpicture}
\node [mybox,title={Livro I, Proposição II  \_Elementos de Euclides\_}] (box){%
\begin{minipage}{0.90\textwidth}
  {\emph{Colocar uma reta igual a uma reta dada com uma extremidade em um ponto dado.}}
\end{minipage}
};
\end{tikzpicture}%

Explicação

Na primeira parte, precisamos encontrar o ponto médio da reta $AB$. Com \TIKZ\ podemos usar a biblioteca calc

\medskip
\vbox{\color{green!50!black} |\coordinate [label=left:$A$] (A) at (0,0);|\\
|\coordinate [label=right:$B$] (B) at (1.25,0.25);|\\
|\draw (A) -- (B);|\\
|\node [fill=red,inner sep=1pt,label=below:$X$] (X) at ($ (A)!.5!(B) $) {};|\\}

Com \pkg{tkz-euclide} temos uma macro \tkzcname{tkzDefMidPoint}, obtemos o ponto X com \tkzcname{tkzGetPoint}, mas não precisamos deste ponto para obter a próxima etapa.


\medskip
\vbox{\red |\tkzDefPoints{0/0/A,0.75/0.25/B,1/1.5/C}|\\  
|\tkzDefMidPoint(A,B) \tkzGetPoint{X}|}

\medskip
Em seguida, precisamos construir um triângulo equilátero. É fácil com \pkg{tkz-euclide}. Com TikZ você precisa de algum esforço porque precisa usar o ponto médio $X$ para obter o ponto $D$ com cálculo trigonométrico.

\medskip
\vbox{\color{green!50!black}
|\node [fill=red,inner sep=1pt,label=below:$X$] (X) at ($ (A)!.5!(B) $) {}; | \\
|\node [fill=red,inner sep=1pt,label=above:$D$] (D) at                      |  \\
|($ (X) ! {sin(60)*2} ! 90:(B) $) {};                                       |  \\
|\draw (A) -- (D) -- (B);                                                   |  \\
}                                                                           

\medskip
\vbox{\color{red} |\tkzDefTriangle[equilateral](A,B) \tkzGetPoint{D}|}

Podemos desenhar o triângulo no final da figura com

\medskip
\vbox{\color{red} |\tkzDrawPolygon{A,B,C}|}

\medskip
Sabemos como desenhar o círculo $\mathcal{H}$ ao redor de $B$ passando por $C$ e como colocar os pontos $E$ e $F$

\medskip
\vbox{\color{green!50!black} 
|\node (H) [label=135:$H$,draw,circle through=(C)] at (B) {};|          \\
|\draw (D) -- ($ (D) ! 3.5 ! (B) $) coordinate [label=below:$F$] (F);|  \\
|\draw (D) -- ($ (D) ! 2.5 ! (A) $) coordinate [label=below:$E$] (E);|} 

\medskip

\vbox{\color{red} |\tkzDrawCircle(B,C)|\\
|\tkzDrawLines[add=0 and 2](D,A D,B)|}

\medskip
Podemos colocar os pontos $E$ e $F$ no final da figura. Não precisamos deles agora.

Interseção de uma Reta e um Círculo: aqui procuramos a interseção do círculo ao redor de $B$ passando por $C$ e a reta $DB$.
A reta infinita $DB$ intercepta o círculo, mas com \TIKZ\ precisamos estender as retas $DB$ e isso pode ser feito usando cálculos parciais. Obtemos o ponto $F$ e $BF$ ou $DF$ intercepta o círculo

\medskip
\vbox{\color{green!50!black}| \node (H) [label=135:$H$,draw,circle through=(C)] at (B) {}; |  \\
|\path let \p1 = ($ (B) - (C) $) in|                                     \\
|  coordinate [label=left:$G$] (G) at ($ (B) ! veclen(\x1,\y1) ! (F) $); |  \\
|\fill[red,opacity=.5] (G) circle (2pt);|}                                

\medskip
Assim como a interseção de dois círculos, é fácil encontrar a interseção de uma reta e um círculo com \pkg{tkz-euclide}. Não precisamos de $F$ 

\medskip
\vbox{\color{red} | \tkzInterLC(B,D)(B,C)\tkzGetFirstPoint{G}|}

\medskip
Não há mais dificuldades. Aqui está o código final com algumas simplificações.
Desenhamos o círculo $\mathcal{K}$ com centro $D$ e passando por $G$. Ele intercepta a reta $AD$ no ponto $L$. $AL = BC$.

\vbox{\color{red} | \tkzDrawCircle(D,G)|}
\vbox{\color{red} | \tkzInterLC(D,A)(D,G)\tkzGetSecondPoint{L}|}

\begin{tkzexample}[latex=7cm,small]
\begin{tikzpicture}[scale=1.5]
\tkzDefPoint(0,0){A}
\tkzDefPoint(0.75,0.25){B}  
\tkzDefPoint(1,1.5){C} 
\tkzDefTriangle[equilateral](A,B)\tkzGetPoint{D}
\tkzInterLC[near](D,B)(B,C)      \tkzGetSecondPoint{G}
\tkzInterLC[near](A,D)(D,G)      \tkzGetFirstPoint{L}
\tkzDrawCircles(B,C D,G)
\tkzDrawLines[add=0 and 2](D,A D,B)
\tkzDrawSegment(A,B) 
\tkzDrawSegments[red](A,L B,C) 
\tkzDrawPoints[red](D,L,G)
\tkzDrawPoints[fill=gray](A,B,C)
\tkzLabelPoints[left,red](A)
\tkzLabelPoints[below right,red](L)
\tkzLabelCircle[above](B,C)(20){$\mathcal{(H)}$}
\tkzLabelPoints[above left](D)
\tkzLabelPoints[above](G)
\tkzLabelPoints[above,red](C)
\tkzLabelPoints[right,red](B)
\tkzLabelCircle[below](D,G)(-90){$\mathcal{(K)}$}
\end{tikzpicture}
\end{tkzexample}

\subsection{\tkzname{\tkznameofpack\ 4}   vs \tkzname{\tkznameofpack\ 3}}

Agora não sou mais professor de Matemática e passo apenas algumas horas estudando geometria. Quis evitar múltiplas complicações tentando tornar \tkzname{tkz-euclide} independente de \tkzname{tkz-base}. Assim nasceu \tkzname{\tkznameofpack} 4. Este último é uma versão simplificada de seu predecessor. As macros do \tkzname{tkz-euclide 3} foram mantidas. A unidade agora é \tkzname{cm}. Se você precisar de algumas macros do \tkzname{tkz-base}, talvez precise usar \tkzcname{tkzInit}.

\subsection{\tkzname{\tkznameofpack\ 5}   vs \tkzname{\tkznameofpack\ 4}}

Nada muda para o usuário. A compilação deve ser realizada usando o motor LuaLaTeX, e os resultados são mais precisos e obtidos mais rapidamente. Basta carregar \tkzname{\tkznameofpack} assim |\usepackage[lua]{tkz-euclide}|.

\subsection{Como usar o pacote \tkzname{\tkznameofpack}?}
\subsubsection{Vejamos um exemplo clássico}
Para mostrar o caminho certo, veremos como construir um triângulo equilátero. Várias possibilidades estão abertas para nós, vamos seguir os passos de Euclides.

\begin{itemize}
\item   Primeiro de tudo, você deve usar uma classe de documento. A melhor escolha para testar seu código é criar uma única figura com a classe \tkzname{standalone}\index{standalone}.
\begin{verbatim}
\documentclass{standalone}
\end{verbatim}
\item Em seguida, carregue o pacote \tkzname{\tkznameofpack}:
\begin{verbatim}
\usepackage{tkz-euclide} or \usepackage[lua]{tkz-euclide}
\end{verbatim}

 Você não precisa carregar \TIKZ\ porque o pacote \tkzname{\tkznameofpack} funciona sobre o TikZ e o carrega.

 \item Inicie o documento e abra um ambiente de figura TikZ:
\begin{verbatim}
\begin{document}
\begin{tikzpicture}
\end{verbatim}

\item Agora definimos dois pontos fixos:
\begin{verbatim}
\tkzDefPoint(0,0){A}
\tkzDefPoint(5,2){B}
\end{verbatim}

\item Dois pontos definem dois círculos, vamos usar esses círculos:

 círculo com centro $A$ passando por $B$ e círculo com centro $B$ passando por $A$. Esses dois círculos têm dois pontos em comum.
\begin{verbatim}
\tkzInterCC(A,B)(B,A)
\end{verbatim}
Podemos obter os pontos de interseção com
\begin{verbatim}
\tkzGetPoints{C}{D}
\end{verbatim}

\item Todos os pontos necessários são obtidos, podemos passar para as etapas finais, incluindo os desenhos.
\begin{verbatim}
\tkzDrawCircles[gray,dashed](A,B B,A)
\tkzDrawPolygon(A,B,C)% The triangle
\end{verbatim}
\item Desenhe todos os pontos $A$, $B$, $C$ e $D$:
\begin{verbatim}
\tkzDrawPoints(A,...,D)
\end{verbatim}

\item A etapa final, imprimimos rótulos aos pontos e usamos opções para posicionamento:\\
\begin{verbatim}
\tkzLabelSegments[swap](A,B){$c$}
\tkzLabelPoints(A,B,D)
\tkzLabelPoints[above](C)
\end{verbatim}
\item Finalmente fechamos ambos os ambientes
\begin{verbatim}
\end{tikzpicture}
\end{document}
\end{verbatim}

\item O código completo

\begin{tkzexample}[latex=8cm,small]
 \begin{tikzpicture}[scale=.5]
   % fixed points
  \tkzDefPoint(0,0){A}
  \tkzDefPoint(5,2){B}
  % calculus
  \tkzInterCC(A,B)(B,A)
  \tkzGetPoints{C}{D}
  % drawings
  \tkzDrawCircles(A,B B,A)
  \tkzDrawPolygon(A,B,C)
  \tkzDrawPoints(A,...,D)
  % marking
  \tkzMarkSegments[mark=s||](A,B B,C C,A)
  % labelling
  \tkzLabelSegments[swap](A,B){$c$}
  \tkzLabelPoints(A,B,D)
  \tkzLabelPoints[above](C)
\end{tikzpicture}
\end{tkzexample}

 \end{itemize}

\subsubsection{ Parte I: triângulo dourado}
\begin{center}
\begin{tikzpicture}
  
\tkzDefPoint(0,0){C} % possible \tkzDefPoint[label=below:$C$](0,0){C} but don't do this
\tkzDefPoint(2,6){B}
% We get D and E with a rotation
\tkzDefPointBy[rotation= center B angle 36](C) \tkzGetPoint{D} 
\tkzDefPointBy[rotation= center B angle 72](C) \tkzGetPoint{E} 
% Toget A we use an intersection of lines
\tkzInterLL(B,E)(C,D) \tkzGetPoint{A}
\tkzInterLL(C,E)(B,D) \tkzGetPoint{H}

% angles 
\tkzMarkAngles[size=2](C,B,D E,A,D) %this is to draw the arcs
\tkzLabelAngles[pos=1.5](C,B,D E,A,D){$\alpha$}
\tkzMarkRightAngle(B,H,C)
\tkzDrawPoints(A,...,E)

% drawing
\tkzDrawArc[delta=10](B,C)(E)
\tkzDrawPolygon(C,B,D)
\tkzDrawSegments(D,A B,A C,E)

% Label only now
\tkzLabelPoints[below left](C,A)
\tkzLabelPoints[below right](D)
\tkzLabelPoints[above](B,E)
\end{tikzpicture}
\end{center}

Vamos analisar a figura
\begin{enumerate}
  \item $CBD$ e $DBE$ são triângulos isósceles;

  \item $BC=BE$ e $(BD)$ é uma bissetriz do ângulo $CBE$;

  \item Disso deduzimos que os ângulos $CBD$ e $DBE$ são iguais e têm a mesma medida $\alpha$
   \[\widehat{BAC} +\widehat{ABC} + \widehat{BCA}=180^\circ \ \text{no triângulo}\ BAC \]
   \[3\alpha + \widehat{BCA}=180^\circ\  \text{no triângulo}\ CBD\]
   então
     \[\alpha + 2\widehat{BCA}=180^\circ \]
   ou
     \[\widehat{BCA}=90^\circ -\alpha/2 \]

    \item  Finalmente   \[\widehat{CBD}=\alpha=36^\circ \]
     o triângulo $CBD$ é um triângulo \code{golden} (dourado).
\end{enumerate}

\vspace*{24pt}
Como construir um triângulo dourado ou um ângulo de $36^\circ$?

\begin{enumerate}
  \item Colocamos os pontos fixos $C$ e $D$. |\tkzDefPoint(0,0){C}| e |\tkzDefPoint(4,0){D}|;
  \item  Construímos um quadrado $CDef$ e construímos o ponto médio $m$ de $[Cf]$;

  Podemos fazer tudo isso com um compasso e uma régua;
  \item Em seguida, traçamos um arco com centro $m$ passando por $e$. Este arco cruza a reta $(Cf)$ em $n$;
  \item Agora os dois arcos com centro $C$ e $D$ e raio $Cn$ definem o ponto $B$.
\end{enumerate}

\begin{tkzexample}[latex=7cm,small]
\begin{tikzpicture}
  \tkzDefPoint(0,0){C}
  \tkzDefPoint(4,0){D}
  \tkzDefSquare(C,D)                     
  \tkzGetPoints{e}{f}
  \tkzDefMidPoint(C,f)                   
  \tkzGetPoint{m}
  \tkzInterLC(C,f)(m,e)                  
  \tkzGetSecondPoint{n}
  \tkzInterCC[with nodes](C,C,n)(D,C,n) 
  \tkzGetFirstPoint{B}
  \tkzDrawSegment[brown,dashed](f,n)
  \pgfinterruptboundingbox% from tikz
  \tkzDrawPolygon[brown,dashed](C,D,e,f)
  \tkzDrawArc[brown,dashed](m,e)(n)
  \tkzCompass[brown,dashed,delta=20](C,B)
  \tkzCompass[brown,dashed,delta=20](D,B)
  \endpgfinterruptboundingbox 
  \tkzDrawPolygon(B,...,D)
  \tkzDrawPoints(B,C,D,e,f,m,n)
  \tkzLabelPoints[above](B)
  \tkzLabelPoints[left](f,m,n)
  \tkzLabelPoints(C,D)
  \tkzLabelPoints[right](e)
\end{tikzpicture}
\end{tkzexample}


Depois de construir o triângulo dourado $BCD$, construímos o ponto $A$ observando que $BD=DA$. Então obtemos o ponto $E$ e finalmente o ponto $F$. Isso é feito com interseções de objetos já definidos (reta e círculo).


\subsubsection{Parte II: dois outros métodos com triângulos dourado e de euclides}

\tkzname{\tkznameofpack} sabe como definir um triângulo \code{golden} (dourado) ou \code{euclide} (de euclides). Podemos definir $BCD$ e $BCA$ como triângulos dourados.


  \begin{center}
    \begin{tkzexample}[code only,small]
      \begin{tikzpicture}
        \tkzDefPoint(0,0){C}
        \tkzDefPoint(4,0){D}
        \tkzDefTriangle[golden](C,D)
        \tkzGetPoint{B}
        \tkzDefTriangle[golden](B,C)
        \tkzGetPoint{A}
        \tkzInterLC[near](A,B)(B,D) \tkzGetFirstPoint{E}
        \tkzInterLL(B,D)(C,E) \tkzGetPoint{F}
        \tkzDrawPoints(C,D,B)
        \tkzDrawPolygon(B,...,D)  
        \tkzDrawPolygon(B,C,D)
        \tkzDrawSegments(D,A A,B C,E)
        \tkzDrawArc[delta=10](B,C)(E)
        \tkzDrawPoints(A,...,F) 
        \tkzMarkRightAngle(B,F,C)  
        \tkzMarkAngles(C,B,D E,A,D)
        \tkzLabelAngles[pos=1.5](C,B,D E,A,D){$\alpha$} 
        \tkzLabelPoints[below](A,C,D,E)
        \tkzLabelPoints[above right](B,F)
      \end{tikzpicture} 
    \end{tkzexample}
  \end{center}

Aqui está um método final que usa rotações:  

\begin{center}
  \begin{tkzexample}[code only,small]
  \begin{tikzpicture} 
  \tkzDefPoint(0,0){C} % possible 
  % \tkzDefPoint[label=below:$C$](0,0){C} 
  % but don't do this
  \tkzDefPoint(2,6){B}
  % We get D and E with a rotation
  \tkzDefPointBy[rotation= center B angle 36](C) \tkzGetPoint{D} 
  \tkzDefPointBy[rotation= center B angle 72](C) \tkzGetPoint{E} 
  % To get A we use an intersection of lines
  \tkzInterLL(B,E)(C,D) \tkzGetPoint{A}
  \tkzInterLL(C,E)(B,D) \tkzGetPoint{H}
  % drawing
  \tkzDrawArc[delta=10](B,C)(E)
  \tkzDrawPolygon(C,B,D)
  \tkzDrawSegments(D,A B,A C,E)
  % angles 
  \tkzMarkAngles(C,B,D E,A,D) %this is to draw the arcs
  \tkzLabelAngles[pos=1.5](C,B,D E,A,D){$\alpha$}
  \tkzMarkRightAngle(B,H,C)
  \tkzDrawPoints(A,...,E)
  % Label only now
  \tkzLabelPoints[below left](C,A)
  \tkzLabelPoints[below right](D)
  \tkzLabelPoints[above](B,E)
  \end{tikzpicture}
  \end{tkzexample}
\end{center}


\subsubsection{Exemplo completo mas mínimo}


Sendo escolhida uma unidade de comprimento, o exemplo mostra como obter um segmento de comprimento $\sqrt{a}$ a partir de um segmento de comprimento $a$, usando uma régua e um compasso.

$IB=a$, $AI=1$

\vspace{12pt}
\hypertarget{firstex}{}
\begin{tkzexample}[vbox,small]
\begin{tikzpicture}[scale=1,ra/.style={fill=gray!20}]
   % fixed points
   \tkzDefPoint(0,0){A}
   \tkzDefPoint(1,0){I}
   % calculation
   \tkzDefPointBy[homothety=center A ratio  10 ](I) \tkzGetPoint{B}  
   \tkzDefMidPoint(A,B)              \tkzGetPoint{M}
   \tkzDefPointWith[orthogonal](I,M) \tkzGetPoint{H}
   \tkzInterLC(I,H)(M,B)             \tkzGetFirstPoint{C}
   \tkzDrawSegment[style=purple](I,C)
   \tkzDrawArc(M,B)(A)
   \tkzDrawSegment[dim={$1$,-16pt,}](A,I)
   \tkzDrawSegment[dim={$(a-1)/2$,-10pt,}](I,M)
   \tkzDrawSegment[dim={$(a+1)/2$,-16pt,}](M,B)   
   \tkzMarkRightAngle[ra](A,I,C)
   \tkzDrawPoints(I,A,B,C,M)  
   \tkzLabelPoint[left](A){$A(0,0)$} 
   \tkzLabelPoints[above right](I,M)
   \tkzLabelPoints[above left](C)
   \tkzLabelPoint[right](B){$B(10,0)$}
   \tkzLabelSegment[right=4pt](I,C){$\sqrt{a^2}=a \ (a>0)$}
\end{tikzpicture}
\end{tkzexample}

\emph{Comentários}

\begin{itemize}
\item O Preâmbulo


 Vamos primeiro olhar para o preâmbulo. Se você precisar, deve carregar \tkzname{xcolor} antes de \tkzname{tkz-euclide}, ou seja, antes de \TIKZ. \TIKZ\ pode causar problemas com os caracteres ativos, mas...
 fornece uma biblioteca em sua versão mais recente que deve resolver esses problemas \NameLib{babel}.
 
\begin{tkzltxexample}[]
\documentclass{standalone} % or another class
   % \usepackage{xcolor} % before tikz or tkz-euclide if necessary
\usepackage{tkz-euclide} % no need to load TikZ
   % \usetkzobj{all}  is no longer necessary 
   % \usetikzlibrary{babel}  if there are problems with the active characters
\end{tkzltxexample}

O código a seguir consiste em várias partes:

   \item  Definição de pontos fixos: a primeira parte inclui as definições dos pontos necessários para a construção, estes são os pontos fixos. As macros \tkzcname{tkzInit} e \tkzcname{tkzClip} na maioria dos casos não são necessárias.

\begin{tkzltxexample}[]
  \tkzDefPoint(0,0){A}
  \tkzDefPoint(1,0){I}
\end{tkzltxexample}
 
  \item A segunda parte é dedicada à criação de novos pontos a partir dos pontos fixos;
  um ponto $B$ é colocado a $10$~cm de $A$. O meio de $[AB]$ é definido por $M$ e então a reta ortogonal à reta $(AB)$ é procurada no ponto $I$. Então procuramos a interseção desta reta com o semicírculo de centro $M$ passando por $A$.  
  
\begin{tkzltxexample}[]
   \tkzDefPointBy[homothety=center A ratio  10 ](I)
      \tkzGetPoint{B}
   \tkzDefMidPoint(A,B)
      \tkzGetPoint{M}
   \tkzDefPointWith[orthogonal](I,M)
      \tkzGetPoint{H}
   \tkzInterLC(I,H)(M,B)             
   \tkzGetSecondPoint{C}
 \end{tkzltxexample}  
     

 \item A terceira inclui os diferentes desenhos;
 \begin{tkzltxexample}[]
   \tkzDrawSegment[style=purple](I,H)
   \tkzDrawPoints(O,I,A,B,M)
   \tkzDrawArc(M,A)(O)
   \tkzDrawSegment[dim={$1$,-16pt,}](A,I)
   \tkzDrawSegment[dim={$a/2$,-10pt,}](I,M)
   \tkzDrawSegment[dim={$a/2$,-16pt,}](M,B)
 \end{tkzltxexample}
 
\item  Marcação: a quarta é dedicada à marcação;


\begin{tkzltxexample}[]
 \tkzMarkRightAngle[ra](A,I,C)
 \end{tkzltxexample}
 
 \item Rotulagem: esta última trata apenas da colocação de rótulos.
\begin{tkzltxexample}[]
   \tkzLabelPoint[left](A){$A(0,0)$} 
   \tkzLabelPoint[right](B){$B(10,0)$}
   \tkzLabelSegment[right=4pt](I,C){$\sqrt{a^2}=a \ (a>0)$}
\end{tkzltxexample}

\end{itemize}

\endinput
\input{TKZdoc-euclide-elements.tex}
\section{Sobre esta documentação e os exemplos}

É obtida compilando com \code{lualatex}. Eu uso uma classe \tkzname{doc.cls} baseada em \tkzname{scrartcl}.

Abaixo a lista de estilos usados na documentação. Para entender como usar os estilos, consulte a seção \ref{custom}

|\tkzSetUpColors[background=white,text=black]  |

|\tkzSetUpCompass[color=orange, line width=.2pt,delta=10]|

|\tkzSetUpArc[color=gray,line width=.2pt]|

|\tkzSetUpPoint[size=2,color=teal]|

|\tkzSetUpLine[line width=.2pt,color=teal]|

|\tkzSetUpStyle[color=orange,line width=.2pt]{new}|

|\tikzset{every picture/.style={line width=.2pt}}|

|\tikzset{label angle style/.append style={color=teal,font=\footnotesize}}|


|\tikzset{label style/.append style={below,color=teal,font=\scriptsize}}|

Alguns exemplos usam estilos predefinidos como

|\tikzset{new/.style={color=orange,line width=.2pt}}  |

\part{Configuração}
\section{Primeiro passo: pontos fixos}

O primeiro passo em uma construção geométrica é definir os pontos fixos a partir dos quais a figura será construída.

A ideia geral é evitar manipular coordenadas e preferir usar as referências dos pontos fixados no primeiro passo ou obtidos usando as ferramentas fornecidas pelo pacote. Mesmo que seja possível, penso que é uma má ideia trabalhar diretamente com coordenadas. Preferível é usar pontos nomeados.

\tkzname{\tkznameofpack} usa macros e vocabulário específico para construção geométrica. É claro que é possível usar as ferramentas do \TIKZ\ mas parece-me mais lógico não misturar as diferentes sintaxes.

Um ponto em \tkzname{\tkznameofpack} é um \code{node} particular para o \TIKZ. Na próxima seção veremos como definir pontos usando coordenadas. O estilo dos pontos (cor e forma) não será discutido. Você encontrará algumas indicações em alguns exemplos; para mais informações você pode ler a seção seguinte \ref{custom}.


\section{Definição de um ponto: \tkzcname{tkzDefPoint} ou \tkzcname{tkzDefPoints}}

Os pontos podem ser especificados de qualquer uma das seguintes formas:
\begin{itemize}
\item Coordenadas cartesianas;
\item Coordenadas polares;
\item Pontos nomeados;
\item Pontos relativos.
\end{itemize}

Um ponto é definido se ele tem um nome vinculado a um par único de números decimais.
 Seja $(x,y)$ ou $(a:d)$ isto é ($x$ abscissa, $y$ ordenada) ou ($a$ ângulo: $d$ distância).
 Isso é possível porque o plano foi fornecido com um sistema de coordenadas cartesianas ortonormado. Os eixos de trabalho são (orto)normados com unidade igual a $1$~cm.

 A coordenada cartesiana $(a,b)$ refere-se ao
 ponto $a$ centímetros na direção $x$ e $b$ centímetros na
 direção $y$.

 Um ponto em coordenadas polares requer um ângulo $\alpha$, em graus,
 e uma distância $d$ da origem com uma unidade
 dimensional, por padrão é o \texttt{cm}.

 A macro \tkzNameMacro{tkzDefPoint} é usada para definir um ponto atribuindo coordenadas a ele. Esta macro é baseada em

 \tkzNameMacro{coordinate}, uma macro do \TIKZ. Ela pode usar opções específicas do \TIKZ\ como \tkzname{shift}. Se cálculos forem necessários, então o pacote \tkzNamePack{xfp} é escolhido. Podemos usar coordenadas cartesianas ou polares.

\begin{minipage}[t]{0.48\textwidth}
 Coordenadas cartesianas
\begin{tkzexample}[code only,small]
\begin{tikzpicture}[scale=1]
  \tkzInit[xmax=5,ymax=5]
  % necessário para limitar
  % o tamanho dos eixos
  \tkzDrawX[>=latex]
  \tkzDrawY[>=latex]
  \tkzDefPoints{0/0/O,1/0/I,0/1/J}
  \tkzDefPoint(3,4){A}
  \tkzDrawPoints(O,A)
  \tkzLabelPoint[above](A){$A_1(x_1,y_1)$}
  \tkzShowPointCoord[xlabel=$x_1$,
                     ylabel=$y_1$](A)
  \tkzLabelPoints(O,I)
  \tkzLabelPoints[left](J)
  \tkzDrawPoints[shape=cross](I,J)
\end{tikzpicture}
\end{tkzexample}%
\end{minipage}
\begin{minipage}[t]{0.45\textwidth}
 Coordenadas polares
\begin{tkzexample}[code only,small]
\begin{tikzpicture}[scale=1]
  \tkzInit[xmax=5,ymax=5]
  \tkzDrawX[>=latex]
  \tkzDrawY[>=latex]
  \tkzDefPoints{0/0/O,1/0/I,0/1/J}
  \tkzDefPoint(40:4){P}
  \tkzDrawSegment[dim={$d$,
                 16pt,above=6pt}](O,P)
  \tkzDrawPoints(O,P)
  \tkzMarkAngle[mark=none,->](I,O,P)
  \tkzFillAngle[opacity=.5](I,O,P)
  \tkzLabelAngle[pos=1.25](I,O,P){%
                              $\alpha$}
  \tkzLabelPoint[right](P){$P(\alpha:d)$}
  \tkzDrawPoints[shape=cross](I,J)
  \tkzLabelPoints(O,I)
  \tkzLabelPoints[left](J)
\end{tikzpicture}
\end{tkzexample}
\end{minipage}%

\begin{minipage}[b]{0.45\textwidth}
\begin{tikzpicture}[scale=1]
  \tkzInit[xmax=5,ymax=5]
  \tkzDrawX[>=latex]
  \tkzDrawY[>=latex]
  \tkzDefPoints{0/0/O,1/0/I,0/1/J}
  \tkzDefPoint(3,4){A}
  \tkzDrawPoints(O,A)
  \tkzLabelPoint[above](A){$A_1 (x_1,y_1)$}
  \tkzShowPointCoord[xlabel=$x_1$,ylabel=$y_1$](A)
  \tkzLabelPoints(O,I)
  \tkzLabelPoints[left](J)
  \tkzDrawPoints[shape=cross](I,J)
\end{tikzpicture}
\end{minipage}
\begin{minipage}[b]{0.45\textwidth}
\begin{tikzpicture}[,scale=1]
  \tkzInit[xmax=5,ymax=5]
  \tkzDrawX[>=latex]
  \tkzDrawY[>=latex]
  \tkzDefPoints{0/0/O,1/0/I,0/1/J}
  \tkzDefPoint(40:4){P}
  \tkzDrawSegment[dim={$d$,
                 16pt,above=6pt}](O,P)
  \tkzDrawPoints(O,P)
  \tkzMarkAngle[mark=none,->](I,O,P)
  \tkzFillAngle[opacity=.5](I,O,P)
  \tkzLabelAngle[pos=1.25](I,O,P){$\alpha$}
  \tkzLabelPoint[right](P){$P  (\alpha : d )$}
  \tkzDrawPoints[shape=cross](I,J)
  \tkzLabelPoints(O,I)
  \tkzLabelPoints[left](J)
\end{tikzpicture}
\end{minipage}%

\subsection{Definindo um ponto nomeado \tkzcname{tkzDefPoint}}

\begin{NewMacroBox}{tkzDefPoint}{\oarg{opções locais}\parg{$x,y$}\marg{ref} ou \parg{$\alpha$:$d$}\marg{ref}}%
\begin{tabular}{lll}%
argumentos &  padrão & definição  \\
\midrule
\TAline{($x,y$)}{sem padrão}{$x$ e $y$ são duas dimensões, por padrão em cm.}
\TAline{($\alpha$:$d$)}{sem padrão}{$\alpha$ é um ângulo em graus, $d$ é uma dimensão}
\TAline{\{ref\}}{sem padrão}{Referência atribuída ao ponto: $A$, $T\_a$ ,$P1$ ou $P_1$}
\bottomrule
\end{tabular}

\medskip
\emph{Os argumentos obrigatórios desta macro são duas dimensões expressas com decimais, no primeiro caso são duas medidas de comprimento, no segundo caso são uma medida de comprimento e a medida de um ângulo em graus. Não confunda a referência com o nome de um ponto. A referência é usada pelos cálculos, mas frequentemente, o nome é idêntico à referência.}

\medskip
\begin{tabular}{lll}%
\toprule
opções             & padrão & definição  \\
\midrule
\TOline{label} {sem padrão} {permite colocar um rótulo a uma distância predefinida}
\TOline{shift} {sem padrão} {adiciona $(x,y)$ ou $(\alpha:d)$ a todas as coordenadas}
\end{tabular}
\end{NewMacroBox}

\subsubsection{Coordenadas cartesianas}

\begin{tkzexample}[latex=6cm,small]
  \begin{tikzpicture}
  \tkzInit[xmax=5,ymax=5] % limita o tamanho dos eixos
  \tkzDrawX[>=latex]
  \tkzDrawY[>=latex]
  \tkzDefPoint(0,0){A}
  \tkzDefPoint(4,0){B}
  \tkzDefPoint(0,3){C}
  \tkzDrawPolygon(A,B,C)
  \tkzDrawPoints(A,B,C)
  \end{tikzpicture}
\end{tkzexample}

\subsubsection{Cálculos com \tkzNamePack{xfp}}

 \begin{tkzexample}[latex=7cm,small]
\begin{tikzpicture}[scale=1]
  \tkzInit[xmax=4,ymax=4]
  \tkzDrawX\tkzDrawY
  \tkzDefPoint(-1+2,sqrt(4)){O}
  \tkzDefPoint({3*ln(exp(1))},{exp(1)}){A}
  \tkzDefPoint({4*sin(pi/6)},{4*cos(pi/6)}){B}
  \tkzDrawPoints(O,B,A)
\end{tikzpicture}
\end{tkzexample}

\subsubsection{Coordenadas polares}

\begin{tkzexample}[latex=7cm,small]
  \begin{tikzpicture}
  \foreach \an [count=\i] in {0,60,...,300}
   { \tkzDefPoint(\an:3){A_\i}}
  \tkzDrawPolygon(A_1,A_...,A_6)
  \tkzDrawPoints(A_1,A_...,A_6)
  \end{tikzpicture}
\end{tkzexample}

\subsubsection{Pontos relativos}
Primeiro, podemos usar o ambiente \tkzNameEnv{scope} do \TIKZ.
No exemplo seguinte, temos uma maneira de definir um triângulo equilátero.

\begin{tkzexample}[latex=7cm,small]
\begin{tikzpicture}[scale=1]
 \begin{scope}[rotate=30]
  \tkzDefPoint(2,3){A}
  \begin{scope}[shift=(A)]
     \tkzDefPoint(90:5){B}
     \tkzDefPoint(30:5){C}
  \end{scope}
 \end{scope}
 \tkzDrawPolygon(A,B,C)
\tkzLabelPoints[above](B,C)
\tkzLabelPoints[below](A)
\tkzDrawPoints(A,B,C)
\end{tikzpicture}
\end{tkzexample}

\subsection{Ponto relativo a outro: \tkzcname{tkzDefShiftPoint}}
\begin{NewMacroBox}{tkzDefShiftPoint}{\oarg{Ponto}\parg{$x,y$}\marg{ref} ou \parg{$\alpha$:$d$}\marg{ref}}%
\begin{tabular}{lll}%
argumentos &  padrão & definição \\
\midrule
\TAline{($x,y$)}{sem padrão}{$x$ e $y$ são duas dimensões, por padrão em cm.}
\TAline{($\alpha$:$d$)}{sem padrão}{$\alpha$ é um ângulo em graus, $d$ é uma dimensão}
\TAline{\{ref\}}{sem padrão}{Referência atribuída ao ponto: $A$, $T\_a$ ,$P1$ ou $P_1$}

\midrule
opções &  padrão & definição \\

\midrule
\TOline{[pt]} {sem padrão} {\tkzcname{tkzDefShiftPoint}[A](0:4)\{B\}}
\end{tabular}
\end{NewMacroBox}

\subsubsection{Triângulo isósceles}
Esta macro permite colocar um ponto relativo a outro. Isso é equivalente a uma translação. Aqui está como construir um triângulo isósceles com vértice principal $A$ e ângulo no vértice de $30^{\circ}$.

\begin{tkzexample}[latex=7cm,small]
\begin{tikzpicture}[rotate=-30]
 \tkzDefPoint(2,3){A}
 \tkzDefShiftPoint[A](0:4){B}
 \tkzDefShiftPoint[A](30:4){C}
 \tkzDrawSegments(A,B B,C C,A)
 \tkzMarkSegments[mark=|](A,B A,C)
 \tkzDrawPoints(A,B,C)
 \tkzLabelPoints[right](B,C)
 \tkzLabelPoints[above left](A)
\end{tikzpicture}
\end{tkzexample}

\subsubsection{Triângulo equilátero}
Vamos ver como obter um triângulo equilátero (há muito mais simples)

\begin{tkzexample}[latex=7cm,small]
\begin{tikzpicture}[scale=1]
 \tkzDefPoint(2,3){A}
 \tkzDefShiftPoint[A](30:3){B}
 \tkzDefShiftPoint[A](-30:3){C}
 \tkzDrawPolygon(A,B,C)
 \tkzDrawPoints(A,B,C)
 \tkzLabelPoints[right](B,C)
 \tkzLabelPoints[above left](A)
 \tkzMarkSegments[mark=|](A,B A,C B,C)
\end{tikzpicture}
\end{tkzexample}

\subsubsection{Paralelogramo}
Há uma maneira mais simples
\begin{tkzexample}[latex=7cm,small]
\begin{tikzpicture}
 \tkzDefPoint(0,0){A}
 \tkzDefPoint(30:3){B}
 \tkzDefShiftPointCoord[B](10:2){C}
 \tkzDefShiftPointCoord[A](10:2){D}
 \tkzDrawPolygon(A,...,D)
 \tkzDrawPoints(A,...,D)
\end{tikzpicture}
\end{tkzexample}

\subsection{Definição de múltiplos pontos: \tkzcname{tkzDefPoints}}

\begin{NewMacroBox}{tkzDefPoints}{\oarg{opções locais}\marg{$x_1/y_1/n_1,x_2/y_2/r_2$, ...}}%
$x_i$ e $y_i$ são as coordenadas de um ponto referenciado $r_i$

\begin{tabular}{lll}%
\toprule
argumentos &  padrão  & exemplo  \\
\midrule
\TAline{$x_i/y_i/r_i$}{}{\tkzcname{tkzDefPoints\{0/0/O,2/2/A\}}}
\end{tabular}

\medskip
\begin{tabular}{lll}%
opções             & padrão & definição   \\
\midrule
\TOline{shift} {sem padrão} {Adiciona $(x,y)$ ou $(\alpha:d)$ a todas as coordenadas}
\end{tabular}
\end{NewMacroBox}

\subsection{Criar um triângulo}
\begin{tkzexample}[latex=6cm,small]
\begin{tikzpicture}[scale=.75]
 \tkzDefPoints{0/0/A,4/0/B,4/3/C}
 \tkzDrawPolygon(A,B,C)
 \tkzDrawPoints(A,B,C)
\end{tikzpicture}
\end{tkzexample}

\subsection{Criar um quadrado}
Note aqui a sintaxe para desenhar o polígono.
\begin{tkzexample}[latex=6cm,small]
\begin{tikzpicture}[scale=1]
 \tkzDefPoints{0/0/A,2/0/B,2/2/C,0/2/D}
 \tkzDrawPolygon(A,...,D)
 \tkzDrawPoints(A,...,D)
\end{tikzpicture}
\end{tkzexample}

\endinput


\part{Calculando}
Agora que os pontos fixos estão definidos, podemos com suas referências usando macros do pacote ou macros que você criará obter novos pontos. Os cálculos podem não ser aparentes, mas geralmente são feitos pelo pacote.
Você pode precisar usar algumas constantes matemáticas, aqui está a lista de constantes definidas pelo pacote.


\section{Ferramentas auxiliares}
\subsection{Constantes}

\tkzname{\tkznameofpack} conhece algumas constantes, aqui está a lista:
\begin{tkzltxexample}[]
  \def\tkzPhi{1.618034}
  \def\tkzInvPhi{0.618034}
  \def\tkzSqrtPhi{1.27202}
  \def\tkzSqrTwo{1.414213}
  \def\tkzSqrThree{1.7320508}
  \def\tkzSqrFive{2.2360679}
  \def\tkzSqrTwobyTwo{0.7071065}
  \def\tkzPi{3.1415926}
  \def\tkzEuler{2.71828182}
\end{tkzltxexample}

\subsection{Novo ponto por cálculo}

Quando uma macro de \tkzname{tkznameofpack} cria um novo ponto, ele é armazenado internamente com a referência \tkzname{tkzPointResult}. Você pode atribuir sua própria referência a ele. Isso é feito com a macro \tkzcname{tkzGetPoint}. Uma nova referência é criada, sua escolha de referência deve ser colocada entre chaves.

\begin{NewMacroBox}{tkzGetPoint}{\marg{ref}}%
Se o resultado está em \tkzname{tkzPointResult}, você pode acessá-lo com \tkzcname{tkzGetPoint}.

 \medskip
\begin{tabular}{lll}%
\toprule
argumentos & padrão & exemplo \\
\midrule
\TAline{ref}{sem padrão}{ \tkzcname{tkzGetPoint\{M\} } veja o próximo exemplo}
\end{tabular}
\end{NewMacroBox}

Às vezes você precisa obter dois pontos. É possível com

\begin{NewMacroBox}{tkzGetPoints}{\marg{ref1}\marg{ref2}}%
O resultado está em \tkzname{tkzPointFirstResult} e \tkzname{tkzPointSecondResult}.

 \medskip
\begin{tabular}{lll}%
\toprule
argumentos & padrão & exemplo \\
\midrule
\TAline{\{ref1,ref2\}}{sem padrão}{ \tkzcname{tkzGetPoints\{M,N\} } É o caso com \tkzcname{tkzInterCC}}
\end{tabular}
\end{NewMacroBox}

Se você precisar apenas do primeiro ou do segundo ponto, você também pode usar:

\begin{NewMacroBox}{tkzGetFirstPoint}{\marg{ref1}}%

 \medskip
\begin{tabular}{lll}%
\toprule
argumentos & padrão & exemplo \\
\midrule
\TAline{ref1}{sem padrão}{ \tkzcname{tkzGetFirstPoint\{M\} }}
\end{tabular}
\end{NewMacroBox}

\begin{NewMacroBox}{tkzGetSecondPoint}{\marg{ref2}}%

 \medskip
\begin{tabular}{lll}%
\toprule
argumentos & padrão & exemplo \\
\midrule
\TAline{ref2}{sem padrão}{ \tkzcname{tkzGetSecondPoint\{M\} }}
\end{tabular}
\end{NewMacroBox}

Às vezes os resultados consistem em um ponto e uma dimensão. Você obtém o ponto com \tkzcname{tkzGetPoint} e a dimensão com \tkzcname{tkzGetLength}.

\begin{NewMacroBox}{tkzGetLength}{\marg{nome de uma macro}}%

 \medskip
\begin{tabular}{lll}%
\toprule
argumentos & padrão & exemplo \\
\midrule
\TAline{nome de uma macro}{sem padrão}{ \tkzcname{tkzGetLength\{rAB\} \tkzcname{rAB} fornece o comprimento em cm}}
\end{tabular}
\end{NewMacroBox}

%\tkzcname{tkzCalcLength}(A,B) Após \tkzcname{tkzGetLength\{dAB\}} \tkzcname{dAB} fornece $AB$ em cm}


\section{Pontos especiais}
Aqui estão alguns pontos especiais.
%<--------------------------------------------------------------------------->
\subsection{Ponto médio de um segmento \tkzcname{tkzDefMidPoint}}
É uma questão de determinar o ponto médio de um segmento.

\begin{NewMacroBox}{tkzDefMidPoint}{\parg{pt1,pt2}}%
O resultado está em \tkzname{tkzPointResult}. Podemos acessá-lo com \tkzcname{tkzGetPoint}.

 \medskip
\begin{tabular}{lll}%
\toprule
argumentos & padrão & definição \\
\midrule
\TAline{(pt1,pt2)}{sem padrão}{pt1 e pt2 são dois pontos}
\end{tabular}
\end{NewMacroBox}

\subsubsection{Uso de \tkzcname{tkzDefMidPoint}}
Revise o uso de \tkzcname{tkzDefPoint}.
\begin{tkzexample}[latex=7cm,small]
\begin{tikzpicture}[scale=1]
 \tkzDefPoint(2,3){A}
 \tkzDefPoint(6,2){B}
 \tkzDefMidPoint(A,B)
 \tkzGetPoint{M}
 \tkzDrawSegment(A,B)
 \tkzDrawPoints(A,B,M)
 \tkzLabelPoints[below](A,B,M)
\end{tikzpicture}
\end{tkzexample}

\subsection{\tkzname{Razão áurea} \tkzcname{tkzDefGoldenRatio}}
Da Wikipedia: Em matemática, duas quantidades estão na razão áurea se sua razão é a mesma que a razão de sua soma para a maior das duas quantidades. Expresso algebricamente, para quantidades $a$, $b$ tal que $a > b > 0$; $a+b$ está para $a$ assim como $a$ está para $b$.

$ \frac{a+b}{a} = \frac{a}{b} = \phi = \frac{1 + \sqrt{5}}{2}$


Uma das duas soluções para a equação $x^2 - x - 1 = 0$
é a razão áurea $\phi$, $\phi = \frac{1 + \sqrt{5}}{2}$.

\begin{NewMacroBox}{tkzDefGoldenRatio}{\parg{pt1,pt2}}%
\begin{tabular}{lll}%
argumentos & padrão & exemplo \\
\midrule
\TAline{(pt1,pt2)}{sem padrão}{\tkzcname{tkzDefGoldenRatio(A,C)} \tkzcname{tkzGetPoint}\{B\}}
\bottomrule
\end{tabular}

\medskip
$AB=a$, $BC=b$ e $\dfrac{AC}{AB} = \dfrac{AB}{BC} =\phi$
\end{NewMacroBox}

\subsubsection{Use a razão áurea para dividir um segmento de linha}
\begin{tkzexample}[latex=7cm,small]
\begin{tikzpicture}
 \tkzDefPoints{0/0/A,6/0/C}
 \tkzDefMidPoint(A,C) \tkzGetPoint{I}
 %\tkzDefPointWith[linear,K=\tkzInvPhi](A,C)
 \tkzDefGoldenRatio(A,C) \tkzGetPoint{B}
 \tkzDrawSegments(A,C)
 \tkzDrawPoints(A,B,C)
 \tkzLabelPoints(A,B,C)
\end{tikzpicture}
\end{tkzexample}

\subsubsection{Arbelos dourado}
\begin{tkzexample}[latex=7cm,small]
\begin{tikzpicture}[scale=.6]
\tkzDefPoints{0/0/A,10/0/B}
\tkzDefGoldenRatio(A,B)     \tkzGetPoint{C}
\tkzDefMidPoint(A,B)        \tkzGetPoint{O_1}
\tkzDefMidPoint(A,C)        \tkzGetPoint{O_2}
\tkzDefMidPoint(C,B)        \tkzGetPoint{O_3}
\tkzDrawSemiCircles[fill=purple!15](O_1,B)
\tkzDrawSemiCircles[fill=teal!15](O_2,C O_3,B)
\end{tikzpicture}
\end{tkzexample}

Também é possível usar a seguinte macro.
\subsection{\tkzname{Coordenadas baricêntricas} com \tkzcname{tkzDefBarycentricPoint}}

$pt_1$, $pt_2$, \dots, $pt_n$ sendo $n$ pontos, eles definem $n$ vetores $\overrightarrow{v_1}$, $\overrightarrow{v_2}$, \dots, $\overrightarrow{v_n}$ com a origem do referencial como o ponto final comum. $\alpha_1$, $\alpha_2$,
\dots $\alpha_n$ são $n$ números, o vetor obtido por:
\begin{align*}
  \frac{\alpha_1 \overrightarrow{v_1} + \alpha_2 \overrightarrow{v_2} + \cdots + \alpha_n \overrightarrow{v_n}}{\alpha_1
    + \alpha_2 + \cdots + \alpha_n}
\end{align*}
define um único ponto.

\begin{NewMacroBox}{tkzDefBarycentricPoint}{\parg{pt1=$\alpha_1$,pt2=$\alpha_2$,\dots}}%
\begin{tabular}{lll}%
argumentos & padrão & definição \\
\midrule
\TAline{(pt1=$\alpha_1$,pt2=$\alpha_2$,\dots)}{sem padrão}{Cada ponto tem um peso atribuído}
\bottomrule
\end{tabular}

\medskip
\emph{Você precisa de pelo menos dois pontos. Resultado em \tkzname{tkzPointResult}.}
\end{NewMacroBox}


\subsubsection{com dois pontos}
No exemplo seguinte, obtemos o baricentro dos pontos $A$ e $B$ com coeficientes $1$ e $2$, em outras palavras:
\[
  \overrightarrow{AI}= \frac{2}{3}\overrightarrow{AB}
\]

\begin{tkzexample}[latex=7cm,small]
\begin{tikzpicture}
  \tkzDefPoint(2,3){A}
  \tkzDefShiftPointCoord[2,3](30:4){B}
  \tkzDefBarycentricPoint(A=1,B=2)
  \tkzGetPoint{G}
  \tkzDrawLine(A,B)
  \tkzDrawPoints(A,B,G)
  \tkzLabelPoints(A,B,G)
\end{tikzpicture}
\end{tkzexample}

\subsubsection{com três pontos}
Desta vez $M$ é simplesmente o centro de gravidade do triângulo.

 Por razões de simplificação e homogeneidade, há também \tkzcname{tkzCentroid}.
\begin{tkzexample}[latex=7cm,small]
\begin{tikzpicture}[scale=.8]
  \tkzDefPoints{2/1/A,5/3/B,0/6/C}
  \tkzDefBarycentricPoint(A=1,B=1,C=1)
  \tkzGetPoint{G}
  \tkzDefMidPoint(A,B)  \tkzGetPoint{C'}
  \tkzDefMidPoint(A,C)  \tkzGetPoint{B'}
  \tkzDefMidPoint(C,B)  \tkzGetPoint{A'}
  \tkzDrawPolygon(A,B,C)
  \tkzDrawLines[add=0 and 1,new](A,G B,G C,G)
  \tkzDrawPoints[new](A',B',C',G)
  \tkzDrawPoints(A,B,C)
  \tkzLabelPoint[above right](G){$G$}
  \tkzAutoLabelPoints[center=G](A,B,C)
  \tkzLabelPoints[above right](A')
  \tkzLabelPoints[below](B',C')
\end{tikzpicture}
\end{tkzexample}


\subsection{\tkzname{Centro de similitude interno e externo}}
Os centros das duas homotetias em que dois círculos correspondem são chamados de centros de similitude externo e interno. Você pode usar \tkzcname{tkzDefIntSimilitudeCenter} e \tkzcname{tkzDefExtSimilitudeCenter}, mas a próxima macro é melhor.

\begin{NewMacroBox}{tkzDefSimilitudeCenter}{\oarg{opções}\parg{O,A}\parg{O',B}}%

\begin{tabular}{lll}%
argumentos           & exemplo & explicação                         \\
\midrule
\TAline{\parg{pt1,pt2}\parg{pt3,pt4}}{$(O,A)(O',B)$} {$r=OA,r'=O'B$}
\end{tabular}

\medskip
\begin{tabular}{lll}%
\toprule
opções             & padrão & definição                         \\
\midrule
\TOline{ext}{ext}{centro externo}
\TOline{int}{ext}{centro interno}
\end{tabular}
\end{NewMacroBox}

\subsubsection{Interno e externo com \tkzname{node}}
\begin{tkzexample}[latex=7.5cm,small]
\begin{tikzpicture}[scale=.7]
 \tkzDefPoints{0/0/O,4/-5/A,3/0/B,5/-5/C}
 \tkzDefSimilitudeCenter[int](O,B)(A,C)
 \tkzGetPoint{I}
 \tkzDefSimilitudeCenter[ext](O,B)(A,C)
 \tkzGetPoint{J}
 \tkzDefLine[tangent from = I](O,B)
 \tkzGetPoints{D}{E}
 \tkzDefLine[tangent from = I](A,C)
 \tkzGetPoints{D'}{E'}
 \tkzDefLine[tangent from = J](O,B)
 \tkzGetPoints{F}{G}
 \tkzDefLine[tangent from = J](A,C)
 \tkzGetPoints{F'}{G'}
 \tkzDrawCircles(O,B A,C)
 \tkzDrawSegments[add = .5 and .5,new](D,D' E,E')
 \tkzDrawSegments[add= 0 and 0.25,new](J,F J,G)
 \tkzDrawPoints(O,A,I,J,D,E,F,G,D',E',F',G')
\end{tikzpicture}
\end{tkzexample}

\subsubsection{Teorema de D'Alembert} % (fold)
\label{ssub:d_alembert_theorem}

\begin{tkzexample}[latex=7cm,small]
 \begin{tikzpicture}[scale=.6,rotate=90]
 \tkzDefPoints{0/0/A,3/0/a,7/-1/B,5.5/-1/b}
 \tkzDefPoints{5/-4/C,4.25/-4/c}
 \tkzDrawCircles(A,a B,b C,c)
 \tkzDefExtSimilitudeCenter(A,a)(B,b) \tkzGetPoint{I}
 \tkzDefExtSimilitudeCenter(A,a)(C,c) \tkzGetPoint{J}
 \tkzDefExtSimilitudeCenter(C,c)(B,b) \tkzGetPoint{K}
 \tkzDefIntSimilitudeCenter(A,a)(B,b) \tkzGetPoint{I'}
 \tkzDefIntSimilitudeCenter(A,a)(C,c) \tkzGetPoint{J'}
 \tkzDefIntSimilitudeCenter(C,c)(B,b) \tkzGetPoint{K'}
 \tkzDrawPoints(A,B,C,I,J,K,I',J',K')
 \tkzDrawSegments[new](I,K A,I A,J B,I B,K C,J C,K)
 \tkzDrawSegments[new](I,J' I',J I',K)
 \end{tikzpicture}
\end{tkzexample}

% subsubsection d_alembert_theorem (end)

Você pode usar \tkzcname{tkzDefBarycentricPoint} para encontrar um centro homotético

|\tkzDefBarycentricPoint(O=\r,A=\R)     \tkzGetPoint{I}| \\
|\tkzDefBarycentricPoint(O={-\r},A=\R)  \tkzGetPoint{J}|

\subsubsection{Exemplo com \tkzname{node}}
\begin{tkzexample}[latex=7cm,small]
\begin{tikzpicture}[rotate=60,scale=.5]
 \tkzDefPoints{0/0/A,5/0/C}
 \tkzDefGoldenRatio(A,C) \tkzGetPoint{B}
 \tkzDefSimilitudeCenter(A,B)(C,B)\tkzGetPoint{J}
 \tkzDefTangent[from = J](A,B)  \tkzGetPoints{F}{G}
 \tkzDefTangent[from = J](C,B)  \tkzGetPoints{F'}{G'}
 \tkzDrawCircles(A,B C,B)
 \tkzDrawSegments[add= 0 and 0.25,cyan](J,F J,G)
 \tkzDrawPoints(A,J,F,G,F',G')
\end{tikzpicture}
\end{tkzexample}
\newpage
%<---------------------------------------------------------------------->
\subsection{ \tkzname{Divisão harmônica} com \tkzcname{tkzDefHarmonic}}
%<---------------------------------------------------------------------->

\begin{NewMacroBox}{tkzDefHarmonic}{\oarg{opções}\parg{pt1,pt2,pt3} ou \parg{pt1,pt2,k}}%

\begin{tabular}{lll}%
opções             & padrão & definição                         \\
\midrule
\TOline{both}{both}{\parg{A,B,2} procuramos C e D tais que $(A,B;C,D) = -1$ e CA=2CB }
\TOline{ext}{both}{\parg{A,B,C} procuramos D tal que $(A,B;C,D) = -1$}
\TOline{int}{both}{\parg{A,B,D} procuramos C tal que $(A,B;C,D) = -1$}
\end{tabular}
\end{NewMacroBox}

\subsubsection{opções \tkzname{ext} e \tkzname{int}}
\begin{tkzexample}[vbox,small]
  \begin{tikzpicture}
  \tkzDefPoints{0/0/A,6/0/B,4/0/C}
  \tkzDefHarmonic[ext](A,B,C) \tkzGetPoint{J}
  \tkzDefHarmonic[int](A,B,J) \tkzGetPoint{I}
  \tkzDrawPoints(A,B,I,J)
  \tkzDrawLine[add=.5 and 1](A,B)
  \tkzLabelPoints(A,B,I,J)
  \end{tikzpicture}
\end{tkzexample}

\subsubsection{Bissetriz e divisão harmônica} % (fold)
\label{ssub:bisector_and_harmonic_division}

\begin{tkzexample}[vbox,small]
  \begin{tikzpicture}[scale=1.25]
  \tkzDefPoints{0/0/A,4/0/C,5/3/X}
  \tkzDefLine[bisector](A,X,C) \tkzGetPoint{x}
  \tkzInterLL(X,x)(A,C)        \tkzGetPoint{B}
  \tkzDefHarmonic[ext](A,C,B)  \tkzGetPoint{D}
  \tkzDrawPolygon(A,X,C)
  \tkzDrawSegments(X,B C,D D,X)
  \tkzDrawPoints(A,B,C,D,X)
  \tkzMarkAngles[mark=s|](A,X,B B,X,C)
  \tkzMarkRightAngle[size=.4,
                     fill=gray!20,
                     opacity=.3](B,X,D)
  \tkzLabelPoints(A,B,C,D)
  \tkzLabelPoints[above right](X)
  \end{tikzpicture}
\end{tkzexample}


% subsubsection bisector_and_harmonic_division (end)
\subsubsection{opção \tkzname{both} }
\tkzname{both} é a opção padrão
\begin{tkzexample}[vbox,small]
\begin{tikzpicture}
 \tkzDefPoints{0/0/A,6/0/B}
 \tkzDefHarmonic(A,B,{1/2})\tkzGetPoints{I}{J}
 \tkzDrawPoints(A,B,I,J)
 \tkzDrawLine[add=1 and .5](A,B)
 \tkzLabelPoints(A,B,I,J)
\end{tikzpicture}
\end{tkzexample}

%<---------------------------------------------------------------------->
\subsection{\tkzname{Pontos equidistantes} com \tkzcname{tkzDefEquiPoints} }
%<---------------------------------------------------------------------->

\begin{NewMacroBox}{tkzDefEquiPoints}{\oarg{opções locais}\parg{pt1,pt2}}%
\begin{tabular}{lll}%
argumentos &  padrão & definição \\
\midrule
\TAline{(pt1,pt2)}{sem padrão}{lista não ordenada de dois itens}
\end{tabular}

\begin{tabular}{lll}%
opções             & padrão & definição  \\
\midrule
\TOline{dist} {2 (cm)} {metade da distância entre os dois pontos}
\TOline{from=pt} {sem padrão} {ponto de referência}
\TOline{show} {false} {se verdadeiro exibe traços de compasso}
\TOline{/compass/delta} {0} {tamanho do traço do compasso }
\end{tabular}

\medskip
\emph{Esta macro torna possível obter dois pontos em uma linha reta equidistantes de um ponto dado.}
\end{NewMacroBox}


\subsubsection{Usando \tkzcname{tkzDefEquiPoints} com opções}
\begin{tkzexample}[latex=7cm,small]
\begin{tikzpicture}
  \tkzSetUpCompass[color=purple,line width=1pt]
  \tkzDefPoints{0/1/A,5/2/B,3/4/C}
  \tkzDefEquiPoints[from=C,dist=1,show,
      /tkzcompass/delta=20](A,B)
   \tkzGetPoints{E}{H}
   \tkzDrawLines[color=blue](C,E C,H A,B)
   \tkzDrawPoints[color=blue](A,B,C)
   \tkzDrawPoints[color=red](E,H)
   \tkzLabelPoints(E,H)
   \tkzLabelPoints[color=blue](A,B,C)
\end{tikzpicture}
\end{tkzexample}
%<---------------------------------------------------------------------->
%                          Middle of an arc                             >
%<---------------------------------------------------------------------->
\subsection{Ponto médio de um arco}
\begin{NewMacroBox}{tkzDefMidArc}{\parg{pt1,pt2,pt3}}%
\begin{tabular}{lll}%
argumentos &  padrão & definição \\
\midrule
\TAline{$pt1,pt2,pt3$}{sem padrão}{$pt1$ é o centro, $\widearc{pt2pt3}$ o arco}
\end{tabular}
\end{NewMacroBox}

\begin{tkzexample}[vbox,small]
  \begin{tikzpicture}[scale=1]
   \tkzDefPoints{0/0/A,10/0/B}
   \tkzDefGoldenRatio(A,B)                              \tkzGetPoint{C}
   \tkzDefMidPoint(A,B)                                 \tkzGetPoint{O_1}
   \tkzDefMidPoint(A,C)                                 \tkzGetPoint{O_2}
   \tkzDefMidPoint(C,B)                                 \tkzGetPoint{O_3}
   \tkzDefMidArc(O_3,B,C)                               \tkzGetPoint{P}
   \tkzDefMidArc(O_2,C,A)                               \tkzGetPoint{Q}
   \tkzDefMidArc(O_1,B,A)                               \tkzGetPoint{L}
   \tkzDefPointBy[rotation=center C angle 90](B)        \tkzGetPoint{c}
   \tkzInterCC[common=B](P,B)(O_1,B)                    \tkzGetFirstPoint{P_1}
   \tkzInterCC[common=C](P,C)(O_2,C)                    \tkzGetFirstPoint{P_2}
   \tkzInterCC[common=C](Q,C)(O_3,C)                    \tkzGetFirstPoint{P_3}
   \tkzInterLC[near](c,C)(O_1,A)                        \tkzGetFirstPoint{D}
   \tkzInterLL(A,P_1)(C,D)                              \tkzGetPoint{P_1'}
   \tkzDefPointBy[inversion = center A through D](P_2)  \tkzGetPoint{P_2'}
   \tkzDefCircle[circum](P_3,P_2,P_1)                   \tkzGetPoint{O_4}
   \tkzInterLL(B,Q)(A,P)                                \tkzGetPoint{S}
   \tkzDefMidPoint(P_2',P_1')                           \tkzGetPoint{o}
   \tkzDefPointBy[inversion = center A through D](S)    \tkzGetPoint{S'}
   \tkzDrawArc[cyan,delta=0](Q,A)(P_1)
   \tkzDrawArc[cyan,delta=0](P,P_1)(B)
   \tkzDrawSemiCircles[teal](O_1,B O_2,C O_3,B)
   \tkzDrawCircles[new](o,P O_4,P_1)
   \tkzDrawSegments(A,B)
   \tkzDrawSegments[cyan](A,P_1 A,S' A,P_2')
   \tkzDrawSegments[purple](B,L C,P_2' B,Q B,L S',P_1')
   \tkzDrawLines[add=0 and .8](B,P_2')
   \tkzDrawLines[add=0 and .4](C,D)
   \tkzDrawPoints(A,B,C,P,Q,P_3,P_2,P_1,P_1',D,P_2',L,S,S')
   \tkzLabelPoints(A,B,C,P_3)
   \tkzLabelPoints[above](P,Q,P_1)
   \tkzLabelPoints[above right](P_2,P_2',D,S')
   \tkzLabelPoints[above left](L,S)
    \tkzLabelPoints[below left](P_1')
  \end{tikzpicture}
\end{tkzexample}

%<---------------------------------------------------------------------->
%                          Point on a line                              >
%<---------------------------------------------------------------------->

\section{Ponto sobre linha ou círculo}
\subsection{Ponto sobre uma linha com \tkzcname{tkzDefPointOnLine}}

\begin{NewMacroBox}{tkzDefPointOnLine}{\oarg{opções locais}\parg{A,B}}%
\begin{tabular}{lll}%
argumentos &  padrão & definição                 \\
\midrule
\TAline{pt1,pt2} {sem padrão}  {Dois pontos para definir uma linha}
\bottomrule
\end{tabular}

\medskip
\begin{tabular}{lll}%
opções       & padrão & definição \\
\midrule
\TOline{pos=nb}  {}{nb é um decimal  }
\end{tabular}
\end{NewMacroBox}

\subsubsection{Uso da opção \tkzname{pos}}
\begin{tkzexample}[latex=7cm,small]
\begin{tikzpicture}
\tkzDefPoints{0/0/A,3/0/B}
\tkzDefPointOnLine[pos=1.2](A,B)\tkzGetPoint{P}
\tkzDefPointOnLine[pos=-0.2](A,B)\tkzGetPoint{R}
\tkzDefPointOnLine[pos=0.5](A,B) \tkzGetPoint{S}
\tkzDrawLine[new](A,B)
\tkzDrawPoints(A,B,P)
\tkzLabelPoints(A,B)
\tkzLabelPoint[above](P){pos=$1.2$}
\tkzLabelPoint[above](R){pos=$-.2$}
\tkzLabelPoint[above](S){pos=$.5$}
\tkzDrawPoints(A,B,P,R,S)
\tkzLabelPoints(A,B)
\end{tikzpicture}
\end{tkzexample}

\subsection{Ponto sobre um círculo com \tkzcname{tkzDefPointOnCircle}}
A ordem dos argumentos mudou: agora é centro, ângulo e ponto ou raio.
Adicionei duas opções para trabalhar com radianos que são \tkzname{through in rad} e \tkzname{R in rad}.

\begin{NewMacroBox}{tkzDefPointOnCircle}{\oarg{opções locais}}%
\begin{tabular}{lll}%
opções   & padrão & exemplos definição \\
\midrule
\TOline{through}  {}{through =  center K1 angle 30 point B]}
\TOline{R} {}{R =  center K1 angle 30 radius \tkzcname{rAp}}
\TOline{through in rad}  {}{through in rad=  center K1 angle pi/4 point B]}
\TOline{R in rad} {}{R in rad =  center K1 angle pi/6 radius \tkzcname{rAp}}
\end{tabular}

\medskip
\emph{A nova ordem para os argumentos são: centro, ângulo e ponto ou raio.}
\end{NewMacroBox}

\subsubsection{Teorema de Altshiller}
 As duas linhas que unem os pontos de interseção de dois círculos ortogonais a um ponto em um dos círculos encontram o outro círculo em dois pontos diametralmente opostos. Altshiller p 176

\begin{tkzexample}[latex=6cm,small]
\begin{tikzpicture}[scale=.4]
\tkzDefPoints{0/0/P,5/0/Q,3/2/I}
\tkzDefCircle[orthogonal from=P](Q,I)
\tkzGetFirstPoint{E}
\tkzDrawCircles(P,E Q,E)
\tkzInterCC[common=E](P,E)(Q,E) \tkzGetFirstPoint{F}
\tkzDefPointOnCircle[through =  center P angle 80 point E]
 \tkzGetPoint{A}
\tkzInterLC[common=E](A,E)(Q,E)  \tkzGetFirstPoint{C}
\tkzInterLL(A,F)(C,Q)  \tkzGetPoint{D}
\tkzDrawLines[add=0 and .75](P,Q)
\tkzDrawLines[add=0 and 2](A,E)
\tkzDrawSegments(P,E E,F F,C A,F C,D)
\tkzDrawPoints(P,Q,E,F,A,C,D)
\tkzLabelPoints(P,Q,F,C,D)
\tkzLabelPoints[above](E,A)
\end{tikzpicture}
\end{tkzexample}

\subsubsection{Uso de  \tkzcname{tkzDefPointOnCircle}}

\begin{tkzexample}[latex=6cm,small]
\begin{tikzpicture}
\tkzDefPoints{0/0/A,4/0/B,0.8/3/C}
\tkzDefPointOnCircle[R = center B  angle 90 radius 1]
\tkzGetPoint{I}
\tkzDefCircle[circum](A,B,C)
\tkzGetPoints{G}{g}
\tkzDefPointOnCircle[through = center G angle 30 point g]
\tkzGetPoint{J}
\tkzDefCircle[R](B,1) \tkzGetPoint{b}
\tkzDrawCircle[teal](B,b)
\tkzDrawCircle(G,J)
\tkzDrawPoints(A,B,C,G,I,J)
\tkzAutoLabelPoints[center=G](A,B,C,J)
\tkzLabelPoints[below](G,I)
\end{tikzpicture}
\end{tkzexample}

\newpage
\section{Pontos especiais relacionados a um triângulo}

\subsection{Centro do triângulo: \tkzcname{tkzDefTriangleCenter}}

\begin{NewMacroBox}{tkzDefTriangleCenter}{\oarg{opções locais}\parg{A,B,C}}%
\tkzHandBomb\ Esta macro permite definir o centro de um triângulo. Cuidado, os argumentos são listas de três pontos. Esta macro é usada em conjunto com \tkzcname{tkzGetPoint} para obter o centro que você está procurando.

 Você pode usar \tkzname{tkzPointResult} se não for necessário manter os resultados.

\medskip
\begin{tabular}{lll}%
\toprule
argumentos & padrão & exemplo \\
\midrule
\TAline{(pt1,pt2,pt3)}{sem padrão}{ \tkzcname{tkzDefTriangleCenter[ortho](B,C,A)}}
\midrule
opções             & padrão & definição                         \\
\midrule
\TOline{ortho}  {circum}{interseção das alturas}
\TOline{orthic}  {circum}{\dots}
\TOline{centroid} {circum}{interseção das medianas}
\TOline{median} {circum}{ \dots }
\TOline{circum}{circum}{centro do círculo circunscrito}
\TOline{in}    {circum}{centro do círculo inscrito em um triângulo }
\TOline{in}    {circum}{interseção das bissetrizes}
\TOline{ex}    {circum}{centro de um círculo ex-inscrito em um triângulo }
\TOline{euler}{circum}{centro do círculo de Euler }
\TOline{gergonne}{circum}{definido com o triângulo de contato}
\TOline{symmedian} {circum}{Ponto de Lemoine ou centro simediano ou Ponto de Grebe }
\TOline{lemoine} {circum}{ \dots}
\TOline{grebe} {circum}{ \dots}
\TOline{spieker} {circum}{centro do círculo de Spieker}
\TOline{nagel}{circum}{Centro de Nagel}
\TOline{mittenpunkt} {circum}{Ou middlespoint}
\TOline{feuerbach}{circum}{Ponto de Feuerbach}

\end{tabular}
\end{NewMacroBox}

\subsubsection{Opção \tkzname{ortho} ou \tkzname{orthic}}
 A interseção $H$ das três alturas de um triângulo é chamada de ortocentro.

\begin{tkzexample}[latex=6cm,small]
\begin{tikzpicture}
  \tkzDefPoint(0,0){A}
  \tkzDefPoint(5,1){B}
  \tkzDefPoint(1,4){C}
  \tkzDefTriangleCenter[ortho](B,C,A)
  \tkzGetPoint{H}
  \tkzDefSpcTriangle[orthic,name=H](A,B,C){a,b,c}
  \tkzDrawPolygon(A,B,C)
  \tkzDrawSegments[new](A,Ha B,Hb C,Hc)
  \tkzDrawPoints(A,B,C,H)
  \tkzLabelPoint(H){$H$}
  \tkzLabelPoints[below](A,B)
  \tkzLabelPoints[above](C)
 \tkzMarkRightAngles(A,Ha,B B,Hb,C C,Hc,A)
\end{tikzpicture}
\end{tkzexample}

\subsubsection{Opção \tkzname{centroid}}
\begin{tkzexample}[latex=6cm,small]
\begin{tikzpicture}[scale=.75]
  \tkzDefPoints{0/0/A,5/0/B,1/4/C}
  \tkzDefTriangleCenter[centroid](A,B,C)
  \tkzGetPoint{G}
  \tkzDrawPolygon(A,B,C)
  \tkzDrawLines[add = 0 and 2/3,new](A,G B,G C,G)
  \tkzDrawPoints(A,B,C,G)
  \tkzLabelPoint(G){$G$}
\end{tikzpicture}
\end{tkzexample}

\subsubsection{Opção \tkzname{circum}}
\begin{tkzexample}[latex=6cm,small]
 \begin{tikzpicture}
  \tkzDefPoints{0/1/A,3/2/B,1/4/C}
  \tkzDefTriangleCenter[circum](A,B,C)
  \tkzGetPoint{O}
  \tkzDrawPolygon(A,B,C)
  \tkzDrawCircle(O,A)
  \tkzDrawPoints(A,B,C,O)
  \tkzLabelPoint(O){$O$}
\end{tikzpicture}
\end{tkzexample}

\subsubsection{Opção \tkzname{in}}
Em geometria, o incírculo ou círculo inscrito de um triângulo é o maior círculo contido no triângulo; ele toca (é tangente a) os três lados. O centro do incírculo é um centro do triângulo chamado incentro do triângulo.
O centro do incírculo, chamado incentro, pode ser encontrado como a interseção das três bissetrizes internas dos ângulos. O centro de um excírculo é a interseção da bissetriz interna de um ângulo (no vértice $A$, por exemplo) e as bissetrizes externas dos outros dois. O centro deste excírculo é chamado de excentro relativo ao vértice $A$, ou excentro de $A$. Como a bissetriz interna de um ângulo é perpendicular à sua bissetriz externa, segue-se que o centro do incírculo juntamente com os três centros dos excírculos formam um sistema ortocêntrico.\\
(Artigo na \href{https://en.wikipedia.org/wiki/Incircle_and_excircles_of_a_triangle}{Wikipedia})

 \medskip
 Obtemos o centro do círculo inscrito do triângulo. O resultado é claro em \tkzname{tkzPointResult}. Podemos recuperá-lo com \tkzcname{tkzGetPoint}.

\begin{tkzexample}[latex=7cm,small]
\begin{tikzpicture}
\tkzDefPoints{0/0/A,6/0/B,0.8/4/C}
\tkzDefTriangleCenter[in](A,B,C)
   \tkzGetPoint{I}
\tkzDrawLines(A,B B,C C,A)
\tkzDefCircle[in](A,B,C) \tkzGetPoints{I}{i}
\tkzDrawCircle(I,i)
\tkzDrawPoint[red](I)
\tkzDrawPoints(A,B,C)
\tkzLabelPoint(I){$I$}
\end{tikzpicture}
\end{tkzexample}

\subsubsection{Opção \tkzname{ex}}
Um excírculo ou círculo ex-inscrito do triângulo é um círculo situado fora do triângulo, tangente a um de seus lados e tangente às extensões dos outros dois. Todo triângulo tem três excírculos distintos, cada um tangente a um dos lados do triângulo.\\
(Artigo na \href{https://en.wikipedia.org/wiki/Incircle_and_excircles_of_a_triangle}{Wikipedia})


 Obtemos o centro de um círculo inscrito do triângulo. O resultado é claro em \tkzname{tkzPointResult}. Podemos recuperá-lo com \tkzcname{tkzGetPoint}.

\begin{tkzexample}[latex=7cm,small]
\begin{tikzpicture}[scale=.5]
  \tkzDefPoints{0/1/A,3/2/B,1/4/C}
  \tkzDefTriangleCenter[ex](B,C,A)
  \tkzGetPoint{J_c}
  \tkzDefPointBy[projection=onto A--B](J_c)
  \tkzGetPoint{Tc}
  \tkzDrawPolygon(A,B,C)
  \tkzDrawCircle[new](J_c,Tc)
  \tkzDrawLines[add=1.5 and 0](A,C B,C)
  \tkzDrawPoints(A,B,C,J_c)
  \tkzLabelPoints(J_c)
\end{tikzpicture}
\end{tkzexample}

\subsubsection{Opção \tkzname{euler}}
Esta macro permite obter o centro do círculo dos nove pontos ou círculo de Euler ou círculo de Feuerbach. O círculo de nove pontos, também chamado círculo de Euler ou círculo de Feuerbach, é o círculo que passa pelos pés perpendiculares $H_A$, $H_B$ e $H_C$ baixados dos vértices de qualquer triângulo de referência $ABC$ sobre os lados opostos a eles. Euler mostrou em 1765 que ele também passa pelos pontos médios $M_A$, $M_B$, $M_C$ dos lados de $ABC$. Pelo teorema de Feuerbach, o círculo de nove pontos também passa pelos pontos médios $E_A$, $E_B$ e $E_C$ dos segmentos que unem os vértices e o ortocentro $H$. Esses pontos são comumente referidos como os pontos de Euler.\\ (\url{https://mathworld.wolfram.com/Nine-PointCircle.html})

\begin{tkzexample}[latex=6cm,small]
\begin{tikzpicture}[scale=1.2,rotate=90]
 \tkzDefPoints{0/0/A,6/0/B,0.8/4/C}
 \tkzDefSpcTriangle[medial,name=M](A,B,C){_A,_B,_C}
 \tkzDefTriangleCenter[euler](A,B,C)\tkzGetPoint{N}
 % I= N nine points
 \tkzDefTriangleCenter[ortho](A,B,C)\tkzGetPoint{H}
 \tkzDefMidPoint(A,H) \tkzGetPoint{E_A}
 \tkzDefMidPoint(C,H) \tkzGetPoint{E_C}
 \tkzDefMidPoint(B,H) \tkzGetPoint{E_B}
 \tkzDefSpcTriangle[ortho,name=H](A,B,C){_A,_B,_C}
 \tkzDrawPolygon(A,B,C)
 \tkzDrawCircle[new](N,E_A)
 \tkzDrawSegments[new](A,H_A B,H_B C,H_C)
 \tkzDrawPoints(A,B,C,N,H)
 \tkzDrawPoints[new](M_A,M_B,M_C)
 \tkzDrawPoints( H_A,H_B,H_C)
 \tkzDrawPoints[green](E_A,E_B,E_C)
 \tkzAutoLabelPoints[center=N,
 font=\scriptsize](A,B,C,M_A,M_B,M_C,H_A,H_B,H_C,%
   E_A,E_B,E_C)
 \tkzLabelPoints[font=\scriptsize](H,N)
 \tkzMarkSegments[mark=s|,size=3pt,
 color=blue,line width=1pt](B,E_B E_B,H)
\end{tikzpicture}
\end{tkzexample}


\subsubsection{Opção \tkzname{symmedian}}

O ponto de concorrência $K$ das simedianas, às vezes também chamado de ponto de Lemoine (na Inglaterra e França) ou ponto de Grebe (na Alemanha).\\
\href{https://mathworld.wolfram.com/SymmedianPoint.html}{Weisstein, Eric W. "Symmedian Point." From MathWorld--A Wolfram Web Resource.}

\begin{tkzexample}[latex=7cm,small]
\begin{tikzpicture}
  \tkzDefPoint(0,0){A}
  \tkzDefPoint(5,0){B}
  \tkzDefPoint(1,4){C}
  \tkzDefTriangleCenter[symmedian](A,B,C)
  \tkzGetPoint{K}
  \tkzDefTriangleCenter[median](A,B,C)
  \tkzGetPoint{G}
  \tkzDefTriangleCenter[in](A,B,C)\tkzGetPoint{I}
  \tkzDefSpcTriangle[centroid,name=M](A,B,C){a,b,c}
  \tkzDefSpcTriangle[incentral,name=I](A,B,C){a,b,c}
  \tkzDrawPolygon(A,B,C)
  \tkzDrawLines[add = 0 and 2/3,new](A,K B,K C,K)
  \tkzDrawSegments[color=cyan](A,Ma B,Mb C,Mc)
  \tkzDrawSegments[color=green](A,Ia B,Ib C,Ic)
  \tkzDrawPoints(A,B,C,K,G,I)
  \tkzLabelPoints[font=\scriptsize](A,B,K,G,I)
  \tkzLabelPoints[above,font=\scriptsize](C)
\end{tikzpicture}
\end{tkzexample}

\subsubsection{Opção \tkzname{spieker}}
O centro de Spieker é o centro $Sp$ do círculo de Spieker, ou seja, o incentro do triângulo medial de um triângulo de referência.\\
\href{https://mathworld.wolfram.com/SpiekerCenter.html}{Weisstein, Eric W. "Spieker Center." From MathWorld--A Wolfram Web Resource. }

\begin{tkzexample}[latex=8cm,small]
\begin{tikzpicture}
 \tkzDefPoints{0/0/A,6/0/B,5/5/C}
 \tkzDefSpcTriangle[medial](A,B,C){Ma,Mb,Mc}
 \tkzDefTriangleCenter[centroid](A,B,C)
 \tkzGetPoint{G}
 \tkzDefTriangleCenter[spieker](A,B,C)
 \tkzGetPoint{Sp}
 \tkzDrawPolygon[](A,B,C)
 \tkzDrawPolygon[new](Ma,Mb,Mc)
 \tkzDefCircle[in](Ma,Mb,Mc) \tkzGetPoints{I}{i}
 \tkzDrawCircle(I,i)
 \tkzDrawPoints(B,C,A,Sp,Ma,Mb,Mc)
 \tkzAutoLabelPoints[center=G,dist=.3](Ma,Mb)
 \tkzLabelPoints[right](Sp)
 \tkzLabelPoints[below](A,B,Mc)
 \tkzLabelPoints[above](C)
\end{tikzpicture}
\end{tkzexample}

\subsubsection{Opção \tkzname{gergonne}}

O Ponto de Gergonne é o ponto de concorrência que resulta da conexão dos vértices de um triângulo aos pontos opostos de tangência do incírculo do triângulo.
(Joseph Gergonne matemático francês)

\begin{tkzexample}[latex=8cm,small]
\begin{tikzpicture}
\tkzDefPoints{0/0/B,3.6/0/C,2.8/4/A}
\tkzDefTriangleCenter[gergonne](A,B,C)
\tkzGetPoint{Ge}
\tkzDefSpcTriangle[intouch](A,B,C){C_1,C_2,C_3}
\tkzDefCircle[in](A,B,C) \tkzGetPoints{I}{i}
\tkzDrawLines[add=.25 and .25,teal](A,B A,C B,C)
\tkzDrawSegments[new](A,C_1 B,C_2 C,C_3)
\tkzDrawPoints(A,...,C,C_1,C_2,C_3)
\tkzDrawPoints[red](Ge)
\tkzLabelPoints(B,C,C_1,Ge)
\tkzLabelPoints[above](A,C_2,C_3)
\end{tikzpicture}
\end{tkzexample}

\subsubsection{Opção \tkzname{nagel}}
Seja $Ta$ o ponto em que o excírculo com centro $Ja$ encontra o lado $BC$ de um triângulo $ABC$, e defina $Tb$ e $Tc$ de forma similar. Então as linhas $ATa$, $BTb$ e $CTc$ concorrem no ponto de Nagel $Na$.\\
\href{https://mathworld.wolfram.com/NagelPoint.html}{Weisstein, Eric W. "Nagel point." From MathWorld--A Wolfram Web Resource. }


\begin{tkzexample}[latex=7cm,small]
  \begin{tikzpicture}[scale=.5]
  \tkzDefPoints{0/0/A,6/0/B,4/6/C}
  \tkzDefSpcTriangle[ex](A,B,C){Ja,Jb,Jc}
  \tkzDefSpcTriangle[extouch](A,B,C){Ta,Tb,Tc}
  \tkzDefTriangleCenter[nagel](A,B,C)
  \tkzGetPoint{Na}
  \tkzDrawPolygon[blue](A,B,C)
  \tkzDrawLines[add=0 and 1](A,Ta B,Tb C,Tc)
  \tkzDrawPoints[new](Ja,Jb,Jc,Ta,Tb,Tc)
  \tkzClipBB
  \tkzDrawLines[add=1 and 1,dashed](A,B B,C C,A)
  \tkzDrawCircles[new](Ja,Ta Jb,Tb Jc,Tc)
  \tkzDrawSegments[new,dashed](Ja,Ta Jb,Tb Jc,Tc)
  \tkzDrawPoints(B,C,A)
  \tkzDrawPoints[new](Na)
  \tkzLabelPoints(B,C,A)
  \tkzLabelPoints[new](Na)
  \tkzLabelPoints[new](Ja,Jb,Jc,Ta,Tb,Tc)
  \tkzMarkRightAngles[fill=gray!20](Ja,Ta,C
              Jb,Tb,A Jc,Tc,B)
  \end{tikzpicture}
\end{tkzexample}


\subsubsection{Opção   \tkzname{mittenpunkt}}

O mittenpunkt (também chamado middlespoint) de um triângulo $ABC$ é o ponto simediano do triângulo excentral, ou seja, o ponto de concorrência M das linhas dos excentros através dos pontos médios dos lados do triângulo correspondentes.\\
\href{https://mathworld.wolfram.com/Mittenpunkt.html}{Weisstein, Eric W. "Mittenpunkt." From MathWorld--A Wolfram Web Resource.}


\begin{tkzexample}[latex=8cm,small]
\begin{tikzpicture}[scale=.5]
 \tkzDefPoints{0/0/A,6/0/B,4/6/C}
 \tkzDefSpcTriangle[centroid](A,B,C){Ma,Mb,Mc}
 \tkzDefSpcTriangle[ex](A,B,C){Ja,Jb,Jc}
 \tkzDefSpcTriangle[extouch](A,B,C){Ta,Tb,Tc}
 \tkzDefTriangleCenter[mittenpunkt](A,B,C)
 \tkzGetPoint{Mi}
 \tkzDrawPoints[new](Ma,Mb,Mc,Ja,Jb,Jc)
 \tkzClipBB
 \tkzDrawPolygon[blue](A,B,C)
 \tkzDrawLines[add=0 and 1](Ja,Ma
               Jb,Mb Jc,Mc)
 \tkzDrawLines[add=1 and 1](A,B A,C B,C)
 \tkzDrawCircles[new](Ja,Ta Jb,Tb Jc,Tc)
 \tkzDrawPoints(B,C,A)
 \tkzDrawPoints[new](Mi)
 \tkzLabelPoints(Mi)
 \tkzLabelPoints[left](Mb)
 \tkzLabelPoints[new](Ma,Mc,Jb,Jc)
 \tkzLabelPoints[above left](Ja,Jc)
\end{tikzpicture}
\end{tkzexample}

\subsubsection{Relação entre  \tkzname{gergonne}, \tkzname{centroid} e \tkzname{mittenpunkt}}

O ponto de Gergonne $Ge$, centroide do triângulo $G$ e mittenpunkt $M$ são colineares, com GeG/GM=2.

\begin{tkzexample}[vbox,small]
\begin{tikzpicture}
\tkzDefPoints{0/0/A,2/2/B,8/0/C}
\tkzDefTriangleCenter[gergonne](A,B,C) \tkzGetPoint{Ge}
\tkzDefTriangleCenter[centroid](A,B,C)
\tkzGetPoint{G}
\tkzDefTriangleCenter[mittenpunkt](A,B,C)
\tkzGetPoint{M}
\tkzDrawLines[add=.25 and .25,teal](A,B A,C B,C)
\tkzDrawLines[add=.25 and .25,new](Ge,M)
\tkzDrawPoints(A,...,C)
\tkzDrawPoints[red,size=2](G,M,Ge)
\tkzLabelPoints(A,...,C,M,G,Ge)
\tkzMarkSegment[mark=s||](Ge,G)
\tkzMarkSegment[mark=s|](G,M)
\end{tikzpicture}
\end{tkzexample}

\endinput

\section{Definição de pontos por transformação}
Essas transformações são:

\begin{itemize}
   \item translação;
   \item homotetia;
   \item reflexão ortogonal ou simetria;
   \item simetria central;
   \item projeção ortogonal;
   \item rotação (graus ou radianos);
   \item inversão em relação a um círculo.
\end{itemize}

\subsection{\tkzcname{tkzDefPointBy}}
A escolha das transformações é feita através das opções. Existem duas macros, uma para a transformação de um único ponto \tkzcname{tkzDefPointBy} e outra para a transformação de uma lista de pontos \tkzcname{tkzDefPointsBy}. Por padrão a imagem de $A$ é $A'$. Por exemplo, escreveremos:
\begin{tkzltxexample}[]
\tkzDefPointBy[translation= from A to A'](B)
\end{tkzltxexample}
O resultado está em \tkzname{tkzPointResult}

\medskip
\begin{NewMacroBox}{tkzDefPointBy}{\oarg{local opções}\parg{pt}}%
O argumento é um simples ponto existente e sua imagem é armazenada em \tkzname{tkzPointResult}. Se você quiser manter este ponto, então a macro \tkzcname{tkzGetPoint\{M\}} permite que você atribua o nome \tkzname{M} ao ponto.

\begin{tabular}{lll}%
\toprule
argumentos &  definição & exemplos               \\
\midrule
\TAline{pt}   {nome do ponto existente}   {$(A)$}
\bottomrule
\end{tabular}

\begin{tabular}{lll}%
opções     &     & exemplos                         \\ 
\midrule
\TOline{translation}{= from \#1 to \#2}{[translation=from A to B](E)}
\TOline{homothety}  {= center \#1 ratio \#2}{[homothety=center A ratio .5](E)}
\TOline{reflection} {= over \#1--\#2}{[reflection=over A--B](E)}
\TOline{symmetry }  {= center \#1}{[symmetry=center A](E)}
\TOline{projection }{= onto \#1--\#2}{[projection=onto A--B](E)}
\TOline{rotation }  {= center \#1 angle \#2}{[rotation=center O angle 30](E)}
\TOline{rotation in rad}{= center \#1 angle \#2}{[rotation in rad=center O angle pi/3](E)} 
\TOline{rotation with nodes}{= center \#1 from \#2 to \#3}{[center O from A to B](E)} 
\TOline{inversion}{= center \#1 through \#2}{[inversion =center O through A](E)} 
\TOline{inversion negative}{= center \#1 through \#2}{...} 
\bottomrule
\end{tabular}

\medskip
\emph{A imagem é apenas definida e não desenhada.}
\end{NewMacroBox} 

\subsubsection{\tkzname{translation}} 

\begin{tkzexample}[latex=7cm,small]
\begin{tikzpicture}[>=latex] 
 \tkzDefPoints{0/0/A,3/1/B,3/0/C}
 \tkzDefPointBy[translation= from B to A](C) 
 \tkzGetPoint{D} 
 \tkzDrawPoints[teal](A,B,C,D)
 \tkzLabelPoints[color=teal](A,B,C,D) 
 \tkzDrawSegments[orange,->](A,B D,C) 
\end{tikzpicture} 
\end{tkzexample}

\subsubsection{\tkzname{reflection} (orthogonal symmetry)} 

\begin{tkzexample}[latex=7cm,small] 
\begin{tikzpicture}[scale=.75]
 \tkzDefPoints{-2/-2/A,-1/-1/C,-4/2/D,-4/0/O}    
 \tkzDrawCircle(O,A)
 \tkzDefPointBy[reflection = over C--D](A)
 \tkzGetPoint{A'}
 \tkzDefPointBy[reflection = over C--D](O)
 \tkzGetPoint{O'}
 \tkzDrawCircle(O',A')
 \tkzDrawLine[add= .5 and .5](C,D)
 \tkzDrawPoints(C,D,O,O')
\end{tikzpicture}
\end{tkzexample}


\subsubsection{\tkzname{homothety} and \tkzname{projection}}

\begin{tkzexample}[latex=7cm,small] 
\begin{tikzpicture}
  \tkzDefPoints{0/1/A,5/3/B,3/4/C}
  \tkzDefLine[bisector](B,A,C) \tkzGetPoint{a} 
  \tkzDrawLine[add=0 and 0,color=magenta!50 ](A,a) 
  \tkzDefPointBy[homothety=center A ratio .5](a)   
  \tkzGetPoint{a'} 
  \tkzDefPointBy[projection = onto A--B](a') 
  \tkzGetPoint{k'}  
  \tkzDefPointBy[projection = onto A--B](a) 
  \tkzGetPoint{k} 
  \tkzDrawLines[add= 0 and .3](A,k A,C)   
  \tkzDrawSegments[blue](a',k' a,k) 
  \tkzDrawPoints(a,a',k,k',A)
  \tkzDrawCircles(a',k' a,k)   
  \tkzLabelPoints(a,a',k,A)
\end{tikzpicture}
\end{tkzexample}  


\subsubsection{\tkzname{projection}}
\begin{tkzexample}[latex=7cm,small] 
\begin{tikzpicture}[scale=1.25]  
 \tkzDefPoints{0/0/A,0/4/B}
 \tkzDefTriangle[pythagore](B,A) \tkzGetPoint{C}
 \tkzDefLine[bisector](B,C,A) \tkzGetPoint{c}
 \tkzInterLL(C,c)(A,B)  \tkzGetPoint{D}
 \tkzDefPointBy[projection=onto B--C](D) 
 \tkzGetPoint{G}
 \tkzInterLC(C,D)(D,A) \tkzGetPoints{E}{F}
 \tkzDrawPolygon(A,B,C)
 \tkzDrawSegment(C,D)
 \tkzDrawCircle(D,A)
 \tkzDrawSegment[new](D,G)
 \tkzMarkRightAngle[fill=orange!10](D,G,B)
 \tkzDrawPoints(A,C,F) \tkzLabelPoints(A,C,F)
 \tkzDrawPoints(B,D,E,G)   
 \tkzLabelPoints[above right](B,D,E)
  \tkzLabelPoints[above](G)
 \end{tikzpicture}
 \end{tkzexample} 

\subsubsection{\tkzname{symmetry} }
\begin{tkzexample}[latex=6cm,small] 
\begin{tikzpicture}[scale=1]
  \tkzDefPoints{2/-1/A,2/2/B,0/0/O}
  \tkzDefPointsBy[symmetry=center O](B,A){}
  \tkzDrawLine(A,A')
  \tkzDrawLine(B,B')
  \tkzMarkAngle[mark=s,arc=lll,
      size=1.5,mkcolor=red](A,O,B) 
  \tkzLabelAngle[pos=2,circle,draw,
    fill=blue!10,font=\scriptsize](A,O,B){$60^{\circ}$}
  \tkzDrawPoints(A,B,O,A',B') 
  \tkzLabelPoints(B,B')
  \tkzLabelPoints[below](A,O,A')  
\end{tikzpicture}   
\end{tkzexample}

\subsubsection{\tkzname{rotation} } 
\begin{tkzexample}[latex=7cm,small] 
\begin{tikzpicture}[scale=0.75] 
 \tkzDefPoints{0/0/A,5/0/B}
 \tkzDrawSegment(A,B)
 \tkzDefPointBy[rotation=center A angle 60](B) 
 \tkzGetPoint{C} 
 \tkzDefPointBy[symmetry=center C](A) 
 \tkzGetPoint{D} 
 \tkzDrawSegment(A,tkzPointResult) 
 \tkzDrawLine(B,D)
 \tkzDrawArc(A,B)(C) \tkzDrawArc(B,C)(A)
 \tkzDrawArc(C,D)(D)  
 \tkzMarkRightAngle(D,B,A) 
 \tkzDrawPoints(A,B) 
 \tkzLabelPoints(A,B)
 \tkzLabelPoints[above](C)
 \tkzLabelPoints[right](D)
\end{tikzpicture}  
\end{tkzexample}  

\subsubsection{\tkzname{rotation in radian}} 
\begin{tkzexample}[latex=6cm,small]
\begin{tikzpicture}
  \tkzDefPoint["$A$" left](1,5){A}
  \tkzDefPoint["$B$" right](4,3){B}
  \tkzDefPointBy[rotation in rad= center A angle pi/3](B)
  \tkzGetPoint{C}  
  \tkzDrawSegment(A,B)
  \tkzDrawPoints(A,B,C) 
  \tkzCompass(A,C)
  \tkzCompass(B,C) 
  \tkzLabelPoints(C)
\end{tikzpicture}
\end{tkzexample} 

\subsubsection{\tkzname{rotation with nodes}} 
\begin{tkzexample}[latex=6cm,small]
\begin{tikzpicture}
 \tkzDefPoint(0,0){O}    
 \tkzDefPoint(0:2){A} 
 \tkzDefPoint(40:2){B}  
 \tkzDefPoint(20:4){C}
 \tkzDrawLine(O,A)
 \tkzDefPointBy[rotation with nodes%
             =center O from A to B](C)  
 \tkzGetPoint{D}
\tkzDrawPoints(A,B,C,D)
\tkzDrawCircle(O,A)
\tkzLabelPoints(A,C,D)
\tkzLabelPoints[above](B)
\end{tikzpicture}
\end{tkzexample} 

\subsubsection{\tkzname{inversion }}

Inversão é o processo de transformar pontos para um conjunto correspondente de pontos conhecidos como seus pontos inversos. Dois pontos $P$ e $P'$ são ditos inversos com respeito a um círculo de inversão tendo centro de inversão $O$ e raio de inversão $k$ se $P'$ é o pé perpendicular da altura de $OQP$, onde $Q$ é um ponto no círculo tal que $OQ$ é perpendicular a $PQ$.\\
 A quantidade $k^2$ é conhecida como a potência do círculo (Coxeter 1969, p. 81).

(\url{https://mathworld.wolfram.com/Inversion.html})

Algumas proposições:
\begin{itemize}
\item O inverso de um círculo (que não passa pelo centro de inversão) é um círculo.
\item O inverso de um círculo que passa pelo centro de inversão é uma reta.
\item O inverso de uma reta (que não passa pelo centro de inversão) é um círculo que passa pelo centro de inversão.
\item Um círculo ortogonal ao círculo de inversão é seu próprio inverso.
\item Uma reta que passa pelo centro de inversão é seu próprio inverso.
\item Ângulos são preservados na inversão.
\end{itemize}

Explicação:

Diretamente
(Centro O potência=$k^2={OA}^2=OP \times OP'$)

\begin{tkzexample}[latex=6cm,small]
\begin{tikzpicture}[scale=.5]
  \tkzDefPoints{4/0/A,6/0/P,0/0/O}
  \tkzDefPointBy[inversion = center O through A](P)
  \tkzGetPoint{P'}
  \tkzDrawSegments(O,P)
  \tkzDrawCircle(O,A)
  \tkzLabelPoints[above right,font=\scriptsize](O,A,P,P')
  \tkzDrawPoints(O,A,P,P')
\end{tikzpicture}
\end{tkzexample} 

\begin{tkzexample}[latex=6cm,small]
\begin{tikzpicture}[scale=.5]
  \tkzDefPoints{4/0/A,6/0/P,0/0/O}
  \tkzDefLine[orthogonal=through P](O,P)
  \tkzGetPoint{L}
  \tkzDefLine[tangent from = P](O,A) \tkzGetPoints{R}{Q}
  \tkzDefPointBy[projection=onto O--A](Q) \tkzGetPoint{P'}
  \tkzDrawSegments(O,P O,A)
  \tkzDrawSegments[new](O,P O,Q P,Q Q,P')
  \tkzDrawCircle(O,A)
  \tkzDrawLines[add=1 and 0](P,L)
  \tkzLabelPoints[below,font=\scriptsize](O,P')
  \tkzLabelPoints[above right,font=\scriptsize](P,Q)
  \tkzDrawPoints(O,P) \tkzDrawPoints[new](Q,P')
  \tkzLabelSegment[above](O,Q){$k$}
  \tkzMarkRightAngles(A,P',Q P,Q,O)
  \tkzLabelCircle[above=.5cm,
      font=\scriptsize](O,A)(100){inversion circle}
  \tkzLabelPoint[left,font=\scriptsize](O){inversion center}
  \tkzLabelPoint[left,font=\scriptsize](L){polar}
\end{tikzpicture}
\end{tkzexample} 


\subsubsection{\tkzname{Inversão de retas} ex 1}
\begin{tkzexample}[latex=6cm,small]  
\begin{tikzpicture}[scale=.5]
\tkzDefPoints{0/0/O,3/0/I,4/3/P,6/-3/Q}
\tkzDrawCircle(O,I)
\tkzDefPointBy[projection= onto P--Q](O) \tkzGetPoint{A}
\tkzDefPointBy[inversion = center O through I](A)
\tkzGetPoint{A'}
\tkzDefPointBy[inversion = center O through I](P)
\tkzGetPoint{P'}
\tkzDefCircle[diameter](O,A')\tkzGetPoint{o}
\tkzDrawCircle[new](o,A')
\tkzDrawLines[add=.25 and .25,red](P,Q)
\tkzDrawLines[add=.25 and .25](O,A)
\tkzDrawSegments(O,P)
\tkzDrawPoints(A,P,O) \tkzDrawPoints[new](A',P')
\end{tikzpicture}
\end{tkzexample} 

\subsubsection{\tkzname{Inversão de retas} ex 2}
\begin{tkzexample}[latex=6cm,small]  
\begin{tikzpicture}[scale=.8]
\tkzDefPoints{0/0/O,3/0/I,3/2/P,3/-2/Q}
\tkzDrawCircle(O,I)
\tkzDefPointBy[projection= onto P--Q](O) \tkzGetPoint{A}
\tkzDefPointBy[inversion = center O through I](A)
\tkzGetPoint{A'}
\tkzDefPointBy[inversion = center O through I](P)
\tkzGetPoint{P'}
\tkzDefCircle[diameter](O,A')\tkzGetPoint{o}
\tkzDrawCircle[new](o,A')
\tkzDrawLines[add=.25 and .25,red](P,Q)
\tkzDrawLines[add=.25 and .25](O,A)
\tkzDrawSegments(O,P)
\tkzDrawPoints(A,P,O) \tkzDrawPoints[new](A',P')
\end{tikzpicture}
\end{tkzexample}

\subsubsection{\tkzname{Inversão de retas} ex 3}
\begin{tkzexample}[latex=6cm,small]  
\begin{tikzpicture}[scale=.8]
\tkzDefPoints{0/0/O,3/0/I,2/1/P,2/-2/Q}
\tkzDrawCircle(O,I)
\tkzDefPointBy[projection= onto P--Q](O) \tkzGetPoint{A}
\tkzDefPointBy[inversion = center O through I](A)
\tkzGetPoint{A'}
\tkzDefPointBy[inversion = center O through I](P)
\tkzGetPoint{P'}
\tkzDefCircle[diameter](O,A')
\tkzDrawCircle[new](I,A')
\tkzDrawLines[add=.25 and .75,red](P,Q)
\tkzDrawLines[add=.25 and .25](O,A')
\tkzDrawSegments(O,P')
\tkzDrawPoints(A,P,O) \tkzDrawPoints[new](A',P')
\end{tikzpicture}
\end{tkzexample}

\subsubsection{\tkzname{Inversão} de círculo e \tkzname{homotetia} }
\begin{tkzexample}[latex=7cm,small]
\begin{tikzpicture}[scale=.7]
\tkzDefPoints{0/0/O,3/2/A,2/1/P}
\tkzDefLine[tangent from = O](A,P) \tkzGetPoints{T}{X}
\tkzDefPointsBy[homothety = center O%
                ratio 1.25](A,P,T){}
\tkzInterCC(A,P)(A',P') \tkzGetPoints{C}{D}
\tkzCalcLength(A,P)
\tkzGetLength{rAP}
\tkzDefPointOnCircle[R=center A angle 190 radius \rAP]
\tkzGetPoint{M}
\tkzDefPointBy[inversion = center O through C](M)
\tkzGetPoint{M'}
\tkzDrawCircles[new](A,P A',P')
\tkzDrawCircle(O,C)
\tkzDrawLines[add=0 and .5](O,T' O,A' O,M' O,P')
\tkzDrawPoints(A,A',P,P',O,T,T',M,M')
\tkzLabelPoints(O,T,T',M,M')
\tkzLabelPoints[below](P,P')
\end{tikzpicture}
\end{tkzexample}


\subsubsection{\tkzname{Inversão} de triângulo em relação ao círculo inscrito}
\begin{tkzexample}[latex=6cm,small] 
\begin{tikzpicture}[scale=1]
\tkzDefPoints{0/0/A,5/1/B,3/6/C}
\tkzDefTriangleCenter[in](A,B,C) \tkzGetPoint{O} 
\tkzDefPointBy[projection= onto A--C](O) \tkzGetPoint{b}
\tkzDefPointBy[projection= onto A--C](O) \tkzGetPoint{b}
\tkzDefPointBy[projection= onto B--C](O) \tkzGetPoint{a}
\tkzDefPointBy[projection= onto A--B](O) \tkzGetPoint{c}
\tkzDefPointsBy[inversion = center O through b](a,b,c)%
                                             {Ia,Ib,Ic}
\tkzDefMidPoint(O,Ia) \tkzGetPoint{Ja}
\tkzDefMidPoint(O,Ib) \tkzGetPoint{Jb}
\tkzDefMidPoint(O,Ic) \tkzGetPoint{Jc}
\tkzInterCC(Ja,O)(Jb,O) \tkzGetPoints{O}{x}
\tkzInterCC(Ja,O)(Jc,O) \tkzGetPoints{y}{O}
\tkzInterCC(Jb,O)(Jc,O) \tkzGetPoints{O}{z}
\tkzDrawPolygon(A,B,C)
\tkzDrawCircle(O,b)\tkzDrawPoints(A,B,C,O)
\tkzDrawCircles[dashed,gray](Ja,y Jb,x Jc,z)
\tkzDrawArc[line width=1pt,orange,delta=0](Jb,x)(z)
\tkzDrawArc[line width=1pt,orange,delta=0](Jc,z)(y)
\tkzDrawArc[line width=1pt,orange,delta=0](Ja,y)(x)
\tkzLabelPoint[below](A){$A$}\tkzLabelPoint[above](C){$C$}
\tkzLabelPoint[right](B){$B$}\tkzLabelPoint[below](O){$O$}
\end{tikzpicture}
\end{tkzexample}

\subsubsection{\tkzname{Inversão}: círculo ortogonal com círculo de inversão}
O próprio círculo de inversão, círculos ortogonais a ele e retas que passam pelo centro de inversão são invariantes sob inversão. Se o círculo encontra o círculo de referência, esses pontos de interseção invariantes também estão no círculo inverso. Veja I e J na próxima figura.

\begin{tkzexample}[latex=5cm,small]
\begin{tikzpicture}[scale=1]
\tkzDefPoint(0,0){O}\tkzDefPoint(1,0){A}
\tkzDefPoint(-1.5,-1.5){z1} 
\tkzDefPoint(1.5,-1.25){z2} 
\tkzDefCircle[orthogonal through=z1 and z2](O,A)
\tkzGetPoint{c} 
\tkzDrawCircle[new](c,z1) 
\tkzDefPointBy[inversion =  center O through A](z1)
\tkzGetPoint{Z1} 
\tkzInterCC(O,A)(c,z1) \tkzGetPoints{I}{J}
\tkzDefPointBy[inversion =  center O through A](I)
\tkzGetPoint{I'}
\tkzDrawCircle(O,A)
\tkzDrawPoints(O,A,z1,z2) 
\tkzDrawPoints[new](c,Z1,I,J) 
\tkzLabelPoints(O,A,z1,z2,c,Z1,I,J)
\end{tikzpicture}
\end{tkzexample}



Para um exemplo mais complexo veja \tkzname{Pappus} \ref{pappus}

\subsubsection{\tkzname{Inversão negativa}}
É uma inversão seguida de uma simetria de centro $O$
\begin{tkzexample}[latex=5cm,small]  
\begin{tikzpicture}[scale=1.5]
  \tkzDefPoints{1/0/A,0/0/O}
  \tkzDefPoint(-1.5,-1.5){z1}
  \tkzDefPoint(0.35,-2){z2} 
  \tkzDefPointBy[inversion negative = center O through A](z1)
  \tkzGetPoint{Z1} 
  \tkzDefPointBy[inversion negative = center O through A](z2)
  \tkzGetPoint{Z2}
  \tkzDrawCircle(O,A)  
  \tkzDrawPoints[color=black, fill=red,size=4](Z1,Z2)    
  \tkzDrawSegments(z1,Z1 z2,Z2)
  \tkzDrawPoints[color=black, fill=red,size=4](O,z1,z2)
  \tkzLabelPoints[font=\scriptsize](O,A,z1,z2,Z1,Z2)  
\end{tikzpicture}
\end{tkzexample} 


\newpage
\subsection{Transformação de múltiplos pontos; \tkzcname{tkzDefPointsBy} }
Variante da macro anterior para definir múltiplas imagens.
Você deve fornecer os nomes das imagens como argumentos, ou indicar que os nomes das imagens são formados a partir dos nomes dos antecedentes, deixando o argumento vazio. 

\begin{tkzltxexample}[]
\tkzDefPointsBy[translation= from A to A'](B,C){}
\end{tkzltxexample}
As imagens são $B'$ e $C'$.

\begin{tkzltxexample}[]
\tkzDefPointsBy[translation= from A to A'](B,C){D,E}
\end{tkzltxexample}
As imagens são $D$ e $E$.

\begin{tkzltxexample}[]
\tkzDefPointsBy[translation= from A to A'](B)
\end{tkzltxexample}
A imagem é $B'$.
\begin{NewMacroBox}{tkzDefPointsBy}{\oarg{local opções}\parg{list of points}\marg{list of points}}%
\begin{tabular}{lll}%
argumentos &  exemplos  &                  \\ 
\midrule
\TAline{\parg{lista de pontos}\marg{lista de pts}}{(A,B)\{E,F\}}{$E$,$F$ imagens de $A$, $B$}   \\
\bottomrule
\end{tabular}

\medskip
Se a lista de imagens estiver vazia, então o nome da imagem é o nome do antecedente ao qual " ' " é adicionado.

\medskip
\begin{tabular}{lll}%
\toprule
opções     &     & exemplos                         \\
\midrule
\TOline{translation = from \#1 to \#2}{}{[translation=from A to B](E)\{\}}
\TOline{homothety = center \#1 ratio \#2}{}{[homothety=center A ratio .5](E)\{F\}}
\TOline{reflection = over \#1--\#2}{}{[reflection=over A--B](E)\{F\}}
\TOline{symmetry = center \#1}{}{[symmetry=center A](E)\{F\}}
\TOline{projection = onto \#1--\#2}{}{[projection=onto A--B](E)\{F\}}
\TOline{rotation = center \#1 angle \#2}{}{[rotation=center  angle 30](E)\{F\}}
\TOline{rotation in rad = center \#1 angle \#2}{}{por exemplo angle pi/3}
\TOline{rotation with nodes = center \#1 from \#2 to \#3}{}{[center O from A to B](E)\{F\}}
\TOline{inversion = center \#1 through \#2}{}{[inversion = center O through A](E)\{F\}}
\TOline{inversion negative = center \#1 through \#2}{}{...}
\bottomrule
\end{tabular}

\medskip
Os pontos são apenas definidos e não desenhados.
\end{NewMacroBox}

\subsubsection{\tkzname{Translação} de múltiplos pontos}
\begin{tkzexample}[latex=7cm,small]
\begin{tikzpicture}[>=latex] 
 \tkzDefPoints{0/0/A,3/0/B,3/1/A',1/2/C}
 \tkzDefPointsBy[translation= from A to A'](B,C){} 
 \tkzDrawPolygon(A,B,C)
 \tkzDrawPolygon[new](A',B',C')
 \tkzDrawPoints(A,B,C)
 \tkzDrawPoints[new](A',B',C') 
 \tkzLabelPoints(A,B,A',B')  
 \tkzLabelPoints[above](C,C')
 \tkzDrawSegments[color = gray,->,
              style=dashed](A,A' B,B' C,C') 
\end{tikzpicture}
\end{tkzexample}

\subsubsection{\tkzname{Simetria} de múltiplos pontos: um oval}

\begin{tkzexample}[latex=7cm,small]
\begin{tikzpicture}[scale=0.4]
  \tkzDefPoint(-4,0){I}
  \tkzDefPoint(4,0){J}
  \tkzDefPoint(0,0){O} 
  \tkzInterCC(J,O)(O,J) \tkzGetPoints{L}{H}
  \tkzInterCC(I,O)(O,I) \tkzGetPoints{K}{G} 
  \tkzInterLL(I,K)(J,H) \tkzGetPoint{M}
  \tkzInterLL(I,G)(J,L) \tkzGetPoint{N}
  \tkzDefPointsBy[symmetry=center J](L,H){D,E} 
  \tkzDefPointsBy[symmetry=center I](G,K){C,F}
  \begin{scope}[line style/.style = {very thin,teal}]
    \tkzDrawLines[add=1.5 and 1.5](I,K I,G J,H J,L) 
    \tkzDrawLines[add=.5 and .5](I,J) 
    \tkzDrawCircles(O,I I,O J,O) 
    \tkzDrawArc[delta=0,orange](N,D)(C) 
    \tkzDrawArc[delta=0,orange](M,F)(E) 
    \tkzDrawArc[delta=0,orange](J,E)(D) 
    \tkzDrawArc[delta=0,orange](I,C)(F) 
  \end{scope}   
\end{tikzpicture} 
\end{tkzexample}

\endinput
\section{Definindo pontos usando um vetor}

\subsection{\tkzcname{tkzDefPointWith}}
Existem várias possibilidades para criar pontos que atendam a certas condições vetoriais.
Isso pode ser feito com

\tkzcname{tkzDefPointWith}. O princípio geral é o seguinte: dois pontos são passados como argumentos, ou seja, um vetor. As diferentes opções permitem obter um novo ponto formando com o primeiro ponto (com algumas exceções) um vetor colinear ou um vetor ortogonal ao primeiro vetor. Em seguida, o comprimento é ou proporcional ao do primeiro, ou proporcional à unidade. Como este ponto é usado apenas temporariamente, ele não precisa ser nomeado imediatamente. O resultado está em \tkzname{tkzPointResult}. A macro \tkzNameMacro{tkzGetPoint} permite recuperar o ponto e nomeá-lo de forma diferente.

 Existem opções para definir a distância entre o ponto dado e o ponto obtido.
No caso geral, esta distância é a distância entre os 2 pontos dados como argumentos; se a opção for do tipo "normed", então a distância entre o ponto dado e o ponto obtido é 1 cm. Então a opção $K$ permite obter múltiplos.

\begin{NewMacroBox}{tkzDefPointWith}{\parg{pt1,pt2}}%
 É na verdade a definição de um ponto atendendo a condições vetoriais.

\medskip

\begin{tabular}{lll}%
\toprule
argumentos             & definição & explicação                         \\
\midrule
\TAline{(pt1,pt2)} {par de pontos}{o resultado é um ponto em \tkzname{tkzPointResult} } \\

\bottomrule
\end{tabular}

\medskip
No que segue, assume-se que o ponto é recuperado por \tkzNameMacro{tkzGetPoint\{C\}}

\begin{tabular}{lll}%
\toprule
opções             & exemplo & explicação                         \\
\midrule
\TOline{orthogonal}{[orthogonal](A,B)}{$AC=AB$ e $\overrightarrow{AC} \perp \overrightarrow{AB}$}
\TOline{orthogonal normed}{[orthogonal normed](A,B)}{$AC=1$ e $\overrightarrow{AC} \perp \overrightarrow{AB}$}
\TOline{linear}{[linear](A,B)}{$\overrightarrow{AC}=K \times \overrightarrow{AB}$}
\TOline{linear normed}{[linear normed](A,B)}{$AC=K$ e $\overrightarrow{AC}=k\times \overrightarrow{AB}$ }
\TOline{colinear= at \#1}{[colinear= at C](A,B)}{$\overrightarrow{CD}= \overrightarrow{AB}$ }
\TOline{colinear normed= at \#1}{[colinear normed= at C](A,B)}{$\overrightarrow{CD}= \overrightarrow{AB}$ }
\TOline{K}{[linear](A,B),K=2}{$\overrightarrow{AC}=2\times \overrightarrow{AB}$}
\end{tabular}
\end{NewMacroBox}

\subsubsection{Opção \tkzname{colinear at}, exemplo simples}
 $(\overrightarrow{AB}=\overrightarrow{CD})$
\begin{tkzexample}[latex=6cm,small]
\begin{tikzpicture}[scale=1.2,
   vect/.style={->,shorten >=1pt,>=latex'}]
  \tkzDefPoint(2,3){A}   \tkzDefPoint(4,2){B}
  \tkzDefPoint(0,1){C}
  \tkzDefPointWith[colinear=at C](A,B)
  \tkzGetPoint{D}
  \tkzDrawPoints[new](A,B,C,D)
  \tkzLabelPoints[above right=3pt](A,B,C,D)
  \tkzDrawSegments[vect](A,B C,D)
\end{tikzpicture}
\end{tkzexample}

\subsubsection{Opção \tkzname{colinear at}, exemplo complexo}
\begin{tkzexample}[vbox,small]
\begin{tikzpicture}[scale=.75]
\tkzDefPoints{0/0/B,3.6/0/C,1.5/4/A}
\tkzDefSpcTriangle[ortho](A,B,C){Ha,Hb,Hc}
\tkzDefTriangleCenter[ortho](A,B,C) \tkzGetPoint{H}
\tkzDefSquare(A,C) \tkzGetPoints{R}{S}
\tkzDefSquare(B,A) \tkzGetPoints{M}{N}
\tkzDefSquare(C,B) \tkzGetPoints{P}{Q}
\tkzDefPointWith[colinear= at M](A,S) \tkzGetPoint{A'}
\tkzDefPointWith[colinear= at P](B,N) \tkzGetPoint{B'}
\tkzDefPointWith[colinear= at Q](C,R) \tkzGetPoint{C'}
\tkzDefPointBy[projection=onto P--Q](Ha) \tkzGetPoint{Pa}
\tkzDrawPolygon[teal,thick](A,C,R,S)\tkzDrawPolygon[teal,thick](A,B,N,M)
\tkzDrawPolygon[teal,thick](C,B,P,Q)
\tkzDrawPoints[teal,size=2](A,B,C,Ha,Hb,Hc,A',B',C')
\tkzDrawSegments[ultra thin,red](M,A' A',S P,B' B',N Q,C' C',R B,S C,M C,N B,R A,P A,Q)
\tkzDrawSegments[ultra thin,teal, dashed](A,Ha B,Hb C,Hc)
\tkzDefPointBy[rotation=center A angle 90](S) \tkzGetPoint{S'}
\tkzDrawSegments[ultra thin,teal,dashed](B,S' A,S' A,A' M,S' B',Q P,C' M,S Ha,Pa)
\tkzDrawArc(A,S)(S')
\end{tikzpicture}
\end{tkzexample}

\subsubsection{Opção \tkzname{colinear at}}
Como usar $K$
\begin{tkzexample}[latex=7cm,small]
\begin{tikzpicture}[vect/.style={->,
               shorten >=1pt,>=latex'}]
  \tkzDefPoints{0/0/A,5/0/B,1/2/C}
  \tkzDefPointWith[colinear=at C](A,B)
  \tkzGetPoint{G}
  \tkzDefPointWith[colinear=at C, K=0.5](A,B)
  \tkzGetPoint{H}
  \tkzLabelPoints(A,B,C,G,H)
  \tkzDrawPoints(A,B,C,G,H)
  \tkzDrawSegments[vect](A,B C,H)
\end{tikzpicture}
\end{tkzexample}

\subsubsection{Opção \tkzname{colinear at} }
Com $K=\frac{\sqrt{2}}{2}$

\begin{tkzexample}[latex=6cm,small]
\begin{tikzpicture}[vect/.style={->,
            shorten >=1pt,>=latex'}]
 \tkzDefPoints{1/1/A,4/2/B,2/2/C}
 \tkzDefPointWith[colinear=at C,K=sqrt(2)/2](A,B)
 \tkzGetPoint{D}
 \tkzDrawPoints[color=red](A,B,C,D)
 \tkzDrawSegments[vect](A,B C,D)
\end{tikzpicture}
\end{tkzexample}

\subsubsection{Opção \tkzname{orthogonal}}
AB=AC já que $K=1$.
\begin{tkzexample}[latex=6cm,small]
\begin{tikzpicture}[scale=1.2,
  vect/.style={->,shorten >=1pt,>=latex'}]
  \tkzDefPoints{2/3/A,4/2/B}
   \tkzDefPointWith[orthogonal,K=1](A,B)
     \tkzGetPoint{C}
   \tkzDrawPoints[color=red](A,B,C)
   \tkzLabelPoints[right=3pt](B,C)
   \tkzLabelPoints[below=3pt](A)
   \tkzDrawSegments[vect](A,B A,C)
   \tkzMarkRightAngle(B,A,C)
\end{tikzpicture}
\end{tkzexample}



\subsubsection{Opção \tkzname{orthogonal}}
 Com $K=-1$
OK=OI já que $\lvert K \rvert=1$ então OI=OJ=OK.

\begin{tkzexample}[latex=7cm,small]
\begin{tikzpicture}[scale=.75]
  \tkzDefPoints{1/2/O,2/5/I}
  \tkzDefPointWith[orthogonal](O,I)
  \tkzGetPoint{J}
  \tkzDefPointWith[orthogonal,K=-1](O,I)
  \tkzGetPoint{K}
  \tkzDrawSegment(O,I)
  \tkzDrawSegments[->](O,J O,K)
  \tkzMarkRightAngles(I,O,J I,O,K)
  \tkzDrawPoints(O,I,J,K)
  \tkzLabelPoints(O,I,J,K)
\end{tikzpicture}
\end{tkzexample}

\subsubsection{Opção \tkzname{orthogonal} exemplo mais complicado}
\begin{tkzexample}[latex=7cm,small]
\begin{tikzpicture}[scale=.75]
  \tkzDefPoints{0/0/A,6/0/B}
  \tkzDefMidPoint(A,B)
    \tkzGetPoint{I}
  \tkzDefPointWith[orthogonal,K=-.75](B,A)
  \tkzGetPoint{C}
  \tkzInterLC(B,C)(B,I)
     \tkzGetPoints{D}{F}
  \tkzDuplicateSegment(B,F)(A,F)
  \tkzGetPoint{E}
  \tkzDrawArc[delta=10](F,E)(B)
  \tkzInterLC(A,B)(A,E)
    \tkzGetPoints{N}{M}
  \tkzDrawArc[delta=10](A,M)(E)
  \tkzDrawLines(A,B B,C A,F)
  \tkzCompass(B,F)
  \tkzDrawPoints(A,B,C,F,M,E)
  \tkzLabelPoints(A,B,C,F,M)
  \tkzLabelPoints[above](E)
\end{tikzpicture}
\end{tkzexample}

\subsubsection{Opções \tkzname{colinear} e \tkzname{orthogonal}}
\begin{tkzexample}[latex=7cm,small]
\begin{tikzpicture}[scale=1.2,
  vect/.style={->,shorten >=1pt,>=latex'}]
  \tkzDefPoints{2/1/A,6/2/B}
  \tkzDefPointWith[orthogonal,K=.5](A,B)
  \tkzGetPoint{C}
  \tkzDefPointWith[colinear=at C,K=.5](A,B)
  \tkzGetPoint{D}
  \tkzMarkRightAngle[fill=gray!20](B,A,C)
  \tkzDrawSegments[vect](A,B A,C C,D)
  \tkzDrawPoints(A,...,D)
\end{tikzpicture}
\end{tkzexample}

\subsubsection{Opção  \tkzname{orthogonal normed}}
 $K=1$ $AC=1$.

\begin{tkzexample}[latex=7cm,small]
\begin{tikzpicture}[scale=1.2,
  vect/.style={->,shorten >=1pt,>=latex'}]
  \tkzDefPoints{2/3/A,4/2/B}
  \tkzDefPointWith[orthogonal normed](A,B)
  \tkzGetPoint{C}
  \tkzDrawPoints[color=red](A,B,C)
  \tkzDrawSegments[vect](A,B A,C)
  \tkzMarkRightAngle[fill=gray!20](B,A,C)
\end{tikzpicture}
\end{tkzexample}

\subsubsection{Opção \tkzname{orthogonal normed} e K=2}
$K=2$ portanto $AC=2$.

\begin{tkzexample}[latex=7cm,small]
\begin{tikzpicture}[scale=1.2,
   vect/.style={->,shorten >=1pt,>=latex'}]
  \tkzDefPoints{2/3/A,5/1/B}
  \tkzDefPointWith[orthogonal normed,K=2](A,B)
  \tkzGetPoint{C}
  \tkzDrawPoints[color=red](A,B,C)
  \tkzDefCircle[R](A,2) \tkzGetPoint{a}
  \tkzDrawCircle(A,a)
  \tkzDrawSegments[vect](A,B A,C)
  \tkzMarkRightAngle[fill=gray!20](B,A,C)
  \tkzLabelPoints[above=3pt](A,B,C)
\end{tikzpicture}
\end{tkzexample}

\subsubsection{Opção \tkzname{linear}}
Aqui $K=0.5$.

Isso equivale a aplicar uma homotetia ou uma multiplicação de um vetor por um real. Aqui está o ponto médio de $[AB]$.

\begin{tkzexample}[latex=7cm,small]
\begin{tikzpicture}[scale=1.2]
  \tkzDefPoints{1/3/A,4/2/B}
  \tkzDefPointWith[linear,K=0.5](A,B)
  \tkzGetPoint{C}
  \tkzDrawPoints[color=red](A,B,C)
  \tkzDrawSegment(A,B)
  \tkzLabelPoints[above right=3pt](A,B,C)
\end{tikzpicture}
\end{tkzexample}

\subsubsection{Opção \tkzname{linear normed}}
No exemplo seguinte $AC=1$ e $C$ pertence a $(AB)$.

\begin{tkzexample}[latex=7cm,small]
\begin{tikzpicture}[scale=1.2]
 \tkzDefPoints{1/3/A,4/2/B}
 \tkzDefPointWith[linear normed](A,B)
 \tkzGetPoint{C}
 \tkzDrawPoints[color=red](A,B,C)
 \tkzDrawSegment(A,B)
 \tkzLabelSegment(A,C){$1$}
 \tkzLabelPoints[above right=3pt](A,B,C)
\end{tikzpicture}
\end{tkzexample}
%<--------------------------------------------------------------------------–>
%         tkzGetVectxy
%<--------------------------------------------------------------------------–>
\subsection{\tkzcname{tkzGetVectxy} }
Recuperando as coordenadas de um vetor.

\begin{NewMacroBox}{tkzGetVectxy}{\parg{$A,B$}\var{text}}%
Permite obter as coordenadas de um vetor.

\medskip
\begin{tabular}{lll}%
\toprule
argumentos    & exemplo & explicação      \\

\midrule

\TAline{(ponto)\{nome da macro\}} {\tkzcname{tkzGetVectxy}(A,B)\{V\}}{\tkzcname{Vx},\tkzcname{Vy}: coordenadas de $\overrightarrow{AB}$}
\end{tabular}
\end{NewMacroBox}

\subsubsection{Transferência de coordenadas com \tkzcname{tkzGetVectxy}}

\begin{tkzexample}[latex=7cm,small]
\begin{tikzpicture}
 \tkzDefPoints{0/0/O,1/1/A,4/2/B}
 \tkzGetVectxy(A,B){v}
 \tkzDefPoint(\vx,\vy){V}
 \tkzDrawSegment[->,color=red](O,V)
 \tkzDrawSegment[->,color=blue](A,B)
 \tkzDrawPoints(A,B,O)
 \tkzLabelPoints(A,B,O,V)
\end{tikzpicture}
\end{tkzexample}
\endinput

\section{Retas}

É claro que é essencial desenhar retas, mas antes que isso possa ser feito, é necessário poder definir certas retas particulares como mediatrizes, bissetrizes, paralelas ou mesmo perpendiculares. O princípio é determinar dois pontos na reta.

\subsection{Definição de retas}

\begin{NewMacroBox}{tkzDefLine}{\oarg{local opções}\parg{pt1,pt2} ou \parg{pt1,pt2,pt3}}%
O argumento é uma lista de dois ou três pontos. Dependendo do caso, a macro define um ou dois pontos necessários para obter a reta procurada. Deve ser usada a macro \tkzcname{tkzGetPoint} ou a macro \tkzcname{tkzGetPoints}.
Usei o termo "mediatriz" para designar a reta bissetriz perpendicular no meio de um segmento de reta.

\medskip
\begin{tabular}{lll}%
\toprule
argumentos           & exemplo & explicação                         \\
\midrule
\TAline{\parg{pt1,pt2}}{[mediator]\parg{A,B}}{mediatriz do segmento $[A,B]$}
\TAline{\parg{pt1,pt2,pt3}}{[bisector]\parg{A,B,C}} {bissetriz de $\widehat{ABC}$}
\TAline{\parg{pt1,pt2,pt3}}{[altitude]\parg{A,B,C}} {altura de $B$}
\TAline{\parg{pt1}}{[tangent at=A]\parg{O}} {tangente em $A$ no círculo de centro $O$}
\TAline{\parg{pt1,pt2}}{[tangent from=A]\parg{O,B}} {círculo de centro $O$ passando por $B$}
\end{tabular}

\medskip
\begin{tabular}{lll}%
\toprule
opções             & padrão & definição                         \\ 
\TOline{mediator}{}{bissetriz perpendicular de um segmento de reta}
\TOline{perpendicular=through\dots}{mediator}{perpendicular a uma reta passando por um ponto}
\TOline{orthogonal=through\dots}{mediator}{veja acima}
\TOline{parallel=through\dots}{mediator}{paralela a uma reta passando por um ponto}
\TOline{bisector}{mediator}{bissetriz de um ângulo definido por três pontos}
\TOline{bisector out}{mediator}{bissetriz do ângulo externo}
\TOline{symmedian}{mediator}{simediana de um vértice}
\TOline{altitude}{mediator}{altura de um vértice}
\TOline{euler}{mediator}{reta de euler de um triângulo}
\TOline{tangent at}{mediator}{tangente em um ponto de um círculo}
\TOline{tangent from}{mediator}{tangente de um ponto exterior}
\TOline{K}{1}{coeficiente para a reta perpendicular}
\TOline{normed}{false}{normaliza o segmento criado}
\end{tabular}
\end{NewMacroBox}  

\subsubsection{With \tkzname{mediator}}  
\begin{tkzexample}[latex=5 cm,small]
\begin{tikzpicture}[rotate=25]
 \tkzDefPoints{-2/0/A,1/2/B}
 \tkzDefLine[mediator](A,B)          \tkzGetPoints{C}{D}
 \tkzDefPointWith[linear,K=.75](C,D) \tkzGetPoint{D}
 \tkzDefMidPoint(A,B)                \tkzGetPoint{I}
 \tkzFillPolygon[color=teal!20](A,C,B,D)
 \tkzDrawSegments(A,B C,D)
 \tkzMarkRightAngle(B,I,C) 
 \tkzDrawSegments(D,B D,A)
 \tkzDrawSegments(C,B C,A)
\end{tikzpicture}
\end{tkzexample}  

\subsubsection{Um envelope com opção \tkzname{mediator}}
Baseado em uma figura de O. Reboux com pst-eucl por D Rodriguez.

\begin{tkzexample}[latex=7cm,small]
\begin{tikzpicture}[scale=.75]
   % necessary
\tkzInit[xmin=-6,ymin=-4,xmax=6,ymax=6]
\tkzClip
\tkzSetUpLine[thin,color=magenta]
\tkzDefPoint(0,0){O} 
\tkzDefPoint(132:4){A}
\tkzDefPoint(5,0){B}
\foreach \ang in {5,10,...,360}{%
 \tkzDefPoint(\ang:5){M}
 \tkzDefLine[mediator](A,M)
 \tkzGetPoints{x}{y}
 \tkzDrawLine[add= 3 and 3](x,y)}
\end{tikzpicture}
\end{tkzexample}


\subsubsection{Uma parábola com opção \tkzname{mediator}}
Baseado em uma figura de O. Reboux com pst-eucl por D Rodriguez.
Não é necessário nomear os dois pontos que definem a mediatriz.

\begin{tkzexample}[latex=8cm,small]
\begin{tikzpicture}[scale=.6]
\tkzInit[xmin=-6,ymin=-4,xmax=6,ymax=6] 
\tkzClip
\tkzSetUpLine[thin,color=teal]
\tkzDefPoint(0,0){O} 
\tkzDefPoint(132:5){A}
\tkzDefPoint(4,0){B}
\foreach \ang in {5,10,...,360}{%
 \tkzDefPoint(\ang:4){M}
 \tkzDefLine[mediator](A,M) 
 \tkzGetPoints{x}{y}
 \tkzDrawLine[add= 3 and 3](x,y)}
\end{tikzpicture}
\end{tkzexample}

\subsubsection{With opções \tkzname{bisector} and \tkzname{normed}} 
\begin{tkzexample}[latex=7 cm,small] 
\begin{tikzpicture}[rotate=25,scale=.75]
 \tkzDefPoints{0/0/C, 2/-3/A, 4/0/B}
 \tkzDefLine[bisector,normed](B,A,C) \tkzGetPoint{a}
 \tkzDrawLines[add= 0 and .5](A,B A,C)
 \tkzShowLine[bisector,gap=4,size=2,color=red](B,A,C)
 \tkzDrawLines[new,dashed,add= 0 and 3](A,a)
\end{tikzpicture}
\end{tkzexample} 

\subsubsection{Com opção \tkzname{parallel=through}} % (fold)
\label{ssub:parallel}
Livro de Lemas de Arquimedes proposição 1

\begin{tkzexample}[latex=7cm,small]
  \begin{tikzpicture}
    \tkzDefPoints{0/0/O_1,0/1/O_2,0/3/A}
    \tkzDefPoint(15:3){F}
    \tkzDefPointBy[symmetry=center O_1](F) 
    \tkzGetPoint{E}
    \tkzDefLine[parallel=through O_2](E,F) 
    \tkzGetPoint{x}   
    \tkzInterLC(x,O_2)(O_2,A) \tkzGetPoints{D}{C} 
    \tkzDrawCircles(O_1,A O_2,A)
    \tkzDrawSegments[new](O_1,A E,F C,D)
    \tkzDrawSegments[purple](A,E A,F)
    \tkzDrawPoints(A,O_1,O_2,E,F,C,D)
    \tkzLabelPoints(A,O_1,O_2,E,F,C,D)
  \end{tikzpicture}
\end{tkzexample}
% subsubsection parallel (end)

\subsubsection{With opção \tkzname{orthogonal} and \tkzname{parallel}}    
\begin{tkzexample}[latex=5 cm,small]
\begin{tikzpicture}
   \tkzDefPoints{-1.5/-0.25/A,1/-0.75/B,-0.7/1/C}
   \tkzDrawLine(A,B)
   \tkzLabelLine[pos=1.25,below left](A,B){$(d_1)$}
   \tkzDrawPoints(A,B,C)
   \tkzDefLine[orthogonal=through C](B,A) \tkzGetPoint{c}
   \tkzDrawLine(C,c) 
   \tkzLabelLine[pos=1.25,left](C,c){$(\delta)$}
   \tkzInterLL(A,B)(C,c) \tkzGetPoint{I}
   \tkzMarkRightAngle(C,I,B) 
   \tkzDefLine[parallel=through C](A,B) \tkzGetPoint{c'}
   \tkzDrawLine(C,c') 
   \tkzLabelLine[pos=1.25,below left](C,c'){$(d_2)$}
   \tkzMarkRightAngle(I,C,c')   
\end{tikzpicture}
\end{tkzexample}

\subsubsection{With opção  \tkzname{altitude}} % (fold)
\label{sub:altitude}
\begin{tkzexample}[latex=7 cm,small]
\begin{tikzpicture}
\tkzDefPoints{0/0/A,6/0/B,0.8/4/C}	
\tkzDefLine[altitude](A,B,C)     \tkzGetPoint{b}
\tkzDefLine[altitude](B,C,A)     \tkzGetPoint{c}
\tkzDefLine[altitude](B,A,C)     \tkzGetPoint{a}
\tkzDrawPolygon(A,B,C)
\tkzDrawPoints[blue](a,b,c)
\tkzDrawSegments[blue](A,a B,b C,c)
\tkzLabelPoints(A,B,c)
\tkzLabelPoints[above](C,a)
\tkzLabelPoints[above left](b)
\end{tikzpicture}
\end{tkzexample}

% subsection altitude (end)


\subsubsection{ With opção \tkzname{euler}} % (fold)
\label{sub:eulerline}
\begin{tkzexample}[latex=7 cm,small]
\begin{tikzpicture}[scale=.75]
\tkzDefPoints{0/0/A,6/0/B,0.8/4/C}			 
\tkzDefLine[euler](A,B,C)             
\tkzGetPoints{h}{e}
\tkzDefTriangleCenter[circum](A,B,C)  
\tkzGetPoint{o}
\tkzDrawPolygon[teal](A,B,C)
\tkzDrawPoints[red](A,B,C,h,e,o)
\tkzDrawLine[add= 2 and 2](h,e)
\tkzLabelPoints(A,B,C,h,e,o)
\tkzLabelPoints[above](C)
\end{tikzpicture}
\end{tkzexample}
% subsection eulerline (end)

\subsubsection{Tangente passando por um ponto no círculo \tkzname{tangent at}} 
\begin{tkzexample}[latex=7cm,small]
\begin{tikzpicture}[scale=.75]
  \tkzDefPoint(0,0){O}
  \tkzDefPoint(6,6){E}
  \tkzDefRandPointOn[circle=center O radius 3]
  \tkzGetPoint{A}
  \tkzDrawSegment(O,A)
  \tkzDrawCircle(O,A)
  \tkzDefLine[tangent at=A](O)
  \tkzGetPoint{h}
  \tkzDrawLine[add = 4 and 3](A,h)
  \tkzMarkRightAngle[fill=teal!30](O,A,h)
\end{tikzpicture}
\end{tkzexample}

\subsubsection{Escolha do ponto de contato com tangentes passando por um ponto externo opção \tkzname{tangent from}}

A tangente não é desenhada. Com opção \tkzname{at}, um ponto da tangente é dado por \tkzname{tkzPointResult}. Com opção \tkzname{from} você obtém dois pontos do círculo com \tkzname{tkzFirstPointResult} e \tkzname{tkzSecondPointResult}. Você pode escolher entre esses dois pontos comparando os ângulos formados com o ponto externo, o ponto de contato e o centro. Os dois ângulos possíveis têm direções diferentes. O ângulo no sentido anti-horário refere-se a \tkzname{tkzFirstPointResult}.

\begin{tkzexample}[latex=7cm,small]
\begin{tikzpicture}[scale=1,rotate=-30]
\tkzDefPoints{0/0/Q,0/2/A,6/-1/O}
\tkzDefLine[tangent from = O](Q,A)  
\tkzGetPoints{R}{S} 
\tkzInterLC[near](O,Q)(Q,A)         
\tkzGetPoints{M}{N}
\tkzDrawCircle(Q,M)
\tkzDrawSegments[new,add = 0 and .2](O,R O,S)
\tkzDrawSegments[gray](N,O R,Q S,Q)
\tkzDrawPoints(O,Q,R,S,M,N)
\tkzMarkAngle[gray,-stealth,size=1](O,R,Q)
\tkzFindAngle(O,R,Q)   \tkzGetAngle{an}
\tkzLabelAngle(O,R,Q){%
    $\pgfmathprintnumber{\an}^\circ$}
\tkzMarkAngle[gray,-stealth,size=1](O,S,Q)
\tkzFindAngle(O,S,Q)   \tkzGetAngle{an}
\tkzLabelAngle(O,S,Q){%
    $\pgfmathprintnumber{\an}^\circ$}
\tkzLabelPoints(Q,O,M,N,R)
\tkzLabelPoints[above,text=red](S)
\end{tikzpicture}
\end{tkzexample}

\subsubsection{Exemplo de tangentes passando por um ponto externo} 
\begin{tkzexample}[latex=7cm,small]
\begin{tikzpicture}[scale=.8] 
\tkzDefPoints{0/0/c,1/0/d,3/0/a0}
\def\tkzRadius{1}
\tkzDrawCircle(c,d) 
 \foreach \an in {0,10,...,350}{
  \tkzDefPointBy[rotation=center c angle \an](a0)  
  \tkzGetPoint{a}
  \tkzDefLine[tangent from = a](c,d) 
  \tkzGetPoints{e}{f}
  \tkzDrawLines(a,f a,e)
  \tkzDrawSegments(c,e c,f)}
\end{tikzpicture} 
\end{tkzexample}

\subsubsection{Exemplo of Andrew Mertz}

\begin{tkzexample}[latex=6cm,small]
 \begin{tikzpicture}[scale=.6] 
 \tkzDefPoint(100:8){A}\tkzDefPoint(50:8){B}  
 \tkzDefPoint(0,0){C} \tkzDefPoint(0,-4){R} 
 \tkzDrawCircle(C,R)
 \tkzDefLine[tangent from = A](C,R)  \tkzGetPoints{D}{E}
\tkzDefLine[tangent from = B](C,R)  \tkzGetPoints{F}{G}
 \tkzDrawSector[fill=teal!20,opacity=0.5](A,E)(D)
 \tkzFillSector[color=teal,opacity=0.5](B,G)(F)
 \end{tikzpicture}
\end{tkzexample}
\url{http://www.texemplo.net/tikz/exemplos/}  

\subsubsection{Desenhando uma tangente opção \tkzname{tangent from}}
\begin{tkzexample}[latex=6cm,small]
\begin{tikzpicture}[scale=.6] 
 \tkzDefPoint(0,0){B} 
 \tkzDefPoint(0,8){A} 
 \tkzDefSquare(A,B)
 \tkzGetPoints{C}{D}
 \tkzDrawPolygon(A,B,C,D)
 \tkzClipPolygon(A,B,C,D)
 \tkzDefPoint(4,8){F}
 \tkzDefPoint(4,0){E}
 \tkzDefPoint(4,4){Q}
 \tkzFillPolygon[color = green](A,B,C,D)
 \tkzDrawCircle[fill = orange](B,A)
 \tkzDrawCircle[fill = purple](E,B)  
 \tkzDefLine[tangent from = B](F,A)
 \tkzInterLL(F,tkzSecondPointResult)(C,D)
 \tkzInterLL(A,tkzPointResult)(F,E) 
 \tkzDrawCircle[fill = yellow](tkzPointResult,Q)  
 \tkzDefPointBy[projection= onto B--A](tkzPointResult)
 \tkzDrawCircle[fill = blue!50!black](tkzPointResult,A)
\end{tikzpicture}
\end{tkzexample}

\endinput
\section{Triângulos}

\subsection{Definição de triângulos \tkzcname{tkzDefTriangle}}
As seguintes macros permitirão que você defina ou construa um triângulo a partir de \tkzname{pelo menos} dois pontos.

 No momento, é possível definir os seguintes triângulos:
 \begin{itemize}
\item  \tkzname{two angles}  determina um triângulo com dois ângulos;
\item  \tkzname{equilateral}  determina um triângulo equilátero;
\item  \tkzname{isosceles right}  determina um triângulo isósceles retângulo;
\item \tkzname{half} determina um triângulo retângulo tal que a razão das medidas dos dois lados adjacentes ao ângulo reto é igual a $2$;
\item \tkzname{pythagore} determina um triângulo retângulo cujas medidas dos lados são proporcionais a 3, 4 e 5;
\item \tkzname{school} determina um triângulo retângulo cujos ângulos são 30, 60 e 90 graus;
\item \tkzname{golden} determina um triângulo retângulo tal que a razão das medidas dos dois lados adjacentes ao ângulo reto é igual a $\Phi=1.618034$, escolhi "triângulo dourado" como denominação porque vem do retângulo dourado e mantive a denominação "triângulo de ouro" ou "triângulo de Euclides" para o triângulo isósceles cujos ângulos na base são 72 graus;

\item  \tkzname{euclid} ou \tkzname{gold} para o triângulo de ouro; na versão anterior a opção era "euclide" com um "e".

\item \tkzname{cheops} determina um terceiro ponto tal que o triângulo é isósceles com medidas de lados proporcionais a $2$, $\Phi$ e $\Phi$.
\end{itemize}

\newpage
\begin{NewMacroBox}{tkzDefTriangle}{\oarg{opções locais}\parg{A,B}}%
Os pontos são ordenados porque o triângulo é construído seguindo a direção direta do círculo trigonométrico. Esta macro é usada em parceria com \tkzcname{tkzGetPoint} ou usando \tkzname{tkzPointResult} se não for necessário manter o nome.

\medskip
\begin{tabular}{lll}%
\toprule
opções             & padrão & definição                        \\
\midrule
\TOline{two angles= \#1 and \#2}{sem padrão}{triângulo conhecendo dois ângulos}
\TOline{equilateral} {equilateral}{triângulo equilátero }
\TOline{half} {equilateral}{B retângulo  $AB=2BC$ $AC$ hipotenusa }
\TOline{isosceles right} {equilateral}{triângulo isósceles retângulo }
\TOline{pythagore}{equilateral}{proporcional ao triângulo pitagórico 3-4-5}
\TOline{pythagoras}{equilateral}{mesmo que acima}
\TOline{egyptian}{equilateral}{mesmo que acima}
\TOline{school} {equilateral}{ângulos de 30, 60 e 90 graus }
\TOline{gold}{equilateral}{B retângulo e $AB/AC = \Phi$}
\TOline{euclid} {equilateral}{ângulos de 72, 72 e 36 graus, $A$ é o ápice}
\TOline{golden} {equilateral}{ângulos de 72, 72 e 36 graus, $C$ é o ápice}
\TOline{sublime} {equilateral}{ângulos de 72, 72 e 36 graus, $C$ é o ápice}
\TOline{cheops} {equilateral}{AC=BC, AC e BC são proporcionais a $2$ e $\Phi$.}
\TOline{swap} {false}{fornece o ponto simétrico em relação a $AB$}
\bottomrule
\end{tabular}

\medskip
\emph{\tkzcname{tkzGetPoint} permite armazenar o ponto, caso contrário \tkzname{tkzPointResult} permite uso imediato.}
\end{NewMacroBox}

\subsubsection{Opção \tkzname{equilateral}}
\begin{tkzexample}[latex=7 cm,small]
\begin{tikzpicture}
  \tkzDefPoint(0,0){A}
  \tkzDefPoint(4,0){B}
  \tkzDefTriangle[equilateral](A,B)
  \tkzGetPoint{C}
  \tkzDrawPolygons(A,B,C)
  \tkzDefTriangle[equilateral](B,A)
  \tkzGetPoint{D}
  \tkzDrawPolygon(B,A,D)
  \tkzMarkSegments[mark=s|](A,B B,C A,C A,D B,D)
\end{tikzpicture}
\end{tkzexample}


\subsubsection{Opção \tkzname{two angles}}
\begin{tkzexample}[latex=6 cm,small]
\begin{tikzpicture}
\tkzDefPoint(0,0){A}
\tkzDefPoint(5,0){B}
\tkzDefTriangle[two angles = 50 and 70](A,B)
\tkzGetPoint{C}
\tkzDrawSegment(A,B)
\tkzDrawPoints(A,B)
\tkzLabelPoints(A,B)
\tkzDrawSegments[new](A,C B,C)
\tkzDrawPoints[new](C)
\tkzLabelPoints[above,new](C)
\tkzLabelAngle[pos=1.4](B,A,C){$50^\circ$}
\tkzLabelAngle[pos=0.8](C,B,A){$70^\circ$}
\end{tikzpicture}
\end{tkzexample}

\subsubsection{Opção \tkzname{school}}
Os ângulos são 30, 60 e 90 graus.

\begin{tkzexample}[latex=6 cm,small]
\begin{tikzpicture}
  \tkzDefPoints{0/0/A,4/0/B}
  \tkzDefTriangle[school](A,B)
  \tkzGetPoint{C}
  \tkzMarkRightAngles(C,B,A)
  \tkzLabelAngle[pos=0.8](B,A,C){$30^\circ$}
  \tkzLabelAngle[pos=0.8](C,B,A){$90^\circ$}
  \tkzLabelAngle[pos=0.8](A,C,B){$60^\circ$}
  \tkzDrawSegments(A,B)
  \tkzDrawSegments[new](A,C B,C)
  \tkzLabelPoints(A,B)
  \tkzLabelPoints[above](C)
\end{tikzpicture}
\end{tkzexample}

\subsubsection{Opção \tkzname{pythagore}}
Este triângulo tem lados cujos comprimentos são proporcionais a 3, 4 e 5.

\begin{tkzexample}[latex=6 cm,small]
\begin{tikzpicture}
  \tkzDefPoints{0/0/A,4/0/B}
  \tkzDefTriangle[pythagore](A,B)
  \tkzGetPoint{C}
  \tkzDrawSegments(A,B)
  \tkzDrawSegments[new](A,C B,C)
  \tkzMarkRightAngles(A,B,C)
  \tkzDrawPoints[new](C)
  \tkzDrawPoints(A,B)
  \tkzLabelPoints[above](A,B)
  \tkzLabelPoints[new](C)
\end{tikzpicture}
\end{tkzexample}

\subsubsection{Opção \tkzname{pythagore} e \tkzname{swap}}
Este triângulo tem lados cujos comprimentos são proporcionais a 3, 4 e 5.

\begin{tkzexample}[latex=6 cm,small]
\begin{tikzpicture}
  \tkzDefPoints{0/0/A,4/0/B}
  \tkzDefTriangle[pythagore,swap](A,B)
  \tkzGetPoint{C}
  \tkzDrawSegments(A,B)
  \tkzDrawSegments[new](A,C B,C)
  \tkzMarkRightAngles(A,B,C)
  \tkzLabelPoint[above,new](C){$C$}
  \tkzDrawPoints[new](C)
  \tkzDrawPoints(A,B)
  \tkzLabelPoints(A,B)
\end{tikzpicture}
\end{tkzexample}

\subsubsection{Opção \tkzname{golden}}
\begin{tkzexample}[latex=6 cm,small]
\begin{tikzpicture}[scale=.8]
\tkzDefPoint(0,0){A} \tkzDefPoint(4,0){B}
\tkzDefTriangle[golden](A,B)\tkzGetPoint{C}
\tkzDefSpcTriangle[in,name=M](A,B,C){a,b,c}
\tkzDrawPolygon(A,B,C)
\tkzDrawPoints(A,B)
\tkzDrawSegment(C,Mc)
\tkzDrawPoints[new](C)
\tkzLabelPoints(A,B)
\tkzLabelPoints[above,new](C)
\end{tikzpicture}
\end{tkzexample}

\subsubsection{Opção \tkzname{euclid}}
\tkzimp{Euclid} e \tkzimp{golden} são idênticos, mas o segmento AB é uma base em um e um lado no outro.

\begin{tkzexample}[latex=7 cm,small]
\begin{tikzpicture}[scale=.75]
 \tkzDefPoint(0,0){A} \tkzDefPoint(4,0){B}
 \tkzDefTriangle[euclid](A,B)\tkzGetPoint{C}
 \tkzDrawPolygon(A,B,C)
 \tkzDrawPoints(A,B,C)
 \tkzLabelPoints(C)
 \tkzLabelPoints[above](A,B)
 \tkzLabelAngle[pos=0.8](A,B,C){$72^\circ$}
 \tkzLabelAngle[pos=0.8](B,C,A){$72^\circ$}
 \tkzLabelAngle[pos=0.8](C,A,B){$36^\circ$}
\end{tikzpicture}
\end{tkzexample}

\subsubsection{Opção \tkzname{isosceles right}}
\begin{tkzexample}[latex=7 cm,small]
\begin{tikzpicture}
  \tkzDefPoint(0,0){A}
  \tkzDefPoint(4,0){B}
  \tkzDefTriangle[isosceles right](A,B)
  \tkzGetPoint{C}
  \tkzDrawPolygons(A,B,C)
  \tkzDrawPoints(A,B,C)
  \tkzMarkRightAngles(A,C,B)
  \tkzLabelPoints(A,B)
  \tkzLabelPoints[above](C)
\end{tikzpicture}
\end{tkzexample}

\subsubsection{Opção \tkzname{gold} }
\begin{tkzexample}[latex=6 cm,small]
\begin{tikzpicture}
 \tkzDefPoints{0/0/A,4/0/B}
 \tkzDefTriangle[gold](A,B)
 \tkzGetPoint{C}
 \tkzDrawPolygon(A,B,C)
 \tkzDrawPoints(A,B,C)
 \tkzLabelPoints[above](A,B)
 \tkzLabelPoints[below](C)
 \tkzMarkRightAngle(A,B,C)
 \tkzText(0,-2){$\dfrac{AC}{AB}=\varphi$}
\end{tikzpicture}
\end{tkzexample}

\clearpage
\subsection{Triângulos específicos com \tkzcname{tkzDefSpcTriangle}}

Os centros de alguns triângulos foram definidos na seção "pontos", aqui é uma questão de determinar os três vértices de triângulos específicos.

\begin{NewMacroBox}{tkzDefSpcTriangle}{\oarg{opções locais}\parg{p1,p2,p3}\marg{r1,r2,r3}}
A ordem dos pontos é importante! p1p2p3 define um triângulo então o resultado é um triângulo cujos vértices têm como referência uma combinação com \tkzname{name} e r1,r2, r3. Se \tkzname{name} estiver vazio então as referências são r1,r2 e r3.

\medskip
\begin{tabular}{lll}%
\toprule
opções             & padrão & definição                        \\
\midrule
\TOline{orthic} {centroid}{determinado pelos pontos finais das alturas ...}
\TOline{centroid ou medial}{centroid}{interseção das três medianas do triângulo}
\TOline{in ou incentral}{centroid}{determinado com as bissetrizes}
\TOline{ex ou excentral} {centroid}{determinado com os excentros}
\TOline{extouch}{centroid}{formado pelos pontos de tangência com os excírculos}
\TOline{intouch ou contact} {centroid}{formado pelos pontos de tangência do incírculo}
\TOline{} {}{cada um dos vértices}
\TOline{euler} {centroid}{formado pelos pontos de Euler no círculo de nove pontos}
\TOline{symmedial} {centroid}{pontos de interseção das simedianas}
\TOline{tangential}{centroid}{formado pelas retas tangentes ao circuncírculo}
\TOline{feuerbach} {centroid}{formado pelos pontos de tangência do círculo de nove ...}
\TOline{} {} {pontos com os excírculos}
\TOline{name} {vazio}{usado para nomear os vértices}
\midrule
\end{tabular}
\end{NewMacroBox}

\subsubsection{Como nomear os vértices}

Com \tkzcname{tkzDefSpcTriangle[medial,name=M](A,B,C)\{\_A,\_B,\_C\}} você obtém três vértices nomeados $M_A$, $M_B$ e $M_C$.

Com \tkzcname{tkzDefSpcTriangle[medial](A,B,C)\{a,b,c\}} você obtém três vértices nomeados e rotulados $a$, $b$ e $c$.

Possível \tkzcname{tkzDefSpcTriangle[medial,name=M\_](A,B,C)\{A,B,C\}} você obtém três vértices nomeados $M_A$, $M_B$ e $M_C$.

\subsection{Opção \tkzname{medial} ou \tkzname{centroid} }
O centroide geométrico dos vértices do polígono de um triângulo é o ponto $G$ (às vezes também denotado $M$) que é também a interseção das três medianas do triângulo. O ponto é, portanto, às vezes chamado de ponto médio. O centroide está sempre no interior do triângulo.
\\

\href{http://mathworld.wolfram.com/TriangleCentroid.html}{Weisstein, Eric W. "Centroid triangle" From MathWorld--A Wolfram Web Resource.}

No exemplo seguinte, obtemos o círculo de Euler que passa pelos pontos previamente definidos.

\begin{tkzexample}[latex=7cm,small]
  \begin{tikzpicture}[rotate=90,scale=.75]
   \tkzDefPoints{0/0/A,6/0/B,0.8/4/C}
   \tkzDefTriangleCenter[centroid](A,B,C)
   \tkzGetPoint{M}
   \tkzDefSpcTriangle[medial,name=M](A,B,C){_A,_B,_C}
   \tkzDrawPolygon(A,B,C)
   \tkzDrawSegments[dashed,new](A,M_A B,M_B C,M_C)
   \tkzDrawPolygon[new](M_A,M_B,M_C)
   \tkzDrawPoints(A,B,C)
   \tkzDrawPoints[new](M,M_A,M_B,M_C)
   \tkzLabelPoints[above](B)
   \tkzLabelPoints[below](A,C,M_B)
   \tkzLabelPoints[right](M_C)
   \tkzLabelPoints[left](M_A)
   \tkzLabelPoints[font=\scriptsize](M)
  \end{tikzpicture}
\end{tkzexample}

\subsubsection{Opção \tkzname{in} ou \tkzname{incentral} }

O triângulo incentral é o triângulo cujos vértices são determinados pelas
interseções das bissetrizes dos ângulos do triângulo de referência com os
respectivos lados opostos.
\\
\href{http://mathworld.wolfram.com/ContactTriangle.html}{Weisstein, Eric W. "Incentral triangle" From MathWorld--A Wolfram Web Resource.}


\begin{tkzexample}[latex=7cm,small]
\begin{tikzpicture}[scale=1]
  \tkzDefPoints{ 0/0/A,5/0/B,2/3/C}
  \tkzDefSpcTriangle[in,name=I](A,B,C){_a,_b,_c}
  \tkzDefCircle[in](A,B,C) \tkzGetPoints{I}{a}
  \tkzDrawCircle(I,a)
  \tkzDrawPolygon(A,B,C)
  \tkzDrawPolygon[new](I_a,I_b,I_c)
  \tkzDrawSegments[dashed,new](A,I_a B,I_b C,I_c)
  \tkzDrawPoints(A,B,C,I,I_a,I_b,I_c)
  \tkzLabelPoints[below](A,B,I_c)
  \tkzLabelPoints[above left](I_b)
  \tkzLabelPoints[above right](C,I_a)
\end{tikzpicture}
\end{tkzexample}

\subsubsection{Opção \tkzname{ex} ou \tkzname{excentral} }

O triângulo excentral de um triângulo $ABC$ é o triângulo $J_aJ_bJ_c$ com vértices correspondentes aos excentros de $ABC$.

\begin{tkzexample}[latex=7cm,small]
\begin{tikzpicture}[scale=.6]
 \tkzDefPoints{0/0/A,6/0/B,0.8/4/C}
 \tkzDefSpcTriangle[excentral,name=J](A,B,C){_a,_b,_c}
 \tkzDefSpcTriangle[extouch,name=T](A,B,C){_a,_b,_c}
 \tkzDrawPolygon(A,B,C)
 \tkzDrawPolygon[new](J_a,J_b,J_c)
 \tkzClipBB
 \tkzDrawPoints(A,B,C)
 \tkzDrawPoints[new](J_a,J_b,J_c)
 \tkzLabelPoints(A,B,C)
 \tkzLabelPoints[new](J_b,J_c)
 \tkzLabelPoints[new,above](J_a)
 \tkzDrawCircles[gray](J_a,T_a J_b,T_b J_c,T_c)
\end{tikzpicture}
\end{tkzexample}


\subsubsection{Opção \tkzname{intouch} ou \tkzname{contact}}
O triângulo de contato de um triângulo $ABC$, também chamado de triângulo intouch, é o triângulo formado pelos pontos de tangência do incírculo de $ABC$ com $ABC$.\\
\href{http://mathworld.wolfram.com/ContactTriangle.html}{Weisstein, Eric W. "Contact triangle" From MathWorld--A Wolfram Web Resource.}

Obtemos as interseções das bissetrizes com os lados.
\begin{tkzexample}[latex=7cm,small]
\begin{tikzpicture}[scale=.75]
 \tkzDefPoints{0/0/A,6/0/B,0.8/4/C}
 \tkzDefSpcTriangle[intouch,name=X](A,B,C){_a,_b,_c}
 \tkzInCenter(A,B,C)\tkzGetPoint{I}
 \tkzDefCircle[in](A,B,C) \tkzGetPoints{I}{i}
 \tkzDrawCircle(I,i)
 \tkzDrawPolygon(A,B,C)
 \tkzDrawPolygon[new](X_a,X_b,X_c)
 \tkzDrawPoints(A,B,C)
 \tkzDrawPoints[new](X_a,X_b,X_c)
 \tkzLabelPoints[right](X_a)
 \tkzLabelPoints[left](X_b)
 \tkzLabelPoints[above](C)
 \tkzLabelPoints[below](A,B,X_c)
\end{tikzpicture}
\end{tkzexample}

\subsubsection{Opção \tkzname{extouch}}
O triângulo extouch $T_aT_bT_c$ é o triângulo formado pelos pontos de tangência de um triângulo $ABC$ com seus excírculos $J_a$, $J_b$ e $J_c$. Os pontos $T_a$, $T_b$ e $T_c$ também podem ser construídos como os pontos que bissectam o perímetro de $A_1A_2A_3$ começando em $A$, $B$ e $C$.\\
\href{http://mathworld.wolfram.com/ExtouchTriangle.html}{Weisstein, Eric W. "Extouch triangle" From MathWorld--A Wolfram Web Resource.}

Obtemos os pontos de contato dos círculos ex-inscritos, bem como o triângulo formado pelos centros dos círculos ex-inscritos.

\begin{tkzexample}[latex=8cm,small]
\begin{tikzpicture}[scale=.7]
\tkzDefPoints{0/0/A,6/0/B,0.8/4/C}
\tkzDefSpcTriangle[excentral,
                 name=J](A,B,C){_a,_b,_c}
\tkzDefSpcTriangle[extouch,
                  name=T](A,B,C){_a,_b,_c}
\tkzDefTriangleCenter[nagel](A,B,C)
\tkzGetPoint{N_a}
\tkzDefTriangleCenter[centroid](A,B,C)
\tkzGetPoint{G}
\tkzDrawPoints[new](J_a,J_b,J_c)
\tkzClipBB \tkzShowBB
\tkzDrawCircles[gray](J_a,T_a J_b,T_b J_c,T_c)
\tkzDrawLines[add=1 and 1](A,B B,C C,A)
\tkzDrawSegments[new](A,T_a B,T_b C,T_c)
\tkzDrawSegments[new](J_a,T_a J_b,T_b J_c,T_c)
\tkzDrawPolygon(A,B,C)
\tkzDrawPolygon[new](T_a,T_b,T_c)
\tkzDrawPoints(A,B,C,N_a)
\tkzDrawPoints[new](T_a,T_b,T_c)
\tkzLabelPoints[below left](A)
\tkzLabelPoints[below](N_a,B)
\tkzLabelPoints[above](C)
\tkzLabelPoints[new,below left](T_b)
\tkzLabelPoints[new,below right](T_c)
\tkzLabelPoints[new,right=6pt](T_a)
\tkzMarkRightAngles[fill=gray!15](J_a,T_a,B
 J_b,T_b,C J_c,T_c,A)
\end{tikzpicture}
\end{tkzexample}

\subsubsection{Opção \tkzname{orthic}}

Dado um triângulo $ABC$, o triângulo $H_AH_BH_C$ cujos vértices são pontos finais das alturas de cada um dos vértices de ABC é chamado de triângulo órtico, ou às vezes triângulo de altitude. As três linhas $AH_A$, $BH_B$ e $CH_C$ são concorrentes no ortocentro H de ABC.

\begin{tkzexample}[latex=7cm,small]
\begin{tikzpicture}[scale=.75]
\tkzDefPoints{1/5/A,0/0/B,7/0/C}
 \tkzDefSpcTriangle[orthic](A,B,C){H_A,H_B,H_C}
 \tkzDefTriangleCenter[ortho](B,C,A)
 \tkzGetPoint{H}
 \tkzDefPointWith[orthogonal,normed](H_A,B)
 \tkzGetPoint{a}
 \tkzDrawSegments[new](A,H_A B,H_B C,H_C)
 \tkzMarkRightAngles[fill=gray!20,
         opacity=.5](A,H_A,C B,H_B,A C,H_C,A)
 \tkzDrawPolygon[fill=teal!20,opacity=.3](A,B,C)
 \tkzDrawPoints(A,B,C)
 \tkzDrawPoints[new](H_A,H_B,H_C)
 \tkzDrawPolygon[new,fill=orange!20,
                opacity=.3](H_A,H_B,H_C)
 \tkzLabelPoints(C)
 \tkzLabelPoints[left](B)
 \tkzLabelPoints[above](A)
 \tkzLabelPoints[new](H_A)
 \tkzLabelPoints[new,above left](H_C)
 \tkzLabelPoints[new,above right](H_B,H)
\end{tikzpicture}
\end{tkzexample}

\subsubsection{Opção \tkzname{feuerbach}}
O triângulo de Feuerbach é o triângulo formado pelos três pontos de tangência do círculo de nove pontos com os excírculos.\\
\href{http://mathworld.wolfram.com/FeuerbachTriangle.html}{Weisstein, Eric W. "Feuerbach triangle" From MathWorld--A Wolfram Web Resource.}

 Os pontos de tangência definem o triângulo de Feuerbach.

\begin{tkzexample}[latex=8cm,small]
\begin{tikzpicture}[scale=1]
  \tkzDefPoint(0,0){A}
  \tkzDefPoint(3,0){B}
  \tkzDefPoint(0.5,2.5){C}
  \tkzDefCircle[euler](A,B,C) \tkzGetPoint{N}
  \tkzDefSpcTriangle[feuerbach,
                       name=F](A,B,C){_a,_b,_c}
  \tkzDefSpcTriangle[excentral,
                       name=J](A,B,C){_a,_b,_c}
  \tkzDefSpcTriangle[extouch,
                        name=T](A,B,C){_a,_b,_c}
  \tkzLabelPoints[below left](J_a,J_b,J_c)
  \tkzClipBB \tkzShowBB
  \tkzDrawCircle[purple](N,F_a)
  \tkzDrawPolygon(A,B,C)
  \tkzDrawPolygon[new](F_a,F_b,F_c)
  \tkzDrawCircles[gray](J_a,F_a J_b,F_b J_c,F_c)
  \tkzDrawPoints[blue](J_a,J_b,J_c,%
          F_a,F_b,F_c,A,B,C)
  \tkzLabelPoints(A,B,F_c)
  \tkzLabelPoints[above](C)
  \tkzLabelPoints[right](F_a)
  \tkzLabelPoints[left](F_b)
\end{tikzpicture}
\end{tkzexample}

\subsubsection{Opção   \tkzname{tangential}}
O triângulo tangencial é o triângulo $T_aT_bT_c$ formado pelas linhas tangentes ao circuncírculo de um dado triângulo $ABC$ em seus vértices. É, portanto, o triângulo antipedal de $ABC$ em relação ao circuncentro $O$.\\
\href{http://mathworld.wolfram.com/TangentialTriangle.html}{Weisstein, Eric W. "Tangential Triangle." From MathWorld--A Wolfram Web Resource. }


\begin{tkzexample}[latex=8cm,small]
\begin{tikzpicture}[scale=.5,rotate=80]
  \tkzDefPoints{0/0/A,6/0/B,1.8/4/C}
  \tkzDefSpcTriangle[tangential,
    name=T](A,B,C){_a,_b,_c}
  \tkzDrawPolygon(A,B,C)
  \tkzDrawPolygon[new](T_a,T_b,T_c)
  \tkzDrawPoints(A,B,C)
  \tkzDrawPoints[new](T_a,T_b,T_c)
  \tkzDefCircle[circum](A,B,C)
  \tkzGetPoint{O}
  \tkzDrawCircle(O,A)
  \tkzLabelPoints(A)
  \tkzLabelPoints[above](B)
  \tkzLabelPoints[left](C)
  \tkzLabelPoints[new](T_b,T_c)
  \tkzLabelPoints[new,left](T_a)
\end{tikzpicture}
\end{tkzexample}

\subsubsection{Opção   \tkzname{euler}}
O triângulo de Euler de um triângulo $ABC$ é o triângulo $E_AE_BE_C$ cujos vértices são os pontos médios dos segmentos que unem o ortocentro $H$ com os respectivos vértices. Os vértices do triângulo são conhecidos como os pontos de Euler e ficam no círculo de nove pontos.
\\
\href{https://mathworld.wolfram.com/EulerTriangle.html}{Weisstein, Eric W. "Euler Triangle." From MathWorld--A Wolfram Web Resource.}

\begin{tkzexample}[latex=7cm,small]
\begin{tikzpicture}[rotate=90,scale=1.25]
 \tkzDefPoints{0/0/A,6/0/B,0.8/4/C}
 \tkzDefSpcTriangle[medial,
     name=M](A,B,C){_A,_B,_C}
 \tkzDefTriangleCenter[euler](A,B,C)
     \tkzGetPoint{N} % I= N nine points
 \tkzDefTriangleCenter[ortho](A,B,C)
        \tkzGetPoint{H}
 \tkzDefMidPoint(A,H) \tkzGetPoint{E_A}
 \tkzDefMidPoint(C,H) \tkzGetPoint{E_C}
 \tkzDefMidPoint(B,H) \tkzGetPoint{E_B}
 \tkzDefSpcTriangle[ortho,name=H](A,B,C){_A,_B,_C}
 \tkzDrawPolygon(A,B,C)
 \tkzDrawCircle(N,E_A)
 \tkzDrawSegments[new](A,H_A B,H_B C,H_C)
 \tkzDrawPoints(A,B,C,N,H)
 \tkzDrawPoints[red](M_A,M_B,M_C)
 \tkzDrawPoints[blue]( H_A,H_B,H_C)
 \tkzDrawPoints[green](E_A,E_B,E_C)
 \tkzAutoLabelPoints[center=N,font=\scriptsize]%
(A,B,C,M_A,M_B,M_C,H_A,H_B,H_C,E_A,E_B,E_C)
\tkzLabelPoints[font=\scriptsize](H,N)
\tkzMarkSegments[mark=s|,size=3pt,
  color=blue,line width=1pt](B,E_B E_B,H)
   \tkzDrawPolygon[color=cyan](M_A,M_B,M_C)
\end{tikzpicture}
\end{tkzexample}

\subsubsection{Opção  \tkzname{euler} e Opção  \tkzname{orthic}}
\begin{tkzexample}[vbox,small]
  \begin{tikzpicture}[scale=1.25]
    \tkzDefPoints{0/0/A,6/0/B,0.8/4/C}
    \tkzDefSpcTriangle[euler,name=E](A,B,C){a,b,c}
    \tkzDefSpcTriangle[orthic,name=H](A,B,C){a,b,c}
    \tkzDefExCircle(A,B,C) \tkzGetPoints{I}{i}
    \tkzDefExCircle(C,A,B) \tkzGetPoints{J}{j}
    \tkzDefExCircle(B,C,A) \tkzGetPoints{K}{k}
    \tkzDrawPoints[orange](I,J,K)
    \tkzLabelPoints[font=\scriptsize](A,B,C,I,J,K)
    \tkzClipBB
    \tkzInterLC(I,C)(I,i) \tkzGetSecondPoint{Fc}
    \tkzInterLC(J,B)(J,j) \tkzGetSecondPoint{Fb}
    \tkzInterLC(K,A)(K,k) \tkzGetSecondPoint{Fa}
    \tkzDrawLines[add=1.5 and 1.5](A,B A,C B,C)
    \tkzDefCircle[euler](A,B,C) \tkzGetPoints{E}{e}
    \tkzDrawCircle[orange](E,e)
    \tkzDrawSegments[orange](E,I E,J E,K)
    \tkzDrawSegments[dashed](A,Ha B,Hb C,Hc)
    \tkzDrawCircles(J,j I,i K,k)
    \tkzDrawPoints(A,B,C)
    \tkzDrawPoints[orange](E,I,J,K,Ha,Hb,Hc,Ea,Eb,Ec,Fa,Fb,Fc)
    \tkzLabelPoints[font=\scriptsize](E,Ea,Eb,Ec,Ha,Hb,Hc,Fa,Fb,Fc)
  \end{tikzpicture}
\end{tkzexample}

\subsubsection{Opção \tkzname{symmedial}}
O triângulo simediano $K_AK_BK_C$ é o triângulo cujos vértices são os pontos de interseção das simedianas com o triângulo de referência $ABC$.

\begin{tkzexample}[latex=7cm,small]
\begin{tikzpicture}
\tkzDefPoint(0,0){A}
\tkzDefPoint(5,0){B}
\tkzDefPoint(.75,4){C}
\tkzDefTriangleCenter[symmedian](A,B,C)\tkzGetPoint{K}
\tkzDefSpcTriangle[symmedial,name=K_](A,B,C){A,B,C}
\tkzDrawPolygon(A,B,C)
\tkzDrawSegments[new](A,K_A B,K_B C,K_C)
\tkzDrawPoints(A,B,C,K,K_A,K_B,K_C)
\tkzLabelPoints(A,B,K,K_C)
\tkzLabelPoints[above](C)
\tkzLabelPoints[right](K_A)
\tkzLabelPoints[left](K_B)
\end{tikzpicture}
\end{tkzexample}

\subsection{Permutação de dois pontos de um triângulo}

\begin{NewMacroBox}{tkzPermute}{\parg{$pt1$,$pt2$,$pt3$}}%
\begin{tabular}{lll}%
argumentos             & exemplo & explicação                         \\
\midrule
\TAline{(pt1,pt2,pt3)} {\tkzcname{tkzPermute}(A,B,C)}{$A$, $\widehat{B,A,C}$ permanecem inalterados, $B$, $C$ trocam de posição}
\midrule
\end{tabular}

\medskip
\emph{O triângulo permanece inalterado.}
\end{NewMacroBox}

\subsubsection{Modificação do triângulo \tkzname{school}}
Este triângulo é construído a partir do segmento $[AB]$ em $[A,x)$.

Se quisermos que o segmento $[AC]$ esteja em $[A,x)$, basta trocar $B$ e $C$.

\begin{tkzexample}[latex=7cm,small]
\begin{tikzpicture}
  \tkzDefPoints{0/0/A,4/0/B,6/0/x}
  \tkzDefTriangle[school](A,B)
  \tkzGetPoint{C}
  \tkzPermute(A,B,C)
  \tkzDrawSegments(A,B C,x)
  \tkzDrawSegments(A,C B,C)
  \tkzDrawPoints(A,B,C)
  \tkzLabelPoints(A,C,x)
  \tkzLabelPoints[above](B)
  \tkzMarkRightAngles(C,B,A)
\end{tikzpicture}
\end{tkzexample}

Observação: Apenas o primeiro ponto permanece inalterado. A ordem dos dois últimos parâmetros não é importante.

\endinput

\section{Definição de polígonos}
\subsection{Definindo os pontos de um quadrado} \label{def_square}
Já vimos as definições de alguns triângulos. Vejamos agora as definições de alguns quadriláteros e polígonos regulares.

\begin{NewMacroBox}{tkzDefSquare}{\parg{pt1,pt2}}%
O quadrado é definido no sentido direto. A partir de dois pontos, dois pontos adicionais são obtidos de forma que os quatro tomados em ordem formam um quadrado. O quadrado é definido no sentido direto. \\Os resultados estão em \tkzname{tkzFirstPointResult} and \tkzname{tkzSecondPointResult}.\\
Podemos renomeá-los com \tkzcname{tkzGetPoints}.

\medskip
\begin{tabular}{lll}%
\toprule
Argumentos             & exemplo & explicação                         \\ 
\midrule
\TAline{\parg{pt1,pt2}}{\tkzcname{tkzDefSquare}\parg{A,B}}{O quadrado é definido no sentido direto.}
\end{tabular}
\end{NewMacroBox}

\subsubsection{Usando \tkzcname{tkzDefSquare} com dois pontos}
Note a inversão dos dois primeiros pontos e o resultado.

\begin{tkzexample}[latex=4cm,small]
\begin{tikzpicture}[scale=.5]
  \tkzDefPoint(0,0){A} \tkzDefPoint(3,0){B}
  \tkzDefSquare(A,B)
  \tkzDrawPolygon[new](A,B,tkzFirstPointResult,%
               tkzSecondPointResult)
  \tkzDefSquare(B,A)
  \tkzDrawPolygon(B,A,tkzFirstPointResult,%
               tkzSecondPointResult) 
\end{tikzpicture} 
\end{tkzexample}

 Podemos precisar apenas de um ponto para desenhar um triângulo retângulo isósceles, então usamos \\ \tkzcname{tkzGetFirstPoint} or \tkzcname{tkzGetSecondPoint}.

\subsubsection{Uso de \tkzcname{tkzDefSquare} para obter um triângulo retângulo isósceles}
\begin{tkzexample}[latex=7cm,small]
\begin{tikzpicture}[scale=1]
  \tkzDefPoint(0,0){A}
  \tkzDefPoint(3,0){B}
  \tkzDefSquare(A,B) \tkzGetFirstPoint{C}
  \tkzDrawSegment(A,B)
  \tkzDrawSegments[new](A,C B,C)
  \tkzMarkRightAngles(A,B,C)
  \tkzDrawPoints(A,B) \tkzDrawPoint[new](C)
  \tkzLabelPoints(A,B)
  \tkzLabelPoints[new,above](C)
\end{tikzpicture}
\end{tkzexample}

\subsubsection{Teorema de Pitágoras e \tkzcname{tkzDefSquare} }
\begin{tkzexample}[latex=8cm,small]
\begin{tikzpicture}[scale=.5]
\tkzDefPoint(0,0){C}
\tkzDefPoint(4,0){A}
\tkzDefPoint(0,3){B} 
\tkzDefSquare(B,A)\tkzGetPoints{E}{F} 
\tkzDefSquare(A,C)\tkzGetPoints{G}{H} 
\tkzDefSquare(C,B)\tkzGetPoints{I}{J} 
\tkzDrawPolygon(A,B,C) 
\tkzDrawPolygon(A,C,G,H) 
\tkzDrawPolygon(C,B,I,J) 
\tkzDrawPolygon(B,A,E,F) 
\tkzLabelSegment(A,C){$a$} 
\tkzLabelSegment[right](C,B){$b$} 
\tkzLabelSegment[swap](A,B){$c$} 
\end{tikzpicture}
\end{tkzexample}

\subsection{Definindo os pontos de um retângulo}
.

\begin{NewMacroBox}{tkzDefRectangle}{\parg{pt1,pt2}}%
O retângulo é definido no sentido direto. A partir de dois pontos, dois pontos adicionais são obtidos de forma que os quatro tomados em ordem formam um retângulo. Os dois pontos passados como argumentos são as extremidades de uma diagonal do retângulo. Os lados são paralelos aos eixos.\\
 Os resultados estão em \tkzname{tkzFirstPointResult} and \tkzname{tkzSecondPointResult}.\\
Podemos renomeá-los com \tkzcname{tkzGetPoints}.

\medskip
\begin{tabular}{lll}%
\toprule
Argumentos             & exemplo & explicação                         \\ 
\midrule
\TAline{\parg{pt1,pt2}}{\tkzcname{tkzDefRectangle}\parg{A,B}}{O retângulo é definido no sentido direto.}
\end{tabular}
\end{NewMacroBox}

\subsubsection{Exemplo de definição de retângulo}
\begin{tkzexample}[latex=7 cm,small]
\begin{tikzpicture}
\tkzDefPoints{0/0/A,5/2/C}
\tkzDefRectangle(A,C) \tkzGetPoints{B}{D}
\tkzDrawPolygon[fill=teal!15](A,...,D)
\end{tikzpicture}
\end{tkzexample}

\subsection{Definição de paralelogramo} 

Definindo os pontos de um paralelogramo. Trata-se de completar três pontos a fim de obter um paralelogramo.
\begin{NewMacroBox}{tkzDefParallelogram}{\parg{pt1,pt2,pt3}}%
\begin{tabular}{lll}%
\toprule
argumentos &  padrão & definição  \\ 
\midrule
\TAline{\parg{pt1,pt2,pt3}}{no padrão}{Três pontos são necessários}
\bottomrule
\end{tabular}
\end{NewMacroBox}

A partir de três pontos, outro ponto é obtido de forma que os quatro tomados em ordem formam um paralelogramo. 
\\ O resultado está em \tkzname{tkzPointResult}. \\
Podemos renomeá-lo com o nome \tkzcname{tkzGetPoint}...


\subsubsection{Example of a parallelogram definição}

\begin{tkzexample}[latex=7 cm,small]
\begin{tikzpicture}[scale=1]
 \tkzDefPoints{0/0/A,3/0/B,4/2/C} 
 \tkzDefParallelogram(A,B,C) 
 % ou    \tkzDefPointWith[colinear= at C](B,A) 
 \tkzGetPoint{D}
 \tkzDrawPolygon(A,B,C,D)
 \tkzLabelPoints(A,B) 
 \tkzLabelPoints[above right](C,D)
 \tkzDrawPoints(A,...,D)
\end{tikzpicture}
\end{tkzexample}


\subsection{O retângulo áureo} 
 \begin{NewMacroBox}{tkzDefGoldenRectangle}{\parg{ponto,ponto}}%
A macro determina um retângulo cuja razão de tamanho é o número $\Phi$.\\
 Os pontos criados estão em \tkzname{tkzFirstPointResult} and \tkzname{tkzSecondPointResult}. \\
 Eles podem ser obtidos com a macro \tkzcname{tkzGetPoints}. A macro seguinte é usada para desenhar o retângulo.

\begin{tabular}{lll}%
\toprule
argumentos             & exemplo & explicação                         \\
\midrule
\TAline{\parg{pt1,pt2}}{\parg{A,B}}{Se C e D são criados então $AB/BC=\Phi$.}
 \end{tabular}
 
 \tkzcname{tkzDefGoldenRectangle} ou   \tkzcname{tkzDefGoldRectangle}
\end{NewMacroBox}


\subsubsection{Retângulos Áureos}
\begin{tkzexample}[latex=6 cm,small]
\begin{tikzpicture}[scale=.6]
 \tkzDefPoint(0,0){A}      \tkzDefPoint(8,0){B}
 \tkzDefGoldRectangle(A,B) \tkzGetPoints{C}{D}
 \tkzDefGoldRectangle(B,C) \tkzGetPoints{E}{F}
 \tkzDefGoldRectangle(C,E) \tkzGetPoints{G}{H}
 \tkzDrawPolygon(A,B,C,D)
 \tkzDrawSegments(E,F G,H)
\end{tikzpicture}
\end{tkzexample}

\subsubsection{Construção do retângulo áureo }
Sem a macro anterior, aqui está como obter o retângulo áureo.

\begin{tkzexample}[latex=8cm,small]
\begin{tikzpicture}[scale=.5] 
\tkzDefPoint(0,0){A}
\tkzDefPoint(8,0){B} 
\tkzDefMidPoint(A,B)
\tkzGetPoint{I} 
\tkzDefSquare(A,B)\tkzGetPoints{C}{D} 
\tkzInterLC(A,B)(I,C)\tkzGetPoints{G}{E} 
\tkzDefPointWith[colinear= at C](E,B) 
 \tkzGetPoint{F}
\tkzDefPointBy[projection=onto D--C ](E) 
 \tkzGetPoint{H}
\tkzDrawArc[style=dashed](I,E)(D)
\tkzDrawPolygon(A,B,C,D) 
\tkzDrawPoints(C,D,E,F,H) 
\tkzLabelPoints(A,B,C,D,E,F,H)
\tkzLabelPoints[above](C,D,F,H)  
\tkzDrawSegments[style=dashed,color=gray]%
(E,F C,F B,E F,H H,C E,H) 
\end{tikzpicture}
\end{tkzexample}




\subsection{Polígono regular} 
 \begin{NewMacroBox}{tkzDefRegPolygon}{\oarg{opções locais}\parg{pt1,pt2}}%
A partir do número de lados, dependendo das opções, esta macro determina um polígono regular de acordo com seu centro ou um lado.

\begin{tabular}{lll}%
\toprule
argumentos             & exemplo & explicação                         \\
\midrule
\TAline{\parg{pt1,pt2}}{\parg{O,A}}{com opção \code{center}, $O$ é o centro do polígono.}
\TAline{\parg{pt1,pt2}}{\parg{A,B}}{com opção \code{side}, $[AB]$ é um lado.}
 \end{tabular}

\medskip
\begin{tabular}{lll}%
\toprule
opções             & padrão & exemplo                         \\
\midrule
\TOline{name}{P}{Os vértices são nomeados $P1$,$P2$,\dots}
\TOline{sides}{5}{número de lados.}
\TOline{center}{center}{O primeiro ponto é o centro.}
\TOline{side}{center}{Os dois pontos são vértices.}
\TOline{Opções TikZ}{...}{}
\end{tabular} 
\end{NewMacroBox}

\subsubsection{Opção \tkzname{center}}
\begin{tkzexample}[latex=7cm, small]   
\begin{tikzpicture}
  \tkzDefPoints{0/0/P0,0/0/Q0,2/0/P1}
  \tkzDefMidPoint(P0,P1) \tkzGetPoint{Q1}
  \tkzDefRegPolygon[center,sides=7](P0,P1)
  \tkzDefMidPoint(P1,P2) \tkzGetPoint{Q1}
  \tkzDefRegPolygon[center,sides=7,name=Q](P0,Q1)
  \tkzFillPolygon[teal!20](Q0,Q1,P2,Q2)
  \tkzDrawPolygon(P1,P...,P7)
  \foreach \j in {1,...,7} {%
  \tkzDrawSegment[black](P0,Q\j)}
\end{tikzpicture}
\end{tkzexample}

\subsubsection{Opção \tkzname{side}}
\begin{tkzexample}[latex=7cm, small]   
\begin{tikzpicture}[scale=1]
    \tkzDefPoints{-4/0/A, -1/0/B}
    \tkzDefRegPolygon[side,sides=5,name=P](A,B)
    \tkzDrawPolygon[thick](P1,P...,P5)
\end{tikzpicture}
\end{tkzexample}
\endinput
\newpage
\section{Círculos}

Entre as macros seguintes, uma permitirá desenhar um círculo, o que não é uma grande façanha. Para fazer isso, você precisará saber o centro do círculo e o raio do círculo ou um ponto na circunferência. Pareceu-me que o uso mais frequente era desenhar um círculo com um centro dado passando por um ponto dado. Este será o método padrão, caso contrário você terá que usar a opção \tkzname{R}. Há um grande número de círculos especiais, por exemplo o círculo circunscrito por um triângulo.

\begin{itemize}
  \item  Criei uma primeira macro \tkzcname{tkzDefCircle} que permite, de acordo com um círculo particular, recuperar seu centro e a medida do raio em cm. Esta recuperação é feita com as macros \tkzcname{tkzGetPoint} e \tkzcname{tkzGetLength};

 \item em seguida, uma macro \tkzcname{tkzDrawCircle};

 \item depois uma macro que permite colorir um disco, mas sem desenhar o círculo \tkzcname{tkzFillCircle};

 \item às vezes, é necessário que um desenho esteja contido em um disco, este é o papel atribuído a \tkzcname{tkzClipCircle};

 \item  finalmente resta poder dar um rótulo para designar um círculo e se várias possibilidades forem oferecidas, veremos aqui \tkzcname{tkzLabelCircle}.
\end{itemize}

\subsection{Características de um círculo: \tkzcname{tkzDefCircle}}

Esta macro permite recuperar as características (centro e raio) de certos círculos.

\begin{NewMacroBox}{tkzDefCircle}{\oarg{local opções}\parg{A,B} ou \parg{A,B,C}}%
\tkzHandBomb\ Atenção, os argumentos são listas de dois ou três pontos. Esta macro é usada em parceria com \\ \tkzcname{tkzGetPoints} para obter o centro e um ponto no círculo, ou usando \\ \tkzname{tkzFirstPointResult} e \tkzname{tkzSecondPointResult} se não for necessário manter os resultados. Você também pode usar \tkzcname{tkzGetLength} para obter o raio.

\medskip
\begin{tabular}{lll}%
\toprule
argumentos           & exemplo & explicação                         \\
\midrule
\TAline{\parg{pt1,pt2} ou \parg{pt1,pt2,pt3}}{\parg{A,B}} {$[AB]$ é o raio $A$ é o centro}
\bottomrule
\end{tabular} 

\medskip
\begin{tabular}{lll}%
\toprule
opções             & padrão & definição                         \\ 
\midrule
\TOline{R}       {circum}{círculo caracterizado por um centro e um raio}
\TOline{diameter}{circum}{círculo caracterizado por dois pontos definindo um diâmetro}
\TOline{circum}       {circum}{círculo circunscrito de um triângulo}
\TOline{in}           {circum}{círculo inscrito em um triângulo}
\TOline{ex}           {circum}{círculo ex-inscrito de um triângulo}
\TOline{euler or nine}{circum}{Círculo de Euler}
\TOline{spieker}      {circum}{Círculo de Spieker}
\TOline{apollonius}   {circum}{círculo de Apolônio}
\TOline{orthogonal from} {circum}{[orthogonal from = A ](O,M)}
\TOline{orthogonal through}{circum}{[orthogonal through = A and B](O,M)}
\TOline{K} {1}{coeficiente usado para um círculo de Apolônio} 
 \bottomrule
\end{tabular}

\medskip
\emph{Nos exemplos seguintes, desenho os círculos com uma macro ainda não apresentada. Você pode precisar apenas do centro e de um ponto no círculo.}
\end{NewMacroBox} 

\subsubsection{Exemplo com opção \tkzname{R}}
Obtemos com a macro \tkzcname{tkzGetPoint} um ponto do círculo que é o polo Leste.

\begin{tkzexample}[latex=7cm,small]  
\begin{tikzpicture}[scale=1]
  \tkzDefPoint(3,3){C}
  \tkzDefPoint(5,5){A}
   \tkzCalcLength(A,C) \tkzGetLength{rAC}
  \tkzDefCircle[R](C,\rAC) \tkzGetPoint{B}
  \tkzDrawCircle(C,B)
  \tkzDrawSegment(C,A)
  \tkzLabelSegment[above left](C,A){$2\sqrt{2}$}
  \tkzDrawPoints(A,B,C)
  \tkzLabelPoints(A,C,B)
\end{tikzpicture} 
\end{tkzexample}     


 \subsubsection{Exemplo com opção \tkzname{diameter}}
 É mais simples aqui procurar diretamente pelo meio de $[AB]$. O resultado é o centro e, se necessário, 
\begin{tkzexample}[latex=7cm,small]  
\begin{tikzpicture}
  \tkzDefPoint(0,0){O}
  \tkzDefPoint(2,2){B}
  \tkzDefCircle[diameter](O,B) \tkzGetPoint{A}
  \tkzDrawCircle(A,B)
  \tkzDrawPoints(O,A,B)
  \tkzDrawSegment(O,B)
  \tkzLabelPoints(O,A,B)
  \tkzLabelSegment[above left](O,A){$\sqrt{2}$}
  \tkzLabelSegment[above left](A,B){$\sqrt{2}$}
  \tkzMarkSegments[mark=s||](O,A A,B)
\end{tikzpicture}
\end{tkzexample}    

\subsubsection{Círculos inscritos e circunscritos para um triângulo dado} 
 
\begin{tkzexample}[latex=7cm,small]  
\begin{tikzpicture}[scale=.75]
 \tkzDefPoint(2,2){A}  \tkzDefPoint(5,-2){B}
 \tkzDefPoint(1,-2){C}
 \tkzDefCircle[in](A,B,C)
 \tkzGetPoints{I}{x}   
 \tkzDefCircle[circum](A,B,C)
 \tkzGetPoint{K}  
 \tkzDrawCircles[new](I,x K,A) 
 \tkzLabelPoints[below](B,C)
 \tkzLabelPoints[above left](A,I,K)
 \tkzDrawPolygon(A,B,C)
 \tkzDrawPoints(A,B,C,I,K) 
\end{tikzpicture} 
\end{tkzexample}

\subsubsection{Exemplo with opção \tkzname{ex}}
We want to define an excircle of a  triangle relatively to point $C$

\begin{tkzexample}[latex=8cm,small]
\begin{tikzpicture}[scale=.75]
  \tkzDefPoints{ 0/0/A,4/0/B,0.8/4/C}
  \tkzDefCircle[ex](B,C,A)                   
  \tkzGetPoints{J_c}{h}
  \tkzDefPointBy[projection=onto A--C ](J_c)   
  \tkzGetPoint{X_c}
  \tkzDefPointBy[projection=onto A--B ](J_c)   
  \tkzGetPoint{Y_c}     
  \tkzDefCircle[in](A,B,C)    
  \tkzGetPoints{I}{y}
  \tkzDrawCircles[color=lightgray](J_c,h I,y)
  \tkzDefPointBy[projection=onto A--C ](I)
  \tkzGetPoint{F}
  \tkzDefPointBy[projection=onto A--B ](I)
  \tkzGetPoint{D}
  \tkzDrawPolygon(A,B,C)
  \tkzDrawLines[add=0 and 1.5](C,A C,B)
  \tkzDrawSegments(J_c,X_c I,D  I,F J_c,Y_c)
  \tkzMarkRightAngles(A,F,I B,D,I J_c,X_c,A%
     J_c,Y_c,B)
  \tkzDrawPoints(B,C,A,I,D,F,X_c,J_c,Y_c)
  \tkzLabelPoints(B,A,J_c,I,D)
  \tkzLabelPoints[above](Y_c)
  \tkzLabelPoints[left](X_c)
  \tkzLabelPoints[above left](C)
  \tkzLabelPoints[left](F)
\end{tikzpicture}  
\end{tkzexample}

\subsubsection{Euler's circle for a given triangle with opção \tkzname{euler}}
 
We verify that this circle passes through the middle of each side.
\begin{tkzexample}[latex=6cm,small]  
\begin{tikzpicture}[scale=.75]
   \tkzDefPoint(5,3.5){A} 
   \tkzDefPoint(0,0){B} \tkzDefPoint(7,0){C}
   \tkzDefCircle[euler](A,B,C)
   \tkzGetPoints{E}{e}
   \tkzDefSpcTriangle[medial](A,B,C){M_a,M_b,M_c}
   \tkzDrawCircle[new](E,e)
   \tkzDrawPoints(A,B,C,E,M_a,M_b,M_c)    
   \tkzDrawPolygon(A,B,C)    
   \tkzLabelPoints[below](B,C)  
   \tkzLabelPoints[left](A,E)   
\end{tikzpicture}
\end{tkzexample}

\subsubsection{Apollonius circles for a given segment opção \tkzname{apollonius}} 
 
\begin{tkzexample}[latex=9cm,small]    
\begin{tikzpicture}[scale=0.75]
  \tkzDefPoint(0,0){A} 
  \tkzDefPoint(4,0){B}
  \tkzDefCircle[apollonius,K=2](A,B)
  \tkzGetPoints{K1}{x}
  \tkzDrawCircle[color = teal!50!black,
      fill=teal!20,opacity=.4](K1,x)
  \tkzDefCircle[apollonius,K=3](A,B)
  \tkzGetPoints{K2}{y}
  \tkzDrawCircle[color=orange!50,
      fill=orange!20,opacity=.4](K2,y) 
  \tkzLabelPoints[below](A,B,K1,K2)
  \tkzDrawPoints(A,B,K1,K2) 
  \tkzDrawLine[add=.2 and 1](A,B)  
\end{tikzpicture}
\end{tkzexample}  

 \subsubsection{Circles exinscribed to a given triangle opção \tkzname{ex}}
 You can also get the center and the projection of it on one side of the triangle. 
 
 with \tkzcname{tkzGetFirstPoint\{Jb\}} and \tkzcname{tkzGetSecondPoint\{Tb\}}.
 
\begin{tkzexample}[latex=8cm,small]  
\begin{tikzpicture}[scale=.6]
  \tkzDefPoint(0,0){A}
  \tkzDefPoint(3,0){B}
  \tkzDefPoint(1,2.5){C}
  \tkzDefCircle[ex](A,B,C) \tkzGetPoints{I}{i}
  \tkzDefCircle[ex](C,A,B) \tkzGetPoints{J}{j}
  \tkzDefCircle[ex](B,C,A) \tkzGetPoints{K}{k}
  \tkzDefCircle[in](B,C,A) \tkzGetPoints{O}{o}
  \tkzDrawCircles[new](J,j I,i K,k O,o) 
  \tkzDrawLines[add=1.5 and 1.5](A,B A,C B,C)
  \tkzDrawPolygon[purple](I,J,K)
  \tkzDrawSegments[new](A,K B,J C,I)
  \tkzDrawPoints(A,B,C)
  \tkzDrawPoints[new](I,J,K)   
  \tkzLabelPoints(A,B,C,I,J,K)
\end{tikzpicture}
\end{tkzexample}
 
\subsubsection{Spieker circle with opção \tkzname{spieker}}   
The incircle of the medial triangle $M_aM_bM_c$ is the Spieker circle:

\begin{tkzexample}[latex=6cm, small]
\begin{tikzpicture}[scale=1.25]
  \tkzDefPoints{ 0/0/A,4/0/B,0.8/4/C}
  \tkzDefSpcTriangle[medial](A,B,C){M_a,M_b,M_c}
  \tkzDefTriangleCenter[spieker](A,B,C) 
  \tkzGetPoint{S_p}
  \tkzDrawPolygon(A,B,C)
  \tkzDrawPolygon[cyan](M_a,M_b,M_c)
  \tkzDrawPoints(B,C,A)
  \tkzDefCircle[spieker](A,B,C)
  \tkzDrawPoints[new](M_a,M_b,M_c,S_p)
  \tkzDrawCircle[new](tkzFirstPointResult,%
    tkzSecondPointResult)
  \tkzLabelPoints[right](M_a)
  \tkzLabelPoints[left](M_b)
  \tkzLabelPoints[below](A,B,M_c,S_p)
  \tkzLabelPoints[above](C)
\end{tikzpicture}
\end{tkzexample}
 
\subsection{Projection of excenters}

\begin{NewMacroBox}{tkzDefProjExcenter}{\oarg{local opções}\parg{A,B,C}\parg{a,b,c}\marg{X,Y,Z}}%
Each excenter has three projections on the sides of the triangle ABC. We can do this with one macro\\ \tkzcname{tkzDefProjExcenter[name=J](A,B,C)(a,b,c)\{Y,Z,X\}}.

\medskip
\begin{tabular}{lll}%
\toprule
opções             & padrão & definição                        \\
\midrule
\TOline{name} {no defaut}{used to name the vertices}
\bottomrule
\end{tabular}

\begin{tabular}{lll}%
argumentos & padrão & definição \\
\midrule
\TAline{(pt1=$\alpha_1$,pt2=$\alpha_2$,\dots)}{no padrão}{Each point has a assigned weight}
\end{tabular}

\medskip
\end{NewMacroBox}

\subsubsection{\tkzname{Excircles}}

\begin{tkzexample}[vbox,small]
\begin{tikzpicture}[scale=.6]
\tikzset{line style/.append style={line width=.2pt}}
\tikzset{label style/.append style={color=teal,font=\footnotesize}}
\tkzDefPoints{0/0/A,5/0/B,0.8/4/C}
\tkzDefSpcTriangle[excentral,name=J](A,B,C){a,b,c} 
\tkzDefSpcTriangle[intouch,name=I](A,B,C){a,b,c}
\tkzDefProjExcenter[name=J](A,B,C)(a,b,c){X,Y,Z}
\tkzDefCircle[in](A,B,C)   \tkzGetPoint{I} \tkzGetSecondPoint{T}  
\tkzDrawCircles[red](Ja,Xa Jb,Yb Jc,Zc)
\tkzDrawCircle(I,T) 
\tkzDrawPolygon[dashed,color=blue](Ja,Jb,Jc)
\tkzDrawLines[add=1.5 and 1.5](A,C A,B B,C)
\tkzDrawSegments(Ja,Xa Ja,Ya Ja,Za
                 Jb,Xb Jb,Yb Jb,Zb
                 Jc,Xc Jc,Yc Jc,Zc
                 I,Ia I,Ib I,Ic)
\tkzMarkRightAngles[size=.2,fill=gray!15](Ja,Za,B Ja,Xa,B Ja,Ya,C Jb,Yb,C)
\tkzMarkRightAngles[size=.2,fill=gray!15](Jb,Zb,B Jb,Xb,C Jc,Yc,A Jc,Zc,B)
\tkzMarkRightAngles[size=.2,fill=gray!15](Jc,Xc,C I,Ia,B I,Ib,C I,Ic,A)
\tkzDrawSegments[blue](Jc,C Ja,A Jb,B)
\tkzDrawPoints(A,B,C,Xa,Xb,Xc,Ja,Jb,Jc,Ia,Ib,Ic,Ya,Yb,Yc,Za,Zb,Zc)
\tkzLabelPoints(A,Ya,Yb,Ja,I)
\tkzLabelPoints[left](Jb,Ib,Yc)
\tkzLabelPoints[below](Zb,Ic,Jc,B,Za,Xa)
\tkzLabelPoints[above right](C,Zc,Yb)
\tkzLabelPoints[right](Xb,Ia,Xc)
\end{tikzpicture}
\end{tkzexample}
 
\subsubsection{\tkzname{Orthogonal from}}
Orthogonal circle of given center. \tkzcname{tkzGetPoints\{z1\}\{z2\}} gives two points of the circle.

\begin{tkzexample}[latex=7cm,small]
\begin{tikzpicture}[scale=.75]
  \tkzDefPoints{0/0/O,1/0/A}
  \tkzDefPoints{1.5/1.25/B,-2/-3/C}
  \tkzDefCircle[orthogonal from=B](O,A)
  \tkzGetPoints{z1}{z2}
  \tkzDefCircle[orthogonal from=C](O,A)
  \tkzGetPoints{t1}{t2}
  \tkzDrawCircle(O,A)
  \tkzDrawCircles[new](B,z1 C,t1)
  \tkzDrawPoints(t1,t2,C)
  \tkzDrawPoints(z1,z2,O,A,B)
  \tkzLabelPoints[right](O,A,B,C)
\end{tikzpicture}
\end{tkzexample}

\subsubsection{\tkzname{Orthogonal through}}
Orthogonal circle passing through two given points.
\begin{tkzexample}[latex=6cm,small]
\begin{tikzpicture}[scale=1]
  \tkzDefPoint(0,0){O}
  \tkzDefPoint(1,0){A}
  \tkzDrawCircle(O,A)
  \tkzDefPoint(-1.5,-1.5){z1}
  \tkzDefPoint(1.5,-1.25){z2}
  \tkzDefCircle[orthogonal through=z1 and z2](O,A)
   \tkzGetPoint{c}
  \tkzDrawCircle[new](tkzPointResult,z1)
  \tkzDrawPoints[new](O,A,z1,z2,c)
  \tkzLabelPoints[right](O,A,z1,z2,c)
\end{tikzpicture}
\end{tkzexample}

\endinput
\subsection{Definição de círculo por transformação; \tkzcname{tkzDefCircleBy} }
Estas transformações são:

\begin{itemize}
   \item translação;
   \item homotetia;
   \item reflexão ortogonal ou simetria;
   \item simetria central;
   \item projeção ortogonal;
   \item rotação (graus);
   \item inversão.
\end{itemize}

A escolha das transformações é feita através das opções. A macro é \tkzcname{tkzDefCircleBy} e a outra para a transformação de uma lista de pontos \tkzcname{tkzDefCirclesBy}. Por exemplo, escreveremos:
\begin{tkzltxexample}[]
\tkzDefCircleBy[translation= from A to A'](O,M)
\end{tkzltxexample}
$O$ é o centro e $M$ é um ponto no círculo.
A imagem é um círculo. O novo centro é |tkzFirstPointResult| e |tkzSecondPointResult| é um ponto no novo círculo. Você pode obter os resultados com a macro \tkzcname{tkzGetPoints}.
\medskip
\begin{NewMacroBox}{tkzDefCircleBy}{\oarg{opções locais}\parg{pt1,pt2}}%
O argumento é um par de pontos. O resultado é um par de pontos. Se você deseja manter esses pontos, então a macro \tkzcname{tkzGetPoints\{O'\}\{M'\}} permite atribuir o nome \tkzname{O'} ao centro e \tkzname{M'} ao ponto no círculo.

\begin{tabular}{lll}%
\toprule
argumentos &  definição & exemplos               \\
\midrule
\TAline{pt1,pt2}   {pontos existentes}   {$(O,M)$}
\bottomrule
\end{tabular}

\begin{tabular}{lll}%
opções     &     & exemplos                         \\
\midrule
\TOline{translation}{= from \#1 to \#2}{[translation=from A to B](O,M)}
\TOline{homothety}  {= center \#1 ratio \#2}{[homothety=center A ratio .5](O,M)}
\TOline{reflection} {= over \#1--\#2}{[reflection=over A--B](O,M)}
\TOline{symmetry }  {= center \#1}{[symmetry=center A](O,M)}
\TOline{projection }{= onto \#1--\#2}{[projection=onto A--B](O,M)}
\TOline{rotation }  {= center \#1 angle \#2}{[rotation=center O angle 30](O,M)}
\TOline{inversion}{= center \#1 through \#2}{[inversion =center O through A](O,M)}
% \TOline{inversion negative}{= center \#1 through \#2}{[inversion negative =center O through A](O,M)}
\bottomrule
\end{tabular}

\medskip
\emph{A imagem é apenas definida e não desenhada.}
\end{NewMacroBox}

\subsubsection{\tkzname{Translation}}
\begin{tkzexample}[latex=7cm,small]
\begin{tikzpicture}[>=latex]
 \tkzDefPoint(0,0){A}  \tkzDefPoint(3,1){B}
 \tkzDefPoint(3,2){C}   \tkzDefPoint(4,3){D}
 \tkzDefCircleBy[translation= from B to A](C,D)
 \tkzGetPoints{C'}{D'}
 \tkzDrawPoints[teal](A,B,C,D,C',D')
 \tkzDrawSegments[orange,->](A,B)
 \tkzDrawCircles(C,D C',D')
 \tkzLabelPoints[color=teal](A,B,C,C')
 \tkzLabelPoints[color=teal,above](D,D')
\end{tikzpicture}
\end{tkzexample}

\subsubsection{\tkzname{Reflection} (simetria ortogonal)}

\begin{tkzexample}[latex=7cm,small]
\begin{tikzpicture}[>=latex]
 \tkzDefPoint(0,0){A}  \tkzDefPoint(3,1){B}
 \tkzDefPoint(3,2){C}   \tkzDefPoint(4,3){D}
 \tkzDefCircleBy[reflection = over A--B](C,D)
 \tkzGetPoints{C'}{D'}
 \tkzDrawPoints[teal](A,B,C,D,C',D')
 \tkzDrawLine[add =0 and 1][orange](A,B)
 \tkzDrawCircles(C,D C',D')
 \tkzLabelPoints[color=teal](A,B,C,C')
 \tkzLabelPoints[color=teal,right](D,D')
\end{tikzpicture}
\end{tkzexample}

\subsubsection{\tkzname{Homothety}}

\begin{tkzexample}[latex=7cm,small]
\begin{tikzpicture}[scale=1.2]
 \tkzDefPoint(0,0){A}   \tkzDefPoint(3,1){B}
 \tkzDefPoint(3,2){C}   \tkzDefPoint(4,3){D}
 \tkzDefCircleBy[homothety=center A ratio .5](C,D)
 \tkzGetPoints{C'}{D'}
 \tkzDrawPoints[teal](A,C,D,C',D')
 \tkzDrawCircles(C,D C',D')
 \tkzLabelPoints[color=teal](A,C,C')
 \tkzLabelPoints[color=teal,right](D,D')
\end{tikzpicture}
\end{tkzexample}

\subsubsection{\tkzname{Symmetry}}
\begin{tkzexample}[latex=7cm,small]
\begin{tikzpicture}[scale=1]
 \tkzDefPoint(3,1){B}
 \tkzDefPoint(3,2){C}   \tkzDefPoint(4,3){D}
 \tkzDefCircleBy[symmetry=center B](C,D)
 \tkzGetPoints{C'}{D'}
 \tkzDrawPoints[teal](B,C,D,C',D')
 \tkzDrawLines[orange](C,C' D,D')
 \tkzDrawCircles(C,D C',D')
 \tkzLabelPoints[color=teal](C,C')
 \tkzLabelPoints[color=teal,above](D)
 \tkzLabelPoints[color=teal,below](D')
\end{tikzpicture}
\end{tkzexample}

\subsubsection{\tkzname{Rotation}}
\begin{tkzexample}[latex=7cm,small]
\begin{tikzpicture}[scale=0.5]
 \tkzDefPoint(3,-1){B}
 \tkzDefPoint(3,2){C}   \tkzDefPoint(4,3){D}
 \tkzDefCircleBy[rotation=center B angle 90](C,D)
 \tkzGetPoints{C'}{D'}
 \tkzDrawPoints[teal](B,C,D,C',D')
 \tkzLabelPoints[color=teal](B,C,D,C',D')
 \tkzDrawCircles(C,D C',D')
\end{tikzpicture}
\end{tkzexample}

\subsubsection{\tkzname{Inversion}}

\begin{tkzexample}[latex=7cm,small]
\begin{tikzpicture}[scale=1.5]
\tkzSetUpPoint[size=3,color=red,fill=red!20]
\tkzSetUpStyle[color=purple,ultra thin]{st1}
\tkzSetUpStyle[color=cyan,ultra thin]{st2}
\tkzDefPoint(2,0){A} \tkzDefPoint(3,0){B}
\tkzDefPoint(3,2){C} \tkzDefPoint(4,2){D}
\tkzDefCircleBy[inversion = center B through A](C,D)
\tkzGetPoints{C'}{D'}
\tkzDrawPoints(A,B,C,D,C',D')
\tkzLabelPoints(A,B,C,D,C',D')
\tkzDrawCircles(B,A)
\tkzDrawCircles[st1](C,D)
\tkzDrawCircles[st2](C',D')
\end{tikzpicture}
\end{tkzexample}

\endinput

\section{\tkzname{Interseções}}

É possível determinar as coordenadas dos pontos de interseção entre duas retas, uma reta e um círculo, e dois círculos.

Os comandos associados não têm argumentos opcionais e o usuário deve determinar por si mesmo a existência dos pontos de interseção.

\subsection{Interseção de duas retas \tkzcname{tkzInterLL}}
\begin{NewMacroBox}{tkzInterLL}{\parg{$A,B$}\parg{$C,D$}}%
Define o ponto de interseção \tkzname{tkzPointResult} das duas retas $(AB)$ e $(CD)$. Os pontos conhecidos são fornecidos em pares (dois por reta) entre parênteses, e o ponto resultante pode ser recuperado com a macro \tkzcname{tkzDefPoint}.
\end{NewMacroBox}

\subsubsection{Exemplo de interseção entre duas retas}

\begin{tkzexample}[latex=7cm,small]
\begin{tikzpicture}[rotate=-45,scale=.75]
  \tkzDefPoint(2,1){A}   
  \tkzDefPoint(6,5){B}
  \tkzDefPoint(3,6){C}   
  \tkzDefPoint(5,2){D}
  \tkzDrawLines(A,B C,D)
  \tkzInterLL(A,B)(C,D)  
     \tkzGetPoint{I}
  \tkzDrawPoints[color=blue](A,B,C,D)
   \tkzDrawPoint[color=red](I)
\end{tikzpicture}
\end{tkzexample}

\subsection{Interseção de uma reta e um círculo \tkzcname{tkzInterLC}}

Como antes, a reta é definida por um par de pontos. O círculo
 também é definido por um par:
\begin{itemize}
\item $(O,C)$ que é um par de pontos, o primeiro é o centro e o segundo é qualquer ponto no círculo.
\item $(O,r)$ A medida $r$ é a medida do raio.
\end{itemize}

\begin{NewMacroBox}{tkzInterLC}{\oarg{opções}\parg{$A,B$}\parg{$O,C$} ou \parg{$O,r$} ou \parg{$O,C,D$}}%
Portanto, os argumentos são dois pares. 

\medskip
\begin{tabular}{lll}%
\toprule
opções            & padrão & definição                         \\ 
\midrule
\TOline{N}         {N}    {(O,C) determina o círculo}
\TOline{R}         {N}    {(O, 1 ) unidade 1 cm}
\TOline{with nodes}{N}    {(O,C,D) CD é um raio}
\TOline{common=pt} {}     {pt é ponto comum; tkzFirstPoint fornece o outro ponto}
\TOline{near}      {}     {tkzFirstPoint é o ponto mais próximo do primeiro ponto da reta}
\bottomrule
\end{tabular}

\medskip
A macro define os pontos de interseção $I$ e $J$ da reta $(AB)$ e do círculo de centro $O$ com raio $r$ se existirem; caso contrário, um erro será relatado no arquivo |.log|. \tkzname{with nodes} evita que você calcule o raio que é o comprimento de $[CD]$.
Se \tkzname{common} e \tkzname{near} não forem usados, então \tkzname{tkzFirstPoint} é o menor ângulo (ângulo com \tkzname{tkzSecondPoint} e o centro do círculo).
\end{NewMacroBox}

\begin{NewMacroBox}{tkzTestInterLC}{\parg{$O,A$}\parg{$O',B$}}%
Portanto, os argumentos são dois pares que definem uma reta e um círculo com um centro e um ponto no círculo. Se houver uma interseção não vazia entre a reta e o círculo, então o teste \tkzcname{iftkzFlagLC} retorna verdadeiro.
\end{NewMacroBox}

\subsubsection{Teste de interseção reta-círculo}

\begin{tkzexample}[latex=7cm,small]
  \begin{tikzpicture}[scale=1]
    \tkzDefPoints{% x   y   name
                    3    /4    /I,
                    3    /2    /P,
                    0    /2    /La,
                    8    /3    /Lb}
  \tkzDrawCircle(I,P)
  \foreach \i in {1,...,3}{%
     \coordinate  (Lb) at (8,\i);
     \tkzDrawLine(La,Lb)
     \tkzTestInterLC(La,Lb)(I,P)
      \iftkzFlagLC
      \tkzInterLC(La,Lb)(I,P)  
      \tkzGetPoints{a}{b}
      \tkzDrawPoints(a,b)
      \fi
     }
  \end{tikzpicture}
\end{tkzexample}


\subsubsection{Interseção reta-círculo}

No exemplo seguinte, o desenho do círculo usa dois pontos e a interseção da reta e do círculo usa dois pares de pontos. Vamos comparar os ângulos $\widehat{D,E,O}$ e $\widehat{E,D,O}$. Esses ângulos estão em direções opostas. \tkzname{tkzFirstPoint} é atribuído ao ponto que forma o ângulo com a menor medida (direção anti-horária). O ângulo anti-horário $\widehat{D,E,O}$ tem uma medida igual a $360\circ$ menos a medida de $\widehat{O,E,D}$.

\begin{tkzexample}[latex=7cm,small]
\begin{tikzpicture}[scale=.75]
 \tkzInit[xmax=5,ymax=4]
 \tkzDefPoint(1,1){O} 
 \tkzDefPoint(-2,4){La} 
 \tkzDefPoint(5,0){Lb} 
 \tkzDefPoint(3,3){C}
 \tkzInterLC(La,Lb)(O,C)  \tkzGetPoints{D}{E}  
 \tkzMarkAngle[->,size=1.5](E,D,O)
 \tkzDrawPolygons[new](O,D,E)
 \tkzMarkAngle[->,size=1.5](D,E,O)
 \tkzDrawCircle(O,C)
 \tkzDrawPoints[color=teal](O,La,Lb,C)
 \tkzDrawPoints[color=red](D,E)
 \tkzDrawLine(La,Lb)
 \tkzLabelPoints[above right](O,La,Lb,C,D,E)
\end{tikzpicture} 
\end{tkzexample}

\subsubsection{Reta passando pelo centro opção \tkzname{common}}
Este caso é especial. Você não pode comparar os ângulos. Neste caso, a opção \tkzname{near} deve ser usada. \tkzname{tkzFirstPoint} é atribuído ao ponto mais próximo do primeiro ponto dado para a reta. Aqui queremos que $A$ esteja mais próximo de $Lb$.

\begin{tkzexample}[latex=8cm,small]
\begin{tikzpicture}
\tkzDefPoints{% x   y   name
             0    /1    /D,
             6    /0    /B,
             3    /3    /O,
             2    /2    /La,
             5    /5    /Lb}
  \tkzDrawCircle(O,D)
  \tkzDrawLine(La,Lb)
  \tkzInterLC[near](Lb,La)(O,D)  
  \tkzGetFirstPoint{A}
  \tkzDrawSegments(O,A)
  \tkzDrawPoints(O,D,A,La,Lb)
  \tkzLabelPoints(O,D,A,La,Lb)
\end{tikzpicture}
\end{tkzexample}

\subsubsection{Interseção reta-círculo com opção \tkzname{common}}
Um caso especial que frequentemente encontramos, um ponto da reta está no círculo e estamos procurando o outro ponto comum.
\begin{tkzexample}[latex=7cm,small]
\begin{tikzpicture}[scale=.5]
 \tkzDefPoints{0/0/O,-5/0/A,2/-2/B,0/5/D}
 \tkzInterLC[common=A](B,A)(O,D)
 \tkzGetFirstPoint{C}
 \tkzDrawPoints(O,A,B)
 \tkzDrawCircle(O,A)
 \tkzDrawLine(A,C)
 \tkzDrawPoint(C)
 \tkzLabelPoints(A,B,C)
\end{tikzpicture}
\end{tkzexample}


\subsubsection{Ordem dos pontos na interseção reta-círculo}
A ideia é comparar os ângulos formados com o primeiro ponto definidor da reta, um ponto resultante e o centro do círculo. O primeiro ponto é aquele que corresponde ao menor ângulo.

Como você pode ver $\widehat{BCO} < \widehat{BEO}$. Para dizer a verdade, $\widehat{BEO}$ é anti-horário.

\begin{tkzexample}[latex=6cm,small]
\begin{tikzpicture}[scale=.5]
  \tkzDefPoints{0/0/O,5/1/A,2/2/B,3/1/D}
  \tkzInterLC[common=A](B,D)(O,A) \tkzGetPoints{C}{E}
  \tkzDrawPoints(O,A,B,D)
  \tkzDrawCircle(O,A) \tkzDrawLine(E,C)
  \tkzDrawSegments[dashed](B,O O,C)
  \tkzMarkAngle[->,size=1.5](B,C,O)
  \tkzDrawSegments[dashed](O,E)
  \tkzMarkAngle[->,size=1.5](B,E,O)
  \tkzDrawPoints(C,E)
  \tkzLabelPoints[above](O,E)
  \tkzLabelPoints[right](A,B,C,D)
\end{tikzpicture}
\end{tkzexample}

\subsubsection{Exemplo with \tkzcname{foreach}}
\begin{tkzexample}[latex=7cm,small]
\begin{tikzpicture}[scale=3,rotate=180]
\tkzDefPoint(0,1){J} 
\tkzDefPoint(0,0){O}
\foreach \i in {0,-5,-10,...,-90}{
 \tkzDefPoint({2.5*cos(\i*pi/180)},%
   {1+2.5*sin(\i*pi/180)}){P}
 \tkzInterLC[R](P,J)(O,1)\tkzGetPoints{N}{M}
 \tkzDrawSegment[color=orange](J,N)
 \tkzDrawPoints[red](N)} 
\foreach \i in {-90,-95,...,-175,-180}{
 \tkzDefPoint({2.5*cos(\i*pi/180)},%
   {1+2.5*sin(\i*pi/180)}){P} 
 \tkzInterLC[R](P,J)(O,1)\tkzGetPoints{N}{M}
 \tkzDrawSegment[color=orange](J,M)
 \tkzDrawPoints[red](M)}   
\end{tikzpicture} 
\end{tkzexample}

\subsubsection{Interseção reta-círculo com opção \tkzname{near}}
$D$ é o ponto mais próximo de $b$.

\begin{tkzexample}[vbox,small]
  \begin{tikzpicture}
    \tkzDefPoints{0/0/A,12/0/C}
    \tkzDefGoldenRatio(A,C)                          \tkzGetPoint{B}
    \tkzDefMidPoint(A,C)                             \tkzGetPoint{O}
    \tkzDefMidPoint(A,B)                             \tkzGetPoint{O_1}
    \tkzDefMidPoint(B,C)                             \tkzGetPoint{O_2}
    \tkzDefPointBy[rotation=center O_2 angle 90](C)  \tkzGetPoint{P}
    \tkzDefPointBy[rotation=center O_1 angle 90](B)  \tkzGetPoint{Q}
    \tkzDefPointBy[rotation=center B angle 90](C)    \tkzGetPoint{b}
    \tkzInterLC[near](b,B)(O,A)                      \tkzGetFirstPoint{D}
    \tkzInterCC(D,B)(O,C)                            \tkzGetPoints{V}{U}
    \tkzDefPointBy[projection=onto U--V](O_1)        \tkzGetPoint{M}
    \tkzDefPointBy[projection=onto U--V](O_2)        \tkzGetPoint{N}  
    \tkzDrawPoints(A,B,C,O,O_1,O_2,D,U,V,M,N,b)
    \tkzDrawSemiCircles[teal](O,C O_1,B O_2,C)
    \tkzDrawSegments(A,C B,D U,V A,D C,D M,B B,N)
    \tkzDrawArc(D,U)(V)
    \tkzLabelPoints(A,B,C,O,O_1,O_2)
    \tkzLabelPoints[above](D,U,V,M,N)
  \end{tikzpicture}
\end{tkzexample}


\subsubsection{Exemplo mais complexo de uma interseção reta-círculo}
Figura de  \url{http://gogeometry.com/problem/p190_tangent_circle}

\begin{tkzexample}[latex=6.5cm,small]
\begin{tikzpicture}[scale=.75]
 \tkzDefPoint(0,0){A}  
 \tkzDefPoint(8,0){B}
 \tkzDefMidPoint(A,B)             \tkzGetPoint{O}
 \tkzDefMidPoint(O,B)             \tkzGetPoint{O'}
 \tkzDefLine[tangent from=A](O',B)\tkzGetFirstPoint{E}
 \tkzInterLC(A,E)(O,B)            \tkzGetFirstPoint{D}
 \tkzDefPointBy[projection=onto A--B](D)  
 \tkzGetPoint{F}
 \tkzDrawCircles(O,B O',B)
 \tkzDrawSegments(A,D A,B D,F) 
 \tkzDrawSegments[color=red,line width=1pt,
     opacity=.4](A,O F,B)
 \tkzDrawPoints(A,B,O,O',E,D) 
 \tkzMarkRightAngle(D,F,B)
 \tkzLabelPoints[below right](A,B,O,O',E,D) 
\end{tikzpicture}
\end{tkzexample}

\subsubsection{Círculo definido por um centro e uma medida, e casos especiais}
Vejamos alguns casos especiais como retas tangentes ao círculo.

\begin{tkzexample}[latex=7cm,small]
\begin{tikzpicture}[scale=.5]
 \tkzDefPoint(0,8){A}      \tkzDefPoint(8,0){B}
 \tkzDefPoint(8,8){C}      \tkzDefPoint(4,4){D}
 \tkzDefPoint(2,4){E}      \tkzDefPoint(4,2){F}
 \tkzDefPoint(8,4){G}
 \tkzInterLC(A,C)(D,G)     \tkzGetPoints{I1}{I2}
 \tkzInterLC(B,C)(D,G)     \tkzGetPoints{J1}{J2}
 \tkzInterLC[near](A,B)(D,G)  \tkzGetPoints{K1}{K2}
 \tkzInterLC(E,F)(D,G)     \tkzGetPoints{E1}{E2}
 \tkzDrawCircle(D,G)
 \tkzDrawPoints[color=red](I1,J1,K1,K2,E1,E2)
 \tkzDrawLines(A,B B,C A,C I2,J2 E1,E2)
 \tkzDrawPoints[color=blue](A,...,F)
 \tkzDrawPoints[color=red](I2,J2)
 \tkzLabelPoints[left](B,D,E,F)
 \tkzLabelPoints[below left](A,C)
 \tkzLabelPoints[below=4pt](I1,K1,K2,E2)
 \tkzLabelPoints[left](J1,E1)
\end{tikzpicture}

\end{tkzexample}

\subsubsection{Cálculo do raio}
 Com \tkzname{pgfmath} e \tkzcname{pgfmathsetmacro}

A medida do raio pode ser o resultado de um cálculo que não é feito dentro da macro de interseção, mas antes.
Um comprimento pode ser calculado de várias maneiras. É possível, claro,
 usar o módulo \tkzname{pgfmath} e a macro \tkzcname{pgfmathsetmacro}. Em alguns casos, os resultados obtidos não são precisos o suficiente, portanto o seguinte cálculo $0.0002 \div 0.0001$ dá $1.98$ com pgfmath enquanto xfp dará $2$.

Com \tkzname{xfp} e \tkzcname{fpeval}:

\begin{tkzexample}[latex=7cm,small]
  \begin{tikzpicture}
  \tkzDefPoint(2,2){A}
  \tkzDefPoint(5,4){B}
  \tkzDefPoint(4,4){O}
  \pgfmathsetmacro\tkzLen{\fpeval{0.0002/0.0001}}
 % or \edef\tkzLen{\fpeval{0.0002/0.0001}}
  \tkzInterLC[R](A,B)(O, \tkzLen)
  \tkzGetPoints{I}{J}
  \tkzDrawCircle(O,I)
  \tkzDrawPoints[color=blue](A,B)
  \tkzDrawPoints[color=red](I,J)
  \tkzDrawLine(I,J)
\end{tikzpicture}
  \end{tkzexample}


\subsubsection{Opção \code{with nodes}}
\begin{tkzexample}[latex=8cm,small]
\begin{tikzpicture}[scale=.75]
\tkzDefPoints{0/0/A,4/0/B,1/1/D,2/0/E}
\tkzDefTriangle[equilateral](A,B)
\tkzGetPoint{C}
\tkzInterLC[with nodes](D,E)(C,A,B)
\tkzGetPoints{F}{G}
\tkzDrawCircle(C,A)
\tkzDrawPolygon(A,B,C)
\tkzDrawPoints(A,...,G)
\tkzDrawLine(F,G)
\end{tikzpicture}
\end{tkzexample}

\subsection{Interseção de dois círculos  \tkzcname{tkzInterCC}}

O caso mais frequente é aquele de dois círculos definidos por seu centro e um ponto, mas como antes a opção \tkzname{R} permite usar as medidas dos raios.

\begin{NewMacroBox}{tkzInterCC}{\oarg{opções}\parg{$O,A$}\parg{$O',A'$} ou \parg{$O,r$}\parg{$O',r'$} ou   \parg{$O,A,B$} \parg{$O',C,D$}}%
\begin{tabular}{lll}%
opções       & padrão & definição                         \\
\midrule
\TOline{N}   {N}    {$OA$ e $O'A'$ são raios, $O$ e $O'$ são os centros.}
\TOline{R}   {N}    {$r$ e $r'$ são dimensões e medem os raios.}
\TOline{with nodes} {N}  {em (A,A,C)(C,B,F) AC e BF fornecem os raios.}
\TOline{common=pt}  {}   {pt é ponto comum; tkzFirstPoint fornece o outro ponto.}
\bottomrule
\end{tabular}

\medskip
Esta macro define o(s) ponto(s) de interseção $I$ e $J$ dos dois círculos de centros $O$ e $O'$. Se os dois círculos não tiverem um ponto em comum, então a macro termina com um erro que não é tratado. Se os centros são $O$ e $O'$ e as interseções são $A$ e $B$, então os ângulos $\widehat{O,A,O'}$ e $\widehat{O,B,O'}$ estão em direções opostas. \tkzname{tkzFirstPoint} é atribuído ao ponto que forma o ângulo \code{clockwise} (horário).
\end{NewMacroBox}

\begin{NewMacroBox}{tkzTestInterCC}{\parg{$O,A$}\parg{$O',B$}}%
Portanto, os argumentos são dois pares que definem dois círculos com um centro e um ponto no círculo. Se houver uma interseção não vazia entre esses dois círculos, então o teste \tkzcname{iftkzFlagCC} retorna verdadeiro.
\end{NewMacroBox}

\subsubsection{Teste de interseção círculo-círculo}

\begin{tkzexample}[latex=7cm,small]
\begin{tikzpicture}[scale=.75]
  \tkzDefPoints{% x   y   name
                   0    /0    /A,
                   2    /0    /B,
                   4    /0    /I,
                   1    /0    /P}
\tkzDrawCircle(A,B)
\foreach \i in {1,...,3}{%
   \coordinate  (P) at (\i,0);
\tkzDrawCircle[new](I,P)
   \tkzTestInterCC(A,B)(I,P)
    \iftkzFlagCC
    \tkzInterCC(A,B)(I,P)  \tkzGetPoints{a}{b}
    \tkzDrawPoints(a,b)
    \fi}
  \end{tikzpicture}
\end{tkzexample}

\subsubsection{Interseção círculo-círculo com ponto \tkzname{common}.}

\begin{tkzexample}[latex=7cm,small]
\begin{tikzpicture}[scale=.5]
  \tkzDefPoints{0/0/O,5/-1/A,2/2/B}
  \tkzDrawPoints(O,A,B)
  \tkzDrawCircles(O,B A,B)
  \tkzInterCC[common=B](O,B)(A,B)
  \tkzGetFirstPoint{C}
  \tkzDrawPoint(C)
  \tkzLabelPoints[above](O,A,B,C)
\end{tikzpicture}
\end{tkzexample}

\subsubsection{Interseção círculo-círculo ordem dos pontos.}
A ideia é comparar os ângulos formados com o primeiro centro, um ponto resultante e o centro do segundo círculo. O primeiro ponto é aquele que corresponde ao menor ângulo.

Como você pode ver $\widehat{ODB} < \widehat{OBE} $

\begin{tkzexample}[latex=6cm,small]
\begin{tikzpicture}[scale=.5]
  \pgfkeys{/pgf/number format/.cd,fixed relative,
     precision=4}
  \tkzDefPoints{0/0/O,5/-1/A,2/2/B,2/-1/C}
  \tkzDrawPoints(O,A,B)
  \tkzDrawCircles(O,A B,C)
  \tkzInterCC(O,A)(B,C)\tkzGetPoints{D}{E}
  \tkzDrawPoints(C,D,E)
  \tkzLabelPoints(O,A,B,C)
  \tkzLabelPoints[above](D,E) 
  \tkzDrawSegments[cyan](D,O D,B)
  \tkzMarkAngle[red,->,size=1.5](O,D,B)
  \tkzFindAngle(O,D,B)   \tkzGetAngle{an}
  \tkzLabelAngle(O,D,B){$ \pgfmathprintnumber{\an}$}
  \tkzDrawSegments[cyan](E,O E,B)
  \tkzMarkAngle[red,->,size=1.5](O,E,B)  
  \tkzFindAngle(O,E,B)   \tkzGetAngle{an}
  \tkzLabelAngle(O,E,B){$ \pgfmathprintnumber{\an}$}  
\end{tikzpicture}
\end{tkzexample}

\subsubsection{Construção de um triângulo equilátero.}
$\widehat{A,C,B}$ é um ângulo horário
\begin{tkzexample}[latex=7cm,small]
\begin{tikzpicture}[trim left=-1cm,scale=.5]
 \tkzDefPoint(1,1){A}
 \tkzDefPoint(5,1){B}
 \tkzInterCC(A,B)(B,A)\tkzGetPoints{C}{D}
 \tkzDrawPoint[color=black](C)
 \tkzDrawCircles(A,B B,A)
 \tkzCompass[color=red](A,C)
 \tkzCompass[color=red](B,C)
 \tkzDrawPolygon(A,B,C)
 \tkzMarkSegments[mark=s|](A,C B,C)
 \tkzLabelPoints[](A,B)
 \tkzLabelPoint[above](C){$C$}
\end{tikzpicture}
\end{tkzexample}


\subsubsection{Trisseção de segmento}
 A ideia aqui é dividir um segmento com uma régua e um compasso em três segmentos de comprimento igual.

\begin{tkzexample}[latex=7cm,small]
\begin{tikzpicture}[scale=.6]
 \tkzDefPoint(0,0){A}
 \tkzDefPoint(3,2){B}
 \tkzInterCC(A,B)(B,A)          \tkzGetSecondPoint{D}
 \tkzInterCC(D,B)(B,A)          \tkzGetPoints{A}{C}
 \tkzInterCC(D,B)(A,B)          \tkzGetPoints{E}{B}
 \tkzInterLC[common=D](C,D)(E,D)\tkzGetFirstPoint{F}
 \tkzInterLL(A,F)(B,C)          \tkzGetPoint{O}
 \tkzInterLL(O,D)(A,B)          \tkzGetPoint{H}
 \tkzInterLL(O,E)(A,B)          \tkzGetPoint{G}
 \tkzDrawCircles(D,E A,B B,A E,A)
 \tkzDrawSegments[](O,F O,B O,D O,E)
 \tkzDrawPoints(A,...,H)
 \tkzDrawSegments(A,B B,D A,D A,E E,F C,F B,C)
 \tkzMarkSegments[mark=s|](A,G G,H H,B)
\end{tikzpicture}
\end{tkzexample}

\subsubsection{Com a opção \code{with nodes}}
\begin{tkzexample}[latex=6cm,small]
\begin{tikzpicture}[scale=.5]
 \tkzDefPoints{0/0/A,0/5/B,5/0/C}
 \tkzDefPoint(54:5){F}
 \tkzInterCC[with nodes](A,A,C)(C,B,F)
 \tkzGetPoints{a}{e}
 \tkzInterCC(A,C)(a,e) \tkzGetFirstPoint{b}
 \tkzInterCC(A,C)(b,a) \tkzGetFirstPoint{c}
 \tkzInterCC(A,C)(c,b) \tkzGetFirstPoint{d}
 \tkzDrawCircle[new](A,C)
 \tkzDrawPoints(a,b,c,d,e)
 \tkzDrawPolygon(a,b,c,d,e)
 \foreach \vertex/\num in {a/36,b/108,c/180,
                          d/252,e/324}{%
 \tkzDrawPoint(\vertex)
 \tkzLabelPoint[label=\num:$\vertex$](\vertex){}
 \tkzDrawSegment(A,\vertex)
 }
\end{tikzpicture}
\end{tkzexample}

\subsubsection{Combinação de interseções}
\begin{tkzexample}[latex=8cm,small]
\begin{tikzpicture}[scale = .7]
  \tkzDefPoint(2,2){A}
  \tkzDefPoint(0,0){B}
  \tkzDefPoint(-2,2){C}
  \tkzDefPoint(0,4){D}
  \tkzDefPoint(4,2){E}
  \tkzCircumCenter(A,B,C)\tkzGetPoint{O}
  \tkzInterCC[R](O,2)(D,2)\tkzGetPoints{M1}{M2}
  \tkzInterCC(O,A)(D,O) \tkzGetPoints{1}{2}
  \tkzInterLC(A,E)(B,M1)\tkzGetSecondPoint{M3}
  \tkzInterLC(O,C)(M3,D)\tkzGetSecondPoint{L}
  \tkzDrawSegments(C,L)
  \tkzDrawPoints(A,B,C,D,E,M1,M2,M3,O,L)
  \tkzDrawSegments(O,E)
  \tkzDrawSegments[new](C,A D,B)
  \tkzDrawPoint(O)
  \tkzDrawCircles[new](M3,D B,M2 D,O)
  \tkzDrawCircle(O,A)
  \tkzLabelPoints[below right](A,B,C,D,E,M1,M2,M3,O,L)
\end{tikzpicture}
\end{tkzexample}


\subsubsection{Teorema de Altshiller-Court}
  As duas retas que unem os pontos de interseção de dois círculos ortogonais a um ponto em um dos círculos encontram o outro círculo em dois pontos diametralmente opostos. Altshiller p 176


\begin{tkzexample}[vbox,small]
\begin{tikzpicture}
  \tkzDefPoints{0/0/P,5/0/Q,3/2/I}
  \tkzDefCircle[orthogonal from=P](Q,I) 
  \tkzGetFirstPoint{E}
  \tkzDrawCircles(P,E Q,E)
  \tkzInterCC[common=E](P,E)(Q,E) \tkzGetFirstPoint{F}
  \tkzDefPointOnCircle[through =  center P angle 80 point E] 
  \tkzGetPoint{A}
  \tkzInterLC[common=E](A,E)(Q,E)  \tkzGetFirstPoint{C}
  \tkzInterLL(A,F)(C,Q)  \tkzGetPoint{D}
  \tkzDrawLines[add=0 and 1](P,Q)
  \tkzDrawLines[add=0 and 2](A,E)
  \tkzDrawSegments(P,E E,F F,C A,F C,D)
  \tkzDrawPoints(P,Q,E,F,A,C,D)
  \tkzLabelPoints(P,Q,F)
  \tkzLabelPoints[above](E,A)
  \tkzLabelPoints[left](D)
  \tkzLabelPoints[above right](C)
\end{tikzpicture}
\end{tkzexample}


\endinput
\section{Ângulos}
\subsection{Definição e uso com \tkzname{tkz-euclide}}
Na geometria euclidiana, um ângulo é a figura formada por dois raios, chamados de lados do ângulo, compartilhando um ponto final comum, chamado de vértice do ângulo.[Wikipedia]. Um raio com \tkzname{tkz-euclide} é definido por dois pontos, assim cada ângulo é definido com três pontos como $\widehat{AOB}$. O vértice $O$ é o segundo ponto. A ordem é importante porque presume-se que o ângulo é especificado na ordem direta (sentido anti-horário).
Na trigonometria e na matemática em geral, os ângulos planos são convencionalmente medidos no sentido anti-horário, começando com $0^\circ$ apontando diretamente para a direita (ou leste), e $90^\circ$ apontando diretamente para cima (ou norte)[Wikipedia].
Vamos concordar que um ângulo medido no sentido anti-horário é positivo.

  \begin{center}
    \begin{tikzpicture}[scale=.75]
      \node {horário};
      \tkzDefPoint(0,0){O} \tkzDefPoint(90:2){A}\tkzDefPoint(180:2){B}
      \tkzDrawArc[black,line width=2pt,arrows = {Stealth-}](O,B)(A)
    \end{tikzpicture}
    \begin{tikzpicture}[scale=.75]
          \node {anti-horário};
      \tkzDefPoint(0,0){O} \tkzDefPoint(90:2){A}\tkzDefPoint(0:2){B}
      \tkzDrawArc[black,line width=2pt,arrows = {-Stealth}](O,A)(B)
    \end{tikzpicture}
  \end{center}

 \tkzname{Ângulos} estão envolvidos em várias macros como \tkzcname{tkzDefPoint},\tkzcname{tkzDefPointBy[rotation = \dots]}, \tkzcname{tkzDrawArc}
 e a próxima \tkzcname{tkzGetAngle}. Com exceção da última, todas essas macros aceitam ângulos negativos.

 \begin{figure}[!ht]
 \centering
 \begin{tabular}{|c|c|}
 \hline
 \tkzsubf{\begin{tikzpicture}
 \tkzDefPoint(0,0){O}    \tkzDefPoint(0:2){A}
 \tkzDefPointBy[rotation=center O angle 80](A)  \tkzGetPoint{B}
 \tkzDrawSegments[-{Stealth}](O,A O,B)
 \tkzMarkAngles[size=2,-Stealth,teal](A,O,B)
 \tkzFindAngle(A,O,B)   \tkzGetAngle{an}
 \tkzLabelAngle[pos=1,teal](A,O,B){$ \pgfmathprintnumber{\an}^\circ$}
 \tkzLabelPoints(A)  \tkzLabelPoints[above](B)
 \end{tikzpicture}}
      {Rotação $80^\circ$ de $(O,A)$ para $(O,B)$\\
    {\textbackslash}tkzDefPointBy[rotation=center O angle 80]}
 &
 \tkzsubf{\begin{tikzpicture}
 \tkzDefPoint(0,0){O}    \tkzDefPoint(0:2){A}
 \tkzDefPointBy[rotation=center O angle -80](A)  \tkzGetPoint{B}
 \tkzDrawSegments[-{Stealth}](O,A O,B)
 \tkzMarkAngles[size=2,Stealth-,red](B,O,A)
 \tkzFindAngle(B,O,A)   \tkzGetAngle{an}
 \tkzLabelAngle[pos=1,red](B,O,A){$-\pgfmathprintnumber{\an}^\circ$}
\tkzLabelPoints[right](A)  \tkzLabelPoints[below](B)
 \end{tikzpicture}}
  {Rotação $-80^\circ$ de $(O,A)$ para $(O,B)$\\
     {\textbackslash}tkzDefPointBy[rotation=center O angle -80]}
 \\ \hline
 \tkzsubf{\begin{tikzpicture}
 \tkzDefPoint(0,0){O}    \tkzDefPoint(0:2){A}
 \tkzDefPointBy[rotation=center O angle 80](A)  \tkzGetPoint{B}
 \tkzDrawSegments[-{Stealth}](O,A O,B)
 \tkzMarkAngles[size=1.5,-Stealth,teal](A,O,B)
 \tkzFindAngle(A,O,B)   \tkzGetAngle{an}
 \tkzLabelAngle[pos=1,teal](A,O,B){$ \pgfmathprintnumber{\an}^\circ$}
\tkzLabelPoints(A)  \tkzLabelPoints[above](B)
 \end{tikzpicture}}
      { {\textbackslash}tkzFindAngle(A,O,B) resulta em $80$}
 &
 \tkzsubf{\begin{tikzpicture}
 \tkzDefPoint(0,0){O}    \tkzDefPoint(0:2){A}
 \tkzDefPointBy[rotation=center O angle -80](A)  \tkzGetPoint{B}
 \tkzDrawSegments[-{Stealth}](O,A O,B)
 \tkzMarkAngles[size=1,-Stealth,red](A,O,B)
 \tkzFindAngle(A,O,B)   \tkzGetAngle{an}
 \tkzLabelAngle[pos=.75,red](A,O,B){$\pgfmathprintnumber{\an}^\circ$}
\tkzLabelPoints[right](A)  \tkzLabelPoints[below](B)
 \end{tikzpicture}}
  {{\textbackslash}tkzFindAngle(A,O,B) resulta em $\pgfmathprintnumber{\an}^\circ$}
 \\\hline
 \end{tabular}
 \end{figure}

Como podemos ver, a rotação $-80^\circ$ define um ângulo horário mas a macro
\tkzcname{tkzFindAngle} recupera um ângulo anti-horário.

\subsection{Recuperando um ângulo \tkzcname{tkzGetAngle}}
\begin{NewMacroBox}{tkzGetAngle}{\parg{nome da macro}}%
Atribui o valor em graus de um ângulo a uma macro. O valor é positivo e está entre $0^\circ$ e $360^\circ$. Esta macro recupera \tkzcname{tkzAngleResult} e armazena o resultado em uma nova macro.

\medskip

\begin{tabular}{lll}%
\toprule
argumentos             & exemplo & explicação             \\
\midrule
\TAline{nome da macro} {\tkzcname{tkzGetAngle}\{ang\}}{\tkzcname{ang} contém o valor do ângulo.}
\end{tabular}
\end{NewMacroBox}

Esta é uma macro auxiliar que permite recuperar o resultado da seguinte macro \tkzcname{tkzFindAngle}.

\subsection{Ângulo formado por três pontos}

\begin{NewMacroBox}{tkzFindAngle}{\parg{pt1,pt2,pt3}}%
O resultado é armazenado em uma macro \tkzcname{tkzAngleResult}.

\medskip

\begin{tabular}{lll}%
\toprule
argumentos     & exemplo & explicação     \\
\midrule
\TAline{(pt1,pt2,pt3)} {\tkzcname{tkzFindAngle}(A,B,C)}{\tkzcname{tkzAngleResult} fornece o ângulo ($\overrightarrow{BA},\overrightarrow{BC}$)}
\bottomrule
\end{tabular}

\medskip
A medida é sempre positiva e está entre $0^\circ$ e $360^\circ$. Com as convenções usuais, um ângulo anti-horário menor que um ângulo reto tem sempre uma medida entre $0^\circ$ e $180^\circ$, enquanto um ângulo horário menor que um ângulo reto terá uma medida maior que $180^\circ$. \tkzcname{tkzGetAngle} pode recuperar o ângulo.
\end{NewMacroBox}

\subsubsection{Verificação da medida de ângulo}

\begin{tkzexample}[latex=7cm,small]
\begin{tikzpicture}[scale=.75]
  \tkzDefPoint(-1,1){A}
  \tkzDefPoint(5,2){B}
  \tkzDefEquilateral(A,B)
  \tkzGetPoint{C}
  \tkzDrawPolygon(A,B,C)
  \tkzFindAngle(B,A,C) \tkzGetAngle{angleBAC}
  \edef\angleBAC{\fpeval{round(\angleBAC)}}
  \tkzDrawPoints(A,B,C)
  \tkzLabelPoints(A,B)
  \tkzLabelPoint[right](C){$C$}
  \tkzLabelAngle(B,A,C){\angleBAC$^\circ$}
  \tkzMarkAngle[size=1.5](B,A,C)
\end{tikzpicture}
\end{tkzexample}

\subsubsection{Determinação dos três ângulos de um triângulo}

\begin{tkzexample}[latex=6cm,small]
\begin{tikzpicture}
\tikzset{label angle style/.append style={pos=1.4}}
\tkzDefPoints{0/0/a,5/3/b,3/6/c}
\tkzDrawPolygon(a,b,c)
\tkzFindAngle(c,b,a)\tkzGetAngle{angleCBA}
\pgfmathparse{round(1+\angleCBA)}
\let\angleCBA\pgfmathresult
\tkzFindAngle(a,c,b)\tkzGetAngle{angleACB}
\pgfmathparse{round(\angleACB)}
\let\angleACB\pgfmathresult
\tkzFindAngle(b,a,c)\tkzGetAngle{angleBAC}
\pgfmathparse{round(\angleBAC)}
\let\angleBAC\pgfmathresult
\tkzMarkAngle(c,b,a)
\tkzLabelAngle(c,b,a){\tiny $\angleCBA^\circ$}
\tkzMarkAngle(a,c,b)
\tkzLabelAngle(a,c,b){\tiny $\angleACB^\circ$}
\tkzMarkAngle(b,a,c)
\tkzLabelAngle(b,a,c){\tiny $\angleBAC^\circ$}
\end{tikzpicture}
\end{tkzexample}

\subsubsection{Ângulo entre dois círculos}
Estamos procurando o ângulo formado pelas tangentes em um ponto de interseção

\begin{tkzexample}[latex=7cm,small]
\begin{tikzpicture}[scale=.4]
\pgfkeys{/pgf/number format/.cd,%
          fixed,precision=1}
\tkzDefPoints{0/0/A,6/0/B,4/2/C}
\tkzDrawCircles(A,C B,C)
\tkzDefLine[tangent at=C](A) \tkzGetPoint{a}
\tkzDefPointsBy[symmetry = center C](a){d}
\tkzDefLine[tangent at=C](B) \tkzGetPoint{b}
\tkzDrawLines[add=1 and 4](a,C  C,b)
\tkzFillAngle[fill=teal,opacity=.2%
                        ,size=2](b,C,d)
\tkzFindAngle(b,C,d)\tkzGetAngle{bcd}
\tkzLabelAngle[pos=1.25](b,C,d){%
  \tiny $\pgfmathprintnumber{\bcd}^\circ$}
\end{tikzpicture}
\end{tkzexample}

\subsection{Ângulo formado por uma reta com o eixo horizontal \tkzcname{tkzFindSlopeAngle}}
Muito mais interessante que a última. O resultado está entre -180 graus e +180 graus.

\begin{NewMacroBox}{tkzFindSlopeAngle}{\parg{A,B}}%
Determina a inclinação da reta (AB). O resultado é armazenado em uma macro \tkzcname{tkzAngleResult}.

\medskip
\begin{tabular}{lll}%
\toprule
argumentos  & exemplo & explicação     \\
\midrule
\TAline{(pt1,pt2)} {\tkzcname{tkzFindSlopeAngle}(A,B)}{}
\bottomrule
\end{tabular}

\medskip
\tkzcname{tkzGetAngle} pode recuperar o resultado. Se a recuperação não for necessária, você pode usar \tkzcname{tkzAngleResult}.
\end{NewMacroBox}

\subsubsection{Como usar \tkzcname{tkzFindSlopeAngle}}

 O ponto aqui é que $(AB)$ é a bissetriz de $\widehat{CAD}$, tal que a inclinação de $AD$ é zero. Recuperamos a inclinação de $(AB)$ e então rotacionamos duas vezes.

\begin{tkzexample}[latex=7cm,small]
\begin{tikzpicture}
 \tkzDefPoint(1,5){A} \tkzDefPoint(5,2){B}
 \tkzFindSlopeAngle(A,B)\tkzGetAngle{tkzang}
 \tkzDefPointBy[rotation= center A angle \tkzang ](B)
 \tkzGetPoint{C}
 \tkzDefPointBy[rotation= center A angle -\tkzang ](B)
 \tkzGetPoint{D}
 \tkzDrawSegment(A,B)
 \tkzDrawSegments[new](A,C A,D)
 \tkzDrawPoints(A,B,C,D)
 \tkzCompass[length=1](A,C)
 \tkzCompass[delta=10,brown](B,C)
 \tkzLabelPoints(B,C,D)
 \tkzLabelPoints[above left](A)
\end{tikzpicture}
\end{tkzexample}

\subsubsection{Uso de \tkzcname{tkzFindSlopeAngle} e \tkzcname{tkzGetAngle}}
Aqui está outra versão da construção de uma mediatriz

\begin{tkzexample}[latex=6cm,small]
\begin{tikzpicture}
 \tkzInit
 \tkzDefPoint(0,0){A}        \tkzDefPoint(3,2){B}
 \tkzDefLine[mediator](A,B)  \tkzGetPoints{I}{J}
 \tkzCalcLength(A,B)         \tkzGetLength{dAB}
 \tkzFindSlopeAngle(A,B)     \tkzGetAngle{tkzangle}
 \begin{scope}[rotate=\tkzangle]
   \tkzSetUpArc[color=gray,line width=0.2pt,%
     /tkzcompass/delta=10]
   \tkzDrawArc[R,arc](B,3/4*\dAB)(120,240)
   \tkzDrawArc[R,arc](A,3/4*\dAB)(-45,60)
   \tkzDrawLine(I,J)         \tkzDrawSegment(A,B)
  \end{scope}
  \tkzDrawPoints(A,B,I,J)    \tkzLabelPoints(A,B)
   \tkzLabelPoints[right](I,J)
\end{tikzpicture}
\end{tkzexample}

\subsubsection{Outro uso de \tkzcname{tkzFindSlopeAngle}}

\begin{tkzexample}[latex=7cm,small]
\begin{tikzpicture}[scale=1.5]
  \tkzDefPoint(1,2){A}    \tkzDefPoint(3,4){B}
  \tkzDefPoint(3,2){C}    \tkzDefPoint(3,1){D}
  \tkzDrawSegments(A,B A,C A,D)
  \tkzDrawPoints[color=red](A,B,C,D)
  \tkzLabelPoints(A,B,C,D)
  \tkzFindSlopeAngle(A,B)\tkzGetAngle{SAB}
  \tkzFindSlopeAngle(A,C)\tkzGetAngle{SAC}
  \tkzFindSlopeAngle(A,D)\tkzGetAngle{SAD}
  \pgfkeys{/pgf/number format/.cd,fixed,precision=2}
  \tkzText(1,5){A inclinação de (AB) é:
     $\pgfmathprintnumber{\SAB}^\circ$}
  \tkzText(1,4.5){A inclinação de (AC) é:
     $\pgfmathprintnumber{\SAC}^\circ$}
  \tkzText(1,4){A inclinação de (AD) é:
     $\pgfmathprintnumber{\SAD}^\circ$}
\end{tikzpicture}
\end{tkzexample}

\endinput

\section{Definição de ponto aleatório}
%<--------------------------------------------------------------------------->
%           points random
%<--------------------------------------------------------------------------->
No momento existem quatro possibilidades:
\begin{enumerate}
  \item ponto em um retângulo;
  \item em um segmento;
  \item em uma reta;
  \item em um círculo.
\end{enumerate}

\subsection{Obtendo pontos aleatórios}
Esta é a nova versão que substitui  \tkzcname{tkzGetRandPointOn}.
\begin{NewMacroBox}{tkzDefRandPointOn}{\oarg{opções locais}}%
{O resultado é um ponto com uma posição aleatória que pode ser nomeado com a macro \tkzcname{tkzGetPoint}. É possível usar \tkzname{tkzPointResult} se não for necessário reter os resultados.}

\medskip
\begin{tabular}{lll}%
\toprule
opções             & padrão & definição                         \\
\midrule
\TOline{rectangle=pt1 and pt2}  {}{[rectangle=A and B]}
\TOline{segment= pt1--pt2} {}{[segment=A--B]}
\TOline{line=pt1--pt2}{}{[line=A--B]}
\TOline{circle =center pt1 radius dim}{}{[circle = center A radius 2]}
\TOline{circle through=center pt1 through pt2}{}{[circle through= center A through B]}
\TOline{disk through=center pt1 through pt2}{}{[disk through=center A through B]}
\end{tabular}
\end{NewMacroBox}

\subsubsection{Ponto aleatório em um retângulo}

\begin{tkzexample}[latex=7cm,small]
\begin{tikzpicture}
  \tkzDefPoints{0/0/A,5/3/C}
  \tkzDefRandPointOn[rectangle = A and C]
  \tkzGetPoint{E}
  \tkzDefRectangle(A,C)\tkzGetPoints{B}{D}
  \tkzDrawPolygon[red](A,...,D)
  \tkzDrawPoints(A,...,E)
  \tkzLabelPoints(A,B)
  \tkzLabelPoints[above](C,D,E)
\end{tikzpicture}
\end{tkzexample}

\subsubsection{Ponto aleatório em um segmento ou uma reta}
\begin{tkzexample}[latex=7cm,small]
\begin{tikzpicture}
  \tkzDefPoints{0/0/A,5/2/C}
  \tkzDefRandPointOn[segment = A--C]\tkzGetPoint{B}
  \tkzDrawLine(A,C)
  \tkzDrawPoints(A,C) \tkzDrawPoint[red](B)
  \tkzLabelPoints(A,C) \tkzLabelPoints[red](B)
\end{tikzpicture}
\end{tkzexample}


\subsubsection{Ponto aleatório em um círculo ou um disco}
\begin{tkzexample}[latex=7cm,small]
\begin{tikzpicture}
\tkzDefPoints{3/2/A,1/1/B}
\tkzCalcLength(A,B) \tkzGetLength{rAB}
\tkzDefRandPointOn[circle = center A radius \rAB]
\tkzGetPoint{C}
\tkzDefRandPointOn[circle through= center A through B]
\tkzGetPoint{D}
\tkzDefRandPointOn[disk through=center A through B]
\tkzGetPoint{E}
\tkzDrawCircle(A,B)
\tkzDrawPoints(A,B)
\tkzLabelPoints(A,B)
\tkzDrawPoints[red](C,D,E)
\tkzLabelPoints[red,right](C,D,E)
\end{tikzpicture}
\end{tkzexample}

\endinput


\part{Desenhando e Preenchendo}
\section{Desenho}
\tkzname{\tkznameofpack} pode desenhar 5 tipos de objetos: ponto, reta ou segmento de reta, círculo, arco e setor.

%<---------------------------------------------------------------------------->
%    POINT(S)
%<---------------------------------------------------------------------------->
\subsection{Desenhar um ponto ou alguns pontos}
Existem duas possibilidades: \tkzcname{tkzDrawPoint} para um único ponto ou \tkzcname{tkzDrawPoints} para um ou mais pontos.

\subsubsection{Desenhando pontos \tkzcname{tkzDrawPoint}} \hypertarget{tdrp}{}

\begin{NewMacroBox}{tkzDrawPoint}{\oarg{local opções}\parg{name}}%
\begin{tabular}{lll}%
argumentos &  padrão & definição                 \\
\midrule
\TAline{name of point} {sem padrão}  {Apenas um nome de ponto é aceito}
\bottomrule
\end{tabular}

\medskip
O argumento é obrigatório. O disco assume a cor do círculo, mas mais claro. É possível alterar tudo. O ponto é um nó e, portanto, é invariante se o desenho for modificado por escala.

\medskip
\begin{tabular}{lll}%
\toprule
opções             & padrão & definição \\
\midrule
\TOline{\TIKZ\ opções}{}{todas as opções \TIKZ\ são válidas.}
\TOline{shape}  {circle}{Possível \tkzname{cross} ou \tkzname{cross out}}
\TOline{size}   {6}{$6 \times$ \tkzcname{pgflinewidth}}
\TOline{color}  {black}{a cor padrão pode ser alterada}
\bottomrule
\end{tabular}

\medskip
{Podemos criar outras formas como \tkzname{cross}}
\end{NewMacroBox}

Por padrão, \tkzname{point style} é definido assim:

\begin{tkzltxexample}[]
  \tikzset{point style/.style = {%
           draw         = black,
           inner sep    = 0pt,
           shape        = circle,
           minimum size = 3 pt,
           fill         = black
                               }
        } 
\end{tkzltxexample}

\subsubsection{Exemplo de desenho de pontos}
Note que \tkzname{scale} não afeta a forma dos pontos. O que é normal. Na maioria das vezes, ficamos satisfeitos com uma única forma de ponto que podemos definir desde o início, seja com uma macro ou modificando um arquivo de configuração.

\begin{tkzexample}[latex=5cm,small]
  \begin{tikzpicture}[scale=.5]
   \tkzDefPoint(1,3){A}
   \tkzDefPoint(4,1){B}
   \tkzDefPoint(0,0){O}
   \tkzDrawPoint[color=red](A)
   \tkzDrawPoint[fill=blue!20,draw=blue](B)
   \tkzDrawPoint[shape=cross,size=8pt,color=teal](O)
  \end{tikzpicture}
\end{tkzexample}

É possível desenhar vários pontos de uma só vez, mas esta macro é um pouco mais lenta que a anterior. Além disso, temos que nos contentar com as mesmas opções para todos os pontos.
\newpage
\hypertarget{tdrps}{}
\begin{NewMacroBox}{tkzDrawPoints}{\oarg{local opções}\parg{liste}}%
\begin{tabular}{lll}%
argumentos &  padrão  & definição \\
\midrule
\TAline{points list}{sem padrão}{exemplo \tkzcname{tkzDrawPoints(A,B,C)}}
\bottomrule
\end{tabular}

\medskip
\begin{tabular}{lll}%
opções             & padrão & definição \\
\midrule
\TOline{shape}  {circle}{Possível \tkzname{cross} ou \tkzname{cross out}}
\TOline{size}  {6}{$6 \times$ \tkzcname{pgflinewidth}}
\TOline{color}  {black}{a cor padrão pode ser alterada}
\bottomrule
\end{tabular}

\medskip
\tkzHandBomb\ Cuidado com o \code{s} final, um descuido leva a erros em cascata se você tentar desenhar vários pontos. As opções são as mesmas da macro anterior.
\end{NewMacroBox}

\subsubsection{Exemplo}

\begin{tkzexample}[latex=7cm,small]
\begin{tikzpicture}
\tkzDefPoints{1/3/A,4/1/B,0/0/C}
\tkzDrawPoints[size=3,color=red,fill=red!50](A,B,C)
\end{tikzpicture}
\end{tkzexample}
%<---------------------------------------------------------------------------->
%    LINE(S)
%<---------------------------------------------------------------------------->
\section{Desenhando as retas}
As macros seguintes são simplesmente usadas para desenhar e nomear retas.
\subsection{Desenhar uma reta}
Para desenhar uma reta normal, basta fornecer um par de pontos. Você pode usar a opção \tkzname{add} para estender a reta (Esta opção é devida a \tkzimp{Mark Wibrow}, veja o código abaixo).

O estilo de uma reta é por padrão:

\begin{tkzltxexample}[]
  \tikzset{line style/.style = {%
    line width = 0.6pt,
    color      = black,
    style      = solid,
    add        = {.2} and  {.2}%
   }}
\end{tkzltxexample}
with
   
\begin{tkzltxexample}[]
  \tikzset{%
    add/.style args={#1 and #2}{
        to path={%
 ($(\tikztostart)!-#1!(\tikztotarget)$)--($(\tikztotarget)!-#2!(\tikztostart)$)%
  \tikztonodes}}}
\end{tkzltxexample}

Você pode modificar este estilo com \tkzcname{tkzSetUpLine} veja \ref{tkzsetupline}

\newpage
\begin{NewMacroBox}{tkzDrawLine}{\oarg{local opções}\parg{pt1,pt2} }%
Os argumentos são uma lista de dois ou três pontos. Seria possível, como para uma semirreta, criar um estilo com \tkzcname{add}.

\begin{tabular}{lll}%
\toprule
opções             & padrão & definição                         \\
\midrule
\TOline{\TIKZ\ opções}{}{todas as opções \TIKZ\ são válidas.}
\TOline{add}{0.2 and 0.2}{add = $kl$ and $kr$, \dots}
\TOline{\dots}{\dots}{permite que o segmento seja estendido para a esquerda e direita.}
\bottomrule
\end{tabular}

\tkzname{add} define o comprimento da reta passando pelos pontos pt1 e pt2. Ambos os números são percentagens. Os estilos do \TIKZ\ estão acessíveis para os desenhos.
\end{NewMacroBox}

\subsubsection{Exemplos  with \tkzname{add}}
\begin{tkzexample}[latex=5cm,small]
\begin{tikzpicture}
 \tkzInit[xmin=-2,xmax=3,ymin=-2.25,ymax=2.25]
 \tkzClip[space=.25]
 \tkzDefPoint(0,0){A} \tkzDefPoint(2,0.5){B}
 \tkzDefPoint(0,-1){C}\tkzDefPoint(2,-0.5){D} 
 \tkzDefPoint(0,1){E} \tkzDefPoint(2,1.5){F} 
 \tkzDefPoint(0,-2){G} \tkzDefPoint(2,-1.5){H}
 \tkzDrawLine(A,B)    \tkzDrawLine[add = 0 and .5](C,D) 
 \tkzDrawLine[add = 1 and 0](E,F)
 \tkzDrawLine[add = 0 and 0](G,H) 
 \tkzDrawPoints(A,B,C,D,E,F,G,H)    
 \tkzLabelPoints(A,B,C,D,E,F,G,H)  
\end{tikzpicture}
\end{tkzexample} 

É possível desenhar várias retas, mas com as mesmas opções.
\begin{NewMacroBox}{tkzDrawLines}{\oarg{local opções}\parg{pt1,pt2 pt3,pt4 ...}}%
Os argumentos são uma lista de pares de pontos separados por espaços. Os estilos do \TIKZ\ estão disponíveis para os desenhos.
\end{NewMacroBox}      

\subsubsection{Exemplo with \tkzcname{tkzDrawLines}}    

\begin{tkzexample}[latex=8cm,small]
\begin{tikzpicture}
  \tkzDefPoint(0,0){A}
  \tkzDefPoint(2,0){B}
  \tkzDefPoint(1,2){C}
  \tkzDefPoint(3,2){D}   
  \tkzDrawLines(A,B C,D A,C B,D)
  \tkzLabelPoints(A,B,C,D)
\end{tikzpicture}
\end{tkzexample}
%<---------------------------------------------------------------------------->
%    SEGMENT(S)
%<---------------------------------------------------------------------------->
\section{Desenhando um segmento}
Há, é claro, uma macro para simplesmente desenhar um segmento.

\subsection{Desenhar um segmento \tkzcname{tkzDrawSegment}}
\begin{NewMacroBox}{tkzDrawSegment}{\oarg{local opções}\parg{pt1,pt2}}%
Os argumentos são uma lista de dois pontos. Os estilos do \TIKZ\ estão disponíveis para os desenhos.
 
\medskip
\begin{tabular}{lll}%
argument    & exemplo & definição    \\
\midrule
\TAline{(pt1,pt2)}{(A,B)}{desenha o segmento $[A,B]$}
\bottomrule 
\end{tabular}
 
\medskip
\begin{tabular}{lll}%
opções    & exemplo & definição    \\
\midrule
\TOline{\TIKZ\ opções}{}{todas as opções \TIKZ\ são válidas.}
\TOline{dim}{sem padrão}{dim = \{label,dim,opção\}, \dots}
\TOline{\dots}{\dots}{permite adicionar dimensões a uma figura.}
\bottomrule
\end{tabular}

Isso é, claro, equivalente a \tkzcname{draw (A)--(B);}. Você também pode usar a opção \tkzname{add}.
\end{NewMacroBox}

\subsubsection{Exemplo com referências de pontos}     

\begin{tkzexample}[latex=6cm,small]
\begin{tikzpicture}[scale=1.5]
  \tkzDefPoint(0,0){A}
  \tkzDefPoint(2,1){B}
  \tkzDrawSegment[color=red,thin](A,B)
  \tkzDrawPoints(A,B)    
  \tkzLabelPoints(A,B)  
\end{tikzpicture}
\end{tkzexample}

\subsubsection{Exemplo de extensão de um segmento com opção \tkzname{add}} 

\begin{tkzexample}[latex=7cm,small]
\begin{tikzpicture}
  \tkzDefPoints{0/0/A,6/0/B,0.8/4/C}
  \tkzDefTriangleCenter[euler](A,B,C) 
  \tkzGetPoint{E}
  \tkzDefCircle[euler](A,B,C)\tkzGetPoints{E}{e}
  \tkzDrawCircle[red](E,e)
  \tkzDrawLines[add=.5 and .5](A,B A,C B,C)
  \tkzDrawPoints(A,B,C,E)
  \tkzLabelPoints(A,B,C,E)
  \end{tikzpicture}
\end{tkzexample}

\subsubsection{Adicionando dimensões com opção \tkzname{dim} novo código de Muzimuzhi Z}
Este código vem de uma resposta a esta pergunta no tex.stackexchange.com
(change-color-and-style-of-dimension-lines-in-tkz-euclide ).
O código do \tkzname{dim} é baseado em opções do TikZ, você deve adicionar as unidades.
Você pode usar agora dois estilos: |dim style| e |dim fence style|. Você tem várias maneiras de usá-los.
Vou deixar você olhar os exemplos para ver o que pode fazer com esses estilos.

\begin{verbatim}
   \tikzset{dim style/.append style={dashed}} % append if you want to keep precedent style.
   or 
   \begin{scope}[ dim style/.append style={orange},
       dim fence style/.style={dashed}]
\end{verbatim}

\begin{tkzexample}[latex=7cm]
\begin{tikzpicture}[scale=.75]
  \tkzDefPoints{0/3/A, 1/-3/B}
  \tkzDrawPoints(A,B)
  \tkzDrawSegment[dim={\(l_0\),1cm,right=2mm}, 
    dim style/.append style={red, 
    dash pattern={on 2pt off 2pt}}](A,B)
  \tkzDrawSegment[dim={\(l_1\),2cm,right=2mm}, 
    dim style/.append style={blue}](A,B)
  \begin{scope}[ dim style/.style={orange},
      dim fence style/.style={dashed}]
    \tkzDrawSegment[dim={\(l_2\),3cm,right=2mm}](A,B)  
    \tkzDrawSegment[dim={\(l_3\),-2cm,right=2mm}](A,B)   
  \end{scope}  
  \tkzLabelPoints[left](A,B)
\end{tikzpicture}
\end{tkzexample}

\subsubsection{Adicionando dimensões com opção \tkzname{dim} parte I} 
\begin{tkzexample}[latex=7cm,small]
\begin{tikzpicture}[scale=2]
\pgfkeys{/pgf/number format/.cd,fixed,precision=2}
\tkzDefPoint(0,0){A}
\tkzDefPoint(3.07,0){B}
\tkzInterCC[R](A,2.37)(B,1.82)
\tkzGetPoints{C}{C'}
\tkzDefCircle[in](A,B,C) \tkzGetPoints{G}{g}
\tkzDrawCircle(G,g)
\tkzDrawPolygon(A,B,C)
\tkzDrawPoints(A,B,C)
\tkzCalcLength(A,B)\tkzGetLength{ABl}
\tkzCalcLength(B,C)\tkzGetLength{BCl}
\tkzCalcLength(A,C)\tkzGetLength{ACl}
\begin{scope}[dim style/.style={dashed,sloped,teal}]
  \tkzDrawSegment[dim={\pgfmathprintnumber\BCl,6pt,%
                                      text=red}](C,B)
  \tkzDrawSegment[dim={\pgfmathprintnumber\ACl,%
                                        6pt,}](A,C)
  \tkzDrawSegment[dim={\pgfmathprintnumber\ABl,%
                                      -6pt,}](A,B)
\end{scope}
\tkzLabelPoints(A,B) \tkzLabelPoints[above](C)
\end{tikzpicture}
\end{tkzexample}

\subsubsection{Adicionando dimensões com opção \tkzname{dim} parte II} 
\begin{tkzexample}[latex=6cm,small]
\begin{tikzpicture}[scale=.5]
  \tkzDefPoints{0/0/O,-2/0/A,2/0/B,
                -2/4/C,2/4/D,2/-4/E,-2/-4/F}
  \tkzDrawPolygon(C,...,F)
  \tkzDrawSegments(A,B)
  \tkzDrawPoints(A,...,F,O)
  \tkzLabelPoints[below left](A,...,F,O)
  \tkzDrawSegment[dim={ $\sqrt{5}$,2cm,}](C,E)
  \tkzDrawSegment[dim={ $\frac{\sqrt{5}}{2}$,1cm,}](O,E)
  \tkzDrawSegment[dim={ $2$,2cm,left=8pt}](F,C)
  \tkzDrawSegment[dim={ $1$,1cm,left=8pt}](F,A)
\end{tikzpicture}
\end{tkzexample}

\subsection{Desenhando segmentos \tkzcname{tkzDrawSegments}}
Se as opções são as mesmas, podemos desenhar vários segmentos com a mesma macro.

\begin{NewMacroBox}{tkzDrawSegments}{\oarg{local opções}\parg{pt1,pt2 pt3,pt4 ...}}%
Os argumentos são uma lista de pares de pontos. Os estilos do \TIKZ\ estão disponíveis para os desenhos.
\end{NewMacroBox}

\begin{tkzexample}[latex=6cm,small]
\begin{tikzpicture}
  \tkzInit[xmin=-1,xmax=3,ymin=-1,ymax=2]
  \tkzClip[space=1]
  \tkzDefPoint(0,0){A}
  \tkzDefPoint(2,1){B} 
  \tkzDefPoint(3,0){C} 
  \tkzDrawSegments(A,B B,C)
  \tkzDrawPoints(A,B,C)    
  \tkzLabelPoints(A,C) 
  \tkzLabelPoints[above](B)  
\end{tikzpicture}
\end{tkzexample}

\subsubsection{Colocar uma seta em um segmento}
\begin{tkzexample}[latex=6cm,small]
\begin{tikzpicture}
\tkzSetUpStyle[postaction=decorate,
    decoration={markings, 
    mark=at position .5 with {\arrow[thick]{#1}}
      }]{myarrow}
  \tkzDefPoint(0,0){A}
  \tkzDefPoint(4,-4){B}
  \tkzDrawSegments[myarrow=stealth](A,B)
  \tkzDrawPoints(A,B) 
\end{tikzpicture}
\end{tkzexample}

\subsection{Desenhando segmentos de reta de um triângulo}

\subsubsection{Como desenhar \tkzname{Altitude}} 
\begin{tkzexample}[latex=7cm,small]
  \begin{tikzpicture}[rotate=-90]
  \tkzDefPoint(0,1){A}
  \tkzDefPoint(2,4){C}
  \tkzDefPointWith[orthogonal normed,K=7](C,A)
  \tkzGetPoint{B}
  \tkzDefSpcTriangle[orthic,name=H](A,B,C){a,b,c}
  \tkzDrawLine[dashed,color=magenta](C,Hc)
  \tkzDrawSegment[green!60!black](A,C)
  \tkzDrawSegment[green!60!black](C,B)
  \tkzDrawSegment[green!60!black](B,A)
  \tkzLabelPoint[left](A){$A$}
  \tkzLabelPoint[right](B){$B$}
  \tkzLabelPoint[above](C){$C$}
  \tkzLabelPoint[left](Hc){$Hc$}
  \tkzLabelSegment[auto](B,A){$c$}
  \tkzLabelSegment[auto,swap](B,C){$a$}
  \tkzLabelSegment[auto,swap](C,A){$b$}
  \tkzMarkAngle[size=1,color=cyan,mark=|](C,B,A)
  \tkzMarkAngle[size=1,color=cyan,mark=|](A,C,Hc)
  \tkzMarkAngle[size=0.75,
                color=orange,mark=||](Hc,C,B)
  \tkzMarkAngle[size=0.75,
                color=orange,mark=||](B,A,C)
  \tkzMarkRightAngle(A,C,B)
  \tkzMarkRightAngle(B,Hc,C)
  \end{tikzpicture} 
\end{tkzexample}

\subsection{Desenhando um polígono}
 \begin{NewMacroBox}{tkzDrawPolygon}{\oarg{local opções}\parg{points list}}%
Basta fornecer uma lista de pontos e a macro desenha o polígono usando as opções \TIKZ\ presentes. Você pode substituir $(A,B,C,D,E)$ por $(A,...,E)$ e $(P_1,P_2,P_3,P_4,P_5)$ por $(P_1,P...,P_5)$

\begin{tabular}{lll}%
\toprule
argumentos             & exemplo & explicação                         \\
\midrule
\TAline{\parg{pt1,pt2,pt3,...}}{|\BS tkzDrawPolygon[gray,dashed](A,B,C)|}{Desenhando um triângulo}
\end{tabular}

\medskip
\begin{tabular}{lll}%
\toprule
opções             & padrão & exemplo                         \\
\midrule
\TOline{Opções TikZ}{...}{|\BS tkzDrawPolygon[red,line width=2pt](A,B,C)|}
 \end{tabular}
\end{NewMacroBox}

\subsubsection{\tkzcname{tkzDrawPolygon}}

\begin{tkzexample}[latex=7cm, small]  
\begin{tikzpicture} [rotate=18,scale=1]
 \tkzDefPoints{0/0/A,2.25/0.2/B,2.5/2.75/C,-0.75/2/D}
 \tkzDrawPolygon(A,B,C,D)
 \tkzDrawSegments[style=dashed](A,C B,D) 
\end{tikzpicture}
\end{tkzexample}

\subsubsection{Opção \tkzname{two angles}}
\begin{tkzexample}[latex=6 cm,small]
\begin{tikzpicture}
\tkzDefPoint(0,0){A} 
\tkzDefPoint(6,0){B} 
\tkzDefTriangle[two angles = 50 and 70](A,B) \tkzGetPoint{C}
\tkzDrawPolygon(A,B,C)
\tkzLabelAngle[pos=1.4](B,A,C){$50^\circ$}
\tkzLabelAngle[pos=0.8](C,B,A){$70^\circ$}
\end{tikzpicture}
\end{tkzexample}

\subsubsection{Estilo de linha}
\begin{tkzexample}[latex=8 cm,small]
\begin{tikzpicture}[scale=.6]
\tkzSetUpLine[line width=5mm,color=teal]
\tkzDefPoint(0,0){O}
\foreach \i in {0,...,5}{%
 \tkzDefPoint({30+60*\i}:4){p\i}}
\tkzDefMidPoint(p1,p3) \tkzGetPoint{m1}
\tkzDefMidPoint(p3,p5) \tkzGetPoint{m3}
\tkzDefMidPoint(p5,p1) \tkzGetPoint{m5}
\tkzDrawPolygon[line join=round](p1,p3,p5)
\tkzDrawPolygon[teal!80,
line join=round](p0,p2,p4)
\tkzDrawSegments(m1,p3 m3,p5 m5,p1)
\tkzDefCircle[R](O,4.8)\tkzGetPoint{o}
\tkzDrawCircle[teal](O,o)
\end{tikzpicture}
\end{tkzexample}

\subsection{Desenhando uma cadeia poligonal}
 \begin{NewMacroBox}{tkzDrawPolySeg}{\oarg{local opções}\parg{points list}}%
Basta fornecer uma lista de pontos e a macro desenha a cadeia poligonal usando as opções \TIKZ\ presentes.

\begin{tabular}{lll}%
\toprule
argumentos             & exemplo & explicação                         \\
\midrule
\TAline{\parg{pt1,pt2,pt3,...}}{|\BS tkzDrawPolySeg[gray,dashed](A,B,C)|}{Desenhando um triângulo}
\end{tabular}

\medskip
\begin{tabular}{lll}%
\toprule
opções             & padrão & exemplo                         \\
\midrule
\TOline{Opções TikZ}{...}{|\BS tkzDrawPolySeg[red,line width=2pt](A,B,C)|}
 \end{tabular}
\end{NewMacroBox}

\subsubsection{Cadeia poligonal}

\begin{tkzexample}[latex=7cm, small]  
\begin{tikzpicture}
 \tkzDefPoints{0/0/A,6/0/B,3/4/C,2/2/D}          
 \tkzDrawPolySeg(A,...,D)
 \tkzDrawPoints(A,...,D)
\end{tikzpicture}
\end{tkzexample}

\subsubsection{A ideia é inscrever dois quadrados em um semicírculo.}
Um visual Sangaku! Trata-se de provar que se pode inscrever em um meio disco, dois quadrados, e determinar o comprimento de seus respectivos lados de acordo com o raio.

\begin{tkzexample}[latex=7 cm,small]
\begin{tikzpicture}[scale=.75] 
  \tkzDefPoints{0/0/A,8/0/B,4/0/I}
  \tkzDefSquare(A,B)    \tkzGetPoints{C}{D} 
  \tkzInterLC(I,C)(I,B) \tkzGetPoints{E'}{E} 
  \tkzInterLC(I,D)(I,B) \tkzGetPoints{F'}{F} 
  \tkzDefPointsBy[projection=onto A--B](E,F){H,G} 
  \tkzDefPointsBy[symmetry = center H](I){J} 
  \tkzDefSquare(H,J)     \tkzGetPoints{K}{L} 
  \tkzDrawSector(I,B)(A) 
  \tkzDrawPolySeg(H,E,F,G) 
  \tkzDrawPolySeg(J,K,L) 
  \tkzDrawPoints(E,G,H,F,J,K,L)
\end{tikzpicture}
\end{tkzexample}

\subsubsection{Cadeia poligonal: notação de índice}

\begin{tkzexample}[latex=7cm, small]  
\begin{tikzpicture}
\foreach \pt in {1,2,...,8} {%
\tkzDefPoint(\pt*20:3){P_\pt}}     
\tkzDrawPolySeg(P_1,P_...,P_8)
\tkzDrawPoints(P_1,P_...,P_8)
\end{tikzpicture}
\end{tkzexample}
%<---------------------------------------------------------------------------->
%    CIRCLE
%<---------------------------------------------------------------------------->
\section{Desenhar um círculo com \tkzcname{tkzDrawCircle}}

\subsection{Desenhar um círculo}
\begin{NewMacroBox}{tkzDrawCircle}{\oarg{local opções}\parg{A,B}}%
\tkzHandBomb\ Atenção, você precisa apenas de dois pontos para definir um raio. Uma opção adicional \tkzname{R} está disponível para fornecer uma medida diretamente.

\medskip
\begin{tabular}{lll}%
\toprule
argumentos           & exemplo & explicação                         \\
\midrule
\TAline{\parg{pt1,pt2}}{\parg{A,B}} {A centro passando por B}
 \bottomrule
\end{tabular}

\medskip
Claro, você tem que adicionar todos os estilos do \TIKZ\ para os traçados...
\end{NewMacroBox}

 \subsubsection{Círculos e estilos, desenhar um círculo e colorir o disco}
 Veremos que é possível colorir um disco enquanto se traça o círculo.
 
\begin{tkzexample}[latex=7cm,small]
\begin{tikzpicture}
  \tkzDefPoint(0,0){O} 
  \tkzDefPoint(3,0){A}
 % circle with center O and passing through A
  \tkzDrawCircle(O,A) 
 % diameter circle $[OA]$
 \tkzDefCircle[diameter](O,A) \tkzGetPoint{I}
 \tkzDrawCircle[new,fill=orange!10,opacity=.5](I,A)
 % circle with center O and radius = exp(1) cm
  \edef\rayon{\fpeval{0.25*exp(1)}}
  \tkzDefCircle[R](O,\rayon) \tkzGetPoint{o}
   \tkzDrawCircle[color=orange](O,o) 
\end{tikzpicture} 
\end{tkzexample}  

\subsection{Desenhando círculos}
\begin{NewMacroBox}{tkzDrawCircles}{\oarg{local opções}\parg{A,B C,D \dots}}%
\tkzHandBomb\ Atenção, os argumentos são listas de dois pontos. Os círculos que podem ser desenhados são os mesmos da macro anterior. Uma opção adicional \tkzname{R} está disponível para fornecer uma medida diretamente.

\medskip
\begin{tabular}{lll}%
\toprule
argumentos           & exemplo & explicação                         \\
\midrule
\TAline{\parg{pt1,pt2 pt3,pt4 ...}}{\parg{A,B C,D}} {Lista de dois pontos}
\bottomrule
\end{tabular}

\medskip
\begin{tabular}{lll}%
\toprule
opções             & padrão & definição                         \\
\midrule
\TOline{through}{through}{círculo com dois pontos definindo um raio}
 \bottomrule
\end{tabular}

\medskip
Você não precisa usar a opção padrão \tkzname{through}.
Claro, você tem que adicionar todos os estilos do \TIKZ\ para os traçados...
\end{NewMacroBox}

 \subsubsection{Círculos definidos por um triângulo.} 
 
\begin{tkzexample}[latex=9cm,small]
\begin{tikzpicture}
  \tkzDefPoints{0/0/A,2/0/B,3/2/C}
  \tkzDrawPolygon(A,B,C)
  \tkzDrawCircles(A,B B,C C,A)
  \tkzDrawPoints(A,B,C)
  \tkzLabelPoints(A,B,C) 
\end{tikzpicture} 
\end{tkzexample}

\subsubsection{Círculos concêntricos.} 
 
\begin{tkzexample}[latex=7cm,small]
\begin{tikzpicture}
   \tkzDefPoints{0/0/A,1/0/a,2/0/b,3/0/c}
   \tkzDrawCircles(A,a A,b A,c)
   \tkzDrawPoint(A)
   \tkzLabelPoints(A)
\end{tikzpicture}
\end{tkzexample}

\subsubsection{Círculos ex-inscritos.} 

\begin{tkzexample}[latex=8cm,small] 
\begin{tikzpicture}[scale=1] 
\tkzDefPoints{0/0/A,4/0/B,1/2.5/C}
\tkzDrawPolygon(A,B,C)
\tkzDefCircle[ex](B,C,A) 
\tkzGetPoint{J_c} \tkzGetSecondPoint{T_c}
\tkzDrawCircle(J_c,T_c)
\tkzDrawLines[add=0 and 1](C,A C,B)
\tkzDrawSegment(J_c,T_c)
\tkzMarkRightAngle(J_c,T_c,B)
\tkzDrawPoints(A,B,C,J_c,T_c)
\end{tikzpicture}
\end{tkzexample}
 
\subsubsection{Cardioide}
Baseado em uma ideia de O. Reboux feita com pst-eucl (módulo Pstricks) por D. Rodriguez.

 Seu nome vem do grego \textit{kardia (coração)}, em referência à sua forma, e foi dado a ela por Johan Castillon (Wikipedia).     
 
\begin{tkzexample}[latex=7cm,small]
\begin{tikzpicture}[scale=.5]
  \tkzDefPoint(0,0){O} 
  \tkzDefPoint(2,0){A}
  \foreach \ang in {5,10,...,360}{%
     \tkzDefPoint(\ang:2){M}
     \tkzDrawCircle(M,A) 
   }  
\end{tikzpicture} 
\end{tkzexample}

\newpage

\subsection{Desenhando semicírculo}
\begin{NewMacroBox}{tkzDrawSemiCircle}{\oarg{local opções}\parg{O,A}}%

\medskip
\begin{tabular}{lll}%
\toprule
argumentos           & exemplo & explicação                         \\
\midrule
\TAline{\parg{pt1,pt2}}{\parg{O,A}} {OA= raio}
\bottomrule
\end{tabular}

$O$ centro $A$ extremidade do semicírculo
\end{NewMacroBox}

\subsubsection{Uso de \tkzcname{tkzDrawSemiCircle}}   

\begin{tkzexample}[latex=7cm,small]
\begin{tikzpicture}
   \tkzDefPoint(0,0){A} \tkzDefPoint(6,0){B}
   \tkzDefMidPoint(A,B)  \tkzGetPoint{O}
   \tkzDrawSemiCircle[blue](O,B)
   \tkzDrawSemiCircle[red](O,A)
   \tkzDrawPoints(O,A,B)
   \tkzLabelPoints[below right](O,A,B)
 \end{tikzpicture}
\end{tkzexample}

\subsection{Desenhando semicírculos}

\begin{NewMacroBox}{tkzDrawSemiCircles}{\oarg{local opções}\parg{A,B C,D \dots}}%

\medskip
\begin{tabular}{lll}%
\toprule
argumentos           & exemplo & explicação                         \\
\midrule
\TAline{\parg{pt1,pt2 pt3,pt4 ...}}{\parg{A,B C,D}} {Lista de dois pontos}
\bottomrule
\end{tabular}

\end{NewMacroBox}

\subsubsection{Uso de \tkzcname{tkzDrawSemiCircles} : Arbelos dourado}  

\begin{tkzexample}[vbox,small]
\begin{tikzpicture}[scale=.75]
\tkzDefPoints{0/0/A,10/0/B}
\tkzDefGoldenRatio(A,B) \tkzGetPoint{C}
\tkzDefMidPoint(A,B)                     \tkzGetPoint{O_0}
\tkzDefMidPoint(A,C)                     \tkzGetPoint{O_1}
\tkzDefMidPoint(C,B)                     \tkzGetPoint{O_2}
\tkzLabelPoints(A,B,C)
\tkzDrawSegment(A,B)
\tkzDrawPoints(A,B,C)
\begin{scope}[local bounding box = graph]
  \tkzDrawSemiCircles[color=black](O_0,B)
\end{scope}
\useasboundingbox (graph.south west) rectangle (graph.north east);
\tkzClipCircle[out](O_1,C)\tkzClipCircle[out](O_2,B)
\tkzDrawSemiCircles[draw=none,fill=teal!15](O_0,B)
\tkzDrawSemiCircles[color=black](O_1,C O_2,B)
\end{tikzpicture}
\end{tkzexample}

%<---------------------------------------------------------------------------->
%    Ellipse
%<---------------------------------------------------------------------------->
\section{Desenhar uma elipse com \tkzcname{tkzDrawEllipse}}

\subsection{Desenhar uma elipse}
\begin{NewMacroBox}{tkzDrawEllipse}{\oarg{local opções}\parg{C,a,b,An}}%


\medskip
\begin{tabular}{lll}%
\toprule
argumentos           & exemplo & explicação                         \\
\midrule
\TAline{\parg{C,a,b,An}}{\parg{C,4,2,45}} {C centro; 4 e 2 comprimentos dos semi-eixos} \\
 & & 45 inclinação do eixo principal  \\
 \bottomrule
\end{tabular}

\medskip
Claro, você tem que adicionar todos os estilos do \TIKZ\ para os traçados...
\end{NewMacroBox}

\subsubsection{Exemplo de desenho de uma elipse}
\begin{tkzexample}[latex=6cm,small]
   \begin{tikzpicture}[scale=.75]
      \tkzDefPoint(0,4){C}
      \tkzDrawEllipse[blue](C,4,2,45)
      \tkzLabelPoints(C)
   \end{tikzpicture}
\end{tkzexample}

%<---------------------------------------------------------------------------->
%    ARC
%<---------------------------------------------------------------------------->
\section{Desenhando arcos}
\subsection{Macro: \tkzcname{tkzDrawArc} }
\begin{NewMacroBox}{tkzDrawArc}{\oarg{local opções}\parg{O,\dots}\parg{\dots}}%
Esta macro traça o arco de centro $O$. Dependendo das opções, os argumentos diferem. Trata-se de determinar um ponto inicial e um ponto final. Ou o ponto inicial é dado, o que é mais simples, ou o raio do arco é dado. Neste último caso, é necessário ter dois ângulos. Ou os ângulos podem ser dados diretamente, ou nós associados ao centro podem ser dados para determiná-los. Os ângulos estão em graus.

\medskip
\begin{tabular}{lll}%
\toprule
opções             & padrão & definição                        \\ 
\midrule
\TOline{towards}{towards}{$O$ é o centro e o arco de $A$ a $(OB)$}
\TOline{rotate} {towards}{o arco começa de $A$ e o ângulo determina seu comprimento}
\TOline{R}{towards}{Damos o raio e dois ângulos}
\TOline{R with nodes}{towards}{Damos o raio e dois pontos}
\TOline{angles}{towards}{Damos o raio e dois pontos}
\TOline{delta}{0}{ângulo adicionado em cada lado}
\TOline{reverse}{false}{inversão do caminho do arco, interessante para inverter seta}
\bottomrule
\end{tabular}

\medskip
Claro, você tem que adicionar todos os estilos do \TIKZ\ para os traçados...

\medskip

\begin{tabular}{lll}%
\toprule
opções             & argumentos & exemplo                         \\ 
\midrule
\TOline{towards}{\parg{pt,pt}\parg{pt}}{\tkzcname{tkzDrawArc[delta=10](O,A)(B)}} 
\TOline{rotate} {\parg{pt,pt}\parg{an}}{\tkzcname{tkzDrawArc[rotate,color=red](O,A)(90)}}
\TOline{R}{\parg{pt,$r$}\parg{an,an}}{\tkzcname{tkzDrawArc[R](O,2)(30,90)}}
\TOline{R with nodes}{\parg{pt,$r$}\parg{pt,pt}}{\tkzcname{tkzDrawArc[R with nodes](O,2)(A,B)}}
\TOline{angles}{\parg{pt,pt}\parg{an,an}}{\tkzcname{tkzDrawArc[angles](O,A)(0,90)}}
\end{tabular}
\end{NewMacroBox}

Aqui estão alguns exemplos:

\subsubsection{Opção \tkzname{towards}}
É inútil colocar \tkzname{towards}. Neste primeiro exemplo, o arco começa de $A$ e vai para $B$. O arco indo de $B$ para $A$ é diferente. O saliente é obtido indo na direção direta do círculo trigonométrico.
\begin{tkzexample}[latex=6cm,small]
\begin{tikzpicture}[scale=.75]
  \tkzDefPoint(0,0){O}
  \tkzDefPoint(2,-1){A}
  \tkzDefPointBy[rotation= center O angle 90](A)
  \tkzGetPoint{B}
  \tkzDrawArc[color=orange,<->](O,A)(B) 
  \tkzDrawArc(O,B)(A)
  \tkzDrawLines[add = 0 and .5](O,A O,B)
  \tkzDrawPoints(O,A,B)
  \tkzLabelPoints[below](O,A,B)  
\end{tikzpicture}
\end{tkzexample}

\subsubsection{Opção \tkzname{towards}}
Neste, o arco começa de A mas para na reta (OB).
 
\begin{tkzexample}[latex=6cm,small]
\begin{tikzpicture}[scale=0.75] 
  \tkzDefPoint(0,0){O}
  \tkzDefPoint(2,-1){A}
  \tkzDefPoint(1,1){B} 
  \tkzDrawArc[color=blue,->](O,A)(B)
  \tkzDrawArc[color=gray](O,B)(A)
  \tkzDrawArc(O,B)(A)
  \tkzDrawLines[add = 0 and .5](O,A O,B) 
  \tkzDrawPoints(O,A,B)
  \tkzLabelPoints[below](O,A,B)  
\end{tikzpicture}
\end{tkzexample}

\subsubsection{Opção \tkzname{rotate}}
\begin{tkzexample}[latex=6cm,small] 
\begin{tikzpicture}[scale=0.75] 
  \tkzDefPoint(0,0){O}
  \tkzDefPoint(2,-2){A}
  \tkzDefPoint(60:2){B}
  \tkzDrawLines[add = 0 and .5](O,A O,B)
  \tkzDrawArc[rotate,color=red](O,A)(180)
  \tkzDrawPoints(O,A,B)
  \tkzLabelPoints[below](O,A,B) 
\end{tikzpicture}
\end{tkzexample} 

\subsubsection{Opção \tkzname{R}} 
\begin{tkzexample}[latex=6cm,small]   
\begin{tikzpicture}[scale=0.75] 
  \tkzDefPoints{0/0/O}
  \tkzSetUpCompass[<->]
  \tkzDrawArc[R,color=teal,double](O,3)(270,360)
  \tkzDrawArc[R,color=orange,double](O,2)(0,270) 
  \tkzDrawPoint(O)
  \tkzLabelPoint[below](O){$O$}  
\end{tikzpicture} 
\end{tkzexample}

\subsubsection{Opção \tkzname{R with nodes}} 
\begin{tkzexample}[latex=6cm,small]
\begin{tikzpicture}[scale=0.75] 
  \tkzDefPoint(0,0){O}
  \tkzDefPoint(2,-1){A}
  \tkzDefPoint(1,1){B}
  \tkzCalcLength(B,A)\tkzGetLength{radius}
  \tkzDrawArc[R with nodes](B,\radius)(A,O)
\end{tikzpicture}
\end{tkzexample}

\subsubsection{Opção \tkzname{delta}}
Esta opção permite um pouco como \tkzcname{tkzCompass} colocar um arco e ultrapassar em cada lado. delta é uma medida em graus.

\begin{tkzexample}[latex=7cm,small] 
\begin{tikzpicture} 
 \tkzDefPoint(0,0){A}
 \tkzDefPoint(3,0){B}
 \tkzDefPointBy[rotation= center A angle 60](B)
 \tkzGetPoint{C} 
 \begin{scope}% style only local
   \tkzDefPointBy[symmetry= center C](A)
   \tkzGetPoint{D} 
   \tkzDrawSegments(A,B A,D)
   \tkzDrawLine(B,D)
   \tkzSetUpCompass[color=orange]
   \tkzDrawArc[orange,delta=10](A,B)(C)
   \tkzDrawArc[orange,delta=10](B,C)(A)
   \tkzDrawArc[orange,delta=10](C,D)(D)
 \end{scope}

 \tkzDrawPoints(A,B,C,D)
 \tkzLabelPoints[below right](A,B,C,D)
 \tkzMarkRightAngle(D,B,A)
\end{tikzpicture}
\end{tkzexample} 

\subsubsection{Opção \tkzname{angles}: exemplo 1}

\begin{tkzexample}[latex=6cm,small]
\begin{tikzpicture}[scale=.75]
  \tkzDefPoint(0,0){A}
  \tkzDefPoint(5,0){B}  
  \tkzDefPoint(2.5,0){O} 
  \tkzDefPointBy[rotation=center O angle 60](B)
  \tkzGetPoint{D}
  \tkzDefPointBy[symmetry=center D](O)
  \tkzGetPoint{E}
  \begin{scope}
    \tkzDrawArc[angles](O,B)(0,180)
    \tkzDrawArc[angles,](B,O)(100,180)  
    \tkzCompass[delta=20](D,E) 
    \tkzDrawLines(A,B O,E B,E)
    \tkzDrawPoints(A,B,O,D,E)
  \end{scope}
  \tkzLabelPoints[below right](A,B,O,D,E)
  \tkzMarkRightAngle(O,B,E) 
\end{tikzpicture} 
\end{tkzexample}

\subsubsection{Opção \tkzname{angles}: exemplo 2}

\begin{tkzexample}[latex=7cm,small]
  \begin{tikzpicture}
   \tkzDefPoint(0,0){O}
   \tkzDefPoint(5,0){I} 
   \tkzDefPoint(0,5){J}
   \tkzInterCC(O,I)(I,O)\tkzGetPoints{B}{C}  
   \tkzInterCC(O,I)(J,O)\tkzGetPoints{D}{A}
   \tkzInterCC(I,O)(J,O)\tkzGetPoints{L}{K}
   \tkzDrawArc[angles](O,I)(0,90)
   \tkzDrawArc[angles,color=gray,
               style=dashed](I,O)(90,180)
   \tkzDrawArc[angles,color=gray,
               style=dashed](J,O)(-90,0)
   \tkzDrawPoints(A,B,K)
   \foreach \point in {I,A,B,J,K}{%
               \tkzDrawSegment(O,\point)} 
  \end{tikzpicture} 
\end{tkzexample}

\subsubsection{Opção \tkzname{reverse}: inversão da seta}

\begin{tkzexample}[latex=6cm,small]
  \begin{tikzpicture}
    \tkzDefPoints{0/0/O,3/0/U}
    \tkzDefPoint(10:1){A}
    \tkzDefPoint(90:1){B}
    \tkzLabelPoints(A,B)
    \tkzDrawArc[reverse,tkz arrow={Stealth}](O,A)(B)
    \tkzDrawPoints(A,B,O)
  \end{tikzpicture}
\end{tkzexample}
%<---------------------------------------------------------------------------->
%    SECTOR
%<---------------------------------------------------------------------------->
\section{Desenhando um setor ou setores}
\subsection{\tkzcname{tkzDrawSector}}
\tkzHandBomb\  Atenção, os argumentos variam de acordo com as opções.
\begin{NewMacroBox}{tkzDrawSector}{\oarg{local opções}\parg{O,\dots}\parg{\dots}}%
\begin{tabular}{SlSlSl}%
opções             & padrão & definição                         \\
\midrule
\TOline{towards}{towards}{$O$ é o centro e o arco de $A$ a $(OB)$}
\TOline{rotate} {towards}{o arco começa de $A$ e o ângulo determina seu comprimento}
\TOline{R}{towards}{Damos o raio e dois ângulos}
\TOline{R with nodes}{towards}{Damos o raio e dois pontos}

\end{tabular}

\medskip
\emph{Você tem que adicionar, é claro, todos os estilos do \TIKZ\ para os traçados...}

\begin{tabular}{lll}%

opções             & argumentos & exemplo                         \\ 
\midrule
\TOline{towards}{\parg{pt,pt}\parg{pt}}{\tkzcname{tkzDrawSector(O,A)(B)}}
\TOline{rotate} {\parg{pt,pt}\parg{an}}{\tkzcname{tkzDrawSector[rotate,color=red](O,A)(90)}} 
\TOline{R}{\parg{pt,$r$}\parg{an,an}}{\tkzcname{tkzDrawSector[R,color=teal](O,2)(30,90)}}
\TOline{R with nodes}{\parg{pt,$r$}\parg{pt,pt}}{\tkzcname{tkzDrawSector[R with nodes](O,2)(A,B)}}
\end{tabular}
\end{NewMacroBox}

Aqui estão alguns exemplos:

\subsubsection{\tkzcname{tkzDrawSector} e \tkzname{towards}}
Não há necessidade de colocar \tkzname{towards}. Você pode usar \tkzname{fill} como uma opção.

\begin{tkzexample}[latex=7cm,small]
\begin{tikzpicture}
  \tkzDefPoint(0,0){O}
  \tkzDefPoint(-30:1){A} 
  \tkzDefPointBy[rotation = center O angle -60](A) 
  \tkzDrawSector[teal](O,A)(tkzPointResult)
 \begin{scope}[shift={(-60:1)}]
  \tkzDefPoint(0,0){O}
  \tkzDefPoint(-30:1){A} 
  \tkzDefPointBy[rotation = center O angle -60](A) 
  \tkzDrawSector[red](O,tkzPointResult)(A)
  \end{scope}
\end{tikzpicture}   
\end{tkzexample}

\subsubsection{\tkzcname{tkzDrawSector} and \tkzname{rotate}}  
\begin{tkzexample}[latex=7cm,small]  
\begin{tikzpicture}[scale=2]
 \tkzDefPoints{0/0/O,2/2/A,2/1/B}
 \tkzDrawSector[rotate,orange](O,A)(20)
 \tkzDrawSector[rotate,teal](O,B)(-20)
\end{tikzpicture} 
\end{tkzexample}  

\subsubsection{\tkzcname{tkzDrawSector} and \tkzname{R}}  
\begin{tkzexample}[latex=7cm,small]
\begin{tikzpicture}[scale=1.25]
 \tkzDefPoint(0,0){O}
 \tkzDefPoint(2,-1){A}
 \tkzDrawSector[R](O,1)(30,90)
 \tkzDrawSector[R](O,1)(90,180)
 \tkzDrawSector[R](O,1)(180,270)
 \tkzDrawSector[R](O,1)(270,360) 
\end{tikzpicture}
\end{tkzexample}

\subsubsection{\tkzcname{tkzDrawSector} e \tkzname{R with nodes}}
Neste exemplo, uso a opção \tkzname{fill}, mas \tkzcname{tkzFillSector} é possível.
\begin{tkzexample}[latex=7cm,small]
\begin{tikzpicture}[scale=1.25]
 \tkzDefPoint(0,0){O}
 \tkzDefPoint(4,-2){A}
 \tkzDefPoint(4,1){B}
 \tkzDefPoint(3,3){C}
 \tkzDrawSector[R with nodes,%
                fill=teal!20](O,1)(B,C)
 \tkzDrawSector[R with nodes,%
                fill=orange!20](O,1.25)(A,B)  
\tkzDrawSegments(O,A O,B O,C)
\tkzDrawPoints(O,A,B,C) 
\tkzLabelPoints(A,B,C) 
\tkzLabelPoints[left](O) 
\end{tikzpicture}
\end{tkzexample}

\subsubsection{\tkzcname{tkzDrawSector} and \tkzname{R with nodes}} 
\begin{tkzexample}[latex=6cm,small]
\begin{tikzpicture} [scale=.4]
 \tkzDefPoints{-1/-2/A,1/3/B}
 \tkzDefRegPolygon[side,sides=6](A,B) 
 \tkzGetPoint{O} 
 \tkzDrawPolygon[fill=black!10, draw=blue](P1,P...,P6) 
 \tkzLabelRegPolygon[sep=1.05](O){A,...,F}
 \tkzDrawCircle[dashed](O,A)
 \tkzLabelSegment[above,sloped,
                  midway](A,B){\(A B = 16m\)}
 \foreach \i  [count=\xi from 1]  in {2,...,6,1}
   {%
    \tkzDefMidPoint(P\xi,P\i)
    \path (O) to [pos=1.1] node {\xi} (tkzPointResult) ;
    }
  \tkzDefRandPointOn[segment = P3--P5] 
  \tkzGetPoint{S}
  \tkzDrawSegments[thick,dashed,red](A,S S,B)
  \tkzDrawPoints(P1,P...,P6,S)
  \tkzLabelPoint[left,above](S){$S$}
  \tkzDrawSector[R with nodes,fill=red!20](S,2)(A,B)
  \tkzLabelAngle[pos=1.5](A,S,B){$\alpha$}
\end{tikzpicture}
\end{tkzexample}

\endinput
\subsection{Colorindo um disco}
Isto era possível com a macro \tkzcname{tkzDrawCircle}, mas o traçado do disco era obrigatório, não é mais o caso.

\begin{NewMacroBox}{tkzFillCircle}{\oarg{opções locais}\parg{A,B}}%
\begin{tabular}{lll}%
opções             & padrão & definição                         \\
\midrule
\TOline{radius}  {radius}{dois pontos definem um raio}
\TOline{R} {radius}{um ponto e a medida de um raio }
\bottomrule
\end{tabular}

\medskip
Você não precisa colocar \tkzname{radius} porque essa é a opção padrão. Claro, você tem que adicionar todos os estilos do \TIKZ\ para os traçados.
\end{NewMacroBox}


\subsubsection{Yin e Yang}
\begin{tkzexample}[latex=8cm,small]
  \begin{tikzpicture}[scale=.75]
    \tkzDefPoint(0,0){O}
    \tkzDefPoint(-4,0){A}
    \tkzDefPoint(4,0){B}
    \tkzDefPoint(-2,0){I}
    \tkzDefPoint(2,0){J}
    \tkzDrawSector[fill=teal](O,A)(B)
    \tkzFillCircle[fill=white](J,B)
    \tkzFillCircle[fill=teal](I,A)
    \tkzDrawCircle(O,A)
  \end{tikzpicture}
\end{tkzexample}


\subsubsection{De um sangaku}

\begin{tkzexample}[latex=7cm,small]
\begin{tikzpicture}
   \tkzDefPoint(0,0){B}  \tkzDefPoint(6,0){C}%
   \tkzDefSquare(B,C)    \tkzGetPoints{D}{A}
   \tkzClipPolygon(B,C,D,A)
   \tkzDefMidPoint(A,D)  \tkzGetPoint{F}
   \tkzDefMidPoint(B,C)  \tkzGetPoint{E}
   \tkzDefMidPoint(B,D)  \tkzGetPoint{Q}
   \tkzDefLine[tangent from = B](F,A)
   \tkzGetPoints{H}{G}
   \tkzInterLL(F,G)(C,D) \tkzGetPoint{J}
   \tkzInterLL(A,J)(F,E) \tkzGetPoint{K}
   \tkzDefPointBy[projection=onto B--A](K)
   \tkzGetPoint{M}
   \tkzDrawPolygon(A,B,C,D)
   \tkzFillCircle[red!20](E,B)
   \tkzFillCircle[blue!20](M,A)
   \tkzFillCircle[green!20](K,Q)
  \tkzDrawCircles(B,A M,A E,B K,Q)
\end{tikzpicture}
\end{tkzexample}

\subsubsection{Recorte e preenchimento parte I}
\begin{tkzexample}[latex=7cm,small]
\begin{tikzpicture}
\tkzDefPoints{0/0/A,4/0/B,2/2/O,3/4/X,4/1/Y,1/0/Z,
              0/3/W,3/0/R,4/3/S,1/4/T,0/1/U}
\tkzDefSquare(A,B)\tkzGetPoints{C}{D}
\tkzDefPointWith[colinear normed=at X,K=1](O,X)
 \tkzGetPoint{F}
\begin{scope}
  \tkzFillCircle[fill=teal!20](O,F)
  \tkzFillPolygon[white](A,...,D)
  \tkzClipPolygon(A,...,D)
  \foreach \c/\t in {S/C,R/B,U/A,T/D}
  {\tkzFillCircle[teal!20](\c,\t)}
\end{scope}
\foreach \c/\t in {X/C,Y/B,Z/A,W/D}
{\tkzFillCircle[white](\c,\t)}
  \foreach \c/\t in {S/C,R/B,U/A,T/D}
  {\tkzFillCircle[teal!20](\c,\t)}
\end{tikzpicture}
\end{tkzexample}

\subsubsection{Recorte e preenchimento parte II}
\begin{tkzexample}[latex=7cm, small]
\begin{tikzpicture}[scale=.75]
\tkzDefPoints{0/0/A,8/0/B,8/8/C,0/8/D}
\tkzDefMidPoint(A,B) \tkzGetPoint{F}
\tkzDefMidPoint(B,C) \tkzGetPoint{E}
\tkzDefMidPoint(D,B) \tkzGetPoint{I}
\tkzDefMidPoint(I,B) \tkzGetPoint{a}
\tkzInterLC(B,I)(B,C) \tkzGetSecondPoint{K}
\tkzDefMidPoint(I,K) \tkzGetPoint{b}
\begin{scope}
  \tkzFillSector[fill=blue!10](B,C)(A)
  \tkzDefMidPoint(A,B) \tkzGetPoint{x}
  \tkzDrawSemiCircle[fill=white](x,B)
  \tkzDefMidPoint(B,C) \tkzGetPoint{y}
  \tkzDrawSemiCircle[fill=white](y,C)
  \tkzClipCircle(E,B)
  \tkzClipCircle(F,B)
  \tkzFillCircle[fill=blue!10](B,A)
\end{scope}
\tkzDrawSemiCircle[thick](F,B)
\tkzDrawSemiCircle[thick](E,C)
\tkzDrawArc[thick](B,C)(A)
\tkzDrawSegments[thick](A,B B,C)
\tkzDrawPoints(A,B,C,E,F)
\tkzLabelPoints[centered](a,b)
\tkzLabelPoints(A,B,C,E,F)
\end{tikzpicture}
\end{tkzexample}

\subsubsection{Recorte e preenchimento parte III}

\begin{tkzexample}[latex=7cm, small]
\begin{tikzpicture}
  \tkzDefPoint(0,0){A} \tkzDefPoint(1,0){B}
  \tkzDefPoint(2,0){C} \tkzDefPoint(-3,0){a}
  \tkzDefPoint(3,0){b}  \tkzDefPoint(0,3){c}
  \tkzDefPoint(0,-3){d}
\begin{scope}
 \tkzClipPolygon(a,b,c,d)
 \tkzFillCircle[teal!20](A,C)
\end{scope}
 \tkzFillCircle[white](A,B)
 \tkzDrawCircle[color=red](A,C)
 \tkzDrawCircle[color=red](A,B)
\end{tikzpicture}
\end{tkzexample}

\subsection{Colorindo um polígono}
 \begin{NewMacroBox}{tkzFillPolygon}{\oarg{opções locais}\parg{lista de pontos}}%
Você pode colorir desenhando o polígono, mas neste caso você colore o interior do polígono sem desenhá-lo.

\medskip
\begin{tabular}{lll}%
\toprule
argumentos                & exemplo & explicação                         \\
\midrule
\TAline{\parg{pt1,pt2,\dots}}{\parg{A,B,\dots}}{}
%\bottomrule
 \end{tabular}
\end{NewMacroBox}

\subsubsection{\tkzcname{tkzFillPolygon}}
\begin{tkzexample}[latex=7cm, small]
\begin{tikzpicture}[scale=.5]
   \tkzDefPoint(0,0){C} \tkzDefPoint(4,0){A}
   \tkzDefPoint(0,3){B}
   \tkzDefSquare(B,A)      \tkzGetPoints{E}{F}
   \tkzDefSquare(A,C)      \tkzGetPoints{G}{H}
   \tkzDefSquare(C,B)       \tkzGetPoints{I}{J}
   \tkzFillPolygon[color  =  orange!30   ](A,C,G,H)
   \tkzFillPolygon[color  =  teal!40  ](C,B,I,J)
   \tkzFillPolygon[color  =  purple!20](B,A,E,F)
   \tkzDrawPolygon[line width  =  1pt](A,B,C)
   \tkzDrawPolygon[line width  =  1pt](A,C,G,H)
   \tkzDrawPolygon[line width  =  1pt](C,B,I,J)
   \tkzDrawPolygon[line width  =  1pt](B,A,E,F)
   \tkzLabelSegment[above](C,A){$a$}
   \tkzLabelSegment[right](B,C){$b$}
   \tkzLabelSegment[below left](B,A){$c$}
\end{tikzpicture}
\end{tkzexample}

\subsection{\tkzcname{tkzFillSector}}
\tkzHandBomb\ Atenção: os argumentos variam de acordo com as opções.
\begin{NewMacroBox}{tkzFillSector}{\oarg{opções locais}\parg{O,\dots}\parg{\dots}}%
\begin{tabular}{lll}%
opções          & padrão & definição      \\
\midrule
\TOline{towards}{towards}{$O$ é o centro e o arco de $A$ a $(OB)$}
\TOline{rotate} {towards}{o arco começa de A e o ângulo determina seu comprimento }
\TOline{R}{towards}{Damos o raio e dois ângulos}
\TOline{R with nodes}{towards}{Damos o raio e dois pontos}
\bottomrule
\end{tabular}

\medskip
Claro, você tem que adicionar todos os estilos do \TIKZ\ para os traçados...

\medskip
\begin{tabular}{lll}%
\toprule
opções             & argumentos & exemplo                         \\
\midrule
\TOline{towards}{\parg{pt,pt}\parg{pt}}{\tkzcname{tkzFillSector(O,A)(B)}}
\TOline{rotate} {\parg{pt,pt}\parg{an}}{\tkzcname{tkzFillSector[rotate,color=red](O,A)(90)}}
\TOline{R}{\parg{pt,$r$}\parg{an,an}}{\tkzcname{tkzFillSector[R,color=blue](O,2)(30,90)}}
\TOline{R with nodes}{\parg{pt,$r$}\parg{pt,pt}}{\tkzcname{tkzFillSector[R with nodes](O,2)(A,B)}}
\end{tabular}
\end{NewMacroBox}

\subsubsection{\tkzcname{tkzFillSector} e \tkzname{towards}}
É inútil colocar \tkzname{towards} e você notará que os contornos não são desenhados, apenas a superfície é colorida.
\begin{tkzexample}[latex=5.75cm,small]
  \begin{tikzpicture}[scale=.6]
  \tkzDefPoint(0,0){O}
  \tkzDefPoint(-30:3){A}
  \tkzDefPointBy[rotation = center O angle -60](A)
  \tkzFillSector[fill=purple!20](O,A)(tkzPointResult)
    \begin{scope}[shift={(-60:1)}]
    \tkzDefPoint(0,0){O}
    \tkzDefPoint(-30:3){A}
    \tkzDefPointBy[rotation = center O angle -60](A)
    \tkzGetPoint{A'}
    \tkzFillSector[color=teal!40](O,A')(A)
      \end{scope}
  \end{tikzpicture}
\end{tkzexample}


\subsubsection{\tkzcname{tkzFillSector} e \tkzname{rotate}}
\begin{tkzexample}[latex=5.75cm,small]
\begin{tikzpicture}[scale=1.5]
 \tkzDefPoint(0,0){O} \tkzDefPoint(2,2){A}
 \tkzFillSector[rotate,color=purple!20](O,A)(30)
 \tkzFillSector[rotate,color=teal!40](O,A)(-30)
\end{tikzpicture}
\end{tkzexample}

\subsection{Colorir um ângulo: \tkzcname{tkzFillAngle}}

A operação mais simples
\begin{NewMacroBox}{tkzFillAngle}{\oarg{opções locais}\parg{A,O,B}}%
$O$ é o vértice do ângulo. $OA$ e $OB$ são os lados. Atenção: o ângulo é determinado pela ordem dos pontos.

\medskip

\begin{tabular}{lll}%
\toprule
opções             & padrão & definição                        \\
\midrule
\TOline{size}{1}{esta opção determina o raio do setor angular colorido.}

\bottomrule
\end{tabular}

\medskip
Claro, você tem que adicionar todos os estilos do \TIKZ, como o uso de fill e shade...
\end{NewMacroBox}

\subsubsection{Exemplo com \tkzname{size}}
\begin{tkzexample}[latex=7cm,small]
\begin{tikzpicture}
   \tkzInit
   \tkzDefPoints{0/0/O,2.5/0/A,1.5/2/B}
   \tkzFillAngle[size=2, fill=gray!10](A,O,B)
   \tkzDrawLines(O,A O,B)
   \tkzDrawPoints(O,A,B)
\end{tikzpicture}
\end{tkzexample}


\subsubsection{Alterando a ordem dos itens}
\begin{tkzexample}[latex=7cm,small]
\begin{tikzpicture}
   \tkzInit
   \tkzDefPoints{0/0/O,2.5/0/A,1.5/2/B}
   \tkzFillAngle[size=2,fill=gray!10](B,O,A)
   \tkzDrawLines(O,A O,B)
   \tkzDrawPoints(O,A,B)
\end{tikzpicture}
\end{tkzexample}

\begin{tkzexample}[latex=7cm,small]
\begin{tikzpicture}
   \tkzInit
   \tkzDefPoints{0/0/O,5/0/A,3/4/B}
   % Don't forget {} to get, () to use
   \tkzFillAngle[size=4,left color=white,
                 right color=red!50](A,O,B)
   \tkzDrawLines(O,A O,B)
   \tkzDrawPoints(O,A,B)
\end{tikzpicture}
\end{tkzexample}

\begin{NewMacroBox}{tkzFillAngles}{\oarg{opções locais}\parg{A,O,B}\parg{A',O',B'}etc.}%
Com opções comuns, há uma macro para múltiplos ângulos.
  \end{NewMacroBox}

\subsubsection{Múltiplos ângulos}
\begin{tkzexample}[latex=5cm,small]
\begin{tikzpicture}[scale=0.5]
  \tkzDefPoints{0/0/B,8/0/C,0/8/A,8/8/D}
  \tkzDrawPolygon(B,C,D,A)
  \tkzDefTriangle[equilateral](B,C) \tkzGetPoint{M}
  \tkzInterLL(D,M)(A,B) \tkzGetPoint{N}
  \tkzDefPointBy[rotation=center N angle -60](D)
  \tkzGetPoint{L}
  \tkzInterLL(N,L)(M,B)     \tkzGetPoint{P}
  \tkzInterLL(M,C)(D,L)     \tkzGetPoint{Q}
  \tkzDrawSegments(D,N N,L L,D B,M M,C)
  \tkzDrawPoints(L,N,P,Q,M,A,D)
  \tkzLabelPoints[left](N,P,Q)
  \tkzLabelPoints[above](M,A,D)
  \tkzLabelPoints(L,B,C)
  \tkzMarkAngles(C,B,M B,M,C M,C,B D,L,N L,N,D N,D,L)
  \tkzFillAngles[fill=red!20,opacity=.2](C,B,M%
      B,M,C M,C,B D,L,N L,N,D N,D,L)
\end{tikzpicture}
\end{tkzexample}
\endinput

\section{Controlando a Caixa Delimitadora}
Do \tkzimp{PgfManual}:

"Quando você adiciona a opção clip, o caminho atual é usado para recortar desenhos subsequentes. O recorte nunca aumenta a área de recorte. Assim, quando você recorta contra um determinado caminho e depois recorta novamente contra outro caminho, você recorta contra a interseção de ambos.
A única maneira de aumentar o caminho de recorte é terminar o {pgfscope} no qual o recorte foi feito. No final de um {pgfscope}, o caminho de recorte que estava em vigor no início do escopo é reinstalado."


Antes de tudo, você não precisa lidar com \TIKZ\ o tamanho da caixa delimitadora. Versões anteriores de \tkzNamePack{tkz-euclide} não controlavam o tamanho da caixa delimitadora, agora com \tkzNamePack{\tkznameofpack} 4 o tamanho da caixa delimitadora é limitado.

A caixa delimitadora inicial após usar a macro \tkzcname{tkzInit} é definida pelo retângulo baseado nos pontos $(0,0)$ e $(10,10)$. A macro \tkzcname{tkzInit} permite que esta caixa delimitadora inicial seja modificada usando os argumentos (\tkzname{xmin}, \tkzname{xmax}, \tkzname{ymin}, e \tkzname{ymax}). Claro que qualquer traço externo modifica a caixa delimitadora. \TIKZ\ mantém essa caixa delimitadora. É possível influenciar este comportamento diretamente com comandos ou opções em \TIKZ\ como um comando como \tkzcname{useasboundingbox} ou a opção \tkzname{use as bounding box}. Uma consequência possível é reservar uma caixa para uma figura, mas a figura pode transbordar a caixa e se espalhar sobre o texto principal.
O seguinte comando \tkzcname{pgfresetboundingbox} limpa uma caixa delimitadora e estabelece uma nova.

\subsection{Utilidade de \tkzcname{tkzInit}}
 No entanto, às vezes é necessário controlar o tamanho do que será exibido.
 Para fazer isso, você precisa ter preparado a caixa delimitadora em que vai trabalhar, este é o papel da macro \tkzNameMacro{tkzInit}. Para alguns desenhos, é interessante fixar os valores extremos (xmin, xmax, ymin e ymax) e \code{clip} o retângulo de definição para controlar o tamanho da figura da melhor maneira possível.

As duas macros que são úteis para controlar a caixa delimitadora:
\begin{itemize}
   \item \tkzcname{tkzInit}
   \item \tkzcname{tkzClip}
\end{itemize}
\vspace{20pt}

A isso, adicionei macros diretamente vinculadas à caixa delimitadora. Você agora pode visualizá-la, fazer backup dela, restaurá-la (veja a seção Bounding Box).

\subsection{\tkzcname{tkzInit}}

\begin{NewMacroBox}{tkzInit}{\oarg{local opções}}\hypertarget{init}{}%
\begin{tabular}{lll}%
opções  & padrão & definição             \\
\midrule
\TOline{xmin} {0} {valor mínimo das abscissas em cm}
\TOline{xmax} {10} {valor máximo das abscissas em cm}
\TOline{xstep}{1} {diferença entre duas graduações em $x$}
\TOline{ymin} {0} {valor mínimo do eixo y em cm}
\TOline{ymax} {10} {valor máximo do eixo y em cm}
\TOline{ystep}{1} {diferença entre duas graduações em $y$}
\bottomrule
\end{tabular}

\medskip

O papel de \tkzcname{tkzInit} é definir um sistema de coordenadas \textcolor{red}{ortogonal} e uma parte retangular do plano na qual você colocará seus desenhos usando coordenadas cartesianas.
Esta macro permite que você defina seu ambiente de trabalho como em uma calculadora. Com \tkzname{\tkznameofpack} 4 \tkzcname{xstep} e \tkzcname{ystep} são sempre 1. Logicamente não é mais útil usar \tkzcname{tkzInit}, exceto para uma ação como \code{Clipping Out}.
\end{NewMacroBox}


\subsection{\tkzcname{tkzClip}}

\begin{NewMacroBox}{tkzClip}{\oarg{local opções}}
O papel desta macro é tornar invisível o que está fora do retângulo definido por (xmin~;~ymin) e (xmax~;~ymax).

\medskip
\begin{tabular}{lll}
\hline
opções  & padrão & definição             \\
\midrule
\TOline{space} {1} {valor adicionado à direita, esquerda, inferior e superior do fundo}
\bottomrule
\end{tabular}

\medskip

O papel da opção \tkzname{space} é aumentar a parte visível do desenho. Esta parte torna-se o retângulo definido por (xmin-space~;~ymin-space) e (xmax+space~;~ymax+space). \tkzname{space} pode ser negativo! A unidade é cm e não deve ser especificada.
\end{NewMacroBox}



O papel desta macro é \code{clip} (recortar) o retângulo inicial para que apenas os caminhos contidos neste retângulo sejam desenhados.

\begin{tkzexample}[latex=8cm,small]
\begin{tikzpicture}
 \tkzInit[xmax=4, ymax=3]
 \tkzDefPoints{-1/-1/A,5/2/B}
 \tkzDrawX \tkzDrawY
 \tkzGrid
 \tkzClip
 \tkzDrawSegment(A,B)
\end{tikzpicture}
\end{tkzexample}

É possível adicionar um pouco de espaço
\begin{tkzltxexample}[]
  \tkzClip[space=1]
\end{tkzltxexample} 

\subsection{\tkzcname{tkzClip} e a opção \tkzname{space}}
Esta opção permite adicionar algum espaço ao redor do retângulo \code{clipped} (recortado).
\begin{tkzexample}[latex=8cm,small]
\begin{tikzpicture}
 \tkzInit[xmax=4, ymax=3]
 \tkzDefPoints{-1/-1/A,5/2/B}
 \tkzDrawX \tkzDrawY 
 \tkzGrid
 \tkzClip[space=1]
 \tkzDrawSegment(A,B)
\end{tikzpicture}
\end{tkzexample}   
As dimensões do retângulo \code{clipped} (recortado) são \tkzname{xmin-1}, \tkzname{ymin-1}, \tkzname{xmax+1} e \tkzname{ymax+1}.

%<--------------------------------------------------------------------------->
%              tkzShowBB
%<--------------------------------------------------------------------------->
\subsection{tkzShowBB}
A macro mais simples.
\begin{NewMacroBox}{tkzShowBB}{\oarg{local opções}}%
Esta macro exibe a caixa delimitadora. Um quadro retangular envolve a caixa delimitadora. Esta macro aceita opções \TIKZ.
\end{NewMacroBox} 


\subsubsection{Exemplo with \tkzcname{tkzShowBB}}
\begin{tkzexample}[latex=8cm,small]
\begin{tikzpicture}[scale=.5]
  \tkzInit[ymax=5,xmax=8]
  \tkzGrid  
  \tkzDefPoint(3,0){A}
   \begin{scope}
    \tkzClipBB
    \tkzDefCircle[R](A,5) \tkzGetPoint{a}
    \tkzDrawCircle(A,a)
    \tkzShowBB[line width = 4pt,fill=teal!10,%
              opacity=.4]
   \end{scope}
\tkzDefCircle[R](A,4) \tkzGetPoint{b}
\tkzDrawCircle[red](A,b)
\end{tikzpicture}
\end{tkzexample}
%<--------------------------------------------------------------------------->
%         tkzClipBB
%<--------------------------------------------------------------------------->
\subsection{tkzClipBB}
\begin{NewMacroBox}{tkzClipBB}{}%
A ideia é limitar construções futuras à caixa delimitadora atual.
\end{NewMacroBox}

\subsubsection{Exemplo com \tkzcname{tkzClipBB} e as bissetrizes}

\begin{tkzexample}[latex=6cm,small]
  \begin{tikzpicture}
  \tkzInit[xmin=-3,xmax=6, ymin=-1,ymax=6]
  \tkzDefPoint(0,0){O}\tkzDefPoint(3,1){I}
  \tkzDefPoint(1,4){J}
  \tkzDefLine[bisector](I,O,J) \tkzGetPoint{i}
  \tkzDefLine[bisector out](I,O,J) \tkzGetPoint{j}
  \tkzDrawPoints(O,I,J,i,j)
  \tkzClipBB
  \tkzDrawLines[add = 1 and 2,color=orange](O,I O,J)
  \tkzDrawLines[add = 1 and 2](O,i O,j)
  \tkzShowBB
  \end{tikzpicture}
\end{tkzexample}


\newpage

\section{Recortando diferentes objetos}

\subsection{Recortando um polígono}
 \begin{NewMacroBox}{tkzClipPolygon}{\oarg{local opções}\parg{lista de pontos}}%
Esta macro torna possível conter os diferentes gráficos no polígono designado.

\medskip
\begin{tabular}{lll}%
\toprule
argumentos       & exemplo & explicação     \\
\midrule
\TAline{\parg{pt1,pt2,pt3,\dots}}{\parg{A,B,C}}{}
\midrule
opções  & padrão & definição             \\
\midrule
\TOline{out} {} {permite recortar o exterior do objeto}
 \end{tabular}
\end{NewMacroBox}

\subsubsection{\tkzcname{tkzClipPolygon}}

\begin{tkzexample}[latex=7cm,small]
  \begin{tikzpicture}[scale=1.25] 
  \tkzDefPoint(0,0){A} 
  \tkzDefPoint(4,0){B} 
  \tkzDefPoint(1,3){C} 
  \tkzDrawPolygon(A,B,C) 
  \tkzDefPoint(0,2){D} 
  \tkzDefPoint(2,0){E} 
  \tkzDrawPoints(D,E) 
  \tkzLabelPoints(D,E) 
  \tkzClipPolygon(A,B,C) 
  \tkzDrawLine[new](D,E)
\end{tikzpicture}
\end{tkzexample}

\subsubsection{\tkzcname{tkzClipPolygon[out]}}

\begin{tkzexample}[latex=7cm,small]
  \begin{tikzpicture}[scale=1]
  \tkzDefPoint(0,0){P1}
  \tkzDefPoint(4,0){P2}
  \tkzDefPoint(4,4){P3}
  \tkzDefPoint(0,4){P4}
  \tkzDefPoint(1,1){Q1}
  \tkzDefPoint(3,1){Q2}
  \tkzDefPoint(3,3){Q3}
  \tkzDefPoint(1,3){Q4}
  \tkzDrawPolygon(P1,P2,P3,P4)
  \begin{scope}
     \tkzClipPolygon[out](Q1,Q2,Q3,Q4)
    \tkzFillPolygon[teal!20](P1,P2,P3,P4)
  \end{scope}
  \tkzDrawPolygon(Q1,Q2,Q3,Q4)
  \end{tikzpicture}
\end{tkzexample}

\subsubsection{Exemplo: uso de \code{Clip} para Sangaku em um quadrado} 
\begin{tkzexample}[latex=7cm, small]  
\begin{tikzpicture}[scale=.75]
 \tkzDefPoint(0,0){A} \tkzDefPoint(8,0){B}
 \tkzDefSquare(A,B)   \tkzGetPoints{C}{D}
 \tkzDefPoint(4,8){F}
 \tkzDefTriangle[equilateral](C,D) 
 \tkzGetPoint{I}
 \tkzDefPointBy[projection=onto B--C](I) 
 \tkzGetPoint{J}
 \tkzInterLL(D,B)(I,J)  \tkzGetPoint{K}
 \tkzDefPointBy[symmetry=center K](B)  
 \tkzGetPoint{M}
 \tkzClipPolygon(B,C,D,A)
 \tkzFillPolygon[color = orange](A,B,C,D)
 \tkzFillCircle[color = yellow](M,I)
 \tkzFillCircle[color = blue!50!black](F,D)
\end{tikzpicture}
\end{tkzexample}

\subsection{Recortando um disco}

\begin{NewMacroBox}{tkzClipCircle}{\oarg{local opções}\parg{A,B}}%
\begin{tabular}{lll}%
\toprule
argumentos           & exemplo & explicação                         \\
\midrule
\TAline{\parg{A,B}}{\parg{A,B}} {raio AB}
\bottomrule
\end{tabular}

\medskip
\begin{tabular}{lll}%
opções             & padrão & definição                         \\
\midrule
\TOline{out} {} {permite recortar o exterior do objeto}
 \bottomrule
\end{tabular}

\medskip
Não é necessário colocar \tkzname{radius} porque essa é a opção padrão.
\end{NewMacroBox}

 \subsubsection{Recorte simples} 
\begin{tkzexample}[latex=6cm,small] 
\begin{tikzpicture}[scale=.5]
  \tkzDefPoint(0,0){A} \tkzDefPoint(2,2){O}
  \tkzDefPoint(4,4){B} \tkzDefPoint(5,5){C}
  \tkzDrawPoints(O,A,B,C) 
  \tkzLabelPoints(O,A,B,C)
  \tkzDrawCircle(O,A) 
  \tkzClipCircle(O,A)
  \tkzDrawLine(A,C)
  \tkzDrawCircle[fill=teal!10,opacity=.5](C,O)
\end{tikzpicture} 
\end{tkzexample}

\subsection{Recorte externo}

\begin{tkzexample}[latex=6cm,small]
\begin{tikzpicture}
  \tkzInit[xmin=-3,ymin=-2,xmax=4,ymax=3]
   \tkzDefPoint(0,0){O}
   \tkzDefPoint(-4,-2){A}
   \tkzDefPoint(3,1){B}
   \tkzDefCircle[R](O,2) \tkzGetPoint{o}
   \tkzDrawPoints(A,B) % to have a good bounding box
   \begin{scope}
    \tkzClipCircle[out](O,o)
    \tkzDrawLines(A,B)
   \end{scope}
\end{tikzpicture}  
\end{tkzexample} 

\subsection{Interseção de discos}

\begin{tkzexample}[latex=6cm,small]
\begin{tikzpicture}
\tkzDefPoints{0/0/O,4/0/A,0/4/B}
\tkzDrawPolygon[fill=teal](O,A,B)
\tkzClipPolygon(O,A,B)
\tkzClipCircle(A,O)
\tkzClipCircle(B,O)
\tkzFillPolygon[white](O,A,B)
\end{tikzpicture}
\end{tkzexample}

veja um exemplo mais complexo sobre recorte de círculos aqui: \ref{About clipping circles}

\subsection{Recortando um setor}
\tkzHandBomb\  Atenção os argumentos variam de acordo com as opções. 
\begin{NewMacroBox}{tkzClipSector}{\oarg{local opções}\parg{O,\dots}\parg{\dots}}%
\begin{tabular}{lll}%
opções             & padrão & definição                         \\
\midrule
\TOline{towards}{towards}{$O$ é o centro e o setor começa de $A$ para $(OB)$}
\TOline{rotate} {towards}{O setor começa de $A$ e o ângulo determina sua amplitude.}
\TOline{R}{towards}{Fornecemos o raio e dois ângulos}
\bottomrule
\end{tabular}

\medskip
Você tem que adicionar, é claro, todos os estilos de \TIKZ\ para os traçados...

\medskip
\begin{tabular}{lll}%
\toprule
opções             & argumentos & exemplo                         \\
\midrule
\TOline{towards}{\parg{pt,pt}\parg{pt}}{\tkzcname{tkzClipSector(O,A)(B)}}
\TOline{rotate} {\parg{pt,pt}\parg{ângulo}}{\tkzcname{tkzClipSector[rotate](O,A)(90)}}
\TOline{R}{\parg{pt,$r$}\parg{ângulo 1,ângulo 2}}{\tkzcname{tkzClipSector[R](O,2)(30,90)}}
\end{tabular}
\end{NewMacroBox}

\subsubsection{Exemplo 1} 
\begin{tkzexample}[latex=7cm,small] 
\begin{tikzpicture}[scale=0.5]
\tkzDefPoint(0,0){a}
\tkzDefPoint(12,0){b}
\tkzDefPoint(4,10){c}
\tkzInterCC[R](a,6)(b,8) 
\tkzGetFirstPoint{AB1} \tkzGetSecondPoint{AB2}
\tkzInterCC[R](a,6)(c,6) 
\tkzGetFirstPoint{AC1} \tkzGetSecondPoint{AC2}
\tkzInterCC[R](b,8)(c,6) 
\tkzGetFirstPoint{BC1} \tkzGetSecondPoint{BC2}
\tkzDrawArc(a,AB2)(AB1)
\tkzDrawArc(b,AB1)(AB2)
\tkzDrawArc(a,AC2)(AC1)
\tkzDrawArc(c,AC1)(AC2)
\tkzDrawArc(b,BC2)(BC1)
\tkzDrawArc(c,BC1)(BC2)
\begin{scope}
\tkzClipSector(b,BC2)(BC1)
\tkzFillSector[teal!40!white](c,BC1)(BC2)
\end{scope}
\begin{scope}
\tkzClipSector(a,AB2)(AB1)
\tkzFillSector[teal!40!white](b,AB1)(AB2)
\end{scope}
\begin{scope}
\tkzClipSector(a,AC2)(AC1)
\tkzFillSector[teal!40!white](c,AC1)(AC2)
\end{scope}
\end{tikzpicture}
\end{tkzexample}

\subsubsection{Exemplo 2} 
\begin{tkzexample}[latex=7cm,small] 
\begin{tikzpicture}[scale=1.5] 
  \tkzDefPoint(0,0){O}
  \tkzDefPoint(2,-1){A}
  \tkzDefPoint(1,1){B} 
  \tkzDrawSector[new,dashed](O,A)(B)
  \tkzDrawSector[new](O,B)(A)
\begin{scope}
\tkzClipSector(O,B)(A)
\tkzDefSquare(O,B) \tkzGetPoints{B'}{O'}
\tkzDrawPolygon[color=teal,fill=teal!20](O,B,B',O')
\end{scope}
\tkzDrawPoints(A,B,O) 
\end{tikzpicture} 
\end{tkzexample}


\subsection{Opções do \TIKZ: trim left ou right}
Veja o \tkzimp{pgfmanual}

\subsection{Controles \TIKZ\ \tkzcname{pgfinterruptboundingbox} e \tkzcname{endpgfinterruptboundingbox}}
Este comando interrompe temporariamente o cálculo da caixa e configura uma nova caixa.
Veja o \tkzimp{pgfmanual}

\subsubsection{Exemplo sobre controle da caixa delimitadora} 
\begin{tkzexample}[latex=7cm,small] 
\begin{tikzpicture}
\tkzDefPoint(0,5){A}\tkzDefPoint(5,4){B}
\tkzDefPoint(0,0){C}\tkzDefPoint(5,1){D}
\tkzDrawSegments(A,B C,D A,C)
\pgfinterruptboundingbox
   \tkzInterLL(A,B)(C,D)\tkzGetPoint{I}
\endpgfinterruptboundingbox
\tkzClipBB
\tkzDrawCircle(I,B)
\end{tikzpicture}
\end{tkzexample}

\subsection{Recorte reverso: tkzreverseclip}

Para usar esta opção, uma caixa delimitadora deve ser definida. 

\begin{tkzltxexample}[]
  \tikzset{tkzreverseclip/.style={insert path={
     (current bounding box.south west) --(current bounding box.north west)
   --(current bounding box.north east) --  (current bounding box.south east)
   -- cycle} }}
\end{tkzltxexample}

\subsubsection{Exemplo with \tkzcname{tkzClipPolygon[out]}}
\tkzcname{tkzClipPolygon[out]}, \tkzcname{tkzClipCircle[out]} use this opção.
\begin{tkzexample}[vbox,small]
\begin{tikzpicture}[scale=1]
\tkzInit[xmin=-5,xmax=5,ymin=-4,ymax=6]
\tkzClip
\tkzDefPoints{-.5/0/P1,.5/0/P2}
\foreach \i [count=\j from 3] in {2,...,7}{%
    \tkzDefShiftPoint[P\i]({45*(\i-1)}:1){P\j}}  
\tkzClipPolygon[out](P1,P...,P8)
\tkzCalcLength(P1,P5)\tkzGetLength{r}
\begin{scope}[blend group=screen]
  \foreach \i in {1,...,8}{%
   \tkzDefCircle[R](P\i,\r) \tkzGetPoint{x}
   \tkzFillCircle[color=teal](P\i,x)}
  \end{scope}
\end{tikzpicture}
\end{tkzexample}

\endinput

\part{Marcando}
\input{TKZdoc-euclide-marking.tex}

\part{Rotulando}
\section{Rotulagem}
\subsection{Rótulo para um ponto}
\hypertarget{tlp}{}
É possível adicionar vários rótulos no mesmo ponto usando esta macro várias vezes.

\begin{NewMacroBox}{tkzLabelPoint}{\oarg{local opções}\parg{point}\var{label}}%
\begin{tabular}{lll}%
argumentos &  exemplo  &                  \\
\midrule
\TAline{point}{\tkzcname{tkzLabelPoint(A)\{\$A\_1\$\}}}{}
opções  & padrão & definição\\
\midrule
\TOline{TikZ opções}{}{cor, posição etc.}
\bottomrule
\end{tabular}

\medskip
Opcionalmente, podemos usar qualquer estilo do \TIKZ, especialmente posicionamento com above, right, etc...
\end{NewMacroBox}

\subsubsection{Exemplo com \tkzcname{tkzLabelPoint}} 
\begin{tkzexample}[latex=7cm,small]  
\begin{tikzpicture}
  \tkzDefPoint(0,0){A}
  \tkzDefPoint(4,0){B}
  \tkzDefPoint(0,3){C}
  \tkzDrawSegments(A,B B,C C,A)
  \tkzDrawPoints(A,B,C)
  \tkzLabelPoint[left,red](A){$A$}
  \tkzLabelPoint[right,blue](B){$B$}
  \tkzLabelPoint[above,purple](C){$C$}  
\end{tikzpicture} 
\end{tkzexample} 

\subsubsection{Rótulo e referência}
 A referência de um ponto é o objeto que permite usar o ponto, o rótulo é o nome do ponto que será exibido.
 
\begin{tkzexample}[latex=6cm,small]
 \begin{tikzpicture}
    \tkzDefPoint(2,0){A} 
    \tkzDrawPoint(A)
    \tkzLabelPoint[above](A){$A_1$}  
  \end{tikzpicture}
 \end{tkzexample}
 
\subsection{Adicionar rótulos aos pontos \tkzcname{tkzLabelPoints}}
É possível colocar vários rótulos rapidamente quando as referências dos pontos são idênticas aos rótulos e quando os rótulos são colocados da mesma maneira em relação aos pontos. Por padrão, \tkzname{below right} é escolhido.
\hypertarget{tlps}{}

\begin{NewMacroBox}{tkzLabelPoints}{\oarg{local opções}\parg{$A_1,A_2,...$}}%
\begin{tabular}{lll}
argumentos &  exemplo & resultado                 \\
\midrule
\TAline{list of points}{\tkzcname{tkzLabelPoints(A,B,C)}}{Exibição de $A$, $B$ e $C$}
\bottomrule
\end{tabular}

\medskip
Esta macro reduz o número de linhas de código, mas não é óbvio que todos os pontos precisem do mesmo posicionamento de rótulo.
\end{NewMacroBox}

\subsubsection{Exemplo com \tkzcname{tkzLabelPoints}}   
\begin{tkzexample}[latex = 6cm,small]  
\begin{tikzpicture}
  \tkzDefPoint(2,3){A}
  \tkzDefShiftPoint[A](30:2){B}
  \tkzDefShiftPoint[A](30:5){C}
  \tkzDrawPoints(A,B,C)
  \tkzLabelPoints(A,B,C) 
\end{tikzpicture} 
\end{tkzexample}
%<--------------------------------------------------------------------------->
%                       tkzAutoLabelPoints
%<--------------------------------------------------------------------------->
\subsection{Posição automática de rótulos \tkzcname{tkzAutoLabelPoints}}
O rótulo de um ponto é colocado em uma direção definida por um centro e um ponto \tkzname{center}. A distância ao ponto é determinada por uma porcentagem da distância entre o centro e o ponto. Esta porcentagem é dada por \tkzname{dist}.
\begin{NewMacroBox}{tkzLabelPoints}{\oarg{local opções}\parg{$A_1,A_2,...$}}%
\begin{tabular}{lll}
argumentos &  exemplo & resultado                 \\
\midrule
\TAline{list of points}{\tkzcname{tkzLabelPoint(A,B,C)}}{Exibição de $A$, $B$ e $C$}
\end{tabular}
\end{NewMacroBox}

\subsubsection{Rótulo para pontos com \tkzcname{tkzAutoLabelPoints}}
Aqui os pontos são posicionados em relação ao centro de gravidade de $A,B,C \text{ e } O$.
\begin{tkzexample}[latex=4cm,small]
\begin{tikzpicture}[scale=1]
 \tkzDefPoint(2,1){O}
 \tkzDefRandPointOn[circle=center O radius 1.5]\tkzGetPoint{A}
 \tkzDefPointBy[rotation=center O angle 100](A)\tkzGetPoint{C}
 \tkzDefPointBy[rotation=center O angle 78](A)\tkzGetPoint{B}
 \tkzDrawCircle(O,A) 
 \tkzDrawPoints(O,A,B,C) 
 \tkzDrawSegments(C,B B,A A,O O,C)
 \tkzDefTriangleCenter[centroid](A,B,C) \tkzGetPoint{O}
 \tkzDrawPoint(tkzPointResult)
 \tkzLabelPoints(O,A,C,B)
\end{tikzpicture}
\end{tkzexample}

\section{Rótulo para um segmento}
\hypertarget{tls}{}
\begin{NewMacroBox}{tkzLabelSegment}{\oarg{local opções}\parg{pt1,pt2}\marg{label}}
Esta macro permite colocar um rótulo ao longo de um segmento ou uma reta. As opções são as do \TIKZ\ por exemplo \tkzname{pos}.

\medskip
\begin{tabular}{lll}%%
argument    & exemplo & definição    \\
\midrule
\TAline{label}{\tkzcname{tkzLabelSegment(A,B)\{$5$\}}}{texto do rótulo}
\TAline{(pt1,pt2)}{(A,B)}{rótulo ao longo de $[AB]$}
\bottomrule
\end{tabular}

\medskip
\begin{tabular}{lll}%
opções  & padrão & definição    \\
\midrule
\TOline{pos}{.5}{posição do rótulo}
\end{tabular}
\end{NewMacroBox}

\subsubsection{Primeiro exemplo}      
\begin{tkzexample}[latex=7 cm,small]
\begin{tikzpicture}
\tkzDefPoint(0,0){A}
\tkzDefPoint(6,0){B}
\tkzDrawSegment(A,B)
\tkzLabelSegment[above,pos=.8](A,B){$a$}
\tkzLabelSegment[below,pos=.2](A,B){$4$}
\end{tikzpicture} 
\end{tkzexample}  

\subsubsection{Exemplo: quadro negro}  
\begin{tkzexample}[latex=6cm,small]
\tikzstyle{background rectangle}=[fill=black]
\begin{tikzpicture}[show background rectangle,scale=.4]
  \tkzDefPoint(0,0){O}
  \tkzDefPoint(1,0){I}
  \tkzDefPoint(10,0){A}
  \tkzDefPointWith[orthogonal normed,K=4](I,A)
   \tkzGetPoint{H}
  \tkzDefMidPoint(O,A) \tkzGetPoint{M}
  \tkzInterLC(I,H)(M,A)\tkzGetPoints{B}{C}   
  \tkzDrawSegments[color=white,line width=1pt](I,H O,A)
  \tkzDrawPoints[color=white](O,I,A,B,M) 
  \tkzMarkRightAngle[color=white,line width=1pt](A,I,B) 
  \tkzDrawArc[color=white,line width=1pt,
              style=dashed](M,A)(O)    
  \tkzLabelSegment[white,right=1ex,pos=.5](I,B){$\sqrt{a}$} 
  \tkzLabelSegment[white,below=1ex,pos=.5](O,I){$1$}   
  \tkzLabelSegment[pos=.6,white,below=1ex](I,A){$a$} 
\end{tikzpicture} 
\end{tkzexample}

\subsubsection{Rótulos e opção: \tkzname{swap}}
\begin{tkzexample}[latex=7cm,small]
\begin{tikzpicture}[rotate=-60]
\tkzSetUpStyle[red,auto]{label style}
\tkzDefPoint(0,1){A}
\tkzDefPoint(2,4){C}
\tkzDefPointWith[orthogonal normed,K=7](C,A)
\tkzGetPoint{B}
\tkzDefSpcTriangle[orthic](A,B,C){N,O,P}
\tkzDefTriangleCenter[circum](A,B,C)
\tkzGetPoint{O}
\tkzDrawPolygon[green!60!black](A,B,C)
\tkzDrawLine[dashed,color=magenta](C,P)
\tkzLabelSegment(B,A){$c$}
\tkzLabelSegment[swap](B,C){$a$}
\tkzLabelSegment[swap](C,A){$b$}
\tkzMarkAngles[size=1,
     color=cyan,mark=|](C,B,A A,C,P)
\tkzMarkAngle[size=0.75,
     color=orange,mark=||](P,C,B)
\tkzMarkAngle[size=0.75,
      color=orange,mark=||](B,A,C)
\tkzMarkRightAngles[german](A,C,B B,P,C)
\tkzAutoLabelPoints[center = O,dist= .1](A,B,C)
 \tkzLabelPoint[below left](P){$P$}
 \end{tikzpicture} 
\end{tkzexample}

\hypertarget{tlss}{}
 \begin{NewMacroBox}{tkzLabelSegments}{\oarg{local opções}\parg{pt1,pt2 pt3,pt4 ...}}%
Os argumentos são uma lista de pares de pontos. Os estilos do \TIKZ\ estão disponíveis para desenho.
\end{NewMacroBox}

\subsubsection{Rótulos para um triângulo isósceles}      
\begin{tkzexample}[latex=6cm,small]
\begin{tikzpicture}[scale=1]
 \tkzDefPoints{0/0/O,2/2/A,4/0/B,6/2/C}
 \tkzDrawSegments(O,A A,B)
 \tkzDrawPoints(O,A,B)
 \tkzDrawLine(O,B)   
 \tkzLabelSegments[color=red,above=4pt](O,A A,B){$a$}
\end{tikzpicture}
\end{tkzexample}  

\section{Adicionar rótulos em uma reta \tkzcname{tkzLabelLine}}%

\begin{NewMacroBox}{tkzLabelLine}{\oarg{local opções}\parg{pt1,pt2}\marg{label}}
\begin{tabular}{lll}%
argumentos &  padrão & definição   \\
\midrule
\TAline{label}{}{\tkzcname{tkzLabelLine(A,B)}\{\$\tkzcname{Delta}\$\}}
\bottomrule
\end{tabular}

\begin{tabular}{lll}%
opções             & padrão & definição   \\
\midrule
\TOline{pos}{.5}{\tkzname{pos} é uma opção para \TIKZ, mas essencial neste caso\dots}
\end{tabular}

Como opção, e além do \tkzname{pos}, você pode usar todos os estilos do \TIKZ, especialmente o posicionamento com \tkzname{above}, \tkzname{right}, \dots
\end{NewMacroBox}

\subsubsection{Exemplo com \tkzcname{tkzLabelLine}}
Uma opção importante é \tkzname{pos}, é a que permite colocar o rótulo ao longo da reta. O valor de \tkzname{pos} pode ser maior que 1 ou negativo.

\begin{tkzexample}[latex=6cm,small]
\begin{tikzpicture}
   \tkzDefPoints{0/0/A,3/0/B,1/1/C}
   \tkzDefLine[perpendicular=through C,K=-1](A,B)
   \tkzGetPoint{c}
   \tkzDrawLines(A,B C,c)
   \tkzLabelLine[pos=1.25,blue,right](C,c){$(\delta)$} 
   \tkzLabelLine[pos=-0.25,red,left](C,c){again $(\delta)$} 
\end{tikzpicture}
\end{tkzexample}

\subsection{Rótulo em um ângulo: \tkzcname{tkzLabelAngle}}

\begin{NewMacroBox}{tkzLabelAngle}{\oarg{local opções}\parg{A,O,B}}%
Há apenas uma opção, dist (com ou sem unidade), que pode ser substituída pela opção pos do TikZ (sem unidade para esta última). Por padrão, o valor está em centímetros.

\begin{tabular}{lll}%
  \toprule
opções             & padrão & definição                        \\
\midrule
\TOline{pos}{1}{ ou dist, controla a distância do vértice ao rótulo.}
\bottomrule
\end{tabular}

\medskip
É possível mover o rótulo com todas as opções do TikZ: rotate, shift, below, etc.
\end{NewMacroBox}

\subsubsection{Exemplo autor js bibra stackexchange} 
\begin{tkzexample}[latex=7cm,small]
\begin{tikzpicture}[scale=.75]
  \tkzDefPoint(0,0){C}
  \tkzDefPoint(20:9){B}
  \tkzDefPoint(80:5){A}
  \tkzDefPointsBy[projection=onto B--C](A){a}
  \tkzDrawPolygon[thick,fill=yellow!15](A,B,C)
  \tkzDrawSegment[dashed, red](A,a)   
  \tkzDrawSegment[style=red, dashed, 
  dim={$10$,15pt,midway,font=\scriptsize,
   rotate=90}](A,a) 
  \tkzMarkAngle(B,C,A)
  \tkzMarkRightAngle(A,a,C)    
  \tkzMarkRightAngle(C,A,B)
  \tkzFillAngle[fill=blue!20, opacity=0.5](B,C,A)
  \tkzFillAngle[fill=red!20, opacity=0.5](A,B,C)
  \tkzLabelAngle[pos=1.25](A,B,C){$\beta$}
  \tkzLabelAngle[pos=1.25](B,C,A){$\alpha$}
  \tkzMarkAngle(A,B,C)
  \tkzDrawPoints(A,B,C)
  \tkzLabelPoints(B,C)
  \tkzLabelPoints[above](A)
\end{tikzpicture}
\end{tkzexample}

\subsubsection{With \tkzname{pos}} 
\begin{tkzexample}[latex=7cm,small]
\begin{tikzpicture}[scale=.75]
  \tkzDefPoints{0/0/O,5/0/A,3/4/B}
  \tkzMarkAngle[size = 4,mark = ||,
      arc=ll,color = red](A,O,B)%     
  \tkzDrawLines(O,A O,B)
  \tkzDrawPoints(O,A,B)
  \tkzLabelAngle[pos=2,draw,circle,
      fill=blue!10](A,O,B){$\alpha$} 
\end{tikzpicture}
\end{tkzexample}

\subsubsection{\tkzname{pos} and \tkzcname{tkzLabelAngles}} 
\begin{tkzexample}[latex=7cm,small]
\begin{tikzpicture}[rotate=30]
  \tkzDefPoint(2,1){S} 
  \tkzDefPoint(7,3){T}
  \tkzDefPointBy[rotation=center S angle 60](T)
  \tkzGetPoint{P} 
  \tkzDefLine[bisector,normed](T,S,P)
  \tkzGetPoint{s}
  \tkzDrawPoints(S,T,P)   
  \tkzDrawPolygon[color=blue](S,T,P) 
  \tkzDrawLine[dashed,color=blue,add=0 and 3](S,s)  
  \tkzLabelPoint[above right](P){$P$}
  \tkzLabelPoints(S,T)
  \tkzMarkAngle[size = 1.8,mark = |,arc=ll,
                    color = blue](T,S,P)
  \tkzMarkAngle[size = 2.1,mark = |,arc=l,
                    color = blue](T,S,s)
  \tkzMarkAngle[size = 2.3,mark = |,arc=l,
                    color = blue](s,S,P)  
 \tkzLabelAngle[pos = 1.5](T,S,P){$60^{\circ}$}%    
 \tkzLabelAngles[pos = 2.7](T,S,s s,S,P){%
                            $30^{\circ}$}%   
\end{tikzpicture}
\end{tkzexample}


\begin{NewMacroBox}{tkzLabelAngles}{\oarg{local opções}\parg{A,O,B}\parg{A',O',B'}etc.}%
Com opções comuns, há uma macro para múltiplos ângulos.
\end{NewMacroBox}

Finalmente resta poder dar um rótulo para designar um círculo e se várias possibilidades são oferecidas, veremos aqui \tkzcname{tkzLabelCircle}.

\subsection{Dando um rótulo a um círculo}
\begin{NewMacroBox}{tkzLabelCircle}{\oarg{tikz opções}\parg{O,A}\parg{angle}\marg{label}}%
\begin{tabular}{lll}%
opções             & padrão & definição                         \\
\midrule
\TOline{tikz opções} {}{círculo $O$ centro passando por $A$}
\bottomrule
\end{tabular}

\medskip
\emph{ Podemos usar os estilos do \TIKZ. O rótulo é criado e, portanto, \code{passado} entre chaves.}
\end{NewMacroBox} 

\subsubsection{Exemplo}  
\begin{tkzexample}[latex=5cm,small] 
\begin{tikzpicture}
 \tkzDefPoint(0,0){O} \tkzDefPoint(2,0){N}
 \tkzDefPointBy[rotation=center O angle 50](N) 
     \tkzGetPoint{M}
 \tkzDefPointBy[rotation=center O angle -20](N) 
      \tkzGetPoint{P}
 \tkzDefPointBy[rotation=center O angle 125](N) 
      \tkzGetPoint{P'}
 \tkzLabelCircle[above=4pt](O,N)(120){$\mathcal{C}$}
 \tkzDrawCircle(O,M) 
 \tkzFillCircle[color=blue!10,opacity=.4](O,M) 
 \tkzLabelCircle[draw,
       text width=2cm,text centered,left=24pt](O,M)(-120)%
          {The circle\\ $\mathcal{C}$}  
 \tkzDrawPoints(M,P)\tkzLabelPoints[right](M,P)   
\end{tikzpicture}      
\end{tkzexample} 

\section{Rótulo para um arco}
\hypertarget{tls}{}
\begin{NewMacroBox}{tkzLabelArc}{\oarg{local opções}\parg{pt1,pt2,pt3}\marg{label}}
Esta macro permite colocar um rótulo ao longo de um arco. As opções são as do \TIKZ\ por exemplo \tkzname{pos}.

\medskip
\begin{tabular}{lll}%%
argument    & exemplo & definição    \\
\midrule
\TAline{label}{\tkzcname{tkzLabelArc(A,B)\{$5$\}}}{texto do rótulo}
\TAline{(pt1,pt2,pt3)}{(O,A,B)}{rótulo ao longo do arco $\widearc{AB}$}
\bottomrule
\end{tabular}

\medskip
\begin{tabular}{lll}%
opções  & padrão & definição    \\
\midrule
\TOline{pos}{.5}{posição do rótulo}
\end{tabular}
\end{NewMacroBox}

\subsubsection{Rótulo em arco}      
\begin{tkzexample}[latex=7 cm,small]
\begin{tikzpicture}
\tkzDefPoint(0,0){O}
\pgfmathsetmacro\r{2}
\tkzDefPoint(30:\r){A}
\tkzDefPoint(85:\r){B}
\tkzDrawCircle(O,A)
\tkzDrawPoints(B,A,O)
\tkzLabelArc[right=2pt](O,A,B){$\widearc{AB}$}
\tkzLabelPoints(A,B,O)
\end{tikzpicture}
\end{tkzexample}

\endinput

\part{Complementos}
\section{Usando o compasso}

\subsection{Macro principal \tkzcname{tkzCompass}}
\begin{NewMacroBox}{tkzCompass}{\oarg{opções locais}\parg{A,B}}%
Esta macro permite deixar um traço de compasso, ou seja, um arco em um ponto designado. O centro deve ser indicado. Várias opções específicas modificarão a aparência do arco, bem como opções do TikZ, como estilo, cor, espessura da linha etc.

Você pode definir o comprimento do arco com a opção |length| ou a opção |delta|.

\medskip
\begin{tabular}{lll}%
\toprule
opções             & padrão & definição                        \\
\midrule
\TOline{delta} {0 (graus)}{Aumenta o ângulo do arco simetricamente}
\TOline{length}{1 (cm)}{Altera o comprimento (em cm)}
\end{tabular}
\end{NewMacroBox}

\subsubsection{Opção \tkzname{length}}
\begin{tkzexample}[latex=7cm,small]
\begin{tikzpicture}
  \tkzDefPoint(1,1){A}
  \tkzDefPoint(6,1){B}
  \tkzInterCC[R](A,4)(B,3)
  \tkzGetPoints{C}{D}
  \tkzDrawPoint(C)
  \tkzCompass[length=1.5](A,C)
  \tkzCompass(B,C)
  \tkzDrawSegments(A,B A,C B,C)
\end{tikzpicture}
\end{tkzexample}

\subsubsection{Opção \tkzname{delta}}
\begin{tkzexample}[latex=7cm,small]
\begin{tikzpicture}
  \tkzDefPoint(0,0){A}
  \tkzDefPoint(5,0){B}
  \tkzInterCC[R](A,4)(B,3)
  \tkzGetPoints{C}{D}
  \tkzDrawPoints(A,B,C)
  \tkzCompass[delta=20](A,C)
  \tkzCompass[delta=20](B,C)
  \tkzDrawPolygon(A,B,C)
  \tkzMarkAngle(A,C,B)
\end{tikzpicture}
\end{tkzexample}

\subsection{Construções múltiplas \tkzcname{tkzCompasss}}
\begin{NewMacroBox}{tkzCompasss}{\oarg{opções locais}\parg{pt1,pt2 pt3,pt4,\dots}}%
\tkzHandBomb\ Atenção: os argumentos são listas de dois pontos. Isso economiza algumas linhas de código.

\medskip
\begin{tabular}{lll}%
\toprule
opções             & padrão & definição                        \\
\midrule
\TOline{delta} {0}{Modifica o ângulo do arco aumentando-o simetricamente}
\TOline{length}{1}{Altera o comprimento}
\end{tabular}
\end{NewMacroBox}

\subsubsection{Uso de \tkzcname{tkzCompasss}} % (fold)
\label{ssub:use_tkzcname_tkzcompasss}

% subsubsection use_tkzcname_tkzcompasss (end)
\begin{tkzexample}[latex=7cm,small]
\begin{tikzpicture}[scale=.6]
 \tkzDefPoint(2,2){A}  \tkzDefPoint(5,-2){B}
 \tkzDefPoint(3,4){C}  \tkzDrawPoints(A,B)
 \tkzDrawPoint[shape=cross out](C)
 \tkzCompasss[new](A,B A,C B,C C,B)
 \tkzShowLine[mediator,new,dashed,length = 2](A,B)
 \tkzShowLine[parallel = through C,
                     color=purple,length=2](A,B)
 \tkzDefLine[mediator](A,B)
  \tkzGetPoints{i}{j}
 \tkzDefLine[parallel=through C](A,B)
   \tkzGetPoint{D}
 \tkzDrawLines[add=.6 and .6](C,D A,C B,D)
 \tkzDrawLines(i,j) \tkzDrawPoints(A,B,C,i,j,D)
 \tkzLabelPoints(A,B,C,i,j,D)
\end{tikzpicture}
\end{tkzexample}

\endinput

\section{Exibindo construções}

\subsection{Mostrar as construções de algumas retas \tkzcname{tkzShowLine}}

\begin{NewMacroBox}{tkzShowLine}{\oarg{opções locais}\parg{pt1,pt2} ou \parg{pt1,pt2,pt3}}%
Essas construções dizem respeito a mediatrizes, retas perpendiculares ou paralelas passando por um ponto dado e bissetrizes. Os argumentos são, portanto, listas de dois ou três pontos. Várias opções permitem o ajuste das construções. A ideia desta macro vem de \tkzimp{Yves Combe}.

\medskip
\begin{tabular}{lll}%
\toprule
opções      & padrão & definição    \\ % <--- CORRIGIDO AQUI (era apenas \)
\midrule
\TOline{mediator}{mediator}{exibe as construções de uma mediatriz}
\TOline{perpendicular}{mediator}{construções para uma perpendicular}
\TOline{orthogonal}{mediator}{idem}
\TOline{bisector}{mediator}{construções para uma bissetriz}
\TOline{K}{1}{círculo dentro de um triângulo }
\TOline{length}{1}{em cm, comprimento de um arco}
\TOline{ratio} {.5}{proporção do comprimento do arco}
\TOline{gap}{2}{posicionamento do ponto de construção}
\TOline{size}{1}{raio de um arco (veja bissetriz)}
\bottomrule
\end{tabular}

\medskip
Você deve adicionar, é claro, todos os estilos do \TIKZ\ para traçados\dots
\end{NewMacroBox}

\subsubsection{Exemplo de \tkzcname{tkzShowLine} e \tkzname{parallel}}
\begin{tkzexample}[latex=7cm,small]
\begin{tikzpicture}
 \tkzDefPoints{-1.5/-0.25/A,1/-0.75/B,-1.5/2/C}
 \tkzDrawLine(A,B)
 \tkzDefLine[parallel=through C](A,B) \tkzGetPoint{c}
 \tkzShowLine[parallel=through C](A,B)
 \tkzDrawLine(C,c) \tkzDrawPoints(A,B,C,c)
\end{tikzpicture}
\end{tkzexample}

\subsubsection{Exemplo de \tkzcname{tkzShowLine} e \tkzname{perpendicular}}
\begin{tkzexample}[latex=6cm,small]
\begin{tikzpicture}
\tkzDefPoints{0/0/A, 3/2/B, 2/2/C}
\tkzDefLine[perpendicular=through C,K=-.5](A,B)
\tkzGetPoint{c}
\tkzShowLine[perpendicular=through C,K=-.5,gap=3](A,B)
\tkzDefPointBy[projection=onto A--B](c)
\tkzGetPoint{h}
\tkzMarkRightAngle[fill=lightgray](A,h,C)
\tkzDrawLines[add=.5 and .5](A,B C,c)
\tkzDrawPoints(A,B,C,h,c)
\end{tikzpicture}
\end{tkzexample}

\subsubsection{Exemplo de \tkzcname{tkzShowLine} e \tkzname{bisector}}
\begin{tkzexample}[latex=7 cm,small]
\begin{tikzpicture}[scale=1.25]
 \tkzDefPoints{0/0/A, 4/2/B, 1/4/C}
 \tkzDrawPolygon(A,B,C)
 \tkzSetUpCompass[color=brown,line width=.1 pt]
 \tkzDefLine[bisector](B,A,C)  \tkzGetPoint{a}
 \tkzDefLine[bisector](C,B,A)  \tkzGetPoint{b}
 \tkzInterLL(A,a)(B,b) \tkzGetPoint{I}
 \tkzDefPointBy[projection = onto A--B](I)
   \tkzGetPoint{H}
 \tkzShowLine[bisector,size=2,gap=3,blue](B,A,C)
 \tkzShowLine[bisector,size=2,gap=3,blue](C,B,A)
 \tkzDrawCircle[color=blue,%
 line width=.2pt](I,H)
 \tkzDrawSegments[color=red!50](I,tkzPointResult)
 \tkzDrawLines[add=0 and -0.3,color=red!50](A,a B,b)
\end{tikzpicture}
\end{tkzexample}

\subsubsection{Exemplo de \tkzcname{tkzShowLine} e \tkzname{mediator}}
\begin{tkzexample}[latex=7 cm,small]
\begin{tikzpicture}
\tkzDefPoint(2,2){A}
\tkzDefPoint(5,4){B}
\tkzDrawPoints(A,B)
\tkzShowLine[mediator,color=orange,length=1](A,B)
\tkzGetPoints{i}{j}
\tkzDrawLines[add=-0.1 and -0.1](i,j)
\tkzDrawLines(A,B)
\tkzLabelPoints[below =3pt](A,B)
\end{tikzpicture}
\end{tkzexample}

\subsection{Construções de certas transformações \addbs{tkzShowTransformation}}
\begin{NewMacroBox}{tkzShowTransformation}{\oarg{opções locais}\parg{pt1,pt2} ou \parg{pt1,pt2,pt3}}%
Essas construções dizem respeito a simetrias ortogonais, simetrias centrais, projeções ortogonais e translações. Várias opções permitem o ajuste das construções. A ideia desta macro vem de \tkzimp{Yves Combe}.

\medskip
\begin{tabular}{lll}%
\toprule
opções             & padrão & definição                         \\ % <--- CORRIGIDO AQUI TAMBÉM
\midrule
\TOline{reflection= over pt1--pt2}{reflection}{construções de simetria ortogonal}
\TOline{symmetry=center pt}{reflection}{construções de simetria central}
\TOline{projection=onto pt1--pt2}{reflection}{construções de uma projeção}
\TOline{translation=from pt1 to pt2}{reflection}{construções de uma translação}
\TOline{K}{1}{círculo dentro de um triângulo }
\TOline{length}{1}{comprimento do arco}
\TOline{ratio} {.5}{proporção do comprimento do arco}
\TOline{gap}{2}{posicionamento do ponto de construção}
\TOline{size}{1}{raio de um arco (veja bissetriz)}
\end{tabular}
\end{NewMacroBox}

\subsubsection{Exemplo de uso de \tkzcname{tkzShowTransformation}}

\begin{tkzexample}[latex=6cm,small]
\begin{tikzpicture}[scale=.5]
  \tkzDefPoint(0,0){O} \tkzDefPoint(2,-2){A}
  \tkzDefPoint(70:4){B} \tkzDrawPoints(A,O,B)
  \tkzLabelPoints(A,O,B)
  \tkzDrawLine[add= 2 and 2](O,A)
  \tkzDefPointBy[translation=from O to A](B)
  \tkzGetPoint{C}
  \tkzDrawPoint[color=orange](C)  \tkzLabelPoints(C)
  \tkzShowTransformation[translation=from O to A,%
             length=2](B)
  \tkzDrawSegments[->,color=orange](O,A B,C)
  \tkzDefPointBy[reflection=over O--A](B) \tkzGetPoint{E}
  \tkzDrawSegment[blue](B,E)
  \tkzDrawPoint[color=blue](E)\tkzLabelPoints(E)
  \tkzShowTransformation[reflection=over O--A,size=2](B)
  \tkzDefPointBy[symmetry=center O](B) \tkzGetPoint{F}
  \tkzDrawSegment[color=green](B,F)
  \tkzDrawPoint[color=green](F)\tkzLabelPoints(F)
  \tkzShowTransformation[symmetry=center O,%
                     length=2](B)
  \tkzDefPointBy[projection=onto O--A](C)
  \tkzGetPoint{H}
  \tkzDrawSegments[color=magenta](C,H)
  \tkzDrawPoint[color=magenta](H)\tkzLabelPoints(H)
  \tkzShowTransformation[projection=onto O--A,%
                          color=red,size=3,gap=-2](C)
\end{tikzpicture}
\end{tkzexample}

\subsubsection{Outro exemplo de uso de \tkzcname{tkzShowTransformation}}

Você encontrará esta figura novamente, mas sem as características de construção.
\begin{tkzexample}[latex=7cm,small]
\begin{tikzpicture}[scale=.6]
  \tkzDefPoints{0/0/A,8/0/B,3.5/10/I}
  \tkzDefMidPoint(A,B) \tkzGetPoint{O}
  \tkzDefPointBy[projection=onto A--B](I)
     \tkzGetPoint{J}
  \tkzInterLC(I,A)(O,A)  \tkzGetPoints{M}{M'}
  \tkzInterLC(I,B)(O,A)  \tkzGetPoints{N}{N'}
  \tkzDefMidPoint(A,B) \tkzGetPoint{M}
  \tkzDrawSemiCircle(M,B)
  \tkzDrawSegments(I,A I,B A,B B,M A,N)
  \tkzMarkRightAngles(A,M,B A,N,B)
  \tkzDrawSegment[style=dashed,color=blue](I,J)
  \tkzShowTransformation[projection=onto A--B,
                  color=red,size=3,gap=-3](I)
  \tkzDrawPoints[color=red](M,N)
  \tkzDrawPoints[color=blue](O,A,B,I,M)
  \tkzLabelPoints(O)
  \tkzLabelPoints[above right](N,I)
  \tkzLabelPoints[below left](M,A)
\end{tikzpicture}
\end{tkzexample}

\endinput
\section{Transferidor}
Baseado em uma ideia de Yves Combe, a seguinte macro permite desenhar um transferidor.
O princípio de operação é ainda mais simples. Basta nomear uma semirreta (um raio). O transferidor será colocado na origem $O$, a direção da semirreta é dada por $A$. O ângulo é medido na direção direta do círculo trigonométrico.
\subsection{A macro \tkzcname{tkzProtractor}} % (fold)
\label{sub:the_macro_tkzcname_tkzprotractor}

% subsection the_macro_tkzcname_tkzprotractor (end)
\begin{NewMacroBox}{tkzProtractor}{\oarg{opções locais}\parg{$O,A$}}%
\begin{tabular}{lll}%
opções    & padrão & definição     \\
\midrule
\TOline{lw}  {0.4 pt} {espessura da linha}
\TOline{scale}  {1} {razão: ajusta o tamanho do transferidor}
\TOline{return} {false} {círculo trigonométrico indireto}
\end{tabular}
\end{NewMacroBox}

\subsubsection{O transferidor circular}
Medindo na direção direta

\begin{tkzexample}[latex=7cm,small]
\begin{tikzpicture}[scale=.5]
\tkzDefPoint(2,0){A}\tkzDefPoint(0,0){O}
\tkzDefShiftPoint[A](31:5){B}
\tkzDefShiftPoint[A](158:5){C}
\tkzDrawPoints(A,B,C)
\tkzDrawSegments[color = red,
    line width = 1pt](A,B A,C)
  \tkzProtractor[scale = 1](A,B)
\end{tikzpicture}
\end{tkzexample}

\subsubsection{O transferidor circular, transparente e invertido}

\begin{tkzexample}[latex=7cm,small]
\begin{tikzpicture}[scale=.5]
  \tkzDefPoint(2,3){A}
  \tkzDefShiftPoint[A](31:5){B}
   \tkzDefShiftPoint[A](158:5){C}
  \tkzDrawSegments[color=red,line width=1pt](A,B A,C)
  \tkzProtractor[return](A,C)
\end{tikzpicture}
\end{tkzexample}
\endinput

\input{TKZdoc-euclide-tools.tex}

\part{Trabalhando com estilo}
\input{TKZdoc-euclide-styles.tex}

\part{Exemplos}
\input{TKZdoc-euclide-others.tex}
\input{TKZdoc-euclide-examples.tex}

\part{FAQ}
\section{FAQ}

\subsection{Erros mais comuns}
 Por enquanto, estou me baseando nos meus próprios erros, porque tendo mudado a sintaxe várias vezes, cometi vários erros. Esta seção será expandida. Com a versão 4.05 novos problemas podem aparecer.

\begin{itemize}\setlength{\itemsep}{10pt}
  \item O erro que cometo com mais frequência é esquecer de colocar um "s" na macro usada para desenhar mais de um objeto: como \tkzcname{tkzDrawSegment(s)} ou \tkzcname{tkzDrawCircle(s)} ou como neste exemplo \tkzcname{tkzDrawPoint(A,B)} quando você precisa de \tkzcname{tkzDrawPoints(A,B)};

  \item Não esqueça que desde a versão 4 a unidade é obrigatoriamente o "cm", portanto é necessário remover a unidade como aqui \tkzcname{tkzDrawCircle[R](O,3cm)} que se torna \tkzcname{tkzDrawCircle[R](O,3)}. As opções tradicionais do \tkzname{TikZ} mantêm suas unidades, exemplo \tkzname{below right = 12pt}, por outro lado escreveremos \tkzname{size=1.2} para posicionar um arco em \tkzcname{tkzMarkAngle};

  \item O seguinte erro ainda acontece comigo de tempos em tempos. Um ponto que é criado tem seu nome entre parênteses, enquanto um ponto que é usado como opção ou como parâmetro tem seu nome entre chaves.

  Exemplo \tkzcname{tkzGetPoint(A)} Ao definir um objeto, use chaves e não parênteses, então escreva: \tkzcname{tkzGetPoint\{A\}};

  \item As mudanças na obtenção dos pontos de interseção entre retas e círculos às vezes trocam as soluções, isso leva ou a uma figura incorreta ou a um erro.

  \item \tkzcname{tkzGetPoint\{A\}} no lugar de \tkzcname{tkzGetFirstPoint\{A\}}. Quando uma macro fornece dois pontos como resultados, ou recuperamos esses pontos usando \tkzcname{tkzGetPoints\{A\}\{B\}}, ou recuperamos apenas um dos dois pontos, usando \tkzcname{tkzGetFirstPoint\{A\}} ou
  \tkzcname{tkzGetSecondPoint\{A\}}. Esses dois pontos podem ser usados com a referência \tkzname{tkzFirstPointResult} ou
  \tkzname{tkzSecondPointResult}. É possível que um terceiro ponto seja fornecido como\ \tkzname{tkzPointResult};

\item Misturar opções e argumentos; todas as macros que usam um círculo precisam conhecer o raio do círculo. Se o raio é fornecido por uma medida, então a opção inclui um \tkzname{R}.


\item Os ângulos são fornecidos em graus, mais raramente em radianos.

\item Se ocorrer um erro em um cálculo ao passar parâmetros, então é melhor fazer esses cálculos antes de chamar a macro.

\item Não misture a sintaxe do \tkzNamePack{pgfmath} e \tkzNamePack{xfp}. Frequentemente escolhi \tkzNamePack{xfp}, mas se você preferir pgfmath, então faça seus cálculos antes de passar os parâmetros.

 \item  Erro "dimension too large"  : Em alguns casos, este erro ocorre. Uma maneira de evitá-lo é usar a opção "\tkzname{veclen}". Quando esta opção é usada em um scope, a função "veclen" é substituída por uma função dependente do "xfp". Não use macros de interseção neste scope. Por exemplo, um erro ocorre se você usar a macro \tkzcname{tkzDrawArc}
 com um ângulo muito pequeno. O erro é produzido pela biblioteca \NameLib{decoration} quando você deseja colocar uma marca em um arco. Mesmo que a marca esteja ausente, o erro ainda está presente.

\end{itemize}
\endinput


\clearpage\newpage
\small\printindex
\end{document}