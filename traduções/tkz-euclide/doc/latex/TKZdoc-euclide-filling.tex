\subsection{Colorindo um disco}
Isto era possível com a macro \tkzcname{tkzDrawCircle}, mas o traçado do disco era obrigatório, não é mais o caso.

\begin{NewMacroBox}{tkzFillCircle}{\oarg{opções locais}\parg{A,B}}%
\begin{tabular}{lll}%
opções             & padrão & definição                         \\
\midrule
\TOline{radius}  {radius}{dois pontos definem um raio}
\TOline{R} {radius}{um ponto e a medida de um raio }
\bottomrule
\end{tabular}

\medskip
Você não precisa colocar \tkzname{radius} porque essa é a opção padrão. Claro, você tem que adicionar todos os estilos do \TIKZ\ para os traçados.
\end{NewMacroBox}


\subsubsection{Yin e Yang}
\begin{tkzexample}[latex=8cm,small]
  \begin{tikzpicture}[scale=.75]
    \tkzDefPoint(0,0){O}
    \tkzDefPoint(-4,0){A}
    \tkzDefPoint(4,0){B}
    \tkzDefPoint(-2,0){I}
    \tkzDefPoint(2,0){J}
    \tkzDrawSector[fill=teal](O,A)(B)
    \tkzFillCircle[fill=white](J,B)
    \tkzFillCircle[fill=teal](I,A)
    \tkzDrawCircle(O,A)
  \end{tikzpicture}
\end{tkzexample}


\subsubsection{De um sangaku}

\begin{tkzexample}[latex=7cm,small]
\begin{tikzpicture}
   \tkzDefPoint(0,0){B}  \tkzDefPoint(6,0){C}%
   \tkzDefSquare(B,C)    \tkzGetPoints{D}{A}
   \tkzClipPolygon(B,C,D,A)
   \tkzDefMidPoint(A,D)  \tkzGetPoint{F}
   \tkzDefMidPoint(B,C)  \tkzGetPoint{E}
   \tkzDefMidPoint(B,D)  \tkzGetPoint{Q}
   \tkzDefLine[tangent from = B](F,A)
   \tkzGetPoints{H}{G}
   \tkzInterLL(F,G)(C,D) \tkzGetPoint{J}
   \tkzInterLL(A,J)(F,E) \tkzGetPoint{K}
   \tkzDefPointBy[projection=onto B--A](K)
   \tkzGetPoint{M}
   \tkzDrawPolygon(A,B,C,D)
   \tkzFillCircle[red!20](E,B)
   \tkzFillCircle[blue!20](M,A)
   \tkzFillCircle[green!20](K,Q)
  \tkzDrawCircles(B,A M,A E,B K,Q)
\end{tikzpicture}
\end{tkzexample}

\subsubsection{Recorte e preenchimento parte I}
\begin{tkzexample}[latex=7cm,small]
\begin{tikzpicture}
\tkzDefPoints{0/0/A,4/0/B,2/2/O,3/4/X,4/1/Y,1/0/Z,
              0/3/W,3/0/R,4/3/S,1/4/T,0/1/U}
\tkzDefSquare(A,B)\tkzGetPoints{C}{D}
\tkzDefPointWith[colinear normed=at X,K=1](O,X)
 \tkzGetPoint{F}
\begin{scope}
  \tkzFillCircle[fill=teal!20](O,F)
  \tkzFillPolygon[white](A,...,D)
  \tkzClipPolygon(A,...,D)
  \foreach \c/\t in {S/C,R/B,U/A,T/D}
  {\tkzFillCircle[teal!20](\c,\t)}
\end{scope}
\foreach \c/\t in {X/C,Y/B,Z/A,W/D}
{\tkzFillCircle[white](\c,\t)}
  \foreach \c/\t in {S/C,R/B,U/A,T/D}
  {\tkzFillCircle[teal!20](\c,\t)}
\end{tikzpicture}
\end{tkzexample}

\subsubsection{Recorte e preenchimento parte II}
\begin{tkzexample}[latex=7cm, small]
\begin{tikzpicture}[scale=.75]
\tkzDefPoints{0/0/A,8/0/B,8/8/C,0/8/D}
\tkzDefMidPoint(A,B) \tkzGetPoint{F}
\tkzDefMidPoint(B,C) \tkzGetPoint{E}
\tkzDefMidPoint(D,B) \tkzGetPoint{I}
\tkzDefMidPoint(I,B) \tkzGetPoint{a}
\tkzInterLC(B,I)(B,C) \tkzGetSecondPoint{K}
\tkzDefMidPoint(I,K) \tkzGetPoint{b}
\begin{scope}
  \tkzFillSector[fill=blue!10](B,C)(A)
  \tkzDefMidPoint(A,B) \tkzGetPoint{x}
  \tkzDrawSemiCircle[fill=white](x,B)
  \tkzDefMidPoint(B,C) \tkzGetPoint{y}
  \tkzDrawSemiCircle[fill=white](y,C)
  \tkzClipCircle(E,B)
  \tkzClipCircle(F,B)
  \tkzFillCircle[fill=blue!10](B,A)
\end{scope}
\tkzDrawSemiCircle[thick](F,B)
\tkzDrawSemiCircle[thick](E,C)
\tkzDrawArc[thick](B,C)(A)
\tkzDrawSegments[thick](A,B B,C)
\tkzDrawPoints(A,B,C,E,F)
\tkzLabelPoints[centered](a,b)
\tkzLabelPoints(A,B,C,E,F)
\end{tikzpicture}
\end{tkzexample}

\subsubsection{Recorte e preenchimento parte III}

\begin{tkzexample}[latex=7cm, small]
\begin{tikzpicture}
  \tkzDefPoint(0,0){A} \tkzDefPoint(1,0){B}
  \tkzDefPoint(2,0){C} \tkzDefPoint(-3,0){a}
  \tkzDefPoint(3,0){b}  \tkzDefPoint(0,3){c}
  \tkzDefPoint(0,-3){d}
\begin{scope}
 \tkzClipPolygon(a,b,c,d)
 \tkzFillCircle[teal!20](A,C)
\end{scope}
 \tkzFillCircle[white](A,B)
 \tkzDrawCircle[color=red](A,C)
 \tkzDrawCircle[color=red](A,B)
\end{tikzpicture}
\end{tkzexample}

\subsection{Colorindo um polígono}
 \begin{NewMacroBox}{tkzFillPolygon}{\oarg{opções locais}\parg{lista de pontos}}%
Você pode colorir desenhando o polígono, mas neste caso você colore o interior do polígono sem desenhá-lo.

\medskip
\begin{tabular}{lll}%
\toprule
argumentos                & exemplo & explicação                         \\
\midrule
\TAline{\parg{pt1,pt2,\dots}}{\parg{A,B,\dots}}{}
%\bottomrule
 \end{tabular}
\end{NewMacroBox}

\subsubsection{\tkzcname{tkzFillPolygon}}
\begin{tkzexample}[latex=7cm, small]
\begin{tikzpicture}[scale=.5]
   \tkzDefPoint(0,0){C} \tkzDefPoint(4,0){A}
   \tkzDefPoint(0,3){B}
   \tkzDefSquare(B,A)      \tkzGetPoints{E}{F}
   \tkzDefSquare(A,C)      \tkzGetPoints{G}{H}
   \tkzDefSquare(C,B)       \tkzGetPoints{I}{J}
   \tkzFillPolygon[color  =  orange!30   ](A,C,G,H)
   \tkzFillPolygon[color  =  teal!40  ](C,B,I,J)
   \tkzFillPolygon[color  =  purple!20](B,A,E,F)
   \tkzDrawPolygon[line width  =  1pt](A,B,C)
   \tkzDrawPolygon[line width  =  1pt](A,C,G,H)
   \tkzDrawPolygon[line width  =  1pt](C,B,I,J)
   \tkzDrawPolygon[line width  =  1pt](B,A,E,F)
   \tkzLabelSegment[above](C,A){$a$}
   \tkzLabelSegment[right](B,C){$b$}
   \tkzLabelSegment[below left](B,A){$c$}
\end{tikzpicture}
\end{tkzexample}

\subsection{\tkzcname{tkzFillSector}}
\tkzHandBomb\ Atenção: os argumentos variam de acordo com as opções.
\begin{NewMacroBox}{tkzFillSector}{\oarg{opções locais}\parg{O,\dots}\parg{\dots}}%
\begin{tabular}{lll}%
opções          & padrão & definição      \\
\midrule
\TOline{towards}{towards}{$O$ é o centro e o arco de $A$ a $(OB)$}
\TOline{rotate} {towards}{o arco começa de A e o ângulo determina seu comprimento }
\TOline{R}{towards}{Damos o raio e dois ângulos}
\TOline{R with nodes}{towards}{Damos o raio e dois pontos}
\bottomrule
\end{tabular}

\medskip
Claro, você tem que adicionar todos os estilos do \TIKZ\ para os traçados...

\medskip
\begin{tabular}{lll}%
\toprule
opções             & argumentos & exemplo                         \\
\midrule
\TOline{towards}{\parg{pt,pt}\parg{pt}}{\tkzcname{tkzFillSector(O,A)(B)}}
\TOline{rotate} {\parg{pt,pt}\parg{an}}{\tkzcname{tkzFillSector[rotate,color=red](O,A)(90)}}
\TOline{R}{\parg{pt,$r$}\parg{an,an}}{\tkzcname{tkzFillSector[R,color=blue](O,2)(30,90)}}
\TOline{R with nodes}{\parg{pt,$r$}\parg{pt,pt}}{\tkzcname{tkzFillSector[R with nodes](O,2)(A,B)}}
\end{tabular}
\end{NewMacroBox}

\subsubsection{\tkzcname{tkzFillSector} e \tkzname{towards}}
É inútil colocar \tkzname{towards} e você notará que os contornos não são desenhados, apenas a superfície é colorida.
\begin{tkzexample}[latex=5.75cm,small]
  \begin{tikzpicture}[scale=.6]
  \tkzDefPoint(0,0){O}
  \tkzDefPoint(-30:3){A}
  \tkzDefPointBy[rotation = center O angle -60](A)
  \tkzFillSector[fill=purple!20](O,A)(tkzPointResult)
    \begin{scope}[shift={(-60:1)}]
    \tkzDefPoint(0,0){O}
    \tkzDefPoint(-30:3){A}
    \tkzDefPointBy[rotation = center O angle -60](A)
    \tkzGetPoint{A'}
    \tkzFillSector[color=teal!40](O,A')(A)
      \end{scope}
  \end{tikzpicture}
\end{tkzexample}


\subsubsection{\tkzcname{tkzFillSector} e \tkzname{rotate}}
\begin{tkzexample}[latex=5.75cm,small]
\begin{tikzpicture}[scale=1.5]
 \tkzDefPoint(0,0){O} \tkzDefPoint(2,2){A}
 \tkzFillSector[rotate,color=purple!20](O,A)(30)
 \tkzFillSector[rotate,color=teal!40](O,A)(-30)
\end{tikzpicture}
\end{tkzexample}

\subsection{Colorir um ângulo: \tkzcname{tkzFillAngle}}

A operação mais simples
\begin{NewMacroBox}{tkzFillAngle}{\oarg{opções locais}\parg{A,O,B}}%
$O$ é o vértice do ângulo. $OA$ e $OB$ são os lados. Atenção: o ângulo é determinado pela ordem dos pontos.

\medskip

\begin{tabular}{lll}%
\toprule
opções             & padrão & definição                        \\
\midrule
\TOline{size}{1}{esta opção determina o raio do setor angular colorido.}

\bottomrule
\end{tabular}

\medskip
Claro, você tem que adicionar todos os estilos do \TIKZ, como o uso de fill e shade...
\end{NewMacroBox}

\subsubsection{Exemplo com \tkzname{size}}
\begin{tkzexample}[latex=7cm,small]
\begin{tikzpicture}
   \tkzInit
   \tkzDefPoints{0/0/O,2.5/0/A,1.5/2/B}
   \tkzFillAngle[size=2, fill=gray!10](A,O,B)
   \tkzDrawLines(O,A O,B)
   \tkzDrawPoints(O,A,B)
\end{tikzpicture}
\end{tkzexample}


\subsubsection{Alterando a ordem dos itens}
\begin{tkzexample}[latex=7cm,small]
\begin{tikzpicture}
   \tkzInit
   \tkzDefPoints{0/0/O,2.5/0/A,1.5/2/B}
   \tkzFillAngle[size=2,fill=gray!10](B,O,A)
   \tkzDrawLines(O,A O,B)
   \tkzDrawPoints(O,A,B)
\end{tikzpicture}
\end{tkzexample}

\begin{tkzexample}[latex=7cm,small]
\begin{tikzpicture}
   \tkzInit
   \tkzDefPoints{0/0/O,5/0/A,3/4/B}
   % Don't forget {} to get, () to use
   \tkzFillAngle[size=4,left color=white,
                 right color=red!50](A,O,B)
   \tkzDrawLines(O,A O,B)
   \tkzDrawPoints(O,A,B)
\end{tikzpicture}
\end{tkzexample}

\begin{NewMacroBox}{tkzFillAngles}{\oarg{opções locais}\parg{A,O,B}\parg{A',O',B'}etc.}%
Com opções comuns, há uma macro para múltiplos ângulos.
  \end{NewMacroBox}

\subsubsection{Múltiplos ângulos}
\begin{tkzexample}[latex=5cm,small]
\begin{tikzpicture}[scale=0.5]
  \tkzDefPoints{0/0/B,8/0/C,0/8/A,8/8/D}
  \tkzDrawPolygon(B,C,D,A)
  \tkzDefTriangle[equilateral](B,C) \tkzGetPoint{M}
  \tkzInterLL(D,M)(A,B) \tkzGetPoint{N}
  \tkzDefPointBy[rotation=center N angle -60](D)
  \tkzGetPoint{L}
  \tkzInterLL(N,L)(M,B)     \tkzGetPoint{P}
  \tkzInterLL(M,C)(D,L)     \tkzGetPoint{Q}
  \tkzDrawSegments(D,N N,L L,D B,M M,C)
  \tkzDrawPoints(L,N,P,Q,M,A,D)
  \tkzLabelPoints[left](N,P,Q)
  \tkzLabelPoints[above](M,A,D)
  \tkzLabelPoints(L,B,C)
  \tkzMarkAngles(C,B,M B,M,C M,C,B D,L,N L,N,D N,D,L)
  \tkzFillAngles[fill=red!20,opacity=.2](C,B,M%
      B,M,C M,C,B D,L,N L,N,D N,D,L)
\end{tikzpicture}
\end{tkzexample}
\endinput
