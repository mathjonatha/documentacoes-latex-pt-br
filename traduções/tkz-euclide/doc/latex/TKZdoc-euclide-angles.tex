\section{Ângulos}
\subsection{Definição e uso com \tkzname{tkz-euclide}}
Na geometria euclidiana, um ângulo é a figura formada por dois raios, chamados de lados do ângulo, compartilhando um ponto final comum, chamado de vértice do ângulo.[Wikipedia]. Um raio com \tkzname{tkz-euclide} é definido por dois pontos, assim cada ângulo é definido com três pontos como $\widehat{AOB}$. O vértice $O$ é o segundo ponto. A ordem é importante porque presume-se que o ângulo é especificado na ordem direta (sentido anti-horário).
Na trigonometria e na matemática em geral, os ângulos planos são convencionalmente medidos no sentido anti-horário, começando com $0^\circ$ apontando diretamente para a direita (ou leste), e $90^\circ$ apontando diretamente para cima (ou norte)[Wikipedia].
Vamos concordar que um ângulo medido no sentido anti-horário é positivo.

  \begin{center}
    \begin{tikzpicture}[scale=.75]
      \node {horário};
      \tkzDefPoint(0,0){O} \tkzDefPoint(90:2){A}\tkzDefPoint(180:2){B}
      \tkzDrawArc[black,line width=2pt,arrows = {Stealth-}](O,B)(A)
    \end{tikzpicture}
    \begin{tikzpicture}[scale=.75]
          \node {anti-horário};
      \tkzDefPoint(0,0){O} \tkzDefPoint(90:2){A}\tkzDefPoint(0:2){B}
      \tkzDrawArc[black,line width=2pt,arrows = {-Stealth}](O,A)(B)
    \end{tikzpicture}
  \end{center}

 \tkzname{Ângulos} estão envolvidos em várias macros como \tkzcname{tkzDefPoint},\tkzcname{tkzDefPointBy[rotation = \dots]}, \tkzcname{tkzDrawArc}
 e a próxima \tkzcname{tkzGetAngle}. Com exceção da última, todas essas macros aceitam ângulos negativos.

 \begin{figure}[!ht]
 \centering
 \begin{tabular}{|c|c|}
 \hline
 \tkzsubf{\begin{tikzpicture}
 \tkzDefPoint(0,0){O}    \tkzDefPoint(0:2){A}
 \tkzDefPointBy[rotation=center O angle 80](A)  \tkzGetPoint{B}
 \tkzDrawSegments[-{Stealth}](O,A O,B)
 \tkzMarkAngles[size=2,-Stealth,teal](A,O,B)
 \tkzFindAngle(A,O,B)   \tkzGetAngle{an}
 \tkzLabelAngle[pos=1,teal](A,O,B){$ \pgfmathprintnumber{\an}^\circ$}
 \tkzLabelPoints(A)  \tkzLabelPoints[above](B)
 \end{tikzpicture}}
      {Rotação $80^\circ$ de $(O,A)$ para $(O,B)$\\
    {\textbackslash}tkzDefPointBy[rotation=center O angle 80]}
 &
 \tkzsubf{\begin{tikzpicture}
 \tkzDefPoint(0,0){O}    \tkzDefPoint(0:2){A}
 \tkzDefPointBy[rotation=center O angle -80](A)  \tkzGetPoint{B}
 \tkzDrawSegments[-{Stealth}](O,A O,B)
 \tkzMarkAngles[size=2,Stealth-,red](B,O,A)
 \tkzFindAngle(B,O,A)   \tkzGetAngle{an}
 \tkzLabelAngle[pos=1,red](B,O,A){$-\pgfmathprintnumber{\an}^\circ$}
\tkzLabelPoints[right](A)  \tkzLabelPoints[below](B)
 \end{tikzpicture}}
  {Rotação $-80^\circ$ de $(O,A)$ para $(O,B)$\\
     {\textbackslash}tkzDefPointBy[rotation=center O angle -80]}
 \\ \hline
 \tkzsubf{\begin{tikzpicture}
 \tkzDefPoint(0,0){O}    \tkzDefPoint(0:2){A}
 \tkzDefPointBy[rotation=center O angle 80](A)  \tkzGetPoint{B}
 \tkzDrawSegments[-{Stealth}](O,A O,B)
 \tkzMarkAngles[size=1.5,-Stealth,teal](A,O,B)
 \tkzFindAngle(A,O,B)   \tkzGetAngle{an}
 \tkzLabelAngle[pos=1,teal](A,O,B){$ \pgfmathprintnumber{\an}^\circ$}
\tkzLabelPoints(A)  \tkzLabelPoints[above](B)
 \end{tikzpicture}}
      { {\textbackslash}tkzFindAngle(A,O,B) resulta em $80$}
 &
 \tkzsubf{\begin{tikzpicture}
 \tkzDefPoint(0,0){O}    \tkzDefPoint(0:2){A}
 \tkzDefPointBy[rotation=center O angle -80](A)  \tkzGetPoint{B}
 \tkzDrawSegments[-{Stealth}](O,A O,B)
 \tkzMarkAngles[size=1,-Stealth,red](A,O,B)
 \tkzFindAngle(A,O,B)   \tkzGetAngle{an}
 \tkzLabelAngle[pos=.75,red](A,O,B){$\pgfmathprintnumber{\an}^\circ$}
\tkzLabelPoints[right](A)  \tkzLabelPoints[below](B)
 \end{tikzpicture}}
  {{\textbackslash}tkzFindAngle(A,O,B) resulta em $\pgfmathprintnumber{\an}^\circ$}
 \\\hline
 \end{tabular}
 \end{figure}

Como podemos ver, a rotação $-80^\circ$ define um ângulo horário mas a macro
\tkzcname{tkzFindAngle} recupera um ângulo anti-horário.

\subsection{Recuperando um ângulo \tkzcname{tkzGetAngle}}
\begin{NewMacroBox}{tkzGetAngle}{\parg{nome da macro}}%
Atribui o valor em graus de um ângulo a uma macro. O valor é positivo e está entre $0^\circ$ e $360^\circ$. Esta macro recupera \tkzcname{tkzAngleResult} e armazena o resultado em uma nova macro.

\medskip

\begin{tabular}{lll}%
\toprule
argumentos             & exemplo & explicação             \\
\midrule
\TAline{nome da macro} {\tkzcname{tkzGetAngle}\{ang\}}{\tkzcname{ang} contém o valor do ângulo.}
\end{tabular}
\end{NewMacroBox}

Esta é uma macro auxiliar que permite recuperar o resultado da seguinte macro \tkzcname{tkzFindAngle}.

\subsection{Ângulo formado por três pontos}

\begin{NewMacroBox}{tkzFindAngle}{\parg{pt1,pt2,pt3}}%
O resultado é armazenado em uma macro \tkzcname{tkzAngleResult}.

\medskip

\begin{tabular}{lll}%
\toprule
argumentos     & exemplo & explicação     \\
\midrule
\TAline{(pt1,pt2,pt3)} {\tkzcname{tkzFindAngle}(A,B,C)}{\tkzcname{tkzAngleResult} fornece o ângulo ($\overrightarrow{BA},\overrightarrow{BC}$)}
\bottomrule
\end{tabular}

\medskip
A medida é sempre positiva e está entre $0^\circ$ e $360^\circ$. Com as convenções usuais, um ângulo anti-horário menor que um ângulo reto tem sempre uma medida entre $0^\circ$ e $180^\circ$, enquanto um ângulo horário menor que um ângulo reto terá uma medida maior que $180^\circ$. \tkzcname{tkzGetAngle} pode recuperar o ângulo.
\end{NewMacroBox}

\subsubsection{Verificação da medida de ângulo}

\begin{tkzexample}[latex=7cm,small]
\begin{tikzpicture}[scale=.75]
  \tkzDefPoint(-1,1){A}
  \tkzDefPoint(5,2){B}
  \tkzDefEquilateral(A,B)
  \tkzGetPoint{C}
  \tkzDrawPolygon(A,B,C)
  \tkzFindAngle(B,A,C) \tkzGetAngle{angleBAC}
  \edef\angleBAC{\fpeval{round(\angleBAC)}}
  \tkzDrawPoints(A,B,C)
  \tkzLabelPoints(A,B)
  \tkzLabelPoint[right](C){$C$}
  \tkzLabelAngle(B,A,C){\angleBAC$^\circ$}
  \tkzMarkAngle[size=1.5](B,A,C)
\end{tikzpicture}
\end{tkzexample}

\subsubsection{Determinação dos três ângulos de um triângulo}

\begin{tkzexample}[latex=6cm,small]
\begin{tikzpicture}
\tikzset{label angle style/.append style={pos=1.4}}
\tkzDefPoints{0/0/a,5/3/b,3/6/c}
\tkzDrawPolygon(a,b,c)
\tkzFindAngle(c,b,a)\tkzGetAngle{angleCBA}
\pgfmathparse{round(1+\angleCBA)}
\let\angleCBA\pgfmathresult
\tkzFindAngle(a,c,b)\tkzGetAngle{angleACB}
\pgfmathparse{round(\angleACB)}
\let\angleACB\pgfmathresult
\tkzFindAngle(b,a,c)\tkzGetAngle{angleBAC}
\pgfmathparse{round(\angleBAC)}
\let\angleBAC\pgfmathresult
\tkzMarkAngle(c,b,a)
\tkzLabelAngle(c,b,a){\tiny $\angleCBA^\circ$}
\tkzMarkAngle(a,c,b)
\tkzLabelAngle(a,c,b){\tiny $\angleACB^\circ$}
\tkzMarkAngle(b,a,c)
\tkzLabelAngle(b,a,c){\tiny $\angleBAC^\circ$}
\end{tikzpicture}
\end{tkzexample}

\subsubsection{Ângulo entre dois círculos}
Estamos procurando o ângulo formado pelas tangentes em um ponto de interseção

\begin{tkzexample}[latex=7cm,small]
\begin{tikzpicture}[scale=.4]
\pgfkeys{/pgf/number format/.cd,%
          fixed,precision=1}
\tkzDefPoints{0/0/A,6/0/B,4/2/C}
\tkzDrawCircles(A,C B,C)
\tkzDefLine[tangent at=C](A) \tkzGetPoint{a}
\tkzDefPointsBy[symmetry = center C](a){d}
\tkzDefLine[tangent at=C](B) \tkzGetPoint{b}
\tkzDrawLines[add=1 and 4](a,C  C,b)
\tkzFillAngle[fill=teal,opacity=.2%
                        ,size=2](b,C,d)
\tkzFindAngle(b,C,d)\tkzGetAngle{bcd}
\tkzLabelAngle[pos=1.25](b,C,d){%
  \tiny $\pgfmathprintnumber{\bcd}^\circ$}
\end{tikzpicture}
\end{tkzexample}

\subsection{Ângulo formado por uma reta com o eixo horizontal \tkzcname{tkzFindSlopeAngle}}
Muito mais interessante que a última. O resultado está entre -180 graus e +180 graus.

\begin{NewMacroBox}{tkzFindSlopeAngle}{\parg{A,B}}%
Determina a inclinação da reta (AB). O resultado é armazenado em uma macro \tkzcname{tkzAngleResult}.

\medskip
\begin{tabular}{lll}%
\toprule
argumentos  & exemplo & explicação     \\
\midrule
\TAline{(pt1,pt2)} {\tkzcname{tkzFindSlopeAngle}(A,B)}{}
\bottomrule
\end{tabular}

\medskip
\tkzcname{tkzGetAngle} pode recuperar o resultado. Se a recuperação não for necessária, você pode usar \tkzcname{tkzAngleResult}.
\end{NewMacroBox}

\subsubsection{Como usar \tkzcname{tkzFindSlopeAngle}}

 O ponto aqui é que $(AB)$ é a bissetriz de $\widehat{CAD}$, tal que a inclinação de $AD$ é zero. Recuperamos a inclinação de $(AB)$ e então rotacionamos duas vezes.

\begin{tkzexample}[latex=7cm,small]
\begin{tikzpicture}
 \tkzDefPoint(1,5){A} \tkzDefPoint(5,2){B}
 \tkzFindSlopeAngle(A,B)\tkzGetAngle{tkzang}
 \tkzDefPointBy[rotation= center A angle \tkzang ](B)
 \tkzGetPoint{C}
 \tkzDefPointBy[rotation= center A angle -\tkzang ](B)
 \tkzGetPoint{D}
 \tkzDrawSegment(A,B)
 \tkzDrawSegments[new](A,C A,D)
 \tkzDrawPoints(A,B,C,D)
 \tkzCompass[length=1](A,C)
 \tkzCompass[delta=10,brown](B,C)
 \tkzLabelPoints(B,C,D)
 \tkzLabelPoints[above left](A)
\end{tikzpicture}
\end{tkzexample}

\subsubsection{Uso de \tkzcname{tkzFindSlopeAngle} e \tkzcname{tkzGetAngle}}
Aqui está outra versão da construção de uma mediatriz

\begin{tkzexample}[latex=6cm,small]
\begin{tikzpicture}
 \tkzInit
 \tkzDefPoint(0,0){A}        \tkzDefPoint(3,2){B}
 \tkzDefLine[mediator](A,B)  \tkzGetPoints{I}{J}
 \tkzCalcLength(A,B)         \tkzGetLength{dAB}
 \tkzFindSlopeAngle(A,B)     \tkzGetAngle{tkzangle}
 \begin{scope}[rotate=\tkzangle]
   \tkzSetUpArc[color=gray,line width=0.2pt,%
     /tkzcompass/delta=10]
   \tkzDrawArc[R,arc](B,3/4*\dAB)(120,240)
   \tkzDrawArc[R,arc](A,3/4*\dAB)(-45,60)
   \tkzDrawLine(I,J)         \tkzDrawSegment(A,B)
  \end{scope}
  \tkzDrawPoints(A,B,I,J)    \tkzLabelPoints(A,B)
   \tkzLabelPoints[right](I,J)
\end{tikzpicture}
\end{tkzexample}

\subsubsection{Outro uso de \tkzcname{tkzFindSlopeAngle}}

\begin{tkzexample}[latex=7cm,small]
\begin{tikzpicture}[scale=1.5]
  \tkzDefPoint(1,2){A}    \tkzDefPoint(3,4){B}
  \tkzDefPoint(3,2){C}    \tkzDefPoint(3,1){D}
  \tkzDrawSegments(A,B A,C A,D)
  \tkzDrawPoints[color=red](A,B,C,D)
  \tkzLabelPoints(A,B,C,D)
  \tkzFindSlopeAngle(A,B)\tkzGetAngle{SAB}
  \tkzFindSlopeAngle(A,C)\tkzGetAngle{SAC}
  \tkzFindSlopeAngle(A,D)\tkzGetAngle{SAD}
  \pgfkeys{/pgf/number format/.cd,fixed,precision=2}
  \tkzText(1,5){A inclinação de (AB) é:
     $\pgfmathprintnumber{\SAB}^\circ$}
  \tkzText(1,4.5){A inclinação de (AC) é:
     $\pgfmathprintnumber{\SAC}^\circ$}
  \tkzText(1,4){A inclinação de (AD) é:
     $\pgfmathprintnumber{\SAD}^\circ$}
\end{tikzpicture}
\end{tkzexample}

\endinput
