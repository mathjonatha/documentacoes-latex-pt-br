\section{Retas}

É claro que é essencial desenhar retas, mas antes que isso possa ser feito, é necessário poder definir certas retas particulares como mediatrizes, bissetrizes, paralelas ou mesmo perpendiculares. O princípio é determinar dois pontos na reta.

\subsection{Definição de retas}

\begin{NewMacroBox}{tkzDefLine}{\oarg{local opções}\parg{pt1,pt2} ou \parg{pt1,pt2,pt3}}%
O argumento é uma lista de dois ou três pontos. Dependendo do caso, a macro define um ou dois pontos necessários para obter a reta procurada. Deve ser usada a macro \tkzcname{tkzGetPoint} ou a macro \tkzcname{tkzGetPoints}.
Usei o termo "mediatriz" para designar a reta bissetriz perpendicular no meio de um segmento de reta.

\medskip
\begin{tabular}{lll}%
\toprule
argumentos           & exemplo & explicação                         \\
\midrule
\TAline{\parg{pt1,pt2}}{[mediator]\parg{A,B}}{mediatriz do segmento $[A,B]$}
\TAline{\parg{pt1,pt2,pt3}}{[bisector]\parg{A,B,C}} {bissetriz de $\widehat{ABC}$}
\TAline{\parg{pt1,pt2,pt3}}{[altitude]\parg{A,B,C}} {altura de $B$}
\TAline{\parg{pt1}}{[tangent at=A]\parg{O}} {tangente em $A$ no círculo de centro $O$}
\TAline{\parg{pt1,pt2}}{[tangent from=A]\parg{O,B}} {círculo de centro $O$ passando por $B$}
\end{tabular}

\medskip
\begin{tabular}{lll}%
\toprule
opções             & padrão & definição                         \\ 
\TOline{mediator}{}{bissetriz perpendicular de um segmento de reta}
\TOline{perpendicular=through\dots}{mediator}{perpendicular a uma reta passando por um ponto}
\TOline{orthogonal=through\dots}{mediator}{veja acima}
\TOline{parallel=through\dots}{mediator}{paralela a uma reta passando por um ponto}
\TOline{bisector}{mediator}{bissetriz de um ângulo definido por três pontos}
\TOline{bisector out}{mediator}{bissetriz do ângulo externo}
\TOline{symmedian}{mediator}{simediana de um vértice}
\TOline{altitude}{mediator}{altura de um vértice}
\TOline{euler}{mediator}{reta de euler de um triângulo}
\TOline{tangent at}{mediator}{tangente em um ponto de um círculo}
\TOline{tangent from}{mediator}{tangente de um ponto exterior}
\TOline{K}{1}{coeficiente para a reta perpendicular}
\TOline{normed}{false}{normaliza o segmento criado}
\end{tabular}
\end{NewMacroBox}  

\subsubsection{With \tkzname{mediator}}  
\begin{tkzexample}[latex=5 cm,small]
\begin{tikzpicture}[rotate=25]
 \tkzDefPoints{-2/0/A,1/2/B}
 \tkzDefLine[mediator](A,B)          \tkzGetPoints{C}{D}
 \tkzDefPointWith[linear,K=.75](C,D) \tkzGetPoint{D}
 \tkzDefMidPoint(A,B)                \tkzGetPoint{I}
 \tkzFillPolygon[color=teal!20](A,C,B,D)
 \tkzDrawSegments(A,B C,D)
 \tkzMarkRightAngle(B,I,C) 
 \tkzDrawSegments(D,B D,A)
 \tkzDrawSegments(C,B C,A)
\end{tikzpicture}
\end{tkzexample}  

\subsubsection{Um envelope com opção \tkzname{mediator}}
Baseado em uma figura de O. Reboux com pst-eucl por D Rodriguez.

\begin{tkzexample}[latex=7cm,small]
\begin{tikzpicture}[scale=.75]
   % necessary
\tkzInit[xmin=-6,ymin=-4,xmax=6,ymax=6]
\tkzClip
\tkzSetUpLine[thin,color=magenta]
\tkzDefPoint(0,0){O} 
\tkzDefPoint(132:4){A}
\tkzDefPoint(5,0){B}
\foreach \ang in {5,10,...,360}{%
 \tkzDefPoint(\ang:5){M}
 \tkzDefLine[mediator](A,M)
 \tkzGetPoints{x}{y}
 \tkzDrawLine[add= 3 and 3](x,y)}
\end{tikzpicture}
\end{tkzexample}


\subsubsection{Uma parábola com opção \tkzname{mediator}}
Baseado em uma figura de O. Reboux com pst-eucl por D Rodriguez.
Não é necessário nomear os dois pontos que definem a mediatriz.

\begin{tkzexample}[latex=8cm,small]
\begin{tikzpicture}[scale=.6]
\tkzInit[xmin=-6,ymin=-4,xmax=6,ymax=6] 
\tkzClip
\tkzSetUpLine[thin,color=teal]
\tkzDefPoint(0,0){O} 
\tkzDefPoint(132:5){A}
\tkzDefPoint(4,0){B}
\foreach \ang in {5,10,...,360}{%
 \tkzDefPoint(\ang:4){M}
 \tkzDefLine[mediator](A,M) 
 \tkzGetPoints{x}{y}
 \tkzDrawLine[add= 3 and 3](x,y)}
\end{tikzpicture}
\end{tkzexample}

\subsubsection{With opções \tkzname{bisector} and \tkzname{normed}} 
\begin{tkzexample}[latex=7 cm,small] 
\begin{tikzpicture}[rotate=25,scale=.75]
 \tkzDefPoints{0/0/C, 2/-3/A, 4/0/B}
 \tkzDefLine[bisector,normed](B,A,C) \tkzGetPoint{a}
 \tkzDrawLines[add= 0 and .5](A,B A,C)
 \tkzShowLine[bisector,gap=4,size=2,color=red](B,A,C)
 \tkzDrawLines[new,dashed,add= 0 and 3](A,a)
\end{tikzpicture}
\end{tkzexample} 

\subsubsection{Com opção \tkzname{parallel=through}} % (fold)
\label{ssub:parallel}
Livro de Lemas de Arquimedes proposição 1

\begin{tkzexample}[latex=7cm,small]
  \begin{tikzpicture}
    \tkzDefPoints{0/0/O_1,0/1/O_2,0/3/A}
    \tkzDefPoint(15:3){F}
    \tkzDefPointBy[symmetry=center O_1](F) 
    \tkzGetPoint{E}
    \tkzDefLine[parallel=through O_2](E,F) 
    \tkzGetPoint{x}   
    \tkzInterLC(x,O_2)(O_2,A) \tkzGetPoints{D}{C} 
    \tkzDrawCircles(O_1,A O_2,A)
    \tkzDrawSegments[new](O_1,A E,F C,D)
    \tkzDrawSegments[purple](A,E A,F)
    \tkzDrawPoints(A,O_1,O_2,E,F,C,D)
    \tkzLabelPoints(A,O_1,O_2,E,F,C,D)
  \end{tikzpicture}
\end{tkzexample}
% subsubsection parallel (end)

\subsubsection{With opção \tkzname{orthogonal} and \tkzname{parallel}}    
\begin{tkzexample}[latex=5 cm,small]
\begin{tikzpicture}
   \tkzDefPoints{-1.5/-0.25/A,1/-0.75/B,-0.7/1/C}
   \tkzDrawLine(A,B)
   \tkzLabelLine[pos=1.25,below left](A,B){$(d_1)$}
   \tkzDrawPoints(A,B,C)
   \tkzDefLine[orthogonal=through C](B,A) \tkzGetPoint{c}
   \tkzDrawLine(C,c) 
   \tkzLabelLine[pos=1.25,left](C,c){$(\delta)$}
   \tkzInterLL(A,B)(C,c) \tkzGetPoint{I}
   \tkzMarkRightAngle(C,I,B) 
   \tkzDefLine[parallel=through C](A,B) \tkzGetPoint{c'}
   \tkzDrawLine(C,c') 
   \tkzLabelLine[pos=1.25,below left](C,c'){$(d_2)$}
   \tkzMarkRightAngle(I,C,c')   
\end{tikzpicture}
\end{tkzexample}

\subsubsection{With opção  \tkzname{altitude}} % (fold)
\label{sub:altitude}
\begin{tkzexample}[latex=7 cm,small]
\begin{tikzpicture}
\tkzDefPoints{0/0/A,6/0/B,0.8/4/C}	
\tkzDefLine[altitude](A,B,C)     \tkzGetPoint{b}
\tkzDefLine[altitude](B,C,A)     \tkzGetPoint{c}
\tkzDefLine[altitude](B,A,C)     \tkzGetPoint{a}
\tkzDrawPolygon(A,B,C)
\tkzDrawPoints[blue](a,b,c)
\tkzDrawSegments[blue](A,a B,b C,c)
\tkzLabelPoints(A,B,c)
\tkzLabelPoints[above](C,a)
\tkzLabelPoints[above left](b)
\end{tikzpicture}
\end{tkzexample}

% subsection altitude (end)


\subsubsection{ With opção \tkzname{euler}} % (fold)
\label{sub:eulerline}
\begin{tkzexample}[latex=7 cm,small]
\begin{tikzpicture}[scale=.75]
\tkzDefPoints{0/0/A,6/0/B,0.8/4/C}			 
\tkzDefLine[euler](A,B,C)             
\tkzGetPoints{h}{e}
\tkzDefTriangleCenter[circum](A,B,C)  
\tkzGetPoint{o}
\tkzDrawPolygon[teal](A,B,C)
\tkzDrawPoints[red](A,B,C,h,e,o)
\tkzDrawLine[add= 2 and 2](h,e)
\tkzLabelPoints(A,B,C,h,e,o)
\tkzLabelPoints[above](C)
\end{tikzpicture}
\end{tkzexample}
% subsection eulerline (end)

\subsubsection{Tangente passando por um ponto no círculo \tkzname{tangent at}} 
\begin{tkzexample}[latex=7cm,small]
\begin{tikzpicture}[scale=.75]
  \tkzDefPoint(0,0){O}
  \tkzDefPoint(6,6){E}
  \tkzDefRandPointOn[circle=center O radius 3]
  \tkzGetPoint{A}
  \tkzDrawSegment(O,A)
  \tkzDrawCircle(O,A)
  \tkzDefLine[tangent at=A](O)
  \tkzGetPoint{h}
  \tkzDrawLine[add = 4 and 3](A,h)
  \tkzMarkRightAngle[fill=teal!30](O,A,h)
\end{tikzpicture}
\end{tkzexample}

\subsubsection{Escolha do ponto de contato com tangentes passando por um ponto externo opção \tkzname{tangent from}}

A tangente não é desenhada. Com opção \tkzname{at}, um ponto da tangente é dado por \tkzname{tkzPointResult}. Com opção \tkzname{from} você obtém dois pontos do círculo com \tkzname{tkzFirstPointResult} e \tkzname{tkzSecondPointResult}. Você pode escolher entre esses dois pontos comparando os ângulos formados com o ponto externo, o ponto de contato e o centro. Os dois ângulos possíveis têm direções diferentes. O ângulo no sentido anti-horário refere-se a \tkzname{tkzFirstPointResult}.

\begin{tkzexample}[latex=7cm,small]
\begin{tikzpicture}[scale=1,rotate=-30]
\tkzDefPoints{0/0/Q,0/2/A,6/-1/O}
\tkzDefLine[tangent from = O](Q,A)  
\tkzGetPoints{R}{S} 
\tkzInterLC[near](O,Q)(Q,A)         
\tkzGetPoints{M}{N}
\tkzDrawCircle(Q,M)
\tkzDrawSegments[new,add = 0 and .2](O,R O,S)
\tkzDrawSegments[gray](N,O R,Q S,Q)
\tkzDrawPoints(O,Q,R,S,M,N)
\tkzMarkAngle[gray,-stealth,size=1](O,R,Q)
\tkzFindAngle(O,R,Q)   \tkzGetAngle{an}
\tkzLabelAngle(O,R,Q){%
    $\pgfmathprintnumber{\an}^\circ$}
\tkzMarkAngle[gray,-stealth,size=1](O,S,Q)
\tkzFindAngle(O,S,Q)   \tkzGetAngle{an}
\tkzLabelAngle(O,S,Q){%
    $\pgfmathprintnumber{\an}^\circ$}
\tkzLabelPoints(Q,O,M,N,R)
\tkzLabelPoints[above,text=red](S)
\end{tikzpicture}
\end{tkzexample}

\subsubsection{Exemplo de tangentes passando por um ponto externo} 
\begin{tkzexample}[latex=7cm,small]
\begin{tikzpicture}[scale=.8] 
\tkzDefPoints{0/0/c,1/0/d,3/0/a0}
\def\tkzRadius{1}
\tkzDrawCircle(c,d) 
 \foreach \an in {0,10,...,350}{
  \tkzDefPointBy[rotation=center c angle \an](a0)  
  \tkzGetPoint{a}
  \tkzDefLine[tangent from = a](c,d) 
  \tkzGetPoints{e}{f}
  \tkzDrawLines(a,f a,e)
  \tkzDrawSegments(c,e c,f)}
\end{tikzpicture} 
\end{tkzexample}

\subsubsection{Exemplo of Andrew Mertz}

\begin{tkzexample}[latex=6cm,small]
 \begin{tikzpicture}[scale=.6] 
 \tkzDefPoint(100:8){A}\tkzDefPoint(50:8){B}  
 \tkzDefPoint(0,0){C} \tkzDefPoint(0,-4){R} 
 \tkzDrawCircle(C,R)
 \tkzDefLine[tangent from = A](C,R)  \tkzGetPoints{D}{E}
\tkzDefLine[tangent from = B](C,R)  \tkzGetPoints{F}{G}
 \tkzDrawSector[fill=teal!20,opacity=0.5](A,E)(D)
 \tkzFillSector[color=teal,opacity=0.5](B,G)(F)
 \end{tikzpicture}
\end{tkzexample}
\url{http://www.texemplo.net/tikz/exemplos/}  

\subsubsection{Desenhando uma tangente opção \tkzname{tangent from}}
\begin{tkzexample}[latex=6cm,small]
\begin{tikzpicture}[scale=.6] 
 \tkzDefPoint(0,0){B} 
 \tkzDefPoint(0,8){A} 
 \tkzDefSquare(A,B)
 \tkzGetPoints{C}{D}
 \tkzDrawPolygon(A,B,C,D)
 \tkzClipPolygon(A,B,C,D)
 \tkzDefPoint(4,8){F}
 \tkzDefPoint(4,0){E}
 \tkzDefPoint(4,4){Q}
 \tkzFillPolygon[color = green](A,B,C,D)
 \tkzDrawCircle[fill = orange](B,A)
 \tkzDrawCircle[fill = purple](E,B)  
 \tkzDefLine[tangent from = B](F,A)
 \tkzInterLL(F,tkzSecondPointResult)(C,D)
 \tkzInterLL(A,tkzPointResult)(F,E) 
 \tkzDrawCircle[fill = yellow](tkzPointResult,Q)  
 \tkzDefPointBy[projection= onto B--A](tkzPointResult)
 \tkzDrawCircle[fill = blue!50!black](tkzPointResult,A)
\end{tikzpicture}
\end{tkzexample}

\endinput