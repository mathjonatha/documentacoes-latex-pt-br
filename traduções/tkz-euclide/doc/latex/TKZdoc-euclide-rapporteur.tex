\section{Transferidor}
Baseado em uma ideia de Yves Combe, a seguinte macro permite desenhar um transferidor.
O princípio de operação é ainda mais simples. Basta nomear uma semirreta (um raio). O transferidor será colocado na origem $O$, a direção da semirreta é dada por $A$. O ângulo é medido na direção direta do círculo trigonométrico.
\subsection{A macro \tkzcname{tkzProtractor}} % (fold)
\label{sub:the_macro_tkzcname_tkzprotractor}

% subsection the_macro_tkzcname_tkzprotractor (end)
\begin{NewMacroBox}{tkzProtractor}{\oarg{opções locais}\parg{$O,A$}}%
\begin{tabular}{lll}%
opções    & padrão & definição     \\
\midrule
\TOline{lw}  {0.4 pt} {espessura da linha}
\TOline{scale}  {1} {razão: ajusta o tamanho do transferidor}
\TOline{return} {false} {círculo trigonométrico indireto}
\end{tabular}
\end{NewMacroBox}

\subsubsection{O transferidor circular}
Medindo na direção direta

\begin{tkzexample}[latex=7cm,small]
\begin{tikzpicture}[scale=.5]
\tkzDefPoint(2,0){A}\tkzDefPoint(0,0){O}
\tkzDefShiftPoint[A](31:5){B}
\tkzDefShiftPoint[A](158:5){C}
\tkzDrawPoints(A,B,C)
\tkzDrawSegments[color = red,
    line width = 1pt](A,B A,C)
  \tkzProtractor[scale = 1](A,B)
\end{tikzpicture}
\end{tkzexample}

\subsubsection{O transferidor circular, transparente e invertido}

\begin{tkzexample}[latex=7cm,small]
\begin{tikzpicture}[scale=.5]
  \tkzDefPoint(2,3){A}
  \tkzDefShiftPoint[A](31:5){B}
   \tkzDefShiftPoint[A](158:5){C}
  \tkzDrawSegments[color=red,line width=1pt](A,B A,C)
  \tkzProtractor[return](A,C)
\end{tikzpicture}
\end{tkzexample}
\endinput
